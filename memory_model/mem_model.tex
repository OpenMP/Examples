
%\pagebreak
\section{OpenMP Memory Model}
\label{sec:mem_model}

The following examples illustrate two major concerns for concurrent thread
execution: ordering of thread execution and memory accesses that may or may not
lead to race conditions.

In the following example, at Print 1, the value of \ucode{xval} could be either 2
or 5, depending on the timing of the threads. The \kcode{atomic} directives are
necessary for the accesses to \ucode{x} by threads 1 and 2 to avoid a data race.
If the atomic write completes before the atomic read, thread 1 is guaranteed to
see 5 in \ucode{xval}. Otherwise, thread 1 is guaranteed to see 2 in \ucode{xval}.

\index{flushes!implicit}
\index{atomic construct@\kcode{atomic} construct}
\index{constructs!atomic@\kcode{atomic}}
The barrier after Print 1 contains implicit flushes on all threads, as well as
a thread synchronization, so the programmer is guaranteed that the value 5 will
be printed by both Print 2 and Print 3. Since neither Print 2 or Print 3 are modifying
\ucode{x}, they may concurrently access \ucode{x} without requiring \kcode{atomic}
directives to avoid a data race.

\cexample[3.1]{mem_model}{1}

\ffreeexample[3.1]{mem_model}{1}

\pagebreak
\index{flushes!flush construct@\kcode{flush} construct}
\index{flush construct@\kcode{flush} construct}
\index{constructs!flush@\kcode{flush}}
The following example demonstrates why synchronization is difficult to perform
correctly through variables. The write to \ucode{flag} on thread 0 and the read
from \ucode{flag} in the loop on thread 1 must be atomic to avoid a data race.
When thread 1 breaks out of the loop, \ucode{flag} will have the value of 1.
However, \ucode{data} will still be undefined at the first print statement. Only
after the flush of both \ucode{flag} and \ucode{data} after the first print
statement will \ucode{data} have the well-defined value of 42.

\cexample[3.1]{mem_model}{2}

\fexample[3.1]{mem_model}{2}

\pagebreak
\index{flushes!flush with a list}
The next example demonstrates why synchronization is difficult to perform
correctly through variables. As in the preceding example, the updates to
\ucode{flag} and the reading of \ucode{flag} in the loops on threads 1 and 2 are
performed atomically to avoid data races on \ucode{flag}. However, the code still
contains data race due to the incorrect use of ``flush with a list'' after the
assignment to \ucode{data1} on thread 1. By not including \ucode{flag} in the
flush-set of that \kcode{flush} directive, the assignment can be reordered with
respect to the subsequent atomic update to \ucode{flag}. Consequentially,
\ucode{data1} is undefined at the print statement on thread 2.

\cexample[3.1]{mem_model}{3}

\fexample[3.1]{mem_model}{3}


The following two examples illustrate the ordering properties of 
the \plc{flush} operation. The \plc{flush} operations are strong flushes 
that are applied to the specified flush lists. 
However, use of a \kcode{flush} construct with a list is extremely error 
prone and users are strongly discouraged from attempting it. 
In the codes the programmer intends to prevent simultaneous 
execution of the protected section by the two threads.
The atomic directives in the codes ensure that the accesses to shared
variables \ucode{a} and \ucode{b} are atomic write and atomic read operations. Otherwise both examples would contain data races and automatically result 
in unspecified behavior. 

In the following incorrect code example, operations on variables \ucode{a} and
\ucode{b} are not ordered with respect to each other. For instance, nothing
prevents the compiler from moving the flush of \ucode{b} on thread 0 or the
flush of \ucode{a} on thread 1 to a position completely after the protected
section (assuming that the protected section on thread 0 does not reference
\ucode{b} and the protected section on thread 1 does not reference \ucode{a}).
If either re-ordering happens, both threads can simultaneously execute the
protected section.
Any shared data accessed in the protected section is not guaranteed to 
be current or consistent during or after the protected section. 

\cexample[3.1]{mem_model}{4a}
\ffreeexample[3.1]{mem_model}{4a}


The following code example correctly ensures that the protected section
is executed by only one thread at a time. Execution of the protected section
by neither thread is considered correct in this example. This occurs if both
flushes complete prior to either thread executing its \bcode{if} statement
for the protected section.
The compiler is prohibited from moving the flush at all for either thread,
ensuring that the respective assignment is complete and the data is flushed
before the \bcode{if} statement is executed.

\cexample[3.1]{mem_model}{4b}
\ffreeexample[3.1]{mem_model}{4b}


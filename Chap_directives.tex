\cchapter{OpenMP Directive Syntax}{directives}
\label{chap:directive_syntax}
\index{directive syntax}

OpenMP \plc{directives} use base-language mechanisms to specify OpenMP program behavior.
In C code, the directives are formed exclusively with pragmas, whereas in C++
code, directives are formed from either pragmas or attributes.
Fortran directives are formed with comments in free form and fixed form sources (codes).
All of these mechanisms allow the compilation to ignore the OpenMP directives if
OpenMP is not supported or enabled.


The OpenMP directive is a combination of the base-language mechanism and a \plc{directive-specification},
as shown below. The \plc{directive-specification} consists
of the \plc{directive-name} which may seldomly have arguments, 
followed by optional \plc{clauses}. Full details of the syntax can be found in the OpenMP Specification.
Illustrations of the syntax is given in the examples.

The formats for combining a base-language mechanism and a \plc{directive-specification} are:

C/C++ pragmas
\begin{indentedcodelist}
\kcode{\#pragma omp} \plc{directive-specification}
\end{indentedcodelist}

C++ attributes
\begin{indentedcodelist}
\kcode{[[omp :: directive( \plc{directive-specification} )]]}
\kcode{[[using omp : directive( \plc{directive-specification} )]]}
\end{indentedcodelist}

Fortran comments
\begin{indentedcodelist}
\scode{!$omp} \plc{directive-specification}
\end{indentedcodelist}

where \scode{c$omp} and \scode{*$omp} may be used in Fortran fixed form sources.

Most OpenMP directives accept clauses that alter the semantics of the directive in some way, 
and some directives also accept parenthesized arguments that follow the directive name. 
A clause may just be a keyword (e.g., \kcode{untied}) or it may also accept argument lists 
(e.g., \kcode{shared(\ucode{x,y,z})}) and/or optional modifiers (e.g., \kcode{tofrom} in 
\kcode{map(tofrom: \ucode{x,y,z})}).
Clause modifiers may be ``simple'' or ``complex'' -- a complex modifier consists of a
keyword followed by one or more parameters, bracketed by parentheses, while a simple 
modifier does not. An example of a complex modifier is the \kcode{iterator} modifier, 
as in \kcode{map(iterator(\ucode{i=0:n}), tofrom: \ucode{p[i]})}, or the \kcode{step} modifier, as in 
\kcode{linear(\ucode{x}: ref, step(\ucode{4}))}. 
In the preceding examples, \kcode{tofrom} and \kcode{ref} are simple modifiers.

For Fortran, a declarative directive (such as \kcode{declare reduction})
must appear after any \bcode{USE}, \bcode{IMPORT}, and \bcode{IMPLICIT} statements
in the specification part.


%===== Examples Sections =====
%\pagebreak
\section{C/C++ Pragmas}
\label{sec:pragmas}
\index{directive syntax!pragma, C/C++}
\index{pragma syntax, C/C++}

OpenMP C and C++ directives can be specified with the C/C++ \code{\#pragma} directive.
An OpenMP directive begins with \code{\#pragma}~\code{omp} and is followed by the 
OpenMP directive name, and required and optional clauses. Lines are continued in the 
usual manner, and comments may be included at the end.
Directives are case sensitive.

The example below illustrates the use of the OpenMP pragma form.
The first pragma (PRAG 1) specifies a combined \code{parallel}~\code{for} 
directive, with a \code{num\_threads} clause, and a comment. 
%The NT macro is expanded in the clause. 
The second pragma (PRAG 2) shows the same directive split
across two lines. The next nested pragmas (PRAG 3 and 4) show the previous combined directive as 
two separate directives. The executable directives above all apply to the next 
statement. The \code{parallel} directive can be applied to a \plc{structured}~\plc{block}
as shown in PRAG 5.

\cexample{directive_syntax_pragma}{1}

\section{C++ Attributes}
\label{sec:attributes}
\index{directive syntax!attribute, C++}
\index{attribute syntax, C++}

OpenMP directives for C++ can also be specified with 
%the implementation-defined 
the \code{directive} extension for the C++11 standard \plc{attributes}.
%https://en.cppreference.com/w/cpp/language/attributes

The C++ example below shows two ways to parallelize a \code{for} loop using the \code{\#pragma} syntax.
The first pragma uses the combined \code{parallel}~\code{for} directive, and the second
applies the uncombined closely nested directives, \code{parallel} and \code{for}, directly to the same statement. 
These are labeled PRAG 1-3.

Using the attribute syntax, the same construct in PRAG 1
is applied two different ways in attribute form, as shown in the ATTR 1 and ATTR 2 sections.
In ATTR 1 the attribute syntax is used with the \code{omp ::} namespace form.
In ATTR 2 the attribute syntax is used with the \code{using omp :} namespace form.

Next, parallelization is attempted by applying directives using two different syntaxes.
For ATTR 3 and PRAG 4, the loop parallelization will fail to compile because multiple directives that
apply to the same statement must all use either the attribute syntax or the pragma syntax.
The lines have been commented out and labeled INVALID.

While multiple attributes may be applied to the same statement,
compilation may fail if the ordering of the directive matters.
For the ATTR 4-5 loop parallelization, the \code{parallel} directive precedes 
the \code{for} directive, but the compiler may reorder consecutive attributes.
If the directives are reversed, compilation will fail.

The attribute directive of the ATTR 6 section resolves the previous problem (in ATTR 4-5).
Here, the \code{sequence} attribute is used to apply ordering to the
directives of ATTR 4-5, using the \code{omp}~\code{::} namespace qualifier. (The
\code{using omp :} namespace form is not available for the \code{sequence} attribute.) 
Note, for the \code{sequence} attribute a comma must separate the \code{directive} extensions.


The last 3 pairs of sections (PRAG DECL 1-2, 3-4, and 5-6) show cases where 
directive ordering does not matter for \code{declare}~\code{simd} directives. 

In section PRAG DECL 1-2, the two loops use different SIMD forms of the \plc{P} function
(one with \code{simdlen(4)} and the other with \code{simdlen(8)}), 
as prescribed by the two different \code{declare}~\code{simd} directives
applied to the \plc{P} function definitions (at the beginning of the code). 
The directives use the pragma syntax, and order is not important. 
For the next set of loops 
(PRAG DECL 3-4) that use the \plc{Q} function, the attribute syntax is 
used for the \code{declare}~\code{simd} directives. 
The result is compliant code since directive order is irrelevant.
Sections ATTR DECL 5-6 are included for completeness. Here, the attribute 
form of the \code{simd} directive is used for loops calling the \plc{Q} function, 
in combination with the attribute form of the \code{declare}~\code{simd} 
directives declaring the variants for \plc{Q}.

\cppexample[5.0]{directive_syntax_attribute}{1}

%\pagebreak
\section{Fortran Comments (Fixed Source Form)}
\label{sec:fortran_fixed_format_comments}
\index{directive syntax!fixed form, Fortran}
\index{fixed form syntax, Fortran}

OpenMP directives in Fortran codes with fixed source form are specified as comments with one of the
\scode{!$omp}, \scode{c$omp}, and \scode{*$omp} sentinels, followed by a
directive name, and required and optional clauses.  The sentinel must begin in column 1.

In the example below the first directive (DIR 1) specifies the %parallel work-sharing 
\kcode{parallel do} combined directive, with a \kcode{num_threads} clause, and a comment.
The second directive (DIR 2) shows the same directive split
across two lines. The next nested directives (DIR 3 and 4) show the previous combined directive as
two separate directives. 
Here, an \kcode{end} directive (\kcode{end parallel}) must be specified to demarcate the range (region)
of the \kcode{parallel} directive.

\fexample{directive_syntax_F_fixed_comment}{1}
\clearpage

%\pagebreak
\begin{fortranspecific}[4ex]
\section{Fortran Comments (Free Source Form)}
\label{sec:fortran_free_format_comments}
\index{directive syntax!free form, Fortran}
\index{free form syntax, Fortran}

OpenMP directives in Fortran codes with free source form are specified as comments
that use the \scode{!$omp} sentinel, followed by the
directive name, and required and optional clauses.  Lines are continued with an ending ampersand (\scode{&}),
and the continued line begins with \scode{!$omp} or \scode{!$omp&}. Comments may appear on the
same line as the directive.  Directives are case insensitive.

In the example below the first directive (DIR 1) specifies the %parallel work-sharing 
\kcode{parallel do} combined directive, with a \kcode{num_threads} clause, and a comment.
The second directive (DIR 2) shows the same directive split across two lines. 
The next nested directives (DIR 3 and 4) show the previous combined directive as
two separate directives. 
Here, an \kcode{end} directive (\kcode{end parallel}) must be specified to demarcate the range (region)
of the \kcode{parallel} directive. 

\ffreenexample{directive_syntax_F_free_comment}{1}
\clearpage

As of OpenMP 5.1, \bcode{block} and \bcode{end block} statements can be used to designate 
a structured block for an OpenMP region, and any paired OpenMP \kcode{end} directive becomes optional,
as shown in the next example.  Note, the variables \ucode{i} and \ucode{thrd_no} are declared within the 
block structure and are hence private.
It was necessary to explicitly declare the \ucode{i} variable, due to the \bcode{implicit none} statement; 
it could have also been declared outside the structured block.

\topmarker{Fortran}
\ffreenexample[5.1]{directive_syntax_F_block}{1}

A Fortran \bcode{BLOCK} construct may eliminate the need for a paired \kcode{end} directive for an OpenMP construct, 
as illustrated in the following example.

The first \kcode{parallel} construct is specified with an OpenMP loosely structured block 
(where the first executable construct is not a Fortran 2008 \bcode{BLOCK} construct). 
A paired \kcode{end} directive must end the OpenMP construct.
The second \kcode{parallel} construct is specified with an OpenMP strictly structured block 
(consists only of a single Fortran \bcode{BLOCK} construct). 
The paired \kcode{end} directive is optional in this case, and is not used here.

The next two \kcode{parallel} directives form an enclosing outer \kcode{parallel} construct 
and a nested inner \kcode{parallel} construct. The first \kcode{end parallel} directive
that subsequently appears terminates the inner \kcode{parallel} construct, 
because a paired \kcode{end} directive immediately following a \bcode{BLOCK} construct that is 
a strictly structured block of an OpenMP construct is treated as the terminating end directive 
of that construct. 
The next \kcode{end parallel} directive is required to terminate the outer \kcode{parallel} construct.

\topmarker{Fortran}
\ffreenexample[5.1]{directive_syntax_F_block}{2}
\end{fortranspecific}




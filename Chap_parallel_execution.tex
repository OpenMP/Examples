\cchapter{Parallel Execution}{parallel_execution}
\label{chap:parallel_execution}

A single thread, the \plc{initial thread}, begins sequential execution of 
an OpenMP enabled program, as if the whole program is in an implicit parallel
region consisting of an implicit task executed by the \plc{initial thread}.

A \code{parallel} construct encloses code, 
forming a parallel region.  An \plc{initial thread} encountering a \code{parallel} 
region forks (creates) a team of threads at the beginning of the 
\code{parallel} region, and joins them (removes from execution) at the 
end of the region.  The initial thread becomes the primary thread of the team in a 
\code{parallel} region with a \plc{thread} number equal to zero, the other 
threads are numbered from 1 to number of threads minus 1. 
A team may be comprised of just a single thread.

Each thread of a team is assigned an implicit task consisting of code within the 
parallel region. The task that creates a parallel region is suspended while the
tasks of the team are executed.  A thread is tied to its task; that is,
only the thread assigned to the task can execute that task.  After completion 
of the \code{parallel} region, the primary thread resumes execution of the generating task.  

%After the \code{parallel} region the primary thread becomes the initial 
%thread again, and continues to execute the \plc{sequential part}.  

Any task within a \code{parallel} region is allowed to encounter another
\code{parallel} region to form a nested \code{parallel} region. The 
parallelism of a nested \code{parallel} region (whether it forks additional 
threads, or is executed serially by the encountering task) can be controlled by the
\code{OMP\_NESTED} environment variable or the \code{omp\_set\_nested()} 
API routine with arguments indicating true or false.

The number of threads of a \code{parallel} region can be set by the \code{OMP\_NUM\_THREADS}
environment variable, the \code{omp\_set\_num\_threads()} routine, or on the \code{parallel} 
directive with the \code{num\_threads}
clause. The routine overrides the environment variable, and the clause overrides all. 
Use the \code{OMP\_DYNAMIC}
or the \code{omp\_set\_dynamic()} function to specify that the OpenMP
implementation dynamically adjust the number of threads for
\code{parallel} regions.  The default setting for dynamic adjustment is implementation
defined. When dynamic adjustment is on and the number of threads is specified,
the number of threads becomes an upper limit for the number of threads to be
provided by the OpenMP runtime.

%\pagebreak
\bigskip
WORKSHARING CONSTRUCTS

A worksharing construct distributes the execution of the associated region
among the members of the team that encounter it.  There is an
implied barrier at the end of the worksharing region
(there is no barrier at the beginning). The worksharing
constructs are:

\begin{compactitem}

\item loop constructs: {\code{for} and \code{do} }
\item \code{sections}
\item \code{single}
\item \code{workshare}

\end{compactitem}

The \code{for} and \code{do} constructs (loop constructs) create a region 
consisting of a loop.  A loop controlled by a loop construct is called 
an \plc{associated} loop.  Nested loops can form a single region when the 
\code{collapse} clause (with an integer argument) designates the number of 
\plc{associated} loops to be executed in parallel, by forming a 
"single iteration space" for the specified number of nested loops.  
The \code{ordered} clause can also control multiple associated loops.

An associated loop must adhere to a "canonical form" (specified in the 
\plc{Canonical Loop Form} of the OpenMP Specifications document) which allows the 
iteration count (of all associated loops) to be computed before the 
(outermost) loop is executed. %[58:27-29].  
Most common loops comply with the canonical form, including C++ iterators.

A \code{single} construct forms a region in which only one thread (any one 
of the team) executes the region. 
The other threads wait at the implied 
barrier at the end, unless the \code{nowait} clause is specified.

The \code{sections} construct forms a region that contains one or more 
structured blocks.  Each block of a \code{sections} directive is 
constructed with a \code{section} construct, and executed once by 
one of the threads (any one) in the team.  (If only one block is 
formed in the region, the \code{section} construct, which is used to
separate blocks, is not required.)
The other threads wait at the implied 
barrier at the end, unless the \code{nowait} clause is specified.


The \code{workshare} construct is a Fortran feature that consists of a
region with a single structure block (section of code). Statements in the
\code{workshare} region are divided into units of work, and executed (once)
by threads of the team.  

\bigskip
MASKED CONSTRUCT

The \code{masked} construct is not a worksharing construct.  The \code{masked} region is
executed only by the primary thread. There is no implicit barrier (and flush) 
at the end of the \code{masked} region; hence the other threads of the team continue
execution beyond code statements beyond the \code{masked} region.
The \code{master} construct, which has been deprecated in OpenMP 5.1, has identical semantics
to the \code{masked} construct with no \code{filter} clause.


%===== Examples Sections =====
\pagebreak
\section{A Simple Parallel Loop}
\label{sec:ploop}
\index{combined constructs!parallel worksharing-loop}
\index{constructs!parallel@\kcode{parallel}}
\index{parallel construct@\kcode{parallel} construct}
\index{worksharing-loop constructs!for@\kcode{for}}
\index{worksharing-loop constructs!do@\kcode{do}}
\index{constructs!for@\kcode{for}}
\index{constructs!do@\kcode{do}}
\index{for construct@\kcode{for} construct}
\index{do construct@\kcode{do} construct}

The following example demonstrates how to parallelize a simple loop 
using the \kcode{parallel} worksharing-loop
construct. The loop iteration variable is private by default, so it is not 
necessary to specify it explicitly in a \kcode{private} clause.

\cexample{ploop}{1}

\fexample{ploop}{1}


\pagebreak
\section{\kcode{parallel} Construct}
\label{sec:parallel}
\index{constructs!parallel@\kcode{parallel}}
\index{parallel construct@\kcode{parallel} construct}

The \kcode{parallel} construct  can be used in coarse-grain parallel programs. 
In the following example, each thread in the \kcode{parallel} region decides what 
part of the global array \ucode{x} to work on, based on the thread number:

\cexample{parallel}{1}

\fexample{parallel}{1}


\pagebreak
\section{\code{teams} Construct on Host}
\label{sec:host_teams}
\index{constructs!teams@\code{teams}}
\index{teams construct@\code{teams} construct}

%{\color{blue} ... } {\color{violet} ... }
Originally the \code{teams} construct was created for devices (such as GPUs)
for independent executions of a structured block by teams within a league (on SMs).
It was only available through offloading with the \code{target} construct,
and the execution of a \code{teams} region could only be directed to host
execution by various means such as \code{if} and \code{device} clauses,
and the \code{OMP\_TARGET\_OFFLOAD} environment variable.

In OpenMP 5.0 the \code{teams} construct was extended to enable the host
to execute a \code{teams} region (without an associated \code{target} construct), 
with anticipation of further affinity and threading controls in future OpenMP releases.
%With additional affinity controls, a team could be
%assigned to execute on a socket or use only a specified number of threads.

In the example below the \code{teams} construct is used to create two
teams, one to execute single precision code, and the other
to execute double precision code. Two teams are required, and
the thread limit for each team is set to 1/2 of the number of 
available processors.

\cexample[5.0]{host_teams}{1}

\ffreeexample[5.0]{host_teams}{1}


%\pagebreak
\section{Controlling the Number of Threads on Multiple Nesting Levels}
\label{sec:nthrs_nesting}
\index{environment variables!OMP_NUM_THREADS@\kcode{OMP_NUM_THREADS}}
\index{OMP_NUM_THREADS@\kcode{OMP_NUM_THREADS}}

The following examples demonstrate how to use the \kcode{OMP_NUM_THREADS} environment 
variable  to control the number of threads on multiple nesting levels:

\cexample{nthrs_nesting}{1}[1]

\fexample{nthrs_nesting}{1}[1]



%\pagebreak
\section{Interaction Between the \kcode{num_threads} Clause and \kcode{omp_set_dynamic}}
\label{sec:nthrs_dynamic}
\index{clauses!num_threads@\kcode{num_threads}}
\index{num_threads clause@\kcode{num_threads} clause}
\index{routines!omp_set_dynamic@\kcode{omp_set_dynamic}}
\index{omp_set_dynamic routine@\kcode{omp_set_dynamic} routine}

The following example demonstrates the \kcode{num_threads} clause  and the effect 
of the \\
\kcode{omp_set_dynamic} routine  on it.

The call to the \kcode{omp_set_dynamic} routine with argument \ucode{0} in 
C/C++, or \ucode{.FALSE.} in Fortran, disables the dynamic adjustment of the number 
of threads in OpenMP implementations that support it. In this case, 10 threads 
are provided. Note that in case of an error the OpenMP implementation is free to 
abort the program or to supply any number of threads available.

\cexample{nthrs_dynamic}{1}

\fexample{nthrs_dynamic}{1}

%\pagebreak
The call to the \kcode{omp_set_dynamic} routine with a non-zero argument in 
C/C++, or \ucode{.TRUE.} in Fortran, allows the OpenMP implementation to choose 
any number of threads between 1 and 10.

\cexample{nthrs_dynamic}{2}

\fexample{nthrs_dynamic}{2}

It is good practice to set the \plc{dyn-var} ICV explicitly by calling the \kcode{omp_set_dynamic} 
routine, as its default setting is implementation defined.



\pagebreak
\section{Fortran Restrictions on the \kcode{do} Construct}
\label{sec:fort_do}
\index{constructs!do@\kcode{do}}
\index{do construct@\kcode{do} construct}
\fortranspecificstart

If an \kcode{end do} directive follows a \plc{do-construct} in which several 
\bcode{DO} statements share a \bcode{DO} termination statement, then a  \kcode{do} 
directive can only be specified for the outermost of these \bcode{DO} statements. 
The following example contains correct usages of 
\kcode{do} constructs:

\fnexample{fort_do}{1}

The following example is non-conforming because the matching \kcode{do} directive 
for the \kcode{end do} does not precede the outermost loop:

\fnexample{fort_do}{2}
\fortranspecificend



%\pagebreak
\section{\kcode{nowait} Clause}
\label{sec:nowait}
\index{clauses!nowait@\kcode{nowait}}
\index{nowait clause@\kcode{nowait} clause}

If there are multiple independent loops within a \kcode{parallel} region, you 
can use the \kcode{nowait} clause to avoid the implied barrier at the end of
the worksharing-loop construct, as follows:

\cexample{nowait}{1}

\fexample{nowait}{1}

\index{loop scheduling!static}
\index{static scheduling}
In the following example, static scheduling distributes the same logical iteration 
numbers to the threads that execute the three loop regions. This allows the \kcode{nowait} 
clause to be used, even though there is a data dependence between the loops. The 
dependence is satisfied as long the same thread executes the same logical iteration 
numbers in each loop.

Note that the iteration count of the loops must be the same. The example satisfies 
this requirement, since the iteration space of the first two loops is from \ucode{0} 
to \ucode{n-1} (from \ucode{1} to \ucode{N} in the Fortran version), while the 
iteration space of the last loop is from \ucode{1} to \ucode{n} (\ucode{2} to 
\ucode{N+1} in the Fortran version).

\cexample{nowait}{2}

\ffreeexample{nowait}{2}


\pagebreak
\section{\code{collapse} Clause}
\label{sec:collapse}
\index{clauses!collapse@\code{collapse}}
\index{collapse clause@\code{collapse} clause}

In the following example, the \code{k} and \code{j} loops are associated with 
the loop construct. So the iterations of the \code{k} and \code{j} loops are 
collapsed into one loop with a larger iteration space, and that loop is then divided 
among the threads in the current team. Since the \code{i} loop is not associated 
with the loop construct, it is not collapsed, and the \code{i} loop is executed 
sequentially in its entirety in every iteration of the collapsed \code{k} and 
\code{j} loop. 

The variable \code{j} can be omitted from the \code{private}  clause when the 
\code{collapse} clause is used since it is implicitly private. However, if the 
\code{collapse} clause is omitted then \code{j} will be shared if it is omitted 
from the \code{private} clause. In either case, \code{k} is implicitly private 
and could be omitted from the \code{private}  clause.

\cexample[3.0]{collapse}{1}

\fexample[3.0]{collapse}{1}

In the next example, the \code{k} and \code{j} loops are associated with the 
loop construct. So the iterations of the \code{k} and \code{j} loops are collapsed 
into one loop with a larger iteration space, and that loop is then divided among 
the threads in the current team.

The sequential execution of the iterations in the \code{k} and \code{j} loops 
determines the order of the iterations in the collapsed iteration space. This implies 
that in the sequentially last iteration of the collapsed iteration space, \code{k} 
will have the value \code{2} and \code{j} will have the value \code{3}. Since 
\code{klast} and \code{jlast} are \code{lastprivate}, their values are assigned 
by the sequentially last iteration of the collapsed \code{k} and \code{j} loop. 
This example prints: \code{2 3}.

\cexample[3.0]{collapse}{2}

\fexample[3.0]{collapse}{2}

\index{clauses!collapse@\code{collapse}}
\index{collapse clause@\code{collapse} clause}
\index{clauses!ordered@\code{ordered}}
\index{ordered clause@\code{ordered} clause}
The next example illustrates the interaction of the \code{collapse} and \code{ordered} 
 clauses.

In the example, the loop construct has both a \code{collapse} clause and an \code{ordered} 
clause. The \code{collapse} clause causes the iterations of the \code{k} and 
\code{j} loops to be collapsed into one loop with a larger iteration space, and 
that loop is divided among the threads in the current team. An \code{ordered} 
clause is added to the loop construct because an ordered region binds to the loop 
region arising from the loop construct.

According to Section 2.12.8 of the OpenMP 4.0 specification, 
a thread must not execute more than one ordered region that binds 
to the same loop region. So the \code{collapse} clause is required for the example 
to be conforming. With the \code{collapse} clause, the iterations of the \code{k} 
and \code{j} loops are collapsed into one loop, and therefore only one ordered 
region will bind to the collapsed \code{k} and \code{j} loop. Without the \code{collapse} 
clause, there would be two ordered regions that bind to each iteration of the \code{k} 
loop (one arising from the first iteration of the \code{j} loop, and the other 
arising from the second iteration of the \code{j} loop).

The code prints

\code{0 1 1}
\\
\code{0 1 2}
\\
\code{0 2 1}
\\
\code{1 2 2}
\\
\code{1 3 1}
\\
\code{1 3 2}

\cexample[3.0]{collapse}{3}

\fexample[3.0]{collapse}{3}
\clearpage


\index{non-rectangular loop nest}
The following example illustrates the collapse of a non-rectangular loop nest,
a new feature in OpenMP 5.0. In a loop nest, a non-rectangular loop has a
loop bound that references the iteration variable of an enclosing loop.

The motivation for this feature is illustrated
in the example below that creates a symmetric correlation matrix for a set of
variables. Note that the initial value of the second loop depends on the index
variable of the first loop for the loops to be collapsed.
Here the data are represented by a 2D array, each row corresponds to a variable
and each column corresponds to a sample of the variable -- the last two columns
are the sample mean and standard deviation (for Fortran, rows and columns are swapped).

\cexample[5.0]{collapse}{4}

\ffreeexample[5.0]{collapse}{4}

\section{\code{linear} Clause in Loop Constructs}
\label{sec:linear_in_loop}
\index{clauses!linear@\code{linear}}
\index{linear clause@\code{linear} clause}

The following example shows the use of the \code{linear} clause in a loop 
construct to allow the proper parallelization of a loop that contains 
an induction variable (\plc{j}).  At the end of the execution of 
the loop construct, the original variable \plc{j} is updated with 
the value \plc{N/2} from the last iteration of the loop.

\cexample[4.5]{linear_in_loop}{1}

\ffreeexample[4.5]{linear_in_loop}{1}


\pagebreak
\section{\code{parallel} \code{sections} Construct}
\label{sec:psections}
\index{combined constructs!parallel sections@\code{parallel}~\code{sections}}
\index{parallel sections construct@\code{parallel}~\code{sections} construct}

In the following example routines \code{XAXIS}, \code{YAXIS}, and \code{ZAXIS} can 
be executed concurrently. The first \code{section} directive is optional. Note 
that all \code{section} directives need to appear in the 
\code{parallel}~\code{sections} construct.

\cexample{psections}{1}

\fexample{psections}{1}


\pagebreak
\section{\kcode{firstprivate} Clause and \kcode{sections} Construct}
\label{sec:fpriv_sections}
\index{constructs!sections@\kcode{sections}}
\index{sections construct@\kcode{sections} construct}
\index{constructs!section@\kcode{section}}
\index{section construct@\kcode{section} construct}
\index{clauses!firstprivate@\kcode{firstprivate}}
\index{firstprivate clause@\kcode{firstprivate} clause}

In the following example of the \kcode{sections} construct  the \kcode{firstprivate} 
clause is used to initialize the private copy of \ucode{section_count} of each 
thread. The problem is that the \kcode{section} constructs modify \ucode{section_count}, 
which breaks the independence of the \kcode{section} constructs. When different 
threads execute each section, both sections will print the value 1. When the same 
thread executes the two sections, one section will print the value 1 and the other 
will print the value 2. Since the order of execution of the two sections in this 
case is unspecified, it is unspecified which section prints which value. 

\cexample{fpriv_sections}{1}

\ffreeexample{fpriv_sections}{1}



\pagebreak
\section{\code{single} Construct}
\label{sec:single}
\index{constructs!single@\code{single}}
\index{single construct@\code{single} construct}

The following example demonstrates the \code{single} construct. In the example, 
only one thread prints each of the progress messages. All other threads will skip 
the \code{single} region and stop at the barrier at the end of the \code{single} 
construct until all threads in the team have reached the barrier. If other threads 
can proceed without waiting for the thread executing the \code{single} region, 
a \code{nowait} clause can be specified, as is done in the third \code{single} 
construct in this example. The user must not make any assumptions as to which thread 
will execute a \code{single} region.

\cexample{single}{1}

\fexample{single}{1}



%\pagebreak
\section{\kcode{workshare} Construct}
\fortranspecificstart
\label{sec:workshare}
\index{constructs!workshare@\kcode{workshare}}
\index{workshare construct@\kcode{workshare} construct}

The following are examples of the \kcode{workshare} construct. 

In the following example, \kcode{workshare} spreads work across the threads executing 
the \kcode{parallel} region, and there is a barrier after the last statement. 
Implementations must enforce Fortran execution rules inside of the \kcode{workshare} 
block.

\fnexample{workshare}{1}

In the following example, the barrier at the end of the first \kcode{workshare} 
region is eliminated with a \kcode{nowait} clause. Threads doing \ucode{CC = 
DD} immediately begin work on \ucode{EE = FF} when they are done with \ucode{CC 
= DD}.

\pagebreak
\fnexample{workshare}{2}
\topmarker{Fortran}

The following example shows the use of an \kcode{atomic} directive inside a \kcode{workshare} 
construct. The computation of \ucode{SUM(AA)} is workshared, but the update to 
\ucode{R} is atomic.

\fnexample{workshare}{3}

Fortran \bcode{WHERE} and \bcode{FORALL} statements are \emph{compound statements}, 
made up of a \emph{control} part and a \emph{statement} part. When \kcode{workshare} 
is applied to one of these compound statements, both the control and the statement 
parts are workshared. The following example shows the use of a \bcode{WHERE} statement 
in a \kcode{workshare} construct.

Each task gets worked on in order by the threads:

\ucode{AA = BB} then
\\
\ucode{CC = DD} then
\\
\ucode{EE .ne. 0} then
\\
\ucode{FF = 1 / EE} then
\\
\ucode{GG = HH}

\fnexample{workshare}{4}
\topmarker{Fortran}

In the following example, an assignment to a shared scalar variable is performed 
by one thread in a \kcode{workshare} while all other threads in the team wait.

\fnexample{workshare}{5}

The following example contains an assignment to a private scalar variable, which 
is performed by one thread in a \kcode{workshare} while all other threads wait. 
It is non-conforming because the private scalar variable is undefined after the 
assignment statement. 

\fnexample{workshare}{6}

Fortran execution rules must be enforced inside a \kcode{workshare} construct. 
In the following example, the same result is produced in the following program 
fragment regardless of whether the code is executed sequentially or inside an OpenMP 
program with multiple threads:

\fnexample{workshare}{7}
\fortranspecificend



\pagebreak
\section{\code{masked} Construct}
\label{sec:masked}
\index{constructs!masked@\code{masked}}
\index{masked construct@\code{masked} construct}
\index{masked construct@\code{masked} construct!filter clause@\code{filter} clause}
\index{clauses!filter@\code{filter}}
\index{filter clause@\code{filter} clause}

The following example demonstrates the \code{masked} construct. 
In the example, the primary thread (thread number 0) 
keeps track of how many iterations have been executed and prints out 
a progress report in the iteration loop.
The other threads skip the \code{masked} region without waiting. 
The \code{filter} clause can be used to specify a thread number other 
than the primary thread to execute a structured block, as illustrated by
the second \code{masked} construct after the iteration loop.
If the thread specified in a \scode{filter} clause does not exist 
in the team then the structured block is not executed by any thread.

\cexample[5.1]{masked}{1}

\fexample[5.1]{masked}{1}



\pagebreak
\section{\code{loop} Construct}
\label{sec:loop}
\index{constructs!loop@\code{loop}}
\index{loop construct@\code{loop} construct}

The following example illustrates the use of the OpenMP 5.0 \code{loop}
construct for the execution of a loop.
The \code{loop} construct asserts to the compiler that the iterations 
of the loop are free of data dependencies and may be executed concurrently.
It allows the compiler to use heuristics to select the parallelization scheme
and compiler-level optimizations for the concurrency. 

\cexample[5.0]{loop}{1}
\ffreeexample[5.0]{loop}{1}

%\pagebreak
\begin{cppspecific}[4ex]
\section{Parallel Random Access Iterator Loop}
\label{sec:pra_iterator}
\index{random access iterator, C++}

The following example shows a parallel random access iterator loop.

\cppnexample[3.0]{pra_iterator}{1}
\end{cppspecific}



\pagebreak
\section{\code{omp\_set\_dynamic} and \\
\code{omp\_set\_num\_threads} Routines}
\label{sec:set_dynamic_nthrs}
\index{routines!omp_set_dynamic@\scode{omp_set_dynamic}}
\index{omp_set_dynamic routine@\scode{omp_set_dynamic} routine}
\index{routines!omp_set_num_threads@\scode{omp_set_num_threads}}
\index{omp_set_num_threads routine@\scode{omp_set_num_threads} routine}

Some programs rely on a fixed, prespecified number of threads to execute correctly. 
Because the default setting for the dynamic adjustment of the number of threads 
is implementation defined, such programs can choose to turn off the dynamic threads 
capability and set the number of threads explicitly to ensure portability. The 
following example shows how to do this using \code{omp\_set\_dynamic}, and \code{omp\_set\_num\_threads}.

In this example, the program executes correctly only if it is executed by 16 threads. 
If the implementation is not capable of supporting 16 threads, the behavior of 
this example is implementation defined. Note that the number of threads executing 
a \code{parallel} region remains constant during the region, regardless of the 
dynamic threads setting. The dynamic threads mechanism determines the number of 
threads to use at the start of the \code{parallel} region and keeps it constant 
for the duration of the region.

\cexample{set_dynamic_nthrs}{1}

\fexample{set_dynamic_nthrs}{1}



\pagebreak
\section{\kcode{omp_get_num_threads} Routine}
\label{sec:get_nthrs}
\index{routines!omp_get_num_threads@\kcode{omp_get_num_threads}}
\index{omp_get_num_threads routine@\kcode{omp_get_num_threads} routine}

In the following example, the \kcode{omp_get_num_threads} call returns 1 in 
the sequential part of the code, so \ucode{np} will always be equal to 1. To determine 
the number of threads that will be deployed for the \kcode{parallel} region, the 
call should be inside the \kcode{parallel} region.

\cexample{get_nthrs}{1}

\fexample{get_nthrs}{1}

\pagebreak
The following example shows how to rewrite this program without including a query 
for the number of threads:

\cexample{get_nthrs}{2}

\fexample{get_nthrs}{2}





\pagebreak
\section{Task Detachment}
\label{sec:task_detachment}
\index{task construct@\kcode{task} construct!detach clause@\kcode{detach} clause}
\index{detach clause@\kcode{detach} clause}
\index{clauses!detach@\kcode{detach}}
\index{routines!omp_fulfill_event@\kcode{omp_fulfill_event}}
\index{omp_fulfill_event routine@\kcode{omp_fulfill_event} routine}

% if used, then generated task must be completed.
% No definition of a detachable task

The \kcode{detach} clause on a \kcode{task} construct provides a mechanism for an asynchronous
routine to be called within a task block, and for the routine's
callback to signal completion to the OpenMP runtime, through an 
event fulfillment, triggered by a call to the \kcode{omp_fulfill_event} routine.
When a \kcode{detach} clause is used on a \kcode{task} construct,
completion of the detachable task occurs when the task's structured
block is completed AND an \plc{allow-completion} event is
fulfilled by a call to the \kcode{omp_fulfill_event} 
routine with the \plc{event-handle} argument.

The first example illustrates the basic components used in a detachable task.
The second example is a program that executes asynchronous IO, and illustrates 
methods that are also inherent in asynchronous messaging within MPI and asynchronous commands in 
streams within GPU codes.
Interfaces to asynchronous operations found in IO, MPI and GPU parallel computing platforms
and their programming models are not standardized. 

-------------------------

The first example creates a detachable task
that executes the asynchronous \ucode{async_work} routine,
passing the \kcode{omp_fulfill_event} function and the (firstprivate) event handle
to the function. Here, the OpenMP \kcode{omp_fulfill_event} procedure is
the ``callback'' function to be executed at the end of the \ucode{async_work} function's
asynchronous operations,
with the associated data, \ucode{event}. 

\cexample[5.0]{task_detach}{1}

\ffreeexample[5.0]{task_detach}{1}
\clearpage

%ASYNCHRONOUS IO

In the following example, text data is written asynchronously to the file \ucode{async_data},
using POSIX asynchronous IO (aio). An aio ``control block'', \ucode{cb}, is set up to
send a signal when IO is complete, and the \ucode{sigaction} function registers
the signal action, a callback to \ucode{callback_aioSigHandler}.

The first task (Task1) starts the asynchronous IO and runs as a detachable task.
The second and third tasks (Task2 and Task3) perform synchronous IO to stdout with print statements.
The difference between the two types of tasks is that the thread for Task1 is freed for 
other execution within the \kcode{parallel} region, while the threads for Task2 and Task3 wait
on the (synchronous) IO to complete, and cannot perform other work while the 
operating system is performing the synchronous IO. 
The \kcode{if} clause ensures that the detachable task is launched 
and the call to the \ucode{aio_write} function returns
before Task2 and Task3 are generated (while the async IO occurs in the ``background'' and eventually
executes the callback function).  The barrier at the end of the \kcode{parallel} region ensures that the
detachable task has completed.

\cexample[5.0]{task_detach}{2}


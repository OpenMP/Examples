%\pagebreak
\section{\kcode{taskloop} Construct}
\label{sec:taskloop}
\index{constructs!taskloop@\kcode{taskloop}}
\index{taskloop construct@\kcode{taskloop} construct}
\index{taskloop construct@\kcode{taskloop} construct!grainsize clause@\kcode{grainsize} clause}
\index{taskloop construct@\kcode{taskloop} construct!nogroup clause@\kcode{nogroup} clause}
\index{clauses!grainsize@\kcode{grainsize}}
\index{grainsize clause@\kcode{grainsize} clause}
\index{clauses!nogroup@\kcode{nogroup}}
\index{nogroup clause@\kcode{nogroup} clause}

The following example illustrates how to execute a long running task concurrently with tasks created
with a \kcode{taskloop} directive for a loop having unbalanced amounts of work for its iterations.

The \kcode{grainsize} clause specifies that each task is to execute at least \ucode{500} iterations of the loop. 

The \kcode{nogroup} clause removes the implicit taskgroup of the \kcode{taskloop} construct; 
the explicit \kcode{taskgroup} construct in the example ensures that the function is not exited 
before the long-running task and the loops have finished execution.

\cexample[4.5]{taskloop}{1}

\ffreeexample[4.5]{taskloop}{1}

%\clearpage

Because a \kcode{taskloop} construct encloses a loop, it is often incorrectly 
perceived as a worksharing construct (when it is directly nested in 
a \kcode{parallel} region).

While a worksharing construct distributes the loop iterations across all threads in a team,
the entire loop of a \kcode{taskloop} construct is executed by every thread of the team.

In the example below the first taskloop occurs closely nested within 
a \kcode{parallel} region and the entire loop is executed by each of the \ucode{T} threads; 
hence the reduction sum is executed \ucode{T}*\ucode{N} times.
 
The loop of the second taskloop is within a \kcode{single} region and is executed
by a single thread so that only \ucode{N} reduction sums occur.  (The other
\ucode{N}-1 threads of the \kcode{parallel} region will participate in executing the 
tasks. This is the common use case for the \kcode{taskloop} construct.)

In the example, the code thus prints \pout{x1 = 16384} (\ucode{T}*\ucode{N}) and 
\pout{x2 = 1024} (\ucode{N}).

\cexample[4.5]{taskloop}{2}

\ffreeexample[4.5]{taskloop}{2}

\pagebreak
\section{\kcode{task} and \kcode{taskwait} Constructs}
\label{sec:task_taskwait}
\index{constructs!task@\kcode{task}}
\index{task construct@\kcode{task} construct}
\index{constructs!taskwait@\kcode{taskwait}}
\index{taskwait construct@\kcode{taskwait} construct}

The following example shows how to traverse a tree-like structure using explicit 
tasks. Note that the \ucode{traverse} function should be called from within a 
\kcode{parallel} region for the different specified tasks to be executed in parallel. Also 
note that the tasks will be executed in no specified order because there are no 
synchronization directives. Thus, assuming that the traversal will be done in post 
order, as in the sequential code, is wrong.

\cexample[3.0]{tasking}{1}

\ffreeexample[3.0]{tasking}{1}

In the next example, we force a postorder traversal of the tree by adding a \kcode{taskwait} 
directive. Now, we can safely assume that the left and right sons have been executed 
before we process the current node.

\cexample[3.0]{tasking}{2}

\ffreeexample[3.0]{tasking}{2}

The following example demonstrates how to use the \kcode{task} construct to process 
elements of a linked list in parallel. The thread executing the \kcode{single} 
region generates all of the explicit tasks, which are then executed by the threads 
in the current team. The pointer \ucode{p} is firstprivate by default 
on the \kcode{task} construct so it is not necessary to specify it in a \kcode{firstprivate} 
clause.
%

\cexample[3.0]{tasking}{3}

\ffreeexample[3.0]{tasking}{3}

The \ucode{fib()} function should be called from within a \kcode{parallel} region 
for the different specified tasks to be executed in parallel. Also, only one thread 
of the \kcode{parallel} region should call \ucode{fib()} unless multiple concurrent 
Fibonacci computations are desired. 

\cexample[3.0]{tasking}{4}

\fexample[3.0]{tasking}{4}
\clearpage

Note: There are more efficient algorithms for computing Fibonacci numbers. This 
classic recursion algorithm is for illustrative purposes.

The following example demonstrates a way to generate a large number of tasks with 
one thread and execute them with the threads in the team. While generating these 
tasks, the implementation may reach its limit on unassigned tasks.  If it does, 
the implementation is allowed to cause the thread executing the task generating 
loop to suspend its task at the task scheduling point in the \kcode{task} directive, 
and start executing unassigned tasks.  Once the number of unassigned tasks is sufficiently 
low, the thread may resume execution of the task generating loop.

\cexample[3.0]{tasking}{5}

\fexample[3.0]{tasking}{5}

\index{task construct@\kcode{task} construct!untied clause@\kcode{untied} clause}
\index{untied clause@\kcode{untied} clause}
\index{clauses!untied@\kcode{untied}}
\index{task scheduling point}
The following example is the same as the previous one, except that the tasks are 
generated in an untied task. While generating the tasks, the implementation may 
reach its limit on unassigned tasks. If it does, the implementation is allowed 
to cause the thread executing the task generating loop to suspend its task at the 
task scheduling point in the \kcode{task} directive, and start executing unassigned 
tasks.  If that thread begins execution of a task that takes a long time to complete, 
the other threads may complete all the other tasks before it is finished.

In this case, since the loop is in an untied task, any other thread is eligible 
to resume the task generating loop. In the previous examples, the other threads 
would be forced to idle until the generating thread finishes its long task, since 
the task generating loop was in a tied task.

\cexample[3.0]{tasking}{6}

\fexample[3.0]{tasking}{6}

The following two examples demonstrate how the scheduling rules illustrated in 
the \docref{Task Scheduling} section of the OpenMP 4.0 specification affect the usage of 
threadprivate variables in tasks. A threadprivate 
variable can be modified by another task that is executed by the same thread. Thus, 
the value of a threadprivate variable cannot be assumed to be unchanged 
across a task scheduling point. In untied tasks, task scheduling points may be 
added in any place by the implementation.

A task switch may occur at a task scheduling point. A single thread may execute 
both of the \kcode{task} regions that modify \ucode{tp}. The parts of these \kcode{task} regions 
in which \ucode{tp} is modified may be executed in any order so the resulting 
value of \ucode{var} can be either 1 or 2.

\cexample[3.0]{tasking}{7}

\fexample[3.0]{tasking}{7}

In this example, scheduling constraints prohibit a thread in the team from executing 
a new task that modifies \ucode{tp} while another such \kcode{task} region tied to the 
same thread is suspended. Therefore, the value written will persist across the 
task scheduling point.

\cexample[3.0]{tasking}{8}

\fexample[3.0]{tasking}{8}

The following two examples demonstrate how the scheduling rules illustrated in 
\docref{Task Scheduling} section of the OpenMP 4.0 specification affect the usage of locks 
and critical sections in tasks.  If a lock is held 
across a task scheduling point, no attempt should be made to acquire the same lock 
in any code that may be interleaved.  Otherwise, a deadlock is possible.

In the example below, suppose the thread executing task 1 defers task 2.  When 
it encounters the task scheduling point at task 3, it could suspend task 1 and 
begin task 2 which will result in a deadlock when it tries to enter \kcode{critical} region 
1.

\cexample[3.0]{tasking}{9}

\fexample[3.0]{tasking}{9}

In the following example, \ucode{lock} is held across a task scheduling point. 
 However, according to the scheduling restrictions, the executing thread can't 
begin executing one of the non-descendant tasks that also acquires \ucode{lock} before 
the \kcode{task} region is complete.  Therefore, no deadlock is possible.

\cexample[3.0]{tasking}{10}

\ffreeexample[3.0]{tasking}{10}
\clearpage

\index{task construct@\kcode{task} construct!mergeable clause@\kcode{mergeable} clause}
\index{clauses!mergeable@\kcode{mergeable}}
\index{mergeable clause@\kcode{mergeable} clause}
The following examples illustrate the use of the \kcode{mergeable} clause in the 
\kcode{task} construct. In this first example, the \kcode{task} construct has 
been annotated with the \kcode{mergeable}  clause. The addition of this clause 
allows the implementation to reuse the data environment (including the ICVs) of 
the parent task for the task inside \ucode{foo} if the task is included or undeferred. 
Thus, the result of the execution may differ depending on whether the task is merged 
or not. Therefore the mergeable clause needs to be used with caution. In this example, 
the use of the mergeable clause is safe. As \ucode{x} is a shared variable the 
outcome does not depend on whether or not the task is merged (that is, the task 
will always increment the same variable and will always compute the same value 
for \ucode{x}).

\cexample[3.1]{tasking}{11}

\ffreeexample[3.1]{tasking}{11}

This second example shows an incorrect use of the \kcode{mergeable} clause. In 
this example, the created task will access different instances of the variable 
\ucode{x} if the task is not merged, as \ucode{x} is firstprivate, but 
it will access the same variable \ucode{x} if the task is merged. As a result, 
the behavior of the program is unspecified, and it can print two different values 
for \ucode{x} depending on the decisions taken by the implementation.

\cexample[3.1]{tasking}{12}

\ffreeexample[3.1]{tasking}{12}

\index{task construct@\kcode{task} construct!final clause@\kcode{final} clause}
\index{clauses!final@\kcode{final}}
\index{final clause@\kcode{final} clause}
\index{routines!omp_in_final@\kcode{omp_in_final}}
\index{omp_in_final routine@\kcode{omp_in_final} routine}
The following example shows the use of the \kcode{final} clause and the \kcode{omp_in_final} 
API call in a recursive binary search program. To reduce overhead, once a certain 
depth of recursion is reached the program uses the \kcode{final} clause to create 
only included tasks, which allow additional optimizations.

The use of the \kcode{omp_in_final} API call allows programmers to optimize 
their code by specifying which parts of the program are not necessary when a task 
can create only included tasks (that is, the code is inside a final task). 
In this example, the use of a different state variable is not necessary so once 
the program reaches the part of the computation that is finalized and copying from 
the parent state to the new state is eliminated. The allocation of \ucode{new_state} 
in the stack could also be avoided but it would make this example less clear. The 
\kcode{final} clause is most effective when used in conjunction with the \kcode{mergeable} 
clause since all tasks created in a final \kcode{task} region are included tasks 
that can be merged if the \kcode{mergeable} clause is present.

\cexample[3.1]{tasking}{13}

\ffreeexample[3.1]{tasking}{13}

\index{task construct@\kcode{task} construct!if clause@\kcode{if} clause}
\index{clauses!if@\kcode{if}}
\index{if clause@\kcode{if} clause}
The following example illustrates the difference between the \kcode{if}  and the 
\kcode{final} clauses. The \kcode{if} clause has a local effect. In the first 
nest of tasks, the one that has the \kcode{if} clause will be undeferred but 
the task nested inside that task will not be affected by the \kcode{if} clause 
and will be created as usual. Alternatively, the \kcode{final} clause affects 
all \kcode{task} constructs in the final \kcode{task} region but not the final 
task itself. In the second nest of tasks, the nested tasks will be created as included 
tasks. Note also that the conditions for the \kcode{if} and \kcode{final} clauses 
are usually the opposite.

\cexample[3.1]{tasking}{14}

\ffreeexample[3.1]{tasking}{14}


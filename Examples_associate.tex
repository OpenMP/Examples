\pagebreak
\section{Fortran \code{ASSOCIATE} Construct}
\fortranspecificstart
\label{sec:associate}

The following is an invalid example of specifying an associate name on a data-sharing attribute 
clause. The constraint in the Data Sharing Attribute Rules section in the OpenMP 
4.0 API Specifications states that an associate name preserves the association 
with the selector established at the \code{ASSOCIATE} statement. The associate 
name \plc{b} is associated with the shared variable \plc{a}. With the predetermined data-sharing 
attribute rule, the associate name \plc{b} is not allowed to be specified on the \code{private} 
clause.

\fnexample[4.0]{associate}{1}

In next example, within the \code{parallel} construct, the association name \plc{thread\_id} 
is associated with the private copy of \plc{i}. The print statement should output the 
unique thread number.

\fnexample[4.0]{associate}{2}

The following example illustrates the effect of specifying a selector name on a data-sharing 
attribute clause. The associate name \plc{u} is associated with \plc{v} and the variable \plc{v} 
is specified on the \code{private} clause of the \code{parallel} construct. 
The construct association is established prior to the \code{parallel} region. 
The association between \plc{u} and the original \plc{v} is retained (see the Data Sharing 
Attribute Rules section in the OpenMP 4.0 API Specifications). Inside the \code{parallel} 
region, \plc{v} has the value of -1 and \plc{u} has the value of the original \plc{v}.

\ffreenexample[4.0]{associate}{3}
\fortranspecificend


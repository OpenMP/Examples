\pagebreak
\section{ \code{declare mapper} Construct}
\label{sec:declare_mapper}

The following examples show how to use the \code{declare mapper}
directive to prescribe a map for later use.
It is also quite useful for pre-defining partitioned and nested 
structure elements.

In the first example the \code{declare mapper} directive specifies 
that any structure of type \plc{myvec\_t} for which implicit data-mapping
rules apply will be mapped according to its \code{map} clause.
The variable \plc{v} is used for referencing the structure and its 
elements within the \code{map} clause. 
Within the \code{map} clause the \plc{v} variable specifies that all
elements of the structure are to be mapped.  Additionally, the
array section \plc{v.data[0:v.len]} specifies that the dynamic 
storage for data is to be mapped. 

Within the main program the \plc{s} variable is typed as \plc{myvec\_t}.
Since the variable is found within the target region and the type has a mapping prescribed by
a \code{declare mapper} directive, it will be automatically mapped according to its prescription: 
full structure, plus the dynamic storage of the \plc{data} element. 

%Note: By default the mapping is \code{tofrom}. 
%The associated Fortran allocatable \plc{data} array is automatically mapped with the derived
%type, it does not require an array section as in the C/C++ example.

\cexample{target_mapper}{1}

\ffreeexample{target_mapper}{1}

\pagebreak
The next example illustrates the use of the \plc{mapper-identifier} and deep copy within a structure. 
The structure, \plc{dzmat\_t},  represents a complex matrix, 
with separate real (\plc{r\_m}) and imaginary (\plc{i\_m}) elements.
Two map identifiers are created for partitioning the \plc{dzmat\_t} structure.

For the C/C++ code the first identifier is named \plc{top\_id} and maps the top half of
two matrices of type \plc{dzmat\_t}; while the second identifier, \plc{bottom\_id},
maps the lower half of two matrices. 
Each identifier is applied to a different \code{target} construct,
as  \code{map(mapper(top\_id), tofrom: a,b)} 
and \code{map(mapper(bottom\_id), tofrom: a,b)}.
Each target offload is allowed to execute concurrently on two different devices 
(\plc{0} and \plc{1}) through the \code{nowait} clause.
The OpenMP 5.0 \code{parallel master} construct creates a region of two threads
for these \code{target} constructs, with a single thread (\plc{master}) generator.

The Fortran code uses the \plc{left\_id} and \plc{right\_id} map identifiers in the
\code{map(mapper(left\_id),tofrom: a,b)} and \code{map(mapper(right\_id),tofrom: a,b)} map clauses.  
The array sections for these left and right contiguous portions of the matrices 
were defined previously in the \code{declare mapper} directive.

Note, the \plc{is} and \plc{ie} scalars are firstprivate 
by default for a target region, but are declared firstprivate anyway
to remind the user of important firstprivate data-sharing properties required here.

\cexample{target_mapper}{2}

\ffreeexample{target_mapper}{2}

\pagebreak
In the third example \plc{myvec} structures are
nested within a \plc{mypoints} structure. The \plc{myvec\_t} type is mapped
as in the first example.  Following the \plc{mypoints} structure declaration, 
the \plc{mypoints\_t} type is mapped by a \code{declare mapper} directive. 
For this structure the \plc{hostonly\_data} element will not be mapped;
also the array section of \plc{x} (\plc{v.x[:1]}) and \plc{x} will be mapped; and
\plc{scratch} will be allocated and used as scratch storage on the device.
The default map-type mapping, \code{tofrom}, applies to the \plc{x} array section,
but not to \plc{scratch} which is explicitly mapped with the \code{alloc} map-type. 
Note: the variable \plc{v} is not included in the map list (otherwise
the \plc{hostonly\_data} would be mapped)-- just the elements 
to be mapped are listed.

The two mappers are combined when a \plc{mypoints\_t} structure type is mapped,
because the mapper \plc{myvec\_t} structure type is used within a \plc{mypoints\_t}
type structure.
%Note, in the main program \plc{P} is an array of \plc{mypoints\_t} type structures, 
%and hence every element of the array is mapped with the mapper prescription.

\cexample{target_mapper}{3}

\ffreeexample{target_mapper}{3}


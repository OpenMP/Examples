\cchapter{Program Control}{program_control}
\label{chap:program_control}

Basic concepts and mechanisms for directing and controlling a program compilation and execution
are provided in this introduction and illustrated in subsequent examples.

\bigskip
CONDITIONAL COMPILATION and EXECUTION

Conditional compilation can be performed with conventional \bcode{\#ifdef} directives
in C, C++, and Fortran, and additionally with OpenMP sentinel (\scode{!$}) in Fortran. 
The \kcode{if} clause on some directives
can direct the runtime to ignore or alter the behavior of the construct.
Of course, the base-language \bcode{if} statements can be used to control the execution
of stand-alone directives (such as \kcode{flush}, \kcode{barrier}, \kcode{taskwait}, 
and  \kcode{taskyield}).
However, the directives must appear in a block structure, and not as a substatement.
The \kcode{metadirective} and \kcode{declare variant} directives provide conditional 
selection of directives and routines for compilation (and use), respectively.
The \kcode{assume} and \kcode{requires} directives provide invariants
for optimizing compilation, and essential features for compilation 
and correct execution, respectively.


\bigskip
CANCELLATION

Cancellation (termination) of the normal sequence of execution for the threads in an OpenMP region can
be  accomplished with the \kcode{cancel} construct.  The construct uses a
\plc{construct-type-clause} to set the region-type to activate for the cancellation. 
That is, inclusion  of one of the \plc{construct-type-clause} names \kcode{parallel}, \kcode{for}, 
\kcode{do}, \kcode{sections} or \kcode{taskgroup} on the directive line 
activates the corresponding region.  
The \kcode{cancel} construct is activated by the first encountering thread,  and it
continues execution at the end of the named region.
The \kcode{cancel} construct is also a cancellation point for any other thread of the team 
to also continue execution at the end of the named region.  

Also, once the specified region has been activated for cancellation any thread that encounters
a \kcode{cancellation point} construct with the same named region (\plc{construct-type-clause}),
continues execution at the end of the region.

For an activated \kcode{cancel taskgroup} construct, the tasks that
belong to the taskgroup set of the innermost enclosing taskgroup region will be canceled. 

A task that encounters a \kcode{cancel taskgroup} construct continues execution at the end of its
task region. Any task of the taskgroup that has already begun execution will run to completion,
unless it encounters a \kcode{cancellation point}; tasks that have not begun execution may be
discarded as completed tasks.

\pagebreak
CONTROL VARIABLES 

  Internal control variables (ICV) are used by implementations to hold values which control the execution
  of OpenMP regions.  Control (and hence the ICVs) may be set as implementation defaults, 
  or set and adjusted through environment variables, clauses, and API functions.  
 %Many of the ICV control values are accessible through API function calls.  
  Initial ICV values are reported by the runtime
  if the \kcode{OMP_DISPLAY_ENV} environment variable has been set to \vcode{TRUE} or \vcode{VERBOSE}. 

 %As an example, the \plc{nthreads-var} is the ICV that holds the number of threads
 %to be used in a \code{parallel} region.  It can be set with the \code{OMP\_NUM\_THREADS} environment variable, 
 %the \code{omp\_set\_num\_threads()} API function, or the \code{num\_threads} clause.  The default \plc{nthreads-var}
 %value is implementation defined.  All of the ICVs are presented in the \plc{Internal Control Variables} section
 %of the \plc{Directives} chapter of the OpenMP Specifications document.  Within the same document section, override 
 %relationships and scoping information can be found for applying user specifications and understanding the 
 %extent of the control.

\bigskip
NESTED CONSTRUCTS

Certain combinations of nested constructs are permitted, giving rise to \plc{combined} constructs
consisting of two or more directives.  These can be used when the two (or several) constructs would be used
immediately in succession (closely nested). A combined construct can use the clauses of the component
constructs without restrictions.
A \plc{composite} construct is a combined construct which has one or more clauses with (an often obviously)
modified or restricted meaning, relative to when the constructs are uncombined. %%[appear separately (singly).

%The combined \code{parallel do} and \code{parallel for} constructs are formed by combining the \code{parallel}
%construct with one of the loops constructs \code{do} or \code{for}.  The
%\code{parallel do SIMD} and \code{parallel for SIMD} constructs are composite constructs (composed from
%the parallel loop constructs and the \code{SIMD} construct), because the \code{collapse} clause must
%explicitly address the ordering of loop chunking \plc{and} SIMD ``combined'' execution.

Certain nestings are forbidden, and often the reasoning is obvious.  For example, worksharing constructs cannot be nested, and
the \kcode{barrier} construct cannot be nested inside a worksharing construct, or a \kcode{critical} construct. 
Also, \kcode{target} constructs cannot be nested, unless the nested target is a reverse offload.

The \kcode{parallel} construct can be nested, as well as the \kcode{task} construct.  
The parallel execution in the nested \kcode{parallel} construct(s) is controlled by the 
\kcode{OMP_MAX_ACTIVE_LEVELS} environment variable, and the \kcode{omp_set_max_active_levels} routine. 
Use the \kcode{omp_get_max_active_levels} routine to determine the maximum levels provided by an implementation.
As of OpenMP 5.0, use of the \kcode{OMP_NESTED} environment variable and the \kcode{omp_set_nested} routine 
has been deprecated.

More details on nesting can be found in the \docref{Nesting of Regions} of the \docref{Directives} 
chapter in the OpenMP Specifications document.


%===== Examples Sections =====
%\pagebreak
\section{Assumption Directives}
\label{sec:assumption}
%\index{assumption!assume directive@\kcode{assume} directive}
%\index{assumption!assumes directive@\kcode{assumes} directive}
\index{directives!assume@\kcode{assume}}
\index{directives!assumes@\kcode{assumes}}
\index{assume directive@\kcode{assume} directive}
\index{assumes directive@\kcode{assumes} directive}

\index{directives!begin assumes@\kcode{begin assumes}}
\index{begin assumes directive@\kcode{begin assumes} directive}

\index{no_parallelism clause@\kcode{no_parallelism} clause}
\index{clauses!no_parallelism@\kcode{no_parallelism}}
\index{holds clause@\kcode{holds} clause}
\index{clauses!holds@\kcode{holds}}

Assumption directives provide additional information about the expected properties of 
the program that may be used by an implementation for optimization. 
Ignoring this information should not alter the behavior of the program. The C/C++ example 
shows the use of delimited scope (Case 1) and block-associated (Case 2) assumption directives.
A similar effect is shown for Fortran where the \kcode{assumes} directive is used in the module (Case 1)
and the block-associated directive uses an \kcode{end assume} termination (Case 2). 
The function \ucode{fun} is annotated with the \kcode{no_parallelism} clause, using the \kcode{begin assumes} 
(C) or \kcode{assumes} (Fortran) directive, to indicate that no implicit/explicit tasks are generated and no 
SIMD constructs are encountered during execution of the function. If the function \ucode{fun} contains 
task-generating or SIMD constructs then the behavior would be undefined. The block-associated 
\kcode{assume} directive is used to indicate that \ucode{N} is a multiple of 8 and 
will always be equal to or greater than 1. This information, if used for optimization,
could eliminate additional checks.

\cexample[5.1]{assumption}{1}
\ffreeexample[5.1]{assumption}{1}

%\pagebreak
\section{Conditional Compilation}
\label{sec:cond_comp}
\index{conditional compilation!_OPENMP macro@\kcode{_OPENMP} macro}
\index{conditional compilation!sentinel}

\ccppspecificstart
The following example illustrates the use of conditional compilation using the 
OpenMP macro \kcode{_OPENMP}. With OpenMP compilation, the \kcode{_OPENMP} 
macro becomes defined.

\cnexample{cond_comp}{1}
\ccppspecificend

\fortranspecificstart
The following example illustrates the use of the conditional compilation sentinel. 
With OpenMP compilation, the conditional compilation sentinel \scode{!$} is recognized 
and treated as two spaces. In fixed form source, statements guarded by the sentinel 
must start after column 6.

\fnexample{cond_comp}{1}
\fortranspecificend


%\pagebreak
\section{Internal Control Variables (ICVs)}
\label{sec:icv}
\index{internal control variables}

According to the \docref{Internal Control Variables} section of the OpenMP 4.0 specification, an OpenMP implementation must act as if there are ICVs that control 
the behavior of the program.  This example illustrates two ICVs, \plc{nthreads-var} 
and \plc{max-active-levels-var}. The \plc{nthreads-var} ICV controls the 
number of threads requested for encountered parallel regions; there is one copy 
of this ICV per task. The \plc{max-active-levels-var} ICV controls the maximum 
number of nested active parallel regions; there is one copy of this ICV for the 
whole program.

In the following example, the \plc{nest-var}, \plc{max-active-levels-var}, 
\plc{dyn-var}, and \plc{nthreads-var} ICVs are modified through calls to 
the runtime library routines \kcode{omp_set_nested},\\ \kcode{omp_set_max_active_levels}, \kcode{omp_set_dynamic}, 
and \kcode{omp_set_num_threads} respectively. These ICVs 
affect the operation of \kcode{parallel} regions. Each implicit task generated 
by a \kcode{parallel} region has its own copy of the \plc{nest-var}, \plc{dyn-var},
and \plc{nthreads-var} ICVs.

In the following example, the new value of \plc{nthreads-var} applies only to 
the implicit tasks that execute the call to \kcode{omp_set_num_threads}. There 
is one copy of the \plc{max-active-levels-var} ICV for the whole program and 
its value is the same for all tasks. This example assumes that nested parallelism 
is supported.

The outer \kcode{parallel} region creates a team of two threads; each of the threads 
will execute one of the two implicit tasks generated by the outer \kcode{parallel} 
region.

Each implicit task generated by the outer \kcode{parallel} region calls \kcode{omp_set_num_threads(\ucode{3})}, 
assigning the value 3 to its respective copy of \plc{nthreads-var}. Then each 
implicit task encounters an inner \kcode{parallel} region that creates a team 
of three threads; each of the threads will execute one of the three implicit tasks 
generated by that inner \kcode{parallel} region.

Since the outer \kcode{parallel} region is executed by 2 threads, and the inner 
by 3, there will be a total of 6 implicit tasks generated by the two inner \kcode{parallel} 
regions.

Each implicit task generated by an inner \kcode{parallel} region will execute 
the call to\\ \kcode{omp_set_num_threads(\ucode{4})}, assigning the value 4 to its respective 
copy of \plc{nthreads-var}.

The print statement in the outer \kcode{parallel} region is executed by only one 
of the threads in the team. So it will be executed only once.

The print statement in an inner \kcode{parallel} region is also executed by only 
one of the threads in the team. Since we have a total of two inner \kcode{parallel} 
regions, the print statement will be executed twice -- once per inner \kcode{parallel} 
region.

\pagebreak
\cexample{icv}{1}

\fexample{icv}{1}


\pagebreak
\section{Placement of \code{flush}, \code{barrier}, \code{taskwait} 
and \code{taskyield} Directives}
\label{sec:standalone}
\index{standalone directive placement}
\index{constructs!flush@\code{flush}}
\index{constructs!barrier@\code{barrier}}
\index{constructs!taskwait@\code{taskwait}}
\index{constructs!taskyield@\code{taskyield}}
\index{flush construct@\code{flush} construct}
\index{barrier construct@\code{barrier} construct}
\index{taskwait construct@\code{taskwait} construct}
\index{taskyield construct@\code{taskyield} construct}

The following example is non-conforming, because the \code{flush}, \code{barrier}, 
\code{taskwait}, and \code{taskyield}  directives are stand-alone directives 
and cannot be the immediate substatement of an \code{if} statement. 

\cexample[3.1]{standalone}{1}

\pagebreak
The following example is non-conforming, because the \code{flush}, \code{barrier}, 
\code{taskwait}, and \code{taskyield}  directives are stand-alone directives 
and cannot be the action statement of an \code{if} statement or a labeled branch 
target.

\ffreeexample[3.1]{standalone}{1}

The following version of the above example is conforming because the \code{flush}, 
\code{barrier}, \code{taskwait}, and \code{taskyield} directives are enclosed 
in a compound statement. 

\cexample[3.1]{standalone}{2}

\pagebreak
The following example is conforming because the \code{flush}, \code{barrier}, 
\code{taskwait}, and \code{taskyield} directives are enclosed in an \code{if} 
construct or follow the labeled branch target.

\ffreeexample[3.1]{standalone}{2}



%\pagebreak
\section{Cancellation Constructs}
\label{sec:cancellation}
\index{cancellation!cancel construct@\kcode{cancel} construct}
\index{constructs!cancel@\kcode{cancel}}
\index{cancel construct@\kcode{cancel} construct}

The examples in this section show how the \kcode{cancel} directive can be used to terminate 
an OpenMP region. Cancellation of the binding region is activated only if the \plc{cancel-var} ICV 
is true, in which case the \kcode{cancel} construct (except \kcode{taskgroup}) causes the encountering 
\kcode{task} to continue execution at the end of the binding. If the \plc{cancel-var} ICV is false, the 
\kcode{cancel} construct is ignored.

\index{cancellation!for parallel region@for \kcode{parallel} region}
\index{cancellation!for worksharing region}

In the following example although the \kcode{cancel} construct terminates the OpenMP 
worksharing region, programmers must still track the exception through the pointer 
\ucode{ex} and issue a cancellation for the \kcode{parallel} region if an exception has
been raised. The primary thread checks the exception pointer to make sure that the 
exception is properly handled in the sequential part. If cancellation of the \kcode{parallel} 
region has been requested, some threads might have executed \ucode{phase_1()}. 
However, it is guaranteed that none of the threads executed \ucode{phase_2()}.

\cppexample[4.0]{cancellation}{1}


\index{cancellation!cancellation point construct@\kcode{cancellation point} construct}
\index{constructs!cancellation point@\kcode{cancellation point}}
\index{cancellation point construct@\kcode{cancellation point} construct}
The following example illustrates the use of the \kcode{cancel} construct in error 
handling. If there is an error condition from the \bcode{allocate} statement, 
the cancellation is activated. The encountering thread sets the shared variable 
\ucode{err} and other threads of the binding thread set proceed to the end of 
the worksharing construct after the cancellation has been activated. 

\ffreeexample[4.0]{cancellation}{1}

\index{cancellation!for taskgroup region@for \kcode{taskgroup} region}
The following example shows how to cancel a parallel search on a binary tree as 
soon as the search value has been detected. The code creates a task to descend 
into the child nodes of the current tree node. If the search value has been found, 
the code remembers the tree node with the found value through an \kcode{atomic} 
write to the result variable (\ucode{found}) and then cancels execution of all search tasks. The
function \ucode{search_tree_parallel} groups all search tasks into a single 
task group to control the effect of the \kcode{cancel taskgroup} directive. The 
\ucode{level} argument is used to create undeferred tasks after the first ten 
levels of the tree.

\cexample[5.1]{cancellation}{2}


The following is the equivalent parallel search example in Fortran.
The code uses the \kcode{atomic write} directive for atomically
updating pointer variables -- a feature defined in OpenMP 6.0.
For earlier versions of OpenMP, the \kcode{critical} directive could 
be used instead.

\ffreeexample[6.0]{cancellation}{2}



%\pagebreak
\section{\kcode{requires} Directive}
\label{sec:requires}
\index{directives!requires@\kcode{requires}}
\index{requires directive@\kcode{requires} directive}

The declarative \kcode{requires} directive can be used to 
specify features that an implementation must provide to compile and 
execute correctly.

\index{requires directive@\kcode{requires} directive!unified_shared_memory clause@\kcode{unified_shared_memory} clause}
\index{clauses!unified_shared_memory@\kcode{unified_shared_memory}}
\index{unified_shared_memory clause@\kcode{unified_shared_memory} clause}
In the following example the \kcode{unified_shared_memory} clause 
of the \kcode{requires} directive ensures that the host and all 
devices accessible through OpenMP provide a \plc{unified address} space
for memory that is shared by all devices.

The example illustrates the use of the \kcode{requires} directive specifying
\plc{unified shared memory} in file scope, before any device 
directives or device routines. No \kcode{map} clause is needed for
the \ucode{p} structure on the device (and its address \ucode{\&p}, for the C++ code,
is the same address on the host and device).
However, scalar variables referenced within the \kcode{target}
construct still have a default data-sharing attribute of \kcode{firstprivate}.
The \ucode{q} scalar is incremented on the device, and its change is
not updated on the host.
% will defaultmap(toform:scalar) make q use shared address space? 
%Or will it be ignored at this point.
% Does before device routines also mean before prototype?

%\pagebreak

\cppexample[5.0]{requires}{1}

\ffreeexample[5.0]{requires}{1}

%\pagebreak
\section{Context-based Variant Selection}
\label{sec:context_based_variants}

\index{directives!declare variant@\kcode{declare variant}}
\index{directives!metadirective@\kcode{metadirective}}

\index{OpenMP context@OpenMP context}
\index{context selector@context selector}
\index{trait selector set@trait selector set}
\index{trait selector@trait selector}
\index{trait property@trait property}

Certain directives, including \kcode{declare variant},
\kcode{begin declare variant}, and \kcode{metadirective}
directives, specify function or directive variants for callsite or directive
substitution. They use \plc{context selectors} to specify the contexts in which
the variant may be selected for substitution. A context selector specifies
various \plc{trait selectors}, grouped into \plc{trait selector sets}. A trait
selector, for a given trait selector set, identifies a corresponding trait
(and, in some cases, its trait properties) that may or may not be active in an
\plc{OpenMP context}. A context selector is considered to be \plc{compatible}
with a given OpenMP context if all traits and trait properties corresponding to
trait selectors are active in that context. 

Each context selector is a comma-separated list of trait selector sets and each
trait selector set has the form \plc{trait-selector-name}~\kcode{=\{}~
\plc{trait-selector-list}~\kcode{\}}, where \plc{trait-selector-list} is a
comma-separated list of trait selectors. Some trait selectors may in turn
specify one or more \plc{trait properties}. Additionally, a trait selector may
optionally specify a \plc{trait score} for explicit control over variant
selection.

Consider this context selector: \kcode{construct=\{teams,parallel,for\},} 
\kcode{device=\{arch(nvptx)\},} \kcode{user=\{condition(\ucode{N>32})\}}. 

The context selector specifies three distinct trait selector sets, a
\kcode{construct} trait selector set, a \kcode{device} trait selector set, and
a \kcode{user} trait selector set. The \kcode{construct} trait selector set
specifies three trait selectors: \kcode{teams}, \kcode{parallel}, and
\kcode{for}. The \kcode{device} trait selector set specifies one trait
selector: \kcode{arch(nvptx)}. And the \kcode{user} trait selector set
specifies one trait selector: \kcode{condition(\ucode{N>32})}. 

The \kcode{teams}, \kcode{parallel}, and \kcode{for} trait selectors
respectively require that the \plc{teams}, \plc{parallel}, and \plc{for} traits
are active in the \plc{construct} trait set of the OpenMP context (i.e., the
\kcode{teams}, \kcode{parallel}, and \kcode{for} constructs are enclosing
constructs that do not appear outside any enclosing \kcode{target} construct at
the program point of interest). The \kcode{arch} trait selector specifies the
\kcode{nvptx} trait property, requiring that \plc{nvptx} is one of the
supported architectures per the \plc{arch} trait of the \plc{device} trait set
of the OpenMP context. Finally, the \kcode{condition} trait selector specifies
the \ucode{N>32} expression as a trait property, requiring that \ucode{N>32}
evaluates to \plc{true} in the OpenMP context.

The remainder of this section presents examples that make use of context
selectors for function and directive variant selection. Sections
\ref{subsec:declare_variant} and \ref{subsec:metadirective} cover cases where
only one context selector is compatible. 
Section \ref{subsec:context_selector_scoring} covers cases where 
multiple compatible context selectors exist and a scoring algorithm 
determines which one of the variants is selected.

\subsection{\kcode{declare variant} Directive}
\label{subsec:declare_variant}
\index{directives!declare variant@\kcode{declare variant}}
\index{declare variant directive@\kcode{declare variant} directive}
\index{declare variant directive@\kcode{declare variant} directive!match clause@\kcode{match} clause}
\index{clauses!match@\kcode{match}}
\index{match clause@\kcode{match} clause}

A \kcode{declare variant} directive specifies an alternate function, 
\plc{function variant}, to be used in place of the \plc{base function} 
%when the trait within the \kcode{match} clause has a valid context.
when the trait within the \kcode{match} clause matches the OpenMP context at a given callsite.
The base function follows the directive in the C and C++ languages.
In Fortran, either a subroutine or function may be used as the base function,
and the \kcode{declare variant} directive must be in the specification 
part of a subroutine or function (unless a \plc{base-proc-name}
modifier is used, as in the case of a procedure declaration statement). See
the OpenMP 5.0 Specification for details on the modifier.

When multiple \kcode{declare variant} directives are used 
a function variant becomes a candidate for replacing the base function if the
%base function call context matches the traits of all selectors in the \kcode{match} clause.
context at the base function call matches the traits of all selectors in the \kcode{match} clause.
If there are multiple candidates, a score is assigned with rules for each
of the selector traits. 
See Section \ref{subsec:context_selector_scoring} for details.
%The scoring algorithm can be found in the OpenMP 5.0 Specification.

In the first example the \ucode{vxv()} function is called within a \kcode{parallel} region,
a \kcode{target} region, and in a sequential part of the program.  Two function variants, \ucode{p_vxv()} and \ucode{t_vxv()},
are defined for the first two regions by using \kcode{parallel} and \kcode{target} selectors (within
the \plc{construct} trait set) in a \kcode{match} clause.  The \ucode{p_vxv()} function variant includes
a \kcode{for} construct (\kcode{do} construct for Fortran) for the \kcode{parallel} region, 
while \ucode{t_vxv()} includes a \kcode{distribute simd} construct for the \kcode{target} region.
The \ucode{t_vxv()} function is explicitly compiled for the device using a declare target directive.

Since the two \kcode{declare variant} directives have no selectors that match traits for the context
of the base function call in the sequential part of the program, the base \ucode{vxv()} function is used there, 
as expected.
(The vectors in the \ucode{p_vxv} and \ucode{t_vxv} functions have been multiplied
by 3 and 2, respectively, for checking the validity of the replacement. Normally
the purpose of a function variant is to produce the same results by a different method.)

%Note: a \code{target teams} construct is used to direct execution onto a device, with a
%\code{distribute simd} construct in the function variant. As of the OpenMP 5.0 implementation
%no intervening code is allowed between a \code{target} and \code{teams} construct. So
%using a \code{target} construct to direct execution onto a device, and including 
%\code{teams distribute simd} in the variant function would produce non conforming code.

\cexample[5.1]{declare_variant}{1}

\ffreeexample[5.0]{declare_variant}{1}

In this example, traits from the \plc{device} set are used to select a function variant.
In the \kcode{declare variant} directive, an \kcode{isa} trait selector
specifies that if the implementation of the ``\vcode{core-avx512}''
instruction set is detected at compile time the \ucode{avx512_saxpy()}
variant function is used for the call to \ucode{base_saxpy()}.  

A compilation of \ucode{avx512_saxpy()} is aware of
the AVX-512 instruction set that supports 512-bit vector extensions.
Within \ucode{avx512_saxpy()}, the \kcode{parallel for simd} construct performs parallel execution, and
takes advantage of 64-byte data alignment. 
When the \ucode{avx512_saxpy()} function variant is not selected, the base \ucode{base_saxpy()} function variant
containing only a basic \kcode{parallel for} construct is used for the call to \ucode{base_saxpy()}.

%Note:
%An allocator is used to set the alignment to 64 bytes when an OpenMP compilation is performed.  
%Details about allocator variable declarations and functions
%can be found in the allocator example of the Memory Management Chapter.

\cexample[5.0]{declare_variant}{2}

\ffreeexample[5.0]{declare_variant}{2}

The \kcode{begin declare variant} with a paired \kcode{end declare variant} directive was introduced 
for C/C++ in the OpenMP 5.1 to allow nesting of declare variant directives. 
This example shows a practical situation where nested declare variant directives can be used 
to include the correct specialized user function based on the underlying vendor \kcode{isa} trait. 
The function name \ucode{my_fun()} is identical in all the header files and the version called will 
differ based on the calling context. The example assumes that either NVIDIA or AMD target devices are used. 

\index{directives!begin declare variant@\kcode{begin declare variant}}
\index{begin declare variant directive@\kcode{begin declare variant} directive}

\cexample[5.1]{declare_variant}{3}

%%%%%%%%%%%%%
\subsection{Metadirectives}
\label{subsec:metadirective}
\index{directives!metadirective@\kcode{metadirective}}
\index{metadirective directive@\kcode{metadirective} directive}

\index{metadirective directive@\kcode{metadirective} directive!when clause@\kcode{when} clause}
\index{metadirective directive@\kcode{metadirective} directive!otherwise clause@\kcode{otherwise} clause}
\index{clauses!when@\kcode{when}}
\index{when clause@\kcode{when} clause}
\index{clauses!otherwise@\kcode{otherwise}}
\index{otherwise clause@\kcode{otherwise} clause}
A \kcode{metadirective} directive provides a mechanism to select a directive in
a \kcode{when} clause to be used, depending upon one or more contexts:  
implementation, available devices and the present enclosing construct. 
The directive in an \kcode{otherwise} clause is used when a directive of the 
\kcode{when} clause is not selected.

\index{context selector!construct@\plc{construct}}
In the \kcode{when} clause the \plc{context selector} (or just \plc{selector}) defines traits that are
evaluated for selection of the directive that follows the selector. 
This ``selectables'' directive is called a \plc{directive variant}.
%Traits are grouped by \plc{construct}, \plc{implementation} and 
%\plc{device} \plc{sets} to be used by a selector of the same name.

\index{context selector!device@\plc{device}}
In the first example the \plc{arch} trait of the
\kcode{device} selector set specifies that if an \ucode{nvptx} architecture is
active in the OpenMP context, then the \kcode{teams loop}
directive variant is selected as the directive; otherwise, the \kcode{parallel loop}
directive variant of the \kcode{otherwise} clause is selected as the directive.
That is, if a device of \ucode{nvptx} architecture is supported by the implementation within
the enclosing \kcode{target} construct, its directive variant is selected.
The architecture names, such as \ucode{nvptx}, are implementation defined.
Also, note that the \kcode{device} clause specified in a \kcode{target} construct specifies
a device number, while \kcode{device}, as used in the \kcode{metadirective}
directive as selector set, has traits of \plc{kind}, \plc{isa} and \plc{arch}.


\cexample[5.2]{metadirective}{1}

\ffreeexample[5.2]{metadirective}{1}
\pagebreak

\index{context selector!implementation@\plc{implementation}}
In the second example, the \kcode{implementation} selector set is specified
in the \kcode{when} clause to distinguish between platforms. 
Additionally, specific architectures are specified with the \kcode{device}
selector set.

In the code, different \kcode{teams} constructs are employed as determined
by the \kcode{metadirective} directive.
The number of teams is restricted by a \kcode{num_teams} clause
and a thread limit is also set by a \kcode{thread_limit} clause for 
vendor platforms and specific architecture
traits.  Otherwise, just the \kcode{teams} construct is used without
any clauses, as prescribed by the \kcode{otherwise} clause.


\cexample[5.2]{metadirective}{2}

\ffreeexample[5.2]{metadirective}{2}

\index{context selector!construct@\plc{construct}}

\index{directives!declare target@\kcode{declare target}}
\index{declare target directive@\kcode{declare target} directive}

\index{directives!begin declare target@\kcode{begin declare target}}
\index{begin declare target directive@\kcode{begin declare target} directive}

In the third example, a \kcode{construct} selector set is specified in the \kcode{when} clause.
Here, a \kcode{metadirective} directive is used within a function that is also
compiled as a function for a target device as directed by a declare target directive.
The \kcode{target} directive name of the \kcode{construct} selector ensures that the
\kcode{distribute parallel for/do} construct is employed for the target compilation.
Otherwise, for the host-compiled version the \kcode{parallel for/do simd} construct is used.

In the first call to the \ucode{exp_pi_diff()} routine the context is a
\kcode{target teams} construct and the \kcode{distribute parallel for/do}
construct version of the function is invoked,
while in the second call the \kcode{parallel for/do simd} construct version is used.

%%%%%%%%
This case illustrates an important point for users that may want to hoist the 
\kcode{target} directive out of a function that contains the usual 
\kcode{target teams distribute parallel for/do} construct
(for providing alternate constructs through the \kcode{metadirective} directive as here).
While this combined construct can be decomposed into a \kcode{target} and
\kcode{teams distribute parallel for/do} constructs, the OpenMP 5.0 specification has the restriction:
``If a \kcode{teams} construct is nested within a \kcode{target} construct, that \kcode{target} construct must
contain no statements, declarations or directives outside of the \kcode{teams} construct''.
So, the \kcode{teams} construct must immediately follow the \kcode{target} construct without any intervening
code statements (which includes function calls).  
Since the \kcode{target} construct alone cannot be hoisted out of a function, 
the \kcode{target teams} construct has been hoisted out of the function, and 
the \kcode{distribute parallel for/do} construct is used
as the variant directive of the \kcode{metadirective} directive within the function.
%%%%%%%%

\cexample[5.2]{metadirective}{3}

\ffreeexample[5.2]{metadirective}{3}
\pagebreak

\index{context selector!user@\plc{user}}
\index{context selector!condition selector@\kcode{condition} selector}

The \kcode{user} selector set can be used in a \kcode{metadirective}
to select directives at execution time when the 
\kcode{condition( \plc{boolean-expr} )} selector expression is not a constant expression.
In this case it is a \plc{dynamic} trait set, and the selection is made at run time, rather
than at compile time.

In the following example the \ucode{foo} function employs the \kcode{condition}
selector to choose a device for execution at run time. 
In the \ucode{bar} routine metadirectives are nested.
At the outer level a selection between serial and parallel execution in performed
at run time, followed by another run time selection on the schedule kind in the inner
level when the active \plc{construct} trait is \kcode{parallel}.

(Note, the variable \ucode{b} in two of the ``selected'' constructs is declared private for the sole purpose 
of detecting and reporting that the construct is used. Since the variable is private, its value 
is unchanged outside of the construct region, whereas it is changed if the ``unselected'' construct
is used.)

%(Note: The value of \plc{b} after the \code{parallel} region remains 0 for the 
%\code{guided} scheduling case, because its \code{parallel} construct also contains
%the \code{private(}~\plc{b}~\code{)} clause. 
%The variable \plc{b} is employed for the sole purpose of distinguishing which 
%\code{parallel} construct is selected-- for testing.)

%While there might be other ways to make these decisions at run time, such as using 
%an \code{if} clause on a \code{parallel} construct, this mechanism is much more general.  
%For instance, an input ``gpu\_type'' string could be used and tested in boolean expressions 
%to select from one of several possible \code{target} constructs.
%Also, setting the scheduling variable (\plc{unbalanced}) within the execution through a 
%``work balance'' function might be a more practical approach for setting the schedule kind.


\cexample[5.2]{metadirective}{4}

\ffreeexample[5.2]{metadirective}{4}
\pagebreak

Metadirectives can be used in conjunction with templates as shown in the C++ code below.
Here the template definition generates two versions of the Fibonacci function.
The \ucode{tasking} boolean is used in the \kcode{condition} selector to enable tasking.
The true form implements a parallel version with \kcode{task} and \kcode{taskwait}
constructs as in the \example{tasking.4.c} code in Section~\ref{sec:task_taskwait}. 
The false form implements a serial version without any tasking constructs.
Note that the serial version is used in the parallel function for optimally
processing numbers less than 8.

\cppexample[5.0]{metadirective}{5}

%\pagebreak
\subsection{Context Selector Scoring}

\label{subsec:context_selector_scoring}
\index{context selector scoring@context selector scoring}

Each context selector for which all specified traits are active in the current
\plc{OpenMP context} is a \plc{compatible context selector}, and the associated
function variant or directive variant for such a context selector is a
\plc{replacement candidate}. The final \plc{score} of each of the compatible
context selectors determine which of the replacement candidates is selected for
substitution. 

For a given compatible context selector, the score is calculated according
to the specified trait selectors and their corresponding traits. If the trait
selectors are a strict subset of the trait selectors specified by another
compatible context selector then the score of the context selector is zero.
Otherwise, the final score is one plus the sum of the score values of each
specified trait selector.

A replacement candidate is selected if no other candidate has a higher scoring
context selector. If multiple replacement candidates have a context selector
with the same highest score, the one specified first on the metadirective is
selected. If multiple function variants are replacement candidates that have
context selectors with the same highest score, the one that is selected is
implementation defined.

If a \kcode{construct} selector set is specified in the context selector, each
active construct trait that is named in that selector set contributes a score
of $2^{p-1}$, where $p$ is the position of that trait in the current
\plc{construct} trait set (the set of traits in the OpenMP context). If a
\kcode{device} or \kcode{target_device} selector set is specified in the
selector, then an active \plc{kind}, \plc{arch}, or \plc{isa} trait that is
named in the selector set contributes a score of $2^l$, $2^{l+1}$, and
$2^{l+2}$, respectively, where $l$ is the number of traits in the
\plc{construct} trait set. For any other active traits that are named in the
context selector that are not implementation-defined extensions, the
contributed score, by default, is zero. 

The default score for any active traits other than \plc{construct} traits and 
the \plc{kind}, \plc{arch}, or \plc{isa} traits may be overridden with an
explicit score expression. Specifying an explicit score is only recommended
for prioritizing replacement candidates for which a selection is not 
dependent on construct traits. That is, none of the compatible context
selectors specify a \kcode{construct} trait selector or a \kcode{kind},
\kcode{arch}, or \kcode{isa} trait selector. 

In the following example, four function variants are declared for the procedure
\ucode{f}: \ucode{fx1}, \ucode{fx2}, \ucode{fx3}, and \ucode{fx4}. Suppose that
the target device for the \kcode{target} region has the \plc{gpu} device kind,
has the \plc{nvptx} architecture, and supports the \splc{sm_70} instruction set
architecture. Hence, the context selectors for all function variants are
compatible with the context at the callsite for \ucode{f} inside the
\kcode{target} region. The \plc{construct} trait set at the callsite,
consisting of all enclosing constructs and having a count of \plc{l=6}, is:
\{\plc{target}, \plc{teams}, \plc{distribute}, \plc{parallel},
\plc{for}/\plc{do}, \plc{task}\}. Note that only \plc{context-matching}
constructs, which does not include \kcode{distribute} or \kcode{task}, may be
named by a \kcode{construct} trait selector as of OpenMP 5.2. The score for
\ucode{fx1} is $1+2^0=2$, for \ucode{fx2} is $1+2^1+2^3+2^4=27$,
for \ucode{fx3} is $1+2^6+2^8=321$, and for \ucode{fx4} is $1+2^7+2^8=385$.
Since \ucode{fx4} is the function variant that has the highest scoring
selector, it is selected by the implementation at the callsite.

\cexample[5.0]{selector_scoring}{1}

\ffreeexample[5.0]{selector_scoring}{1}

In the next example, three function variants are declared for the procedure
\ucode{kernel}: \ucode{kernel_target_ua}, \ucode{kernel_target_usm}, and 
\ucode{kernel_target_usm_v2}. Suppose that the implementation supports the 
\splc{unified_address} and \splc{unified_shared_memory} requirements, so that
the context selectors for all function variants are compatible. The score for 
\ucode{kernel_target_ua} is 1, which is one plus the zero score associated with
the active \splc{unified_address} requirement. The score for 
\ucode{kernel_target_usm} is 0, as the selector is a strict subset of the 
selector for \ucode{kernel_target_usm_v2}. The score for 
\ucode{kernel_target_usm_v2} is 2, which is one plus the explicit score of 1
for the \plc{condition} trait and the zero score associated with the acive
\splc{unified_shared_memory} requirement . Since \ucode{kernel_target_usm_v2}
is the function variant that has the highest scoring selector, it is selected
by the implementation at the callsite.

\cexample[5.0]{selector_scoring}{2}

\ffreeexample[5.0]{selector_scoring}{2}

\pagebreak
\section{\kcode{dispatch} Construct}
\label{sec:dispatch}

\index{construct!dispatch@\kcode{dispatch}}
\index{dispatch construct@\kcode{dispatch} construct}

\index{dispatch construct@\kcode{dispatch} construct!novariants clause@\kcode{novariants} clause}
\index{clauses!novariants@\kcode{novariants}}
\index{novariants clause@\kcode{novariants} clause}

\index{dispatch construct@\kcode{dispatch} construct!nocontext clause@\kcode{nocontext} clause}
\index{clauses!nocontext@\kcode{nocontext}}
\index{nocontext clause@\kcode{nocontext} clause}

The \kcode{dispatch} directive can be applied to a statement that performs a
procedure call to control variant substitution for the called procedure.

In the example below, the \ucode{foo_variant1()} and \ucode{foo_variant2()}
procedures are declared as variants for \ucode{foo()} using the
\kcode{declare variant} directive, with matching requirements specified
by the \kcode{match} clause's context selector. To be selected for
substitution, both variants require that the condition \ucode{foo_sub} evaluates
to \plc{true}.

In Cases 1 and 2, the calls to \ucode{foo()} are not controlled by a
\kcode{dispatch} construct. Hence, there can be no match for the
\ucode{foo_variant2()} variant. A \ucode{foo_variant1()} call is substituted for
the call to \ucode{foo()} in Case 1, as the matching requirement
is satisfied by \ucode{foo_sub} being \plc{true}. In Case 2, there is
no variant substitution as \ucode{foo_sub} is \plc{false}.

Cases 3 through 6 illustrate some uses of the \kcode{dispatch} construct,
including uses of the \kcode{novariants} and \kcode{nocontext} clauses on the
directive.

In Case 3, variant substitution does not occur as \ucode{foo_sub} is \plc{false}.

In Case 4, \ucode{foo_sub} is \plc{true} and the \kcode{dispatch} construct is
part of the OpenMP context; therefore, the matching requirements for both
variants to \ucode{foo()} are satisfied. As the matching requirements for the
\ucode{foo_variant1()} variant are a subset of the matching requirements for the
\ucode{foo_variant2()} variant (per the OpenMP specification, its computed score
for matching purposes is smaller), \ucode{foo_variant2()} is selected for
variant substitution. (Note that prior to OpenMP 6.0, which of the two variants
are selected for substitution is implementation defined since the earlier
specifications did not require an implementation to treat the \kcode{dispatch}
construct as part of the OpenMP context at the call site.)

In Case 5, the \kcode{novariants} clause disables variant
substitution for the call to \ucode{foo()}, despite the matching requirements
being satisfied for both variants.

In Case 6, the  \kcode{nocontext} clause directs the implementation to not
include the \kcode{dispatch} construct in the OpenMP context at the call site
for \ucode{foo()}. Hence, the \ucode{foo_variant2()} variant is not considered
and \ucode{foo_variant1()} is instead selected for variant substitution.

      \cexample[6.0]{dispatch}{1}
  \ffreeexample[6.0]{dispatch}{1}

\pagebreak
\section{Nested Loop Constructs}
\label{sec:nested_loop}
\index{nested loop constructs}

The following example of loop construct nesting is conforming because the inner 
and outer loop regions bind to different \code{parallel} regions:

\cexample{nested_loop}{1}

\fexample{nested_loop}{1}

The following variation of the preceding example is also conforming:

\cexample{nested_loop}{2}

\fexample{nested_loop}{2}



%\pagebreak
\section{Restrictions on Nesting of Regions}
\label{sec:nesting_restrict}

\index{region nesting rules}
The examples in this section illustrate the region nesting rules. 

The following example is non-conforming because the inner and outer loop regions 
are closely nested:

\cexample{nesting_restrict}{1}

\fexample{nesting_restrict}{1}

The following orphaned version of the preceding example is also non-conforming:

\cexample{nesting_restrict}{2}

\fexample{nesting_restrict}{2}

The following example is non-conforming because the loop and \kcode{single} regions 
are closely nested:

\cexample{nesting_restrict}{3}
\pagebreak

\fexample{nesting_restrict}{3}

The following example is non-conforming because a \kcode{barrier} region cannot 
be closely nested inside a loop region:

\cexample{nesting_restrict}{4}

\fexample{nesting_restrict}{4}

The following example is non-conforming because the \kcode{barrier} region cannot 
be closely nested inside the \kcode{critical} region. If this were permitted, 
it would result in deadlock due to the fact that only one thread at a time can 
enter the \kcode{critical} region:

\cexample{nesting_restrict}{5}

\fexample{nesting_restrict}{5}

The following example is non-conforming because the \kcode{barrier} region cannot 
be closely nested inside the \kcode{single} region. If this were permitted, it 
would result in deadlock due to the fact that only one thread executes the \kcode{single} 
region:

\cexample{nesting_restrict}{6}

\fexample{nesting_restrict}{6}



\pagebreak
\section{Target Offload}
\label{sec:target_offload}
\index{environment variables!OMP_TARGET_OFFLOAD@\scode{OMP_TARGET_OFFLOAD}}
\index{OMP_TARGET_OFFLOAD@\scode{OMP_TARGET_OFFLOAD}}

In the OpenMP 5.0 implementation the \code{OMP\_TARGET\_OFFLOAD}
environment variable was defined to change \plc{default} offload behavior. 
By \plc{default} the target code (region) is executed on the host if the target device 
does not exist or the implementation does not support the target device.  
%Last sentence uses words of the 5.0 spec pg. 21 lines 7-8

In an OpenMP 5.0 compliant implementation, setting the 
\code{OMP\_TARGET\_OFFLOAD} variable to \code{MANDATORY} will 
force the program to terminate execution when a \code{target} 
construct is encountered and the target device is not supported or is not available.
With a value \code{DEFAULT} the target region will execute on a device if the 
device exists and is supported by the implementation,
otherwise it will execute on the host.
Support for the \code{DISABLED}
value is optional; when it is supported the behavior is as if only the 
host device exists (other devices are considered non-existent to the runtime), 
and target regions are executed on the host.  

The following example reports execution behavior for different 
values of the \code{OMP\_TARGET\_OFFLOAD} variable. A handy routine 
for extracting the \code{OMP\_TARGET\_OFFLOAD} environment variable
value is deployed here, because the OpenMP API does not have a routine 
for obtaining the value. %(\texit{yet}).

Note: 
The example issues a warning when a pre-5.0 implementation is used,
indicating that the \code{OMP\_TARGET\_OFFLOAD} is ignored.
The value of the \code{OMP\_TARGET\_OFFLOAD} variable is reported 
when the \code{OMP\_DISPLAY\_ENV} 
environment variable is set to \code{TRUE} or \code{VERBOSE}.

%\pagebreak
\cexample[5.0]{target_offload_control}{1}

%\pagebreak
\ffreeexample[5.0]{target_offload_control}{1}


% OMP 4.5 target offload  15:9-11
%If the target device does not exist or the
%implementation does not support the target device, all target regions associated with that device
%execute on the host device.

\pagebreak
\section{\scode{omp_pause_resource} and \\
  \scode{omp_pause_resource_all} Routines}
\label{sec:pause_resource}
\index{routines!omp_pause_resource@\scode{omp_pause_resource}}
\index{omp_pause_resource routine@\scode{omp_pause_resource} routine}
\index{routines!omp_get_max_threads@\scode{omp_get_max_threads}}
\index{omp_get_max_threads routine@\scode{omp_get_max_threadsi} routine}
\index{routines!omp_get_initial_device@\scode{omp_get_initial_device}}
\index{omp_get_initial_device routine@\scode{omp_get_initial_device} routine}

Sometimes, it is necessary to relinquish resources created or allocated
for the OpenMP runtime environment to avoid interference with subsequent
actions as illustrated by the following example.  In the beginning 
either a call to the \scode{omp_get_max_threads} routine 
or the subsequent \code{parallel} construct may trigger resource allocation
by the OpenMP runtime, which may cause unexpected side effects 
for the subsequent \plc{fork} call.
It is desirable to relinquish OpenMP resources allocated before 
the fork by using the \scode{omp_pause_resource} routine for a given
device, in this case the host device.  The host device number is returned by 
the \scode{omp_get_initial_device} routine.
The \scode{omp_pause_hard} value is used here to free as many
OpenMP resources as possible.
After the fork, the child process will initialize its OpenMP runtime
environment when encountering the \code{parallel} construct.

\cexample[5.0]{pause_resource}{1}

\index{routines!omp_pause_resource_all@\scode{omp_pause_resource_all}}
\index{omp_pause_resource_all routine@\scode{omp_pause_resource_all} routine}
The following example illustrates a different use case. 
After executing the first parallel code (parallel region 1), 
the \plc{relinquish} program switches to executing an external parallel program
(called \plc{subprogram}, which is compiled from \splc{pause_resource.2b}).  
In order to make resources available for the external
subprogram, \plc{relinquish} calls \scode{omp_pause_resource_all}
to relinquish OpenMP resources used by the current program before
calling \scode{execute_command_line} to execute \plc{subprogram}.
The \scode{omp_pause_soft} value is used here to allow subsequent
OpenMP regions (parallel region 2) to restart more quickly.

\ffreeexample[5.0]{pause_resource}{2a}
\ffreeexample{pause_resource}{2b}

\pagebreak
\section{Controlling Concurrency and Reproducibility with 
the \code{order} Clause}
\label{sec:reproducible_modifier}

\index{clauses!order(concurrent)@\code{order(concurrent)}}
\index{order(concurrent) clause@\code{order(concurrent)} clause}

The \code{order} clause is used for controlling the parallel execution of 
loop iterations for one or more loops that are associated with a directive. 
It is specified with a clause argument and optional modifier. 
The only supported argument, introduced in OpenMP 5.0, is the keyword 
\code{concurrent} which indicates that the loop iterations may execute 
concurrently, including iterations in the same chunk per the loop schedule. 
Because of the relaxed execution permitted with an \code{order(concurrent)} 
clause, codes must not assume that any cross-iteration data dependences 
would be preserved or that any two iterations may execute on the same thread.

The following example in this section demonstrates the use of 
the \code{order(concurrent)} clause, without any modifiers, for controlling 
the parallel execution of loop iterations.
The \code{order(concurrent)} clause cannot be used for the second and third 
\code{parallel}~\code{for}/\code{do} constructs because of either having 
data dependences or accessing threadprivate variables.

\cexample[5.0]{reproducible}{1}

\ffreeexample[5.0]{reproducible}{1}

\index{order(concurrent) clause@\code{order(concurrent)} clause!reproducible modifier@\code{reproducible} modifier}
\index{order(concurrent) clause@\code{order(concurrent)} clause!unconstrained modifier@\code{unconstrained} modifier}
Modifiers to the \code{order} clause, introduced in OpenMP 5.1, may be 
specified to control the reproducibility of the loop schedule for 
the associated loop(s). A reproducible loop schedule will consistently 
yield the same mapping of iterations to threads (or SIMD lanes) if the 
directive name, loop schedule, iteration space, and binding region remain 
the same. The \code{reproducible} modifier indicates the loop schedule must 
be reproducible, while the \code{unconstrained} modifier indicates that 
the loop schedule is not reproducible.
If a modifier is not specified, then the \code{order} clause does not affect 
the reproducibility of the loop schedule.

The next example demonstrates the use of the \code{order(concurrent)} clause 
with modifiers for additionally controlling the reproducibility of a loop's 
schedule.
The two worksharing-loop constructs in the first \code{parallel} construct
specify that the loops have reproducible schedules, thus memory effects from iteration \plc{i} from the first loop will be observable to iteration \plc{i}
in the second loop. 
In the second \code{parallel} construct, the \code{order} clause does not 
control reproducibility for the loop schedules. However, since both loops 
specify the same static schedules, the schedules are reproducible and the 
data dependences between the loops are preserved by the execution.
In the third \code{parallel} construct, the \code{order} clause indicates 
that the loops are not reproducible, overriding the default reproducibility
prescribed by the specified static schedule. Consequentially, 
the \code{nowait} clause on the first worksharing-loop construct should not 
be used to ensure that the data dependences are preserved by the execution.

\cexample[5.1]{reproducible}{2}

\ffreeexample[5.1]{reproducible}{2}


\pagebreak
\section{\code{interop} Construct}
\label{sec:interop}

The \scode{interop} construct allows OpenMP to interoperate with foreign runtime environments.
In the example below, asynchronous cuda memory copies and a \splc{cublasDaxpy} routine are executed 
in a cuda stream. Also, an asynchronous target task execution (having a \scode{nowait} clause) 
and two explicit tasks are executed through OpenMP directives.  Scheduling dependences (synchronization) are
imposed on the foreign stream and the OpenMP tasks through \scode{depend} clauses. 

First, an interop object, \splc{obj}, is initialized for synchronization by including the
\scode{targetsync} \splc{interop-type} in the interop \scode{init} clause 
(\scode{init(}~\scode{targetsync,obj}~\scode{)}).  
The object provides access to the foreign runtime.
The \scode{depend} clause provides a dependence behavior
for foreign tasks associated with a valid object.

Next, the \scode{omp_get_interop_int} routine is used to extract the foreign 
runtime id (\scode{omp_ipr_fr_id}), and a test in the next statement ensures 
that the cuda runtime (\scode{omp_ifr_cuda}) is available.

Within the block for executing the \splc{cublasDaxpy} routine, a stream is acquired 
with the \scode{omp_get_interop_ptr} routine, which returns a cuda stream (\splc{s}).
The stream is included in the cublas handle, and used directly in the asynchronous memory
routines.  The following \scode{interop} construct, with the \scode{destroy} clause, 
ensures that the foreign tasks have completed.

\cexample[5.1]{interop}{1}

\pagebreak
\section{Utilities}
\label{sec:utilities}
This section contains examples of utility routines and features.

%---------------------------
\subsection{Timing Routines}
\label{subsec:get_wtime}
\index{routines!omp_get_wtime@\kcode{omp_get_wtime}}
\index{omp_get_wtime routine@\kcode{omp_get_wtime} routine}
\index{routines!omp_get_wtick@\kcode{omp_get_wtick}}
\index{omp_get_wtick routine@\kcode{omp_get_wtick} routine}

The \kcode{omp_get_wtime} routine can be used to measure the elapsed wall
clock time (in seconds) of code execution in a program.
The routine is thread safe and can be executed by multiple threads concurrently.
The precision of the timer can be obtained by a call to
the \kcode{omp_get_wtick} routine. The following example shows a use case.

\cexample{get_wtime}{1}

\ffreeexample{get_wtime}{1}


%---------------------------
\subsection{Environment Display}
\label{subsec:display_env}
\index{environment display!OMP_DISPLAY_ENV@\kcode{OMP_DISPLAY_ENV}}
\index{environment variables!OMP_DISPLAY_ENV@\kcode{OMP_DISPLAY_ENV}}
\index{OMP_DISPLAY_ENV@\kcode{OMP_DISPLAY_ENV}}
\index{environment display!omp_display_env routine@\kcode{omp_display_env} routine}
\index{routines!omp_display_env@\kcode{omp_display_env}}
\index{omp_display_env routine@\kcode{omp_display_env} routine}

The OpenMP version number and the values of ICVs associated with the relevant
environment variables can be displayed at runtime by setting 
the \kcode{OMP_DISPLAY_ENV} environment variable to either 
\vcode{TRUE} or \vcode{VERBOSE}.
The information is displayed once by the runtime.

A more flexible or controllable approach is to call 
the \kcode{omp_display_env} API routine at any desired
point of a code to display the same information.
This OpenMP 5.1 API routine takes a single \ucode{verbose} argument.
A value of 0 or \bcode{.false.} (for C/C++ or Fortran) indicates
the required OpenMP ICVs associated with environment variables be displayed,
and a value of 1 or \bcode{.true.} (for C/C++ or Fortran) will include
vendor-specific ICVs that can be modified by environment variables.

The following example illustrates the conditional execution of the API
\kcode{omp_display_env} routine.  Typically it would be invoked in
various debug modes of an application. 
An important use case is to have a single MPI process (e.g., rank = 0) 
of a hybrid (MPI+OpenMP) code execute the routine,
instead of all MPI processes, as would be done by 
setting the \kcode{OMP_DISPLAY_ENV} to \vcode{TRUE} or \vcode{VERBOSE}.

\cexample[5.1]{display_env}{1}

\ffreeexample[5.1]{display_env}{1}
\clearpage

A sample output from the execution of the code might look like:
{\small\begin{verbatim}
 OPENMP DISPLAY ENVIRONMENT BEGIN
    _OPENMP='202011'
   [host] OMP_AFFINITY_FORMAT='(null)'
   [host] OMP_ALLOCATOR='omp_default_mem_alloc'
   [host] OMP_CANCELLATION='FALSE'
   [host] OMP_DEFAULT_DEVICE='0'
   [host] OMP_DISPLAY_AFFINITY='FALSE'
   [host] OMP_DISPLAY_ENV='FALSE'
   [host] OMP_DYNAMIC='FALSE'
   [host] OMP_MAX_ACTIVE_LEVELS='1'
   [host] OMP_MAX_TASK_PRIORITY='0'
   [host] OMP_NESTED: deprecated; max-active-levels-var=1
   [host] OMP_NUM_THREADS: value is not defined
   [host] OMP_PLACES: value is not defined
   [host] OMP_PROC_BIND: value is not defined
   [host] OMP_SCHEDULE='static'
   [host] OMP_STACKSIZE='4M'
   [host] OMP_TARGET_OFFLOAD=DEFAULT
   [host] OMP_THREAD_LIMIT='0'
   [host] OMP_TOOL='enabled'
   [host] OMP_TOOL_LIBRARIES: value is not defined
 OPENMP DISPLAY ENVIRONMENT END
\end{verbatim}}


%---------------------------
\subsection{\kcode{error} Directive}
\label{subsec:error}
\index{directives!error@\kcode{error}}
\index{error directive@\kcode{error} directive}
\index{error directive@\kcode{error} directive!at clause@\kcode{at} clause}
\index{clauses!at@\kcode{at}}
\index{at clause@\kcode{at} clause}
\index{error directive@\kcode{error} directive!severity clause@\kcode{severity} clause}
\index{clauses!severity@\kcode{severity}}
\index{severity clause@\kcode{severity} clause}

The \kcode{error} directive provides a consistent method for C, C++, and Fortran to emit a \kcode{fatal} or
\kcode{warning} message at \kcode{compilation} or \kcode{execution} time, as determined by a \kcode{severity}
or an \kcode{at} clause, respectively. When \kcode{severity(fatal)} is present, the compilation 
or execution is aborted. Without any clauses the default behavior is as if \kcode{at(compilation)} 
and \kcode{severity(fatal)} were specified.

The C, C++, and Fortran examples below show all the cases for reporting messages.

\cexample[5.2]{error}{1}
\ffreeexample[5.2]{error}{1}





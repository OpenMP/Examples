%\pagebreak
\section{Declare Target Directive}
\label{sec:declare_target}

%\index{declare target directive@\code{declare target} directive!enter clause@\code{enter} clause}
%\index{enter clause@\code{enter} clause}
%\index{clauses!enter@\code{enter}}

\subsection{Declare Target Directive for a Procedure}

\label{subsec:declare_target_function}

\index{directives!declare target@\kcode{declare target}}
\index{declare target directive@\kcode{declare target} directive}

\index{directives!begin declare target@\kcode{begin declare target}}
\index{begin declare target directive@\kcode{begin declare target} directive}

The following example shows how the declare target directive 
is used to indicate that the corresponding call inside a \kcode{target} region 
is to a \ucode{fib} procedure that can execute on the default target device.

A version of the function is also available on the host device. When the \kcode{if} 
clause conditional expression on the \kcode{target} construct evaluates to \plc{false}, 
the \kcode{target} region (thus \ucode{fib}) will execute on the host device.

For the following C/C++ code the declaration of the function \ucode{fib} appears between the 
\kcode{begin declare target} and \kcode{end declare target} directives. 
In the corresponding Fortran code, the \kcode{declare target} directive appears at the
end of the specification part of the subroutine.

\cexample[5.1]{declare_target}{1}

The Fortran \ucode{fib} subroutine contains a \kcode{declare target} declaration 
to indicate to the compiler to create an device executable version of the procedure. 
The subroutine name has not been included on the \kcode{declare target} 
directive and is, therefore, implicitly assumed.

The program uses the \ucode{module_fib} module, which presents an explicit interface to 
the compiler with the \kcode{declare target} declarations for processing 
the \ucode{fib} call.

\ffreeexample[4.0]{declare_target}{1}

\pagebreak
The next Fortran example shows the use of an external subroutine. As the subroutine
is neither use associated nor an internal procedure, the \kcode{declare target}
declarations within a external subroutine are unknown to the main program unit; 
therefore, a \kcode{declare target} must be provided within the program
scope for the compiler to determine that a target binary should be available.

\ffreeexample[4.0]{declare_target}{2}

\subsection{Declare Target Directive for Indirect Procedure Call}
\label{subsec:indirect}

\index{clauses!indirect@\kcode{indirect}}
\index{indirect clause@\kcode{indirect} clause}

In the OpenMP 5.1 Specification the \kcode{indirect} clause was added to allow
indirect procedure calls, via function pointers, in a \kcode{target} region.
The functions to be allowed indirect invocation are specified in an \kcode{enter} 
clause of a declare target directive, along with the \kcode{indirect} clause.
The clause has an optional enabling/disabling argument (default enabled). In the 
absence of the indirect clause the function pointer would be mapped as a scalar 
(firstprivate) that would point to the host versions of the functions. 
Indirect clause informs the compiler that the function can potentially be 
used via function pointers and to use device versions of the same within 
the target region.

Only with an enabled \kcode{indirect} clause and a function specification in an \kcode{enter} clause
of a declare target directive may a function be called with an indirect invocation in a \kcode{target} region.
(Note: this feature limits the number of functions that can be used by function 
pointers in the \kcode{target} region to a restricted list for the compiler.)
%% KFM should be "to a restricted...   -> to those listed in the \code{enter} clause.

In the following example, the \kcode{declare target} \kcode{enter(\ucode{fun1,fun2})} 
\kcode{indirect} directive specifies that the \ucode{fun1} and \ucode{fun2} functions may 
be invoked with a function pointer in the \kcode{target} region.
Either the \ucode{fun1} or \ucode{fun2} function is invoked by the \ucode{fptr} function 
pointer in the \kcode{target} construct, as determined by the value of \ucode{count}.

\cexample[5.2]{declare_target_indirect_call}{1}
\ffreeexample[5.2]{declare_target_indirect_call}{1}

\subsection{Declare Target Directive for Class Type}
\label{subsec:declare_target_class}

\index{directives!begin declare target@\kcode{begin declare target}}
\index{begin declare target directive@\kcode{begin declare target} directive}

The following example shows the use of the \kcode{begin declare target}
and \kcode{end declare target} pair to designate the beginning and
end of the affected declarations, as introduced in OpenMP 5.1.
The \kcode{begin declare target} directive was defined
to symmetrically complement the terminating (``end'') directive.

\cppspecificstart

The example also shows 3 different ways to use a \kcode{declare target} directive for a 
class and an external member-function definition (for the \ucode{XOR1}, \ucode{XOR2}, 
and \ucode{XOR3} classes and definitions for their corresponding \ucode{foo()} member functions).

For \ucode{XOR1}, a \kcode{begin declare target} and 
\kcode{end declare target} directive
enclose both the class and its member function definition. The compiler immediately
knows to create a device version of the function for execution in a \kcode{target} region.

For \ucode{XOR2}, the class member function definition is not specified with a
\kcode{declare target} directive.
An implicit declare target is created for the member function definition.
The same applies if this declaration arrangement for the class and function 
are included through a header file.

For \ucode{XOR3}, the class and its member function are not enclosed by \kcode{begin declare target}
and \kcode{end declare target} directives,
but there is an implicit declare target since the class, its function
and the \kcode{target} construct are in the same file scope. That is, the class
and its function are treated as if delimited by a \kcode{declare target} directive.
The same applies if the class and function are included through a header file.

\cppnexample[5.1]{declare_target}{2a}

\topmarker{C++}

Often class definitions and their function definitions are included in separate files,
as shown in \example{declare_target.2b_classes.hpp} and \example{declare_target.2b_functions.cpp} example code files below.
In this case, it is necessary to specify a declare target directive for the classes.
However, as long as the \example{2b_functions.cpp} file includes the corresponding declare target classes,
there is no need to specify the functions with a declare target directive.
The functions are treated as if they are specified with a declare target directive.
Compiling the \example{declare_target.2b_functions.cpp} and \example{declare_target.2b_main.cpp} files 
separately and linking them, will create appropriate executable device functions for the target device.

\srcnexample[5.1]{declare_target}{2b_classes}{hpp}
\smallskip
\cppnexample[5.1]{declare_target}{2b_functions}[1]
\smallskip
\cppnexample[5.1]{declare_target}{2b_main}[1]

%\cppspecificend
\topmarker{C++}

%\cppspecificstart
The following example shows how the \kcode{begin declare target} and \kcode{end declare target} directives are used to enclose the declaration 
of a variable \ucode{varY} with a class type \ucode{typeY}. 
%Prior to OpenMP 5.0, the member function \code{typeY::foo()} cannot 
%be accessed on a target device because its declaration did not appear between \code{begin declare} 
%\code{target} and \code{end declare target} directives.

This example shows pre-OpenMP 5.0 behavior for the \ucode{varY.foo()} function call (an error).
The member function \ucode{typeY::foo()} cannot be accessed on a target device because its 
declaration does not appear between \kcode{begin declare target} and 
\kcode{end declare target} directives. As of OpenMP 5.0, the
function is implicitly declared with a declare target directive 
and will successfully execute the function on the device.  See previous examples.
%as if it were included in list or block of a declare target directive,

\cppnexample[5.1]{declare_target}{2c}
\cppspecificend

\subsection{Declare Target Directive for Variables}
\label{subsec:declare_target_variables}

\index{directives!declare target@\kcode{declare target}}
\index{declare target directive@\kcode{declare target} directive}

\index{directives!begin declare target@\kcode{begin declare target}}
\index{begin declare target directive@\kcode{begin declare target} directive}

The following examples show how the declare target directive is used to indicate that
global variables are mapped to the implicit device data environment of each target device.

In the following example, the declarations of the variables \ucode{p}, \ucode{v1}, and \ucode{v2} appear 
between \kcode{begin declare target} and \kcode{end declare target} 
directives indicating that the variables are mapped to the implicit device data 
environment of each target device. The \kcode{target update} directive 
is then used to manage the consistency of the variables \ucode{p}, \ucode{v1}, and \ucode{v2} between the 
data environment of the encountering host device task and the implicit device data 
environment of the default target device.

\cexample[5.1]{declare_target}{3}

The Fortran version of the above C code uses a different syntax. Fortran modules 
use a list syntax on the \kcode{declare target} directive to declare 
mapped variables.

\ffreeexample[4.0]{declare_target}{3}

\pagebreak
The following example also indicates that the function \ucode{Pfun()} is available on the 
target device, as well as the variable \ucode{Q}, which is mapped to the implicit device 
data environment of each target device. The \kcode{target update} directive 
is then used to manage the consistency of the variable \ucode{Q} between the data environment 
of the encountering host device task and the implicit device data environment of 
the default target device.

In the following example, the function and variable declarations appear between 
the \kcode{begin declare target} and \kcode{end declare target} 
directives.

\cexample[5.1]{declare_target}{4}

The Fortran version of the above C code uses a different syntax. In Fortran modules 
a list syntax on the \kcode{declare target} directive is used to declare 
mapped variables and procedures. The \ucode{N} and \ucode{Q} variables are declared as a comma 
separated list. When the \kcode{declare target} directive is used to 
declare just the procedure, the procedure name need not be listed -- it is implicitly 
assumed, as illustrated in the \ucode{Pfun()} function.

\ffreeexample[4.0]{declare_target}{4}

\subsection{Declare Target Directive with \kcode{declare simd}}
\label{subsec:declare_target_simd}

\index{directives!declare target@\kcode{declare target}}
\index{declare target directive@\kcode{declare target} directive}

\index{directives!begin declare target@\kcode{begin declare target}}
\index{begin declare target directive@\kcode{begin declare target} directive}

\index{directives!declare simd@\kcode{declare simd}}
\index{declare simd directive@\kcode{declare simd} directive}

The following example shows how the \kcode{begin declare target} and 
\kcode{end declare target} directives are used to indicate that a function 
is available on a target device. The \kcode{declare simd} directive indicates 
that there is a SIMD version of the function \ucode{P()} that is available on the target 
device as well as one that is available on the host device.

\cexample[5.1]{declare_target}{5}

The Fortran version of the above C code uses a different syntax. Fortran modules 
use a list syntax of the \kcode{declare target} declaration for the mapping. 
%%KFM use a list syntax in the \kcode{declare target} directive for the mapping. 
Here the \ucode{N} and \ucode{Q} variables are declared in the list form as a comma separated list. 
The function declaration does not use a list and implicitly assumes the function 
name. In this Fortran example row and column indices are reversed relative to the 
C/C++ example, as is usual for codes optimized for memory access.

\ffreeexample[4.0]{declare_target}{5}


\subsection{Declare Target Directive with \kcode{link} Clause}
\label{subsec:declare_target_link}

\index{directives!declare target@\kcode{declare target}}
\index{declare target directive@\kcode{declare target} directive}

\index{directives!begin declare target@\kcode{begin declare target}}
\index{begin declare target directive@\kcode{begin declare target} directive}

\index{clauses!link@\kcode{link}}
\index{link clause@\kcode{link} clause}

In the OpenMP 4.5 standard the \kcode{declare target} directive was extended to allow static
data to be mapped, \emph{when needed}, through a \kcode{link} clause.

Data storage for items listed in the \kcode{link} clause becomes available on the device
when it is mapped implicitly or explicitly in a \kcode{map} clause, and it persists for the scope of
the mapping (as specified by a \kcode{target} construct, 
a \kcode{target data} construct, or 
\kcode{target enter/exit data} constructs).

Tip: When all the global data items will not fit on a device and are not needed
simultaneously, use the \kcode{link} clause and map the data only when it is needed.
%%KFM simultaneously, use the \kcode{link} clause and map sections of the data only when it is needed.

The following C and Fortran examples show two sets of data (single precision and double precision)
that are global on the host for the entire execution on the host; but are only used
globally on the device for part of the program execution. The single precision data
are allocated and persist only for the first \kcode{target} region. Similarly, the
double precision data are in scope on the device only for the second \kcode{target} region.

\cexample[5.1]{declare_target}{6}
\ffreeexample[4.5]{declare_target}{6}


\subsection{Declare Target Directive with \kcode{device_type} Clause}
\label{subsec:declare_target_device_type}

\index{directives!declare target@\kcode{declare target}}
\index{declare target directive@\kcode{declare target} directive}

\index{directives!begin declare target@\kcode{begin declare target}}
\index{begin declare target directive@\kcode{begin declare target} directive}

\index{clauses!device_type@\kcode{device_type}}
\index{device_type clause@\kcode{device_type} clause}

The \kcode{declare target} directives apply to procedures to ensure that they can be executed or accessed on a device. 
The \kcode{device_type} clause specifies whether a version of the procedure or variable should be made available on the host, device or both. 
This example uses \kcode{nohost} for a procedure \ucode{foo()}. Only a device version of the procedure \ucode{foo()} is made available. 
If the variant function \ucode{foo_onhost()} is not specified for the host fallback execution, the call to \ucode{foo()} from the \kcode{target} region will result in a link time error due to the code generated for host execution of the target region.
This is because host symbol for the device routine \ucode{foo()} marked as \kcode{nohost} is not required to be present in the host environment.

\cexample[5.2]{declare_target}{7}
\ffreeexample[5.2]{declare_target}{7}


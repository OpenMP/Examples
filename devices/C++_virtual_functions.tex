\pagebreak
\section{C++ Virtual Functions}
\label{sec:virtual_functions}

\index{mapping!virtual functions, C++}

The 5.2 OpenMP Specification clarified restrictions on the use 
of polymorphic classes and virtual functions when used within 
\scode{target} regions.  The following example identifies 
problem cases in which the restrictions are not followed
(for Unified Shared Memory, as prescribed by the \scode{requires}
directive).

The first section illustrates the restriction
that when mapping an object for the first time, 
the static and dynamic types must match.

For the first target region the behavior of the implicit map of \splc{ar} 
is not specified-- its static type (A) doesn't match its dynamic type (D).  
Hence access to the virtual functions is undefined.
However, the second target region can access \splc{D::vf()} 
since the object to which \splc{ap} points is not mapped and 
therefore the restriction does not apply.

The second section illustrates the restriction:

\emph{"Invoking a virtual member function of an object on a device other than the device on which the
object was constructed results in unspecified behavior, unless the object is accessible and was
constructed on the host device."}

An instantiation of a polymorphic class (\splc{A}) occurs in the 
\scode{target} region, and access of its virtual function
is incorrectly attempted on the host (another device).
However, once the object is deleted on
the target device and instantiated on the host, access within
the next \scode{target} region is permitted.

\cppexample[5.2]{virtual_functions}{1}

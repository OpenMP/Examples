\section{Array Shaping}
\label{sec:array-shaping}
\index{array shaping!in motion-clause@in \plc{motion-clause}}
\index{constructs!target update@\kcode{target update}}
\index{target update construct@\kcode{target update} construct!to clause@\kcode{to} clause}
\index{target update construct@\kcode{target update} construct!from clause@\kcode{from} clause}

\index{directives!declare target@\kcode{declare target}}
\index{declare target directive@\kcode{declare target} directive}

\index{directives!begin declare target@\kcode{begin declare target}}
\index{begin declare target directive@\kcode{begin declare target} directive}

\begin{ccppspecific}
A pointer variable can be shaped to a multi-dimensional array to facilitate
data access. This is achieved by a \plc{shape-operator} casted in front of 
a pointer (lvalue expression):
\begin{description}
\item[]\hspace*{5mm}\code{([$s_1$][$s_2$]...[$s_n$])}\plc{pointer}
\end{description}
where each $s_i$ is an integral-type expression of positive value.
The shape-operator can appear in either the \plc{motion-clause}
of the \kcode{target update} directive or the \kcode{depend} clause.

The following example shows the use of the shape-operator in the 
\kcode{target update} directive. The shape-operator \ucode{([nx][ny+2])}
casts pointer variable \ucode{a} to a 2-dimensional array of size
%\ucode{nx}$\times$\ucode{(ny+2)}.  The resulting array is then accessed as
\ucode{nx}*\ucode{(ny+2)}.  The resulting array is then accessed as
array sections (such as \ucode{[0:nx][1]} and \ucode{[0:nx][ny]}) 
in the \kcode{from} or \kcode{to} clause for transferring two columns of 
noncontiguous boundary data from or to the device.  
Note the use of additional parentheses
around the shape-operator and \ucode{a} to ensure the correct precedence 
over array-section operations.

\cnexample[5.1]{array_shaping}{1}
\end{ccppspecific}
%\clearpage

\begin{fortranspecific}
The shape operator is not defined for Fortran.  Explicit array shaping
of procedure arguments can be used instead to achieve a similar goal.
Below is the Fortran equivalent of the above example that illustrates
the support of transferring two rows of noncontiguous boundary
data in the \kcode{target update} directive.
 
\ffreenexample[5.2]{array_shaping}{1}
\end{fortranspecific}

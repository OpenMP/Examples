%\pagebreak
\section{\kcode{target update} Construct}
\label{sec:target_update}

\subsection{Simple \kcode{target data} and \kcode{target update} Constructs}
\label{subsec:target_data_and_update}
\index{constructs!target data@\kcode{target data}}
\index{target data construct@\kcode{target data} construct}
\index{constructs!target update@\kcode{target update}}
\index{target update construct@\kcode{target update} construct}
\index{target update construct@\kcode{target update} construct!to clause@\kcode{to} clause}
\index{target update construct@\kcode{target update} construct!from clause@\kcode{from} clause}
\index{target update construct@\kcode{target update} construct!motion-clause@\plc{motion-clause}}
\index{clauses!motion-clause@\plc{motion-clause}}
\index{clauses!to@\kcode{to}}
\index{clauses!from@\kcode{from}}
\index{motion-clause@\plc{motion-clause}!to clause@\kcode{to} clause}
\index{motion-clause@\plc{motion-clause}!from clause@\kcode{from} clause}

The following example shows how the \kcode{target update} construct updates 
variables in a device data environment.

The \kcode{target data} construct maps array sections \ucode{v1[:N]} and \ucode{v2[:N]} 
(arrays \ucode{v1} and \ucode{v2} in the Fortran code) into a device data environment.

The task executing on the host device encounters the first \kcode{target} region 
and waits for the completion of the region.

After the execution of the first \kcode{target} region, the task executing on 
the host device then assigns new values to \ucode{v1[:N]} and \ucode{v2[:N]} (\ucode{v1} and \ucode{v2} arrays 
in Fortran code) in the task's data environment by calling the function \ucode{init_again()}.

The \kcode{target update} construct assigns the new values of \ucode{v1} and 
\ucode{v2} from the task's data environment to the corresponding mapped array sections 
in the device data environment of the \kcode{target data} construct.

The task executing on the host device then encounters the second \kcode{target} 
region and waits for the completion of the region.

The second \kcode{target} region uses the updated values of \ucode{v1[:N]} and \ucode{v2[:N]}.

\cexample[4.0]{target_update}{1}

\ffreeexample[4.0]{target_update}{1}

\subsection{\kcode{target update} Construct with \kcode{if} Clause}
\label{subsec:target_update_if}
\index{target update construct@\kcode{target update} construct!if clause@\kcode{if} clause}
\index{clauses!if@\kcode{if}}
\index{if clause@\kcode{if} clause}

The following example shows how the \kcode{target update} construct updates 
variables in a device data environment.

The \kcode{target data} construct maps array sections \ucode{v1[:N]} and \ucode{v2[:N]} 
(arrays \ucode{v1} and \ucode{v2} in the Fortran code) into a device data environment. In between 
the two \kcode{target} regions, the task executing on the host device conditionally 
assigns new values to \ucode{v1} and \ucode{v2} in the task's data environment. The function \ucode{maybe_init_again()} 
returns \vcode{true} if new data is written.

When the conditional expression (the return value of \ucode{maybe_init_again()}) in the 
\kcode{if} clause is \plc{true}, the \kcode{target update} construct 
assigns the new values of \ucode{v1} and \ucode{v2} from the task's data environment to the corresponding 
mapped array sections in the \kcode{target data} construct's device data 
environment.

\cexample[4.0]{target_update}{2}

\ffreeexample[4.0]{target_update}{2}

\subsection{\kcode{target update} Construct with Mapper}
\label{subsec:target_update_mapper}
\index{target update construct@\kcode{target update} construct!modifier@mapper}

The following example shows how the \kcode{target update} construct can be used with a \kcode{mapper} (\ucode{custom}). 
The \ucode{custom} mapper maps members of structure \ucode{T} with different map-type modifiers. Inside the 
\kcode{target data} region the \kcode{target update} with the \kcode{to} \plc{motion-clause} is equivalent to an update of \ucode{x} on the device. After the \kcode{target} region the \kcode{target update} with the \kcode{from} motion-clause is equivalent to an update of \ucode{y} on the host.

\cexample[5.1]{target_update}{3}

\ffreeexample[5.1]{target_update}{3}


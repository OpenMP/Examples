\pagebreak
\section{Structure Mapping}
\label{sec:structure_mapping}
\index{mapping!structure}

\index{directives!begin declare target@\kcode{begin declare target}}
\index{begin declare target directive@\kcode{begin declare target} directive}


%%%%%%%%%%%%%%%%%%%%%%%%%%%%%%%%%%%%%%%%%%%%%%%%%%%%%%%%%%%%%%%%%%%%%%%%%%%%%%%%%
In the example below, only structure elements \ucode{S.a}, \ucode{S.b} and \ucode{S.p} 
of the \ucode{S} structure appear in \kcode{map} clauses of a \kcode{target} construct.
Only these components have corresponding variables and storage on the device.  
Hence, the large arrays, \ucode{S.buffera} and \ucode{S.bufferb}, and the \ucode{S.x} component have no storage 
on the device and cannot be accessed.  

Also, since the pointer member \ucode{S.p} is used in an array section of a 
\kcode{map} clause, the array storage of the array section on the device, 
\ucode{S.p[:N]}, is \plc{attached} to the pointer member \ucode{S.p} on the device.
Explicitly mapping the pointer member \ucode{S.p} is optional in this case.

Note: The buffer arrays and the \ucode{x} variable have been grouped together, so that
the components that will reside on the device are all together (without gaps).
This allows the runtime to optimize the transfer and the storage footprint on the device.

\cexample[5.1]{target_struct_map}{1}


%%%%%%%%%%%%%%%%%%%%%%%%%%%%%%%%%%%%%%%%%%%%%%%%%%%%%%%%%%%%%%%%%%%%%%%%%%%%%%%%%
The following example is a slight modification of the above example for 
a C++ class.  In the member function \ucode{SAXPY::driver} 
the array section \ucode{p[:N]} is attached to the pointer member \ucode{p}
on the device.
 
\cppexample[5.1]{target_struct_map}{2}

%%%%%%%%%%%%%%%%%%%%%%%%%%%%%%%%%%%%%%%%%%%%%%%%%%%%%%%%%%%%%%%%%%%%%%%%%%%%%%%%%

%In this example a pointer, \plc{p}, is mapped in a 
%\code{target data} construct (\code{map(p)}) and remains 
%persistent throughout the \code{target data} region. The address stored
%on the host is not assigned to the device pointer variable, and 
%the device value is not copied back to the host at the end of the
%region (for a pointer, it is as though \code{map(alloc:p}) is effectively
%used).  The array section, \plc{p[:N]}, is mapped on both \code{target}
%constructs, and the pointer \plc{p} on the device is attached at the
%beginning and detached at the end of the regions to the newly created
%array section on the device.
%
%Also, in the following example the global variable, \plc{a}, becomes 
%allocated when it is first used on the device in a \code{target} region, 
%and persists on the device for all target regions.  The value on the
%device and host may be different, as shown by the print statements.
%The values may be made consistent with the \code{update} construct,
%as shown in the \plc{declare\_target.3.c} and \plc{declare\_target.3.f90} 
%examples.
%
%\cexample{target_struct_map}{2}

%%%%%%%%%%%%%%%%%%%%%%%%%%%%%%%%%%%%%%%%%%%%%%%%%%%%%%%%%%%%%%%%%%%%%%%%%%%%%%%%%

The next example shows two ways in which the structure may be
\emph{incorrectly} mapped.

In Case 1, the array section \ucode{S1.p[:N]} is first mapped in an enclosing
\kcode{target data} construct, and the \kcode{target} construct then
implicitly maps the structure \ucode{S1}. The initial map of the array section
does not map the base pointer \ucode{S1.p} -- it only maps the elements of the
array section.  Furthermore, the implicit map is not sufficient to ensure
pointer attachment for the structure member \ucode{S1.p} (refer to the conditions
for pointer attachment described in Section~\ref{sec:pointer_mapping}).
Consequentially, the dereference operation \ucode{S1.p[i]} in the call to
\ucode{saxpyfun} will probably fail because \ucode{S1.p} contains a host address.

In Case 2, again an array section is mapped on an enclosing
\kcode{target data} construct. This time, the nested \kcode{target}
construct explicitly maps \ucode{S2.p}, \ucode{S2.a}, and \ucode{S2.b}. But as in
Case 1, this does not satisfy the conditions for pointer attachment since the
construct must map a list item for which \ucode{S2.p} is a base pointer, and it
must do so when the \ucode{S2.p} is already present on the device or will be
created on the device as a result of the same construct.

 
\cexample[5.1]{target_struct_map}{3}

%%%%%%%%%%%%%%%%%%%%%%%%%%%%%%%%%%%%%%%%%%%%%%%%%%%%%%%%%%%%%%%%%%%%%%%%%%%%%%%%%

The following example correctly implements pointer attachment cases that
involve implicit structure maps.

In Case 1, members \ucode{p}, \ucode{a}, and \ucode{b} of the structure \ucode{S1}
are explicitly mapped by the \kcode{target data} construct, to avoid
mapping parts of \ucode{S1} that aren't required on the device. The mapped
\ucode{S1.p} is attached to the array section \ucode{S1.p[:N]}, and remains
attached while it exists on the device (for the duration of
\kcode{target data} region).  Due to the \ucode{S1} reference inside the
nested \kcode{target} construct, the construct implicitly maps \ucode{S1} so that
the reference refers to the corresponding storage created by the enclosing
\kcode{target data} region. Note that only the members \ucode{a},
\ucode{b}, and \ucode{p} may be accessed from this storage.

In Case 2, only the storage for the array section \ucode{S2.p[:N]} is mapped
by the \kcode{target data} construct.  The nested \kcode{target}
construct explicitly maps \ucode{S2.a} and \ucode{S2.b} and explicitly
maps an array section for which \ucode{S2.p} is a base pointer. This satisfies
the conditions for \ucode{S2.p} becoming an attached pointer. The array
section in this case is zero-length, but the effect would be the same if the
length was a positive integer less than or equal to \ucode{N}. There is also an
implicit map of the containing structure \ucode{S2}, again due to the reference
to \ucode{S2} inside the construct. The effect of this implicit map permits
access only to members \ucode{a}, \ucode{b}, and \ucode{p}, as for Case 1. 

In Case 3, there is no \kcode{target data} construct. The \kcode{target}
construct explicitly maps \ucode{S3.a} and \ucode{S3.b} and explicitly
maps an array section for which \ucode{S3.p} is a base pointer. Again, there is
an implicit map of the structure referenced in the construct, \ucode{S3}. This
implicit map also causes \ucode{S3.p} to be implicitly mapped, because no other
part of \ucode{S3} is present prior to the construct being encountered. The
result is an attached pointer \ucode{S3.p} on the device. As for Cases 1 and 2,
this implicit map only ensures that storage for the members \ucode{a}, \ucode{b},
and \ucode{p} are accessible within the corresponding \ucode{S3} that is created
on the device.

\cexample[5.1]{target_struct_map}{4}


\pagebreak
\section{Unified Shared Memory}
\label{sec:usm}

\index{requires directive@\kcode{requires} directive}
\index{directives!requires@\kcode{requires}}

\index{unified_shared_memory clause@\kcode{unified_shared_memory} clause}
\index{clauses!unified_shared_memory@\kcode{unified_shared_memory}}

The following examples show behavior of scalars, pointers, references (C++) and 
associate names (Fortran) in \kcode{target} constructs when
unified shared memory (USM) is required throughout the scope of the program by the 
\kcode{unified_shared_memory} clause in a \kcode{requires} directive.
USM assumes a unified address space.

In the C++ code of the first example, a scalar (\ucode{x}), a pointer 
(\ucode{ptr}), and a reference (\ucode{ref}) are used in a \kcode{target} construct in Cases 1, 2 and 3, respectively.
For the scalar variable \ucode{x}, the predetermined data-sharing attribute is still
firstprivate under the USM requirement and, hence, any manipulation of the local variable on the device is never reflected on the host.
With the USM requirement, pointers always refer to the same location in memory on the host and devices.
Hence, the value of \ucode{x} (in the host data environment) can be modified with \ucode{ptr} in 
the \kcode{target} construct, as seen in Case 2. 
For the reference \ucode{ref}, the object to which it refers is mapped for  
the \kcode{target} construct, as seen in Case 3. 

In Case 1 of the Fortran example, the scalar \ucode{x} is firstprivate under the USM requirement
in the \kcode{target} construct, and modification of the local variable on the device is 
never updated to the host data environment.
Also, in Case 2, the use of \ucode{ax}, which is associated with \ucode{x}, 
will update the \ucode{x} value in the host data environment.
In Case 3, the Fortran pointer \ucode{ptr} and its target \ucode{y} are not firstprivate, 
but implicitly mapped.  Hence, updates to the value of \ucode{y} appear in the host data environment.

%Hence, updates to \ucode{y} in the \kcode{target} construct appear in the data environment of the host.

%\pagebreak
\cppexample[5.2]{usm_scalar_ptr_ref_asc}{1}
\ffreeexample[5.2]{usm_scalar_ptr_ref_asc}{1}

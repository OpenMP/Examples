\cchapter{Memory Model}{memory_model}
\label{chap:memory_model}

OpenMP provides a shared-memory model that allows all threads on a given
device shared access to \emph{memory}. For a given OpenMP region that may be
executed by more than one thread or SIMD lane, variables in memory may be
\emph{shared} or \emph{private} with respect to those threads or SIMD lanes. A
variable's data-sharing attribute indicates whether it is shared (the
\emph{shared} attribute) or private (the \emph{private}, \emph{firstprivate},
\emph{lastprivate}, \emph{linear}, and \emph{reduction} attributes) in the data
environment of an OpenMP region. While private variables in an OpenMP region
are new copies of the original variable (with same name) that may then be
concurrently accessed or modified by their respective threads or SIMD lanes, a
shared variable in an OpenMP region is the same as the variable of the same
name in the enclosing region. Concurrent accesses or modifications to a
shared variable may therefore require synchronization to avoid data races.

OpenMP's memory model also includes a \emph{temporary view} of memory that is
associated with each thread. Two different threads may see different values for
a given variable in their respective temporary views. Threads may employ flush
operations for the purposes of making their temporary view of a variable
consistent with the value of the variable in memory. The effect of a given
flush operation is characterized by its flush properties -- some combination of
\emph{strong}, \emph{release}, and \emph{acquire} -- and, for \emph{strong}
flushes, a \emph{flush-set}. 

A \emph{strong} flush will force consistency between the temporary view and the
memory for all variables in its \emph{flush-set}.  Furthermore, all strong flushes in a
program that have intersecting flush-sets will execute in some total order, and
within a thread strong flushes may not be reordered with respect to other
memory operations on variables in its flush-set. \emph{Release} and
\emph{acquire} flushes operate in pairs. A release flush may ``synchronize''
with an acquire flush, and when it does so the local memory operations that
precede the release flush will appear to have been completed before the local
memory operations on the same variables that follow the acquire flush.

Flush operations arise from explicit \code{flush} directives, implicit
\code{flush} directives, and also from the execution of \code{atomic}
constructs. The \code{flush} directive forces a  consistent view of local
variables of the thread executing the \code{flush}.  When a list is supplied on
the directive, only the items (variables) in the list are guaranteed to be
flushed.  Implied flushes exist at prescribed locations of certain constructs.
For the complete list of these locations and associated constructs, please
refer to the \plc{flush Construct} section of the OpenMP Specifications
document.

In this chapter, examples illustrate how race conditions may arise for accesses
to variables with a \plc{shared} data-sharing attribute when flush operations
are not properly employed.  A race condition can exist when two or more threads
are involved in accessing a variable and at least one of the accesses modifies
the variable.  In particular, a data race will arise when conflicting accesses
do not have a well-defined \emph{completion order}.  The existence of data
races in OpenMP programs result in undefined behavior, and so they should
generally be avoided for programs to be correct.  The completion order of
accesses to a shared variable is guaranteed in OpenMP through a set of memory
consistency rules that are described in the \plc{OpenMP Memory Consistency}
section of the OpenMP Specifications document.

%This chapter also includes examples that exhibit non-sequentially consistent
%(\emph{non-SC}) behavior. Sequential consistency (\emph{SC}) is the desirable
%property that the results of a multi-threaded program are as if all operations
%are performed in some total order, consistent with the program order of
%operations performed by each thread. OpenMP guarantees that a correct program
%(i.e. a program that does not have a data race) will exhibit SC behavior
%so long as the only \code{atomic} constructs it uses are SC atomic directives.



% The following table lists construct in which implied flushes exist, and the
% location of their execution.
% 
% %\begin{table}[hb]
% \begin{center}
% %\caption {Execution Location for Implicit Flushes. } 
% \begin{tabular}{ | p{0.6\linewidth} | l | } 
% \hline
% \code{CONSTRUCT}                                   & \makecell{\code{EXECUTION} \\ \code{LOCATION}} \\
% \hline
% \code{parallel}                                    & upon entry and exit \\
% \hline
% \makecell[l]{worksharing \\ \hspace{1.5em}\code{for}, \code{do} 
%                          \\ \hspace{1.5em}\code{sections} 
%                          \\ \hspace{1.5em}\code{single} 
%                          \\ \hspace{1.5em}\code{workshare} }  
%                                                    & upon exit \\ 
% \hline
% \code{critical}                                    & upon entry and exit \\
% \hline
% \code{target}                                      & upon entry and exit \\
% \hline
% \code{barrier}                                     & during \\
% \hline
% \code{atomic} operation with \plc{seq\_cst} clause & upon entry and exit \\
% \hline
% \code{ordered}*                                    & upon entry and exit \\
% \hline
% \code{cancel}** and \code{cancellation point}**    & during \\
% \hline
% \code{target data}                                 & upon entry and exit \\
% \hline
% \code{target update} + \code{to} clause,   
% \code{target enter data}                           & on entry \\
% \hline
% \code{target update} + \code{from} clause, 
% \code{target exit data}                            & on exit \\
% \hline
% \code{omp\_set\_lock}                              & during \\
% \hline
% \makecell[l]{ \code{omp\_set/unset\_lock}, \code{omp\_test\_lock}*** 
%            \\ \code{omp\_set/unset/test\_nest\_lock}*** }
%                                                    & during \\
% \hline
% task scheduling point                              & \makecell[l]{immediately \\ before and after} \\
% \hline
% \end{tabular}
% %\caption {Execution Location for Implicit Flushes. } 
% 
% \end{center}
% %\end{table}
% 
% * without clauses and with \code{threads} or \code{depend} clauses \newline
% ** when \plc{cancel-var} ICV is \plc{true} (cancellation is turned on) and cancellation is activated \newline
% *** if the region causes the lock to be set or unset
% 
% A flush with a list is implied for non-sequentially consistent \code{atomic} operations
% (\code{atomic} directive without a \code{seq\_cst} clause), where the list item is the
% specific storage location accessed atomically (specified as the \plc{x} variable
% in \plc{atomic Construct} subsection of the OpenMP Specifications document).

% Examples 1-3 show the difficulty of synchronizing threads through \code{flush} and \code{atomic} directives.


%===== Examples Sections =====

%\pagebreak
\section{OpenMP Memory Model}
\label{sec:mem_model}

The following examples illustrate two major concerns for concurrent thread
execution: ordering of thread execution and memory accesses that may or may not
lead to race conditions.

In the following example, at Print 1, the value of \ucode{xval} could be either 2
or 5, depending on the timing of the threads. The \kcode{atomic} directives are
necessary for the accesses to \ucode{x} by threads 1 and 2 to avoid a data race.
If the atomic write completes before the atomic read, thread 1 is guaranteed to
see 5 in \ucode{xval}. Otherwise, thread 1 is guaranteed to see 2 in \ucode{xval}.

\index{flushes!implicit}
\index{atomic construct@\kcode{atomic} construct}
\index{constructs!atomic@\kcode{atomic}}
The barrier after Print 1 contains implicit flushes on all threads, as well as
a thread synchronization, so the programmer is guaranteed that the value 5 will
be printed by both Print 2 and Print 3. Since neither Print 2 or Print 3 are modifying
\ucode{x}, they may concurrently access \ucode{x} without requiring \kcode{atomic}
directives to avoid a data race.

\cexample[3.1]{mem_model}{1}

\ffreeexample[3.1]{mem_model}{1}

\pagebreak
\index{flushes!flush construct@\kcode{flush} construct}
\index{flush construct@\kcode{flush} construct}
\index{constructs!flush@\kcode{flush}}
The following example demonstrates why synchronization is difficult to perform
correctly through variables. The write to \ucode{flag} on thread 0 and the read
from \ucode{flag} in the loop on thread 1 must be atomic to avoid a data race.
When thread 1 breaks out of the loop, \ucode{flag} will have the value of 1.
However, \ucode{data} will still be undefined at the first print statement. Only
after the flush of both \ucode{flag} and \ucode{data} after the first print
statement will \ucode{data} have the well-defined value of 42.

\cexample[3.1]{mem_model}{2}

\fexample[3.1]{mem_model}{2}

\pagebreak
\index{flushes!flush with a list}
The next example demonstrates why synchronization is difficult to perform
correctly through variables. As in the preceding example, the updates to
\ucode{flag} and the reading of \ucode{flag} in the loops on threads 1 and 2 are
performed atomically to avoid data races on \ucode{flag}. However, the code still
contains data race due to the incorrect use of ``flush with a list'' after the
assignment to \ucode{data1} on thread 1. By not including \ucode{flag} in the
flush-set of that \kcode{flush} directive, the assignment can be reordered with
respect to the subsequent atomic update to \ucode{flag}. Consequentially,
\ucode{data1} is undefined at the print statement on thread 2.

\cexample[3.1]{mem_model}{3}

\fexample[3.1]{mem_model}{3}


The following two examples illustrate the ordering properties of 
the \plc{flush} operation. The \plc{flush} operations are strong flushes 
that are applied to the specified flush lists. 
However, use of a \kcode{flush} construct with a list is extremely error 
prone and users are strongly discouraged from attempting it. 
In the codes the programmer intends to prevent simultaneous 
execution of the protected section by the two threads.
The atomic directives in the codes ensure that the accesses to shared
variables \ucode{a} and \ucode{b} are atomic write and atomic read operations. Otherwise both examples would contain data races and automatically result 
in unspecified behavior. 

In the following incorrect code example, operations on variables \ucode{a} and
\ucode{b} are not ordered with respect to each other. For instance, nothing
prevents the compiler from moving the flush of \ucode{b} on thread 0 or the
flush of \ucode{a} on thread 1 to a position completely after the protected
section (assuming that the protected section on thread 0 does not reference
\ucode{b} and the protected section on thread 1 does not reference \ucode{a}).
If either re-ordering happens, both threads can simultaneously execute the
protected section.
Any shared data accessed in the protected section is not guaranteed to 
be current or consistent during or after the protected section. 

\cexample[3.1]{mem_model}{4a}
\ffreeexample[3.1]{mem_model}{4a}


The following code example correctly ensures that the protected section
is executed by only one thread at a time. Execution of the protected section
by neither thread is considered correct in this example. This occurs if both
flushes complete prior to either thread executing its \bcode{if} statement
for the protected section.
The compiler is prohibited from moving the flush at all for either thread,
ensuring that the respective assignment is complete and the data is flushed
before the \bcode{if} statement is executed.

\cexample[3.1]{mem_model}{4b}
\ffreeexample[3.1]{mem_model}{4b}


\pagebreak
\section{Memory Allocators}
\label{sec:allocators}

\index{memory allocators!allocator traits}
\index{memory allocators!memory space}
\index{memory allocators!omp_alloc routine@\scode{omp_alloc} routine}
\index{memory allocators!allocators directive@\scode{allocators} directive}

\index{omp_alloc routine@\scode{omp_alloc} routine}
\index{routines!omp_alloc@\scode{omp_alloc}}

\index{directives!allocators@\code{allocators}}
\index{allocators directive@\code{allocators} directive}
\index{allocators directive@\code{allocators} directive!allocator clause@\code{allocator} clause}

\index{clauses!allocator@\code{allocator}}
\index{allocator clause@\code{allocator} clause}
\index{omp_init_allocator routine@\scode{omp_init_allocator} routine}
\index{routines!omp_init_allocator@\scode{omp_init_allocator}}

OpenMP memory allocators can be used to allocate memory with
specific allocator traits.  In the following example an OpenMP allocator is used to
specify an alignment for arrays \plc{x} and \plc{y}. The
general approach for attributing traits to variables allocated by
OpenMP is to create or specify a pre-defined \plc{memory space}, create an
array of \plc{traits}, and then form an \plc{allocator} from the
memory space and trait.
The allocator is then specified
in an OpenMP allocation (using an API \plc{omp\_alloc()} function
for C/C++ code and an \code{allocators} directive for Fortran code
in the \splc{allocators.1} example).

In the example below the \plc{xy\_memspace} variable is declared
and assigned the default memory space (\plc{omp\_default\_mem\_space}).
Next, an array for \plc{traits} is created. Since only one
trait will be used, the array size is \plc{1}.
A trait is a structure in C/C++ and a derived type in Fortran,
containing 2 components: a key and a corresponding value (key-value pair).
The trait key used here is \plc{omp\_atk\_alignment} (an enum for C/C++
and a parameter for Fortran)
and the trait value of 64 is specified in the \plc{xy\_traits} declaration.
These declarations are followed by a call to the
\plc{omp\_init\_allocator()} function to combine the memory
space (\plc{xy\_memspace}) and the traits (\plc{xy\_traits})
to form an allocator (\plc{xy\_alloc}).

%In the C/C++ code the API  \plc{omp\_allocate()} function is used 
%to allocate space, similar to \plc{malloc}, except that the allocator 
%is specified as the second argument.  
%In Fortran an API allocation function is not available. 
%An \code{allocate} construct is used (with \plc{x} and \plc{y} 
%listed as the variables to be allocated), along
%with an \code{allocator} clause (specifying the \plc{xy\_alloc} as the allocator)
%for the following Fortran \plc{allocate} statement.

In the C/C++ code the API  \plc{omp\_allocate()} function is used
to allocate space, similar to \plc{malloc}, except that the allocator
is specified as the second argument.
In Fortran an \code{allocators} directive is used to specify an allocator
for the following Fortran \plc{allocate} statement.
A variable list in the \scode{allocate} clause may be supplied if the allocator
is to be applied to a subset of variables in the Fortran allocate
statement.
Here, the \plc{xy\_alloc} allocator is specified
in the modifier of the \code{allocator} clause,
and the set of all variables used in the \plc{allocate} statement is specified in the list.

%"for a following Fortran allocation statement" (no using "immediately" here)
% it looks like if you have a list, the allocation statement does not need
% to follow immediately.(?)
% spec5.0 157:19-20 The allocate directive must appear in the same scope as
% the declarations of each of its list items and must follow all such declarations.

%\pagebreak

\cexample[5.0]{allocators}{1}
\ffreeexample[5.2]{allocators}{1}


When using the \scode{allocators} construct with optional clauses in Fortran code, 
users should be aware of the behavior of a reallocation.

In the following example, the \splc{a} variable is allocated with 64-byte
alignment through the \scode{align} clause of the \scode{allocators} construct.
%The alignment of the newly allocated object, \splc{a}, in the (reallocation)
%assignment \splc{a = b} may not be the same as before.  
The alignment of the newly allocated object, \splc{a}, in the (reallocation)
assignment \splc{a = b} will not be reallocated with the 64-byte alignment, but
with the 32-byte alignment prescribed by the trait of the \splc{my_alloctr} 
allocator. It is best to avoid this problem by constructing and using an
allocator (not the \scode{align} clause) with the required alignment in 
the \scode{allocators} construct.
Note that in the subsequent
deallocation of \splc{a} the deallocation must precede the destruction
of the allocator used in the allocation of \splc{a}.

\ffreeexample[5.2]{allocators}{2}

When creating and using an \scode{allocators} construct within a Fortran procedure
for allocating storage (and subsequently freeing the allocator storage with an 
\scode{omp_destroy_allocator} construct), users should be aware of the necessity
of using an explicit Fortran deallocation instead of relying on auto-deallocation.

In the following example, a user-defined allocator is used in the allocation
of the \splc{c} variable, and then the allocator is destroyed.
Auto-deallocation at the end of the \splc{broken_auto_deallocation} procedure
will fail without the allocator, hence an explicit deallocation should be used 
(before the \scode{omp_destroy_allocator} construct).
Note that an allocator may be specified directly in the \scode{allocate} clause
without using the \scode{allocator} complex modifier, so long as no other modifier 
is specified in the clause.

\ffreeexample[5.2]{allocators}{3}

\index{directives!allocate@\code{allocate}}
\index{allocate directive@\code{allocate} directive}
\index{allocate directive@\code{allocate} directive!allocator clause@\code{allocator} clause}

The \scode{allocate} directive is a convenient way to apply an OpenMP 
allocator to the allocation of declared variables.

This example illustrates the allocation of specific types of storage in a program 
for use in libraries, privatized variables, and with offloading.

Two groups of variables, \{\plc{v1, v2}\} and \{\plc{v3, v4}\}, are used with the \scode{allocate} 
directive, and the \{\plc{v5, v6}\} pair is used with the \scode{allocate} clause. 
Here we explicitly use predefined allocators \scode{omp_high_bw_mem_alloc} and \scode{omp_default_mem_alloc}
with the \scode{allocate} directive in CASE 1. Similar effects are achieved for private variables of a task
by using the \scode{allocate} clause, as shown in CASE 2.  

Note, when the \scode{allocate} directive does not specify an \scode{allocator} clause, an
implementation-defined default, stored in the \splc{def-allocator-var} ICV, is used
(not illustrated here).
Users can set and get the default allocator with the \scode{omp_set_default_allocator}
and \scode{omp_get_default_allocator} API routines. 

\cexample[5.1]{allocators}{4}
\ffreeexample[5.1]{allocators}{4}

\pagebreak
\index{uses_allocators clause@\scode{uses_allocators} clause}
\index{clauses!uses_allocators@\scode{uses_allocators}}

The use of allocators in \scode{target} regions is facilitated by the
\scode{uses_allocators} clause as shown in the cases below.

In CASE 1, the predefined \scode{omp_cgroup_mem_alloc} allocator is made available on the
device in the first \scode{target} construct as specified in the \scode{uses_allocators} clause.
The allocator is then used in the \scode{allocate}
clause of the \scode{teams} construct to allocate a private array for each
team (contention group). The private \splc{xbuf} arrays that are filled by each
team are reduced as specified in the \scode{reduction} clause on the \scode{teams} construct.

In CASE 2, user-defined traits are specified in the \splc{cgroup_traits} variable.
An allocator is initialized for the \scode{target} region in the \scode{uses_allocators} clause,
and the traits specified in \splc{cgroup_traits} are included by the \scode{traits} modifier.

In CASE 3, the \splc{cgroup_alloc} variable is initialized on the host with traits
and a memory space. However, these are ignored by the \scode{uses_allocators} clause
and a new allocator for the \scode{target} region is initialized with default traits.

\cexample[5.2]{allocators}{5}
\ffreeexample[5.2]{allocators}{5}

\index{dynamic_allocators clause@\scode{dynamic_allocators} clause}
\index{clauses!dynamic_allocators@\scode{dynamic_allocators}}

The following example shows how to make an allocator available in a \scode{target} region 
without specifying a \scode{uses_allocators} clause.

In CASE 1, the predefined \scode{omp_cgroup_mem_alloc} allocator is used in the \scode{target}
region as in CASE 1 of the previous example, but without specifying a \scode{uses_allocators} clause.
This is accomplished by specifying the \scode{requires} directive with a
\scode{dynamic_allocators} clause in the same compilation unit, to remove
restrictions on allocator usage in \scode{target} regions.

CASE 2 also uses the \scode{dynamic_allocators} clause to remove allocator
restrictions in \scode{target} regions. Here, an allocator is initialized
by calling the \scode{omp_init_allocator} routine in the \code{target} region.
The allocator is then used for the allocations of array \plc{xbuf} in 
an \scode{allocate} clause of the \code{target}~\code{teams} construct 
for each team and destroyed after its use.
The use of separate \code{target} regions is needed here since
no statement is allowed between a \code{target} directive and 
its nested \code{teams} construct.

\cexample[5.2]{allocators}{6}
\ffreeexample[5.2]{allocators}{6}

\pagebreak
\section{Race Conditions Caused by Implied Copies of Shared Variables in Fortran}
\fortranspecificstart
\label{sec:fort_race}
\index{shared variables!race conditions}

The following example contains a race condition, because the shared variable, which 
is an array section, is passed as an actual argument to a routine that has an assumed-size 
array as its dummy argument. The subroutine call passing an array section argument 
may cause the compiler to copy the argument into a temporary location prior to 
the call and copy from the temporary location into the original variable when the 
subroutine returns. This copying would cause races in the \code{parallel} region.

\ffreenexample{fort_race}{1}
\fortranspecificend





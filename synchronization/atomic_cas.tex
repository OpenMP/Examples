%\pagebreak
\section{Atomic Compare}
\label{sec:cas}

\index{constructs!atomic@\kcode{atomic}}
\index{atomic construct@\kcode{atomic} construct}
\index{clauses!capture@\kcode{capture}}
\index{clauses!compare@\kcode{compare}}
\index{capture clause@\kcode{capture} clause}
\index{compare clause@\kcode{compare} clause}

In OpenMP 5.1 the \kcode{compare} clause was added to the \plc{extended-atomic} clauses.
The \kcode{compare} clause extends the semantics to perform the \kcode{atomic}
update conditionally. 

In the following C/C++ example, two formats of structured blocks
are shown for associated \kcode{atomic} constructs with the \kcode{compare} clause.
In the first \kcode{atomic} construct, the format forms a \plc{conditional update} statement.
In the second \kcode{atomic} construct the format forms a \plc{conditional expression} statement.
The ``greater than'' and ``less than'' forms are not available with the Fortran \kcode{compare}
clause.  One can use the \vcode{max} and \vcode{min} functions with the \kcode{atomic update}
construct to perform the C/C++ example operations.

\cexample[5.1]{cas}{1}
%\ffreeexample[5.1]{cas}{1}

In OpenMP 5.1 the \kcode{compare} clause was also added to support \emph{Compare And
Swap} (CAS) semantics. In the following example the \ucode{enqueue} routine
(a naive implementation of a Michael and Scott enqueue function), uses the
\kcode{compare} clause, with the \kcode{capture} clause, to perform and compare
(\ucode{q->head == node->next}) and swap (\plc{if-else} assignments) of the
form: 
\begin{description}[noitemsep,labelindent=5mm,widest=f90]
\item \splc{{ r = x == e; if(r) { x = d; } else { v = x; } }}.
\end{description}
The example program concurrently enqueues nodes from an array of nodes (\ucode{nodes[N]}).
Since the equivalence of Fortran pointers can be determined only with a function (such as associated),
no Fortran version is provided here. The use of the associated function in an \kcode{atomic compare} syntax is
being considered in a future release.

\cexample[5.1]{cas}{2}
%\ffreeexample[5.1]{cas}{2}

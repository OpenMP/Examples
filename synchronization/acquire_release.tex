%\pagebreak
\section{Synchronization Based on Acquire/Release Semantics}
\label{sec:acquire_and_release_semantics}

%OpenMP 5.0 introduced ``release/acquire'' memory ordering semantics to the
%specification. The memory ordering behavior of OpenMP constructs and routines
%that permit two threads to synchronize with each other are defined in terms of
%\emph{release flushes} and \emph{acquire flushes}, where a release flush
%must occur at the source of the synchronization and an acquire flush must occur
%at the sink of the synchronization. Flushes resulting from a \code{flush}
%directive without a list may function as a release flush, an acquire flush, or
%both a release and acquire flush. Flushes implied on entry to or exit from an
%atomic operation (specified by an \code{atomic} construct) may also function as
%a release flush or an acquire flush, depending on if a memory ordering clause
%appears on a construct. Flushes implied by other OpenMP constructs or routines
%also function as either a release flush or an acquire flush, according to the
%synchronization semantics of the construct.

%%%%%%%%%%%%%%%%%%

\index{flushes!acquire}
\index{flushes!release}
\index{clauses!memory ordering clauses}
\index{memory ordering clauses!acquire@\kcode{acquire}}
\index{acquire clause@\kcode{acquire} clause}
\index{memory ordering clauses!release@\kcode{release}}
\index{release clause@\kcode{release} clause}
\index{memory ordering clauses!acq_rel@\kcode{acq_rel}}
\index{acq_rel clause@\kcode{acq_rel} clause}
\index{flush construct@\kcode{flush} construct}
\index{atomic construct@\kcode{atomic} construct}
\index{clauses!acquire@\kcode{acquire}}
\index{clauses!release@\kcode{release}}
\index{clauses!acq_rel@\kcode{acq_rel}}
As explained in the Memory Model chapter of this document, a flush operation
may be an \plc{acquire flush} and/or a \plc{release flush}, and OpenMP 5.0
defines acquire/release semantics in terms of these fundamental flush
operations.  For any synchronization between two threads that is specified by
OpenMP, a release flush logically occurs at the source of the synchronization
and an acquire flush logically occurs at the sink of the synchronization.
OpenMP 5.0 added memory ordering clauses -- \kcode{acquire}, \kcode{release}, and
\kcode{acq_rel} -- to the \kcode{flush} and \kcode{atomic} constructs for
explicitly requesting acquire/release semantics.  Furthermore, implicit flushes
for all OpenMP constructs and runtime routines that synchronize OpenMP threads
in some manner were redefined in terms of synchronizing release and acquire
flushes to avoid the requirement of strong memory fences (see the \docref{Flush
Synchronization and Happens Before} and \docref{Implicit Flushes} sections of the
OpenMP Specifications document).

The examples that follow in this section illustrate how acquire and release
flushes may be employed, implicitly or explicitly, for synchronizing threads. A
\kcode{flush} directive without a list and without any memory ordering clause
can also function as both an acquire and release flush for facilitating thread
synchronization.  Flushes implied on entry to, or exit from, an atomic
operation (specified by an \kcode{atomic} construct) may function as an acquire
flush or a release flush if a memory ordering clause appears on the construct.
On entry to and exit from a \kcode{critical} construct there is now an implicit
acquire flush and release flush, respectively.

%%%%%%%%%%%%%%%%%%

\index{constructs!critical@\kcode{critical}}
\index{critical construct@\kcode{critical} construct}
\index{flushes!implicit}
The first example illustrates how the release and acquire flushes implied by a
\kcode{critical} region guarantee a value written by the first thread is visible
to a read of the value on the second thread. Thread 0 writes to \ucode{x} and
then executes a \kcode{critical} region in which it writes to \ucode{y}; the write
to \ucode{x} happens before the execution of the \kcode{critical} region,
consistent with the program order of the thread.  Meanwhile, thread 1 executes a
\kcode{critical} region in a loop until it reads a non-zero value from
\ucode{y} in the \kcode{critical} region, after which it prints the value of
\ucode{x}; again, the execution of the \kcode{critical} regions happen before the
read from \ucode{x} based on the program order of the thread. The \kcode{critical}
regions executed by the two threads execute in a serial manner, with a
pair-wise synchronization from the exit of one \kcode{critical} region to the
entry to the next \kcode{critical} region.  These pair-wise synchronizations
result from the implicit release flushes that occur on exit from
\kcode{critical} regions and the implicit acquire flushes that occur on entry to
\kcode{critical} regions; hence, the execution of each \kcode{critical} region in
the sequence happens before the execution of the next \kcode{critical} region.
A ``happens before'' order is therefore established between the assignment to \ucode{x}
by thread 0 and the read from \ucode{x} by thread 1, and so thread 1 must see that
\ucode{x} equals 10.

%\pagebreak
\cexample[5.0]{acquire_release}{1}
\ffreeexample[5.0]{acquire_release}{1}

\index{constructs!atomic@\kcode{atomic}}
\index{atomic construct@\kcode{atomic} construct}
\index{atomic construct@\kcode{atomic} construct!write clause@\kcode{write} clause}
\index{atomic construct@\kcode{atomic} construct!read clause@\kcode{read} clause}
\index{atomic construct@\kcode{atomic} construct!memory ordering clauses}
\index{write clause@\kcode{write} clause}
\index{read clause@\kcode{read} clause}
\index{clauses!write@\kcode{write}}
\index{clauses!read@\kcode{read}}
\index{memory ordering clauses!seq_cst@\kcode{seq_cst}}
\index{seq_cst clause@\kcode{seq_cst} clause}
\index{clauses!seq_cst@\kcode{seq_cst}}
In the second example, the \kcode{critical} constructs are exchanged with
\kcode{atomic} constructs that have \emph{explicit} memory ordering specified. When the
\plc{atomic read} operation on thread 1 reads a non-zero value from \ucode{y}, this
results in a release/acquire synchronization that in turn implies that the
assignment to \ucode{x} on thread 0 happens before the read of \ucode{x} on thread
1. Therefore, thread 1 will print ``x = 10''.

\cexample[5.0]{acquire_release}{2}
\ffreeexample[5.0]{acquire_release}{2}

\pagebreak
\index{constructs!atomic@\kcode{atomic}}
\index{atomic construct@\kcode{atomic} construct!relaxed atomic operations}
\index{flush construct@\kcode{flush} construct}
In the third example, \kcode{atomic} constructs that specify relaxed atomic
operations are used with explicit \kcode{flush} directives to enforce memory
ordering between the two threads. The explicit \kcode{flush} directive on thread
0 must specify a release flush and the explicit \kcode{flush} directive on
thread 1 must specify an acquire flush to establish a release/acquire
synchronization between the two threads. The \kcode{flush} and \kcode{atomic}
constructs encountered by thread 0 can be replaced by the \kcode{atomic} construct used in
Example 2 for thread 0, and similarly the \kcode{flush} and \kcode{atomic}
constructs encountered by thread 1 can be replaced by the \kcode{atomic}
construct used in Example 2 for thread 1.

%%%%%%%%%%%%%%%%%%%%%%%%%%%%%%3
%{\color{violet}
%For this example, the implicit release flush of the \code{flush} directive for thread 0 creates
%a source synchronization with release memory ordering, while the implicit release flush of the
%\code{flush} directive for thread 1 creates a sink synchronization with acquire memory ordering.
%The code performs the same thread synchronization of the previous example, with only a slight 
%coding change.
%The explicit \code{release} and \code{acquire} clauses of the atomic construct has been 
%replaced with implicit release and acquire flushes of explicit \code{flush} constructs.
%(Here, the \code{atomic} constructs have \plc{relaxed} operations.)
%}
%%%%%%%%%%%%%%%%%%%%%%%%%%%%%%3

\cexample[5.0]{acquire_release}{3}
\ffreeexample[5.0]{acquire_release}{3}

Example 4 will fail to order the write to \ucode{x} on thread 0 before the read
from \ucode{x} on thread 1. Importantly, the implicit release flush on exit from
the \kcode{critical} region will not synchronize with the acquire flush that
occurs on the \plc{atomic read} operation performed by thread 1. This is because
implicit release flushes that occur on a given construct may only synchronize
with implicit acquire flushes on a compatible construct (and vice-versa) that
internally makes use of the same synchronization variable. For a
\kcode{critical} construct, this might correspond to a \plc{lock} object that
is used by a given implementation (for the synchronization semantics of other
constructs due to implicit release and acquire flushes, refer to the \docref{Implicit
Flushes} section of the OpenMP Specifications document).  Either an explicit \kcode{flush}
directive that provides a release flush (i.e., a flush without a list that does
not have the \kcode{acquire} clause) must be specified between the
\kcode{critical} construct and the \plc{atomic write}, or an atomic operation that
modifies \ucode{y} and provides release semantics must be specified.

%{\color{violet}
%In the following example synchronization between the acquire flush of the atomic read
%of \plc{y} by thread 1 is not synchronized with the relaxed atomic construct that
%assigns a value to \plc{y} by thread 0.
%While there is a \code{critical} construct and implicit release flush
%for the \plc{x} assignment of thread 0,
%a release flush association with the \plc{y} assignment of
%thread 0 is not formed.  A \code{release} or \code{acq-rel} clause on the
%\code{atomic write} construct or a \code{flush} directive after the assignment to \plc{y}
%will form a synchronization and will guarantee  memory ordering of the x and y assignments
%by thread 0.
%}

\cexample[5.0]{acquire_release_broke}{4}
\ffreeexample[5.0]{acquire_release_broke}{4}

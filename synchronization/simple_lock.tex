\subsection{Simple Lock Routines}
\label{subsec:simple_lock}

\index{routines!omp_set_lock@\kcode{omp_set_lock}}
\index{omp_set_lock routine@\kcode{omp_set_lock} routine}
\index{routines!omp_test_lock@\kcode{omp_test_lock}}
\index{omp_test_lock routine@\kcode{omp_test_lock} routine}
In the following example, the lock routines cause the threads to be idle while 
waiting for entry to the first critical section, but to do other work while waiting 
for entry to the second. The \kcode{omp_set_lock} function blocks, but the \kcode{omp_test_lock} 
function does not, allowing the work in \ucode{skip} to be done. 

Note that the argument to the lock routines should have type 
\kcode{omp_lock_t} (or \kcode{omp_lock_kind} in Fortran), 
and that there is no need to flush the lock variable (\ucode{lck}). 

\cexample{simple_lock}{1}

\fexample{simple_lock}{1}



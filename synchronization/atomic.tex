%\pagebreak
\section{\kcode{atomic} Construct}
\label{sec:atomic}
\index{constructs!atomic@\kcode{atomic}}
\index{atomic construct@\kcode{atomic} construct}
\index{atomic construct@\kcode{atomic} construct!update clause@\kcode{update} clause}
\index{clauses!update@\kcode{update}}
\index{update clause@\kcode{update} clause}

The following example avoids race conditions (simultaneous updates of an element 
of \ucode{x} by multiple threads) by using the \kcode{atomic} construct .

The advantage of using the \kcode{atomic} construct in this example is that it 
allows updates of two different elements of \ucode{x} to occur in parallel. If 
a \kcode{critical} construct were used instead, then all updates to elements of 
\ucode{x} would be executed serially (though not in any guaranteed order).

Note that the \kcode{atomic} directive applies only to the statement immediately 
following it. As a result, elements of \ucode{y} are not updated atomically in 
this example.

\cexample[3.1]{atomic}{1}

\fexample[3.1]{atomic}{1}

\index{atomic construct@\kcode{atomic} construct!write clause@\kcode{write} clause}
\index{atomic construct@\kcode{atomic} construct!read clause@\kcode{read} clause}
\index{write clause@\kcode{write} clause}
\index{clauses!write@\kcode{write}}
\index{read clause@\kcode{read} clause}
\index{clauses!read@\kcode{read}}
The following example illustrates the \kcode{read} and \kcode{write}  clauses 
for the \kcode{atomic} directive. These clauses ensure that the given variable 
is read or written, respectively, as a whole. Otherwise, some other thread might 
read or write part of the variable while the current thread was reading or writing 
another part of the variable. Note that most hardware provides atomic reads and 
writes for some set of properly aligned variables of specific sizes, but not necessarily 
for all the variable types supported by the OpenMP API.

\cexample[3.1]{atomic}{2}

\fexample[3.1]{atomic}{2}

\index{atomic construct@\kcode{atomic} construct!capture clause@\kcode{capture} clause}
\index{capture clause@\kcode{capture} clause}
\index{clauses!capture@\kcode{capture}}
The following example illustrates the \kcode{capture} clause for the \kcode{atomic} 
directive. In this case the value of a variable is captured, and then the variable 
is incremented. These operations occur atomically. This example could 
be implemented using the \emph{fetch-and-add} instruction available on many kinds of hardware.
The example also shows a way to implement a spin lock using the \kcode{capture} 
 and \kcode{read} clauses.

\cexample[3.1]{atomic}{3}

\fexample[3.1]{atomic}{3}



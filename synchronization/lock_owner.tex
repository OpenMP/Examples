\subsection{Ownership of Locks}
\label{subsec:lock_owner}

\index{routines!omp_unset_lock@\scode{omp_unset_lock}}
\index{omp_unset_lock routine@\scode{omp_unset_lock} routine}
Ownership of locks has changed since OpenMP 2.5. In OpenMP 2.5, locks are owned 
by threads; so a lock released by the \code{omp\_unset\_lock} routine must be 
owned by the same thread executing the routine.  Beginning with OpenMP 3.0, locks are owned 
by task regions; so a lock released by the \code{omp\_unset\_lock} routine in 
a task region must be owned by the same task region.

This change in ownership requires extra care when using locks. The following program 
is conforming in OpenMP 2.5 because the thread that releases the lock \code{lck} 
in the parallel region is the same thread that acquired the lock in the sequential 
part of the program (primary thread of parallel region and the initial thread are 
the same). However, it is not conforming beginning with OpenMP 3.0, because the task 
region that releases the lock \code{lck} is different from the task region that 
acquires the lock.

\cexample[5.1]{lock_owner}{1}

\fexample[5.1]{lock_owner}{1}



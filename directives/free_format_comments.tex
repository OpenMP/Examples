%\pagebreak
\begin{fortranspecific}[4ex]
\section{Fortran Comments (Free Source Form)}
\label{sec:fortran_free_format_comments}
\index{directive syntax!free form, Fortran}
\index{free form syntax, Fortran}

OpenMP directives in Fortran codes with free source form are specified as comments
that use the \scode{!$omp} sentinel, followed by the
directive name, and required and optional clauses.  Lines are continued with an ending ampersand (\scode{&}),
and the continued line begins with \scode{!$omp} or \scode{!$omp&}. Comments may appear on the
same line as the directive.  Directives are case insensitive.

In the example below the first directive (DIR 1) specifies the %parallel work-sharing 
\kcode{parallel do} combined directive, with a \kcode{num_threads} clause, and a comment.
The second directive (DIR 2) shows the same directive split across two lines. 
The next nested directives (DIR 3 and 4) show the previous combined directive as
two separate directives. 
Here, an \kcode{end} directive (\kcode{end parallel}) must be specified to demarcate the range (region)
of the \kcode{parallel} directive. 

\ffreenexample{directive_syntax_F_free_comment}{1}
\clearpage

As of OpenMP 5.1, \bcode{block} and \bcode{end block} statements can be used to designate 
a structured block for an OpenMP region, and any paired OpenMP \kcode{end} directive becomes optional,
as shown in the next example.  Note, the variables \ucode{i} and \ucode{thrd_no} are declared within the 
block structure and are hence private.
It was necessary to explicitly declare the \ucode{i} variable, due to the \bcode{implicit none} statement; 
it could have also been declared outside the structured block.

\topmarker{Fortran}
\ffreenexample[5.1]{directive_syntax_F_block}{1}

A Fortran \bcode{BLOCK} construct may eliminate the need for a paired \kcode{end} directive for an OpenMP construct, 
as illustrated in the following example.

The first \kcode{parallel} construct is specified with an OpenMP loosely structured block 
(where the first executable construct is not a Fortran 2008 \bcode{BLOCK} construct). 
A paired \kcode{end} directive must end the OpenMP construct.
The second \kcode{parallel} construct is specified with an OpenMP strictly structured block 
(consists only of a single Fortran \bcode{BLOCK} construct). 
The paired \kcode{end} directive is optional in this case, and is not used here.

The next two \kcode{parallel} directives form an enclosing outer \kcode{parallel} construct 
and a nested inner \kcode{parallel} construct. The first \kcode{end parallel} directive
that subsequently appears terminates the inner \kcode{parallel} construct, 
because a paired \kcode{end} directive immediately following a \bcode{BLOCK} construct that is 
a strictly structured block of an OpenMP construct is treated as the terminating end directive 
of that construct. 
The next \kcode{end parallel} directive is required to terminate the outer \kcode{parallel} construct.

\topmarker{Fortran}
\ffreenexample[5.1]{directive_syntax_F_block}{2}
\end{fortranspecific}

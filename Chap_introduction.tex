% This is the introduction for the OpenMP Examples document.
% This is an included file. See the main file (openmp-examples.tex) for more information.
%
% When editing this file:
%
%    1. To change formatting, appearance, or style, please edit openmp.sty.
%
%    2. Custom commands and macros are defined in openmp.sty.
%
%    3. Be kind to other editors -- keep a consistent style by copying-and-pasting to
%       create new content.
%
%    4. We use semantic markup, e.g. (see openmp.sty for a full list):
%         \code{}     % for bold monospace keywords, code, operators, etc.
%         \plc{}      % for italic placeholder names, grammar, etc.
%
%    5. Other recommendations:
%         Use the convenience macros defined in openmp.sty for the minor headers
%         such as Comments, Syntax, etc.
%
%         To keep items together on the same page, prefer the use of 
%         \begin{samepage}.... Avoid \parbox for text blocks as it interrupts line numbering.
%         When possible, avoid \filbreak, \pagebreak, \newpage, \clearpage unless that's
%         what you mean. Use \needspace{} cautiously for troublesome paragraphs.
%
%         Avoid absolute lengths and measures in this file; use relative units when possible.
%         Vertical space can be relative to \baselineskip or ex units. Horizontal space
%         can be relative to \linewidth or em units.
%
%         Prefer \emph{} to italicize terminology, e.g.:
%             This is a \emph{definition}, not a placeholder.
%             This is a \plc{var-name}.
%

\cchapter{Introduction}{introduction}
\label{chap:introduction}

This collection of programming examples supplements the OpenMP API for Shared
Memory Parallelization specifications, and is not part of the formal specifications. It
assumes familiarity with the OpenMP specifications, and shares the typographical
conventions used in that document.

The OpenMP API specification provides a model for parallel programming that is
portable across shared memory architectures from different vendors. Compilers from
numerous vendors support the OpenMP API.

The directives, library routines, and environment variables demonstrated in this
document allow users to create and manage parallel programs while permitting
portability. The directives extend the C, C++ and Fortran base languages with single
program multiple data (SPMD) constructs, tasking constructs, device constructs,
worksharing constructs, and synchronization constructs, and they provide support for
sharing and privatizing data. The functionality to control the runtime environment is
provided by library routines and environment variables. Compilers that support the
OpenMP API often include a command line option to the compiler that activates and
allows interpretation of all OpenMP directives.

The documents and source codes for OpenMP Examples can be downloaded from
\href{https://github.com/OpenMP/Examples}{https://github.com/OpenMP/Examples}.
Each directory holds the contents of a chapter and has a \splc{sources} subdirectory of its codes. 
The codes for this OpenMP \VER{} Examples document have the tag 
\href{https://github.com/OpenMP/Examples/tree/v\VER}{\plc{v\PVER}}.

Complete information about the OpenMP API and a list of the compilers that support
the OpenMP API can be found at the OpenMP.org web site

\code{https://www.openmp.org}

\clearpage

\section{Examples Organization}
\label{chap:examples}
\label{sec:examples}
\index{example label}
\index{example label!omp_verno@\kcode{omp_\plc{verno}}}
\index{omp_verno@\kcode{omp_\plc{verno}}}

This document includes examples of the OpenMP API directives, constructs, and routines.

Each example is labeled with \plc{ename.seq-id.ext}, where \plc{ename} is 
the example name, \plc{seq-id} is the sequence identifier in a section, and 
\plc{ext} is the source file extension to indicate the code type and 
source form.  \plc{ext} is one of the following:
\begin{description}[noitemsep,labelindent=5mm,widest=f90]
\item[\plc{c}] -- \ C code,
\item[\plc{cpp}] -- \ C++ code,
\item[\plc{f}] -- \ Fortran code in fixed form, and
\item[\plc{f90}] -- \ Fortran code in free form.
\end{description}

Example labels include version information of the form
\verlabel{\plc{verno}} to indicate features that are illustrated
by an example for a specific OpenMP version, such as 
``\example{scan.1.c} \;\verlabel{5.0}.''
Some of the example labels include version information of the form
\verlabel[pre\_]{3.0} to indicate features that are specified 
prior to OpenMP version 3.0, such as
``\example{ploop.1.c} \;\verlabel[pre\_]{3.0}.''

Language markers may be used to indicate text or codes that are specific 
to a particular base language.
\begin{ccppspecific}
This is C/C++ specific: 
A statement following a directive is compound only when necessary, and a 
non-compound statement is indented with respect to a directive preceding it.
\end{ccppspecific}
\begin{fortranspecific}
This is Fortran specific...
\end{fortranspecific}
\linewitharrows{-1}{dashed}{Fortran (cont.)}{8em}
This marks the continuation of language specific page.

\medskip
Throughout the examples document we assume that the number of threads 
used for a \kcode{parallel} region is the same as 
the number of threads requested, unless explicitly specified otherwise.



% This is the end of introduction.tex of the OpenMP Examples document.


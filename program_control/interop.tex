\pagebreak
\section{\code{interop} Construct}
\label{sec:interop}

The \scode{interop} construct allows OpenMP to interoperate with foreign runtime environments.
In the example below, asynchronous cuda memory copies and a \splc{cublasDaxpy} routine are executed 
in a cuda stream. Also, an asynchronous target task execution (having a \scode{nowait} clause) 
and two explicit tasks are executed through OpenMP directives.  Scheduling dependences (synchronization) are
imposed on the foreign stream and the OpenMP tasks through \scode{depend} clauses. 

First, an interop object, \splc{obj}, is initialized for synchronization by including the
\scode{targetsync} \splc{interop-type} in the interop \scode{init} clause 
(\scode{init(}~\scode{targetsync,obj}~\scode{)}).  
The object provides access to the foreign runtime.
The \scode{depend} clause provides a dependence behavior
for foreign tasks associated with a valid object.

Next, the \scode{omp_get_interop_int} routine is used to extract the foreign 
runtime id (\scode{omp_ipr_fr_id}), and a test in the next statement ensures 
that the cuda runtime (\scode{omp_ifr_cuda}) is available.

Within the block for executing the \splc{cublasDaxpy} routine, a stream is acquired 
with the \scode{omp_get_interop_ptr} routine, which returns a cuda stream (\splc{s}).
The stream is included in the cublas handle, and used directly in the asynchronous memory
routines.  The following \scode{interop} construct, with the \scode{destroy} clause, 
ensures that the foreign tasks have completed.

\cexample[5.1]{interop}{1}

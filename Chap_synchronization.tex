\cchapter{Synchronization}{synchronization}
\label{chap:synchronization}

The \code{barrier} construct is a stand-alone directive that requires all threads
of a team (within a contention group) to execute the barrier and complete
execution of all tasks within the region, before continuing past the barrier.

The \code{critical} construct is a directive that contains a structured block. 
The construct allows only a single thread at a time to execute the structured block (region).
Multiple critical regions may exist in a parallel region, and may
act cooperatively (only one thread at a time in all \code{critical} regions),
or separately (only one thread at a time in each \code{critical} regions when
a unique name is supplied on each \code{critical} construct).
An optional (lock) \code{hint} clause may be specified on a named \code{critical} 
construct to provide the OpenMP runtime guidance in selection a locking 
mechanism.

On a finer scale the \code{atomic} construct allows only a single thread at 
a time to have atomic access to a storage location involving a single read, 
write, update or capture statement, and a limited number of combinations 
when specifying the \code{capture} \plc{atomic-clause} clause.  The
\plc{atomic-clause} clause is required for some expression statements, but is
not required for \code{update} statements. The \plc{memory-order} clause can be
used to specify the degree of memory ordering enforced by an \code{atomic}
construct. From weakest to strongest, they are \code{relaxed} (the default),
acquire and/or release clauses (specified with \code{acquire}, \code{release},
or \code{acq\_rel}), and \code{seq\_cst}.  Please see the details in the
\plc{atomic Construct} subsection of the \plc{Directives} chapter in the OpenMP
Specifications document.

% The following three sentences were stolen from the spec.
The \code{ordered} construct either specifies a structured block in a loop, 
simd, or loop SIMD region that will be executed in the order of the loop 
iterations.  The ordered construct sequentializes and orders the execution 
of ordered regions while allowing code outside the region to run in parallel.

Since OpenMP 4.5 the \code{ordered} construct can also be a stand-alone 
directive that specifies cross-iteration dependences in a doacross loop nest.  
The \code{depend} clause uses a \code{sink} \plc{dependence-type}, along with an 
iteration vector argument (vec) to indicate the iteration that satisfies the 
dependence.  The \code{depend} clause with a \code{source}
\plc{dependence-type} specifies dependence satisfaction.

The \code{flush} directive is a stand-alone construct for enforcing consistency
between a thread's view of memory and the view of memory for other threads (see
the Memory Model chapter of this document for more details). When the construct
is used with an explicit variable list, a \plc{strong flush} that forces a
thread's temporary view of memory to be consistent with the actual memory is
applied to all listed variables. When the construct is used without an explicit
variable list and without a \plc{memory-order} clause, a strong flush is
applied to all locally thread-visible data as defined by the base language, and
additionally the construct provides both acquire and release memory ordering
semantics.  When an explicit variable list is not present and a
\plc{memory-order} clause is present, the construct provides acquire and/or
release memory ordering semantics according to the \plc{memory-order} clause,
but no strong flush is performed. A resulting strong flush that applies to a
set of variables effectively ensures that no memory (load or store)
operation for the affected variables may be reordered across the \code{flush}
directive.

General-purpose routines provide mutual exclusion semantics through locks, 
represented by lock variables.  
The semantics allows a task to \plc{set}, and hence 
\plc{own} a lock, until it is \plc{unset} by the task that set it. A 
\plc{nestable} lock can be set multiple times by a task, and is used
when in code requires nested control of locks.  A \plc{simple lock} can
only be set once by the owning task. There are specific calls for the two
types of locks, and the variable of a specific lock type cannot be used by the
other lock type.  

Any explicit task will observe the synchronization prescribed in a 
\code{barrier} construct and an implied barrier.  Also, additional synchronizations 
are available for tasks.  All children of a task will wait at a \code{taskwait} (for 
their siblings to complete).  A \code{taskgroup} construct creates a region in which the
current task is suspended at the end of the region until all sibling tasks, 
and their descendants, have completed. 
Scheduling constraints on task execution can be prescribed by the \code{depend}
clause to enforce dependence on previously generated tasks.
More details on controlling task executions can be found in the \plc{Tasking} Chapter
in the OpenMP Specifications document. %(DO REF. RIGHT.)


%===== Examples Sections =====
%\pagebreak
\section{\kcode{critical} Construct}
\label{sec:critical}
\index{constructs!critical@\kcode{critical}}
\index{critical construct@\kcode{critical} construct}
\index{critical construct@\kcode{critical} construct!hint clause@\kcode{hint} clause}
\index{clauses!hint@\kcode{hint}}
\index{hint clause@\kcode{hint} clause}

The following example includes several \kcode{critical} constructs. The example 
illustrates a queuing model in which a task is dequeued and worked on. To guard 
against multiple threads dequeuing the same task, the dequeuing operation must 
be in a \kcode{critical} region. Because the two queues in this example are independent, 
they are protected by \kcode{critical} constructs with different names, \ucode{xaxis} 
and \ucode{yaxis}.

\cexample{critical}{1}

\fexample{critical}{1}

The following example extends the previous example by adding the \kcode{hint} clause to the \kcode{critical} constructs.

\cexample[5.0]{critical}{2}

\fexample[5.0]{critical}{2}

\pagebreak
\section{Worksharing Constructs Inside a \code{critical} Construct}
\label{sec:worksharing_critical}
\index{constructs!worksharing}
\index{constructs!critical@\code{critical}}
\index{critical construct@\code{critical} construct}

The following example demonstrates using a worksharing construct inside a \code{critical} 
construct. This example is conforming because the worksharing \code{single}  
region is not closely nested inside the \code{critical} region. A single thread 
executes the one and only section in the \code{sections} region, and executes 
the \code{critical} region. The same thread encounters the nested \code{parallel} 
region, creates a new team of threads, and becomes the primary thread of the new team. 
One of the threads in the new team enters the \code{single} region and increments 
\code{i} by \code{1}. At the end of this example \code{i} is equal to \code{2}.

\cexample{worksharing_critical}{1}

\fexample{worksharing_critical}{1}



\pagebreak
\section{Binding of \code{barrier} Regions}
\label{sec:barrier_regions}
\index{binding!barrier regions@\code{barrier} regions}

The binding rules call for a \code{barrier} region to bind to the closest enclosing 
\code{parallel} region. 

In the following example, the call from the main program to \plc{sub2} is conforming 
because the \code{barrier} region (in \plc{sub3}) binds to the \code{parallel} 
region in \plc{sub2}. The call from the main program to \plc{sub1} is conforming 
because the \code{barrier} region binds to the \code{parallel} region in subroutine 
\plc{sub2}.

The call from the main program to \plc{sub3} is conforming because the \code{barrier} 
region binds to the implicit inactive \code{parallel} region enclosing the sequential 
part. Also note that the \code{barrier} region in \plc{sub3} when called from 
\plc{sub2} only synchronizes the team of threads in the enclosing \code{parallel} 
region and not all the threads created in \plc{sub1}.

\cexample{barrier_regions}{1}

\fexample{barrier_regions}{1}



\pagebreak
\section{\code{atomic} Construct}
\label{sec:atomic}
\index{constructs!atomic@\code{atomic}}
\index{atomic construct@\code{atomic} construct}
\index{atomic construct@\code{atomic} construct!update clause@\code{update} clause}
\index{clauses!update@\code{update}}
\index{update clause@\code{update} clause}

The following example avoids race conditions (simultaneous updates of an element 
of \plc{x} by multiple threads) by using the \code{atomic} construct .

The advantage of using the \code{atomic} construct in this example is that it 
allows updates of two different elements of \plc{x} to occur in parallel. If 
a \code{critical} construct were used instead, then all updates to elements of 
\plc{x} would be executed serially (though not in any guaranteed order).

Note that the \code{atomic} directive applies only to the statement immediately 
following it. As a result, elements of \plc{y} are not updated atomically in 
this example.

\cexample[3.1]{atomic}{1}

\fexample[3.1]{atomic}{1}

\index{atomic construct@\code{atomic} construct!write clause@\code{write} clause}
\index{atomic construct@\code{atomic} construct!read clause@\code{read} clause}
\index{write clause@\code{write} clause}
\index{clauses!write@\code{write}}
\index{read clause@\code{read} clause}
\index{clauses!read@\code{read}}
The following example illustrates the \code{read} and \code{write}  clauses 
for the \code{atomic} directive. These clauses ensure that the given variable 
is read or written, respectively, as a whole. Otherwise, some other thread might 
read or write part of the variable while the current thread was reading or writing 
another part of the variable. Note that most hardware provides atomic reads and 
writes for some set of properly aligned variables of specific sizes, but not necessarily 
for all the variable types supported by the OpenMP API.

\cexample[3.1]{atomic}{2}

\fexample[3.1]{atomic}{2}

\index{atomic construct@\code{atomic} construct!capture clause@\code{capture} clause}
\index{capture clause@\code{capture} clause}
\index{clauses!capture@\code{capture}}
The following example illustrates the \code{capture} clause for the \code{atomic} 
directive. In this case the value of a variable is captured, and then the variable 
is incremented. These operations occur atomically. This example could 
be implemented using the fetch-and-add instruction available on many kinds of hardware. 
The example also shows a way to implement a spin lock using the \code{capture} 
 and \code{read} clauses.

\cexample[3.1]{atomic}{3}

\fexample[3.1]{atomic}{3}



\pagebreak
\section{Atomic Compare}
\label{sec:cas}

\index{constructs!atomic@\code{atomic}}
\index{atomic construct@\code{atomic} construct}
\index{clauses!capture@\code{capture}}
\index{clauses!compare@\code{compare}}
\index{capture clause@\code{capture} clause}
\index{compare clause@\code{compare} clause}

In OpenMP 5.1 the \scode{compare} clause was added to the extended-atomic clauses.
The \scode{compare} clause extends the semantics to perform the \scode{atomic}
update conditionally. 

In the following C/C++ example, two formats of structured blocks
are shown for associated \scode{atomic} constructs with the \scode{compare} clause.
In the first \scode{atomic} construct, the format forms a conditional update statement.
In the second \scode{atomic} construct the format forms a conditional expression statement.
The ``greater than'' and ``less than'' forms are not available with the Fortran \scode{compare}
clause.  One can use the \splc{max} and \splc{min} functions with the \scode{atomic}~\scode{update}
construct to perform the C/C++ example operations.

\cexample[5.1]{cas}{1}
%\ffreeexample[5.1]{cas}{1}

In OpenMP 5.1 the \scode{compare} clause was also added to support Compare And
Swap (CAS) semantics. In the following example the \splc{enqueue} routine
(a naive implementation of a Michael and Scott enqueue function), uses the
\scode{compare} clause, with the \scode{capture} clause, to perform and compare
(\splc{q->head == node->next}) and swap (\splc{if-else} assignments) of the
form: 
\begin{description}[noitemsep,labelindent=5mm,widest=f90]
\item \splc{{ r = x == e; if(r) { x = d; } else { v = x; } }}.
\end{description}
The example program concurrently enqueues nodes from an array of nodes (\splc{nodes[N]}).
Since the equivalence of Fortran pointers can be determined only with a function (such as associated),
no Fortran version is provided here. The use of the associated function in an atomic compare syntax is
being considered in a future release.

\cexample[5.1]{cas}{2}
%\ffreeexample[5.1]{cas}{2}

%\pagebreak
\section{Restrictions on the \kcode{atomic} Construct}
\label{sec:atomic_restrict}
\index{constructs!atomic@\kcode{atomic}}
\index{atomic construct@\kcode{atomic} construct}

The following non-conforming examples illustrate the restrictions on the \kcode{atomic} 
construct. 

\cexample[3.1]{atomic_restrict}{1}

\fexample[3.1]{atomic_restrict}{1}

\cexample[3.1]{atomic_restrict}{2}

\begin{fortranspecific}
The following example is non-conforming because \ucode{I} and \ucode{R} reference 
the same location but have different types.

\fnexample[3.1]{atomic_restrict}{2}

Although the following example might work on some implementations, this is also 
non-conforming:

\fnexample[3.1]{atomic_restrict}{3}
\end{fortranspecific}


\pagebreak
\section{\code{flush} Construct without a List}
\label{sec:flush_nolist}
\index{constructs!flush@\code{flush}}
\index{flush construct@\code{flush} construct}
\index{flushes!flush without a list}

The following example distinguishes the shared variables affected by a \code{flush} 
construct with no list from the shared objects that are not affected:

\cexample{flush_nolist}{1}

\fexample{flush_nolist}{1}



\pagebreak
\section{Synchronization Based on Acquire/Release Semantics}
\label{sec:acquire_and_release_semantics}

%OpenMP 5.0 introduced ``release/acquire'' memory ordering semantics to the
%specification. The memory ordering behavior of OpenMP constructs and routines
%that permit two threads to synchronize with each other are defined in terms of
%\textit{release flushes} and \textit{acquire flushes}, where a release flush
%must occur at the source of the synchronization and an acquire flush must occur
%at the sink of the synchronization. Flushes resulting from a \code{flush}
%directive without a list may function as a release flush, an acquire flush, or
%both a release and acquire flush. Flushes implied on entry to or exit from an
%atomic operation (specified by an \code{atomic} construct) may also function as
%a release flush or an acquire flush, depending on if a memory ordering clause
%appears on a construct. Flushes implied by other OpenMP constructs or routines
%also function as either a release flush or an acquire flush, according to the
%synchronization semantics of the construct.

%%%%%%%%%%%%%%%%%%

\index{flushes!acquire}
\index{flushes!release}
\index{clauses!memory ordering clauses}
\index{memory ordering clauses!acquire@\code{acquire}}
\index{acquire clause@\code{acquire} clause}
\index{memory ordering clauses!release@\code{release}}
\index{release clause@\code{release} clause}
\index{memory ordering clauses!acq_rel@\scode{acq_rel}}
\index{acq_rel clause@\scode{acq_rel} clause}
\index{flush construct@\code{flush} construct}
\index{atomic construct@\code{atomic} construct}
\index{clauses!acquire@\code{acquire}}
\index{clauses!release@\code{release}}
\index{clauses!acq_rel@\scode{acq_rel}}
As explained in the Memory Model chapter of this document, a flush operation
may be an \emph{acquire flush} and/or a \emph{release flush}, and OpenMP 5.0
defines acquire/release semantics in terms of these fundamental flush
operations.  For any synchronization between two threads that is specified by
OpenMP, a release flush logically occurs at the source of the synchronization
and an acquire flush logically occurs at the sink of the synchronization.
OpenMP 5.0 added memory ordering clauses -- \code{acquire}, \code{release}, and
\code{acq\_rel} -- to the \code{flush} and \code{atomic} constructs for
explicitly requesting acquire/release semantics.  Furthermore, implicit flushes
for all OpenMP constructs and runtime routines that synchronize OpenMP threads
in some manner were redefined in terms of synchronizing release and acquire
flushes to avoid the requirement of strong memory fences (see the \plc{Flush
Synchronization and Happens Before} and \plc{Implicit Flushes} sections of the
OpenMP Specifications document).

The examples that follow in this section illustrate how acquire and release
flushes may be employed, implicitly or explicitly, for synchronizing threads. A
\code{flush} directive without a list and without any memory ordering clause
can also function as both an acquire and release flush for facilitating thread
synchronization.  Flushes implied on entry to, or exit from, an atomic
operation (specified by an \code{atomic} construct) may function as an acquire
flush or a release flush if a memory ordering clause appears on the construct.
On entry to and exit from a \code{critical} construct there is now an implicit
acquire flush and release flush, respectively.

%%%%%%%%%%%%%%%%%%

\index{constructs!critical@\code{critical}}
\index{critical construct@\code{critical} construct}
\index{flushes!implicit}
The first example illustrates how the release and acquire flushes implied by a
\code{critical} region guarantee a value written by the first thread is visible
to a read of the value on the second thread. Thread 0 writes to \plc{x} and
then executes a \code{critical} region in which it writes to \plc{y}; the write
to \plc{x} happens before the execution of the \code{critical} region,
consistent with the program order of the thread.  Meanwhile, thread 1 executes a
\code{critical} region in a loop until it reads a non-zero value from
\plc{y} in the \code{critical} region, after which it prints the value of
\plc{x}; again, the execution of the \code{critical} regions happen before the
read from \plc{x} based on the program order of the thread. The \code{critical}
regions executed by the two threads execute in a serial manner, with a
pairwise synchronization from the exit of one \code{critical} region to the
entry to the next \code{critical} region.  These pairwise synchronizations
result from the implicit release flushes that occur on exit from
\code{critical} regions and the implicit acquire flushes that occur on entry to
\code{critical} regions; hence, the execution of each \code{critical} region in
the sequence happens before the execution of the next \code{critical} region.
A ``happens before'' order is therefore established between the assignment to \plc{x}
by thread 0 and the read from \plc{x} by thread 1, and so thread 1 must see that
\plc{x} equals 10.

\pagebreak
\cexample[5.0]{acquire_release}{1}
\ffreeexample[5.0]{acquire_release}{1}

\index{constructs!atomic@\code{atomic}}
\index{atomic construct@\code{atomic} construct}
\index{atomic construct@\code{atomic} construct!write clause@\code{write} clause}
\index{atomic construct@\code{atomic} construct!read clause@\code{read} clause}
\index{atomic construct@\code{atomic} construct!memory ordering clauses}
\index{write clause@\code{write} clause}
\index{read clause@\code{read} clause}
\index{clauses!write@\code{write}}
\index{clauses!read@\code{read}}
\index{memory ordering clauses!seq_cst@\scode{seq_cst}}
\index{seq_cst clause@\scode{seq_cst} clause}
\index{clauses!seq_cst@\scode{seq_cst}}
In the second example, the \code{critical} constructs are exchanged with
\code{atomic} constructs that have \textit{explicit} memory ordering specified. When the
atomic read operation on thread 1 reads a non-zero value from \plc{y}, this
results in a release/acquire synchronization that in turn implies that the
assignment to \plc{x} on thread 0 happens before the read of \plc{x} on thread
1. Therefore, thread 1 will print ``x = 10''.

\cexample[5.0]{acquire_release}{2}
\ffreeexample[5.0]{acquire_release}{2}

\pagebreak
\index{constructs!atomic@\code{atomic}}
\index{atomic construct@\code{atomic} construct!relaxed atomic operations}
\index{flush construct@\code{flush} construct}
In the third example, \code{atomic} constructs that specify relaxed atomic
operations are used with explicit \code{flush} directives to enforce memory
ordering between the two threads. The explicit \code{flush} directive on thread
0 must specify a release flush and the explicit \code{flush} directive on
thread 1 must specify an acquire flush to establish a release/acquire
synchronization between the two threads. The \code{flush} and \code{atomic}
constructs encountered by thread 0 can be replaced by the \code{atomic} construct used in
Example 2 for thread 0, and similarly the \code{flush} and \code{atomic}
constructs encountered by thread 1 can be replaced by the \code{atomic}
construct used in Example 2 for thread 1.

%%%%%%%%%%%%%%%%%%%%%%%%%%%%%%3
%{\color{violet}
%For this example, the implicit release flush of the \code{flush} directive for thread 0 creates
%a source synchronization with release memory ordering, while the implicit release flush of the
%\code{flush} directive for thread 1 creates a sink synchronization with acquire memory ordering.
%The code performs the same thread synchronization of the previous example, with only a slight 
%coding change.
%The explicit \code{release} and \code{acquire} clauses of the atomic construct has been 
%replaced with implicit release and acquire flushes of explicit \code{flush} constructs.
%(Here, the \code{atomic} constructs have \plc{relaxed} operations.)
%}
%%%%%%%%%%%%%%%%%%%%%%%%%%%%%%3

\cexample[5.0]{acquire_release}{3}
\ffreeexample[5.0]{acquire_release}{3}

Example 4 will fail to order the write to \plc{x} on thread 0 before the read
from \plc{x} on thread 1. Importantly, the implicit release flush on exit from
the \code{critical} region will not synchronize with the acquire flush that
occurs on the atomic read operation performed by thread 1. This is because
implicit release flushes that occur on a given construct may only synchronize
with implicit acquire flushes on a compatible construct (and vice-versa) that
internally makes use of the same synchronization variable. For a
\code{critical} construct, this might correspond to a \plc{lock} object that
is used by a given implementation (for the synchronization semantics of other
constructs due to implicit release and acquire flushes, refer to the \plc{Implicit
Flushes} section of the OpenMP Specifications document).  Either an explicit \code{flush}
directive that provides a release flush (i.e., a flush without a list that does
not have the \code{acquire} clause) must be specified between the
\code{critical} construct and the atomic write, or an atomic operation that
modifies \plc{y} and provides release semantics must be specified.

%{\color{violet}
%In the following example synchronization between the acquire flush of the atomic read
%of \plc{y} by thread 1 is not synchronized with the relaxed atomic construct that
%assigns a value to \plc{y} by thread 0.
%While there is a \code{critical} construct and implicit release flush
%for the \plc{x} assignment of thread 0,
%a release flush association with the \plc{y} assignment of
%thread 0 is not formed.  A \code{release} or \code{acq-rel} clause on the
%\code{atomic write} construct or a \code{flush} directive after the assignment to \plc{y}
%will form a synchronization and will guarantee  memory ordering of the x and y assignments
%by thread 0.
%}

\cexample[5.0]{acquire_release_broke}{4}
\ffreeexample[5.0]{acquire_release_broke}{4}

\pagebreak
\section{\code{ordered} Clause and \code{ordered} Construct}
\label{sec:ordered}
\index{clauses!ordered@\code{ordered}}
\index{ordered clause@\code{ordered} clause}
\index{constructs!ordered@\code{ordered}}
\index{ordered construct@\code{ordered} construct}

Ordered constructs  are useful for sequentially ordering the output from work that 
is done in parallel. The following program prints out the indices in sequential 
order:

\cexample{ordered}{1}

\fexample{ordered}{1}

It is possible to have multiple \code{ordered} constructs within a loop region 
with the \code{ordered} clause specified. The first example is non-conforming 
because all iterations execute two \code{ordered} regions. An iteration of a 
loop must not execute more than one \code{ordered} region:

\cexample{ordered}{2}

\fexample{ordered}{2}

The following is a conforming example with more than one \code{ordered} construct. 
Each iteration will execute only one \code{ordered} region:

\cexample{ordered}{3}

\fexample{ordered}{3}


\pagebreak
\section{\code{depobj} Construct}
\label{sec:depobj}
\index{constructs!depobj@\code{depobj}}
\index{depobj construct@\code{depobj} construct}
\index{depobj construct@\code{depobj} construct!depend clause@\code{depend} clause}
\index{depend clause@\code{depend} clause}
\index{clauses!depend@\code{depend}}

The stand-alone \code{depobj} construct provides a mechanism 
to create a \plc{depend object} that expresses a dependence to be 
used subsequently in the \code{depend} clause of another construct.
The dependence is created from a dependence type and a storage location,
within a \code{depend} clause of an \code{depobj} construct; 
%just as one would find directly on a \code{task} construct.  
and it is stored in the depend object.
The depend object is represented by a variable of type \code{omp\_depend\_t} 
in C/C++ (by a scalar variable of integer kind \code{omp\_depend\_kind} in Fortran).

\index{depobj construct@\code{depobj} construct!update clause@\code{update} clause}
\index{update clause@\code{update} clause}
\index{clauses!update@\code{update}}
\index{depobj construct@\code{depobj} construct!destroy clause@\code{destroy} clause}
\index{destroy clause@\code{destroy} clause}
\index{clauses!destroy@\code{destroy}}
In the example below the stand-alone \code{depobj} construct uses the 
\code{depend}, \code{update} and \code{destroy} clauses to 
\plc{initialize}, \plc{update} and \plc{uninitialize}
a depend object (\code{obj}).

The first \code{depobj} construct initializes the \code{obj} 
depend object with 
an \code{inout} dependence type with a storage 
location defined by variable \code{a}.  
This dependence is passed into the \plc{driver} 
routine via the \code{obj} depend object.

In the first \plc{driver} routine call, \emph{Task 1} uses
the dependence of the object (\code{inout}), 
while \emph{Task 2} uses an \code{in} dependence specified 
directly in a \code{depend} clause.
For these task dependences \emph{Task 1} must execute and 
complete before \emph{Task 2} begins.

Before the second call to \plc{driver}, \code{obj} is updated 
using the \code{depobj} construct to represent an \code{in} dependence. 
Hence, in the second call to \plc{driver}, \emph{Task 1}
will have an \code{in} dependence; and \emph{Task 1} and 
\emph{Task 2} can execute simultaneously. Note: in an \code{update}
clause, only the dependence type can be (is) updated.

The third \code{depobj} construct uses the \code{destroy} clause.
It frees resources as it puts the depend object in an uninitialized state--
effectively destroying the depend object.
After an object has been uninitialized it can be initialized again
with a new dependence type \emph{and} a new variable.

\cexample[5.2]{depobj}{1}

\ffreeexample[5.2]{depobj}{1}

\pagebreak
\section{Doacross Loop Nest}
\label{sec:doacross}
\index{dependences!doacross loop nest}
\index{doacross loop nest!ordered construct@\code{ordered} construct}
\index{ordered construct@\code{ordered} construct!doacross loop nest}
\index{doacross loop nest!doacross clause@\code{doacross} clause}
\index{constructs!ordered@\code{ordered}}
\index{clauses!doacross@\code{doacross}}
\index{doacross clause@\code{doacross} clause}

An \code{ordered} clause can be used on a loop construct with an integer
parameter argument to define the number of associated loops within 
a \plc{doacross loop nest} where cross-iteration dependences exist.
A \code{doacross} clause on an \code{ordered} construct within an ordered 
loop describes the dependences of the \plc{doacross} loops. 

In the code below, the \code{doacross(sink:i-1)} clause defines an \plc{i-1} 
to \plc{i} cross-iteration dependence that specifies a wait point for 
the completion of computation from iteration \plc{i-1} before proceeding 
to the subsequent statements. The \scode{doacross(source:omp_cur_iteration)} 
or \scode{doacross(source:)} clause indicates 
the completion of computation from the current iteration (\plc{i}) 
to satisfy the cross-iteration dependence that arises from the iteration.
The \scode{omp_cur_iteration} keyword is optional for the \scode{source}
dependence type.
For this example the same sequential ordering could have been achieved 
with an \code{ordered} clause without a parameter, on the loop directive, 
and a single \code{ordered} directive without the \code{doacross} clause
specified for the statement executing the \plc{bar} function.

\cexample[5.2]{doacross}{1}

\ffreeexample[5.2]{doacross}{1}

The following code is similar to the previous example but with 
\plc{doacross loop nest} extended to two nested loops, \plc{i} and \plc{j}, 
as specified by the \code{ordered(2)} clause on the loop directive. 
In the C/C++ code, the \plc{i} and \plc{j} loops are the first and
second associated loops, respectively, whereas
in the Fortran code, the \plc{j} and \plc{i} loops are the first and
second associated loops, respectively.
The \code{doacross(sink:i-1,j)} and \code{doacross(sink:i,j-1)} clauses in 
the C/C++ code define cross-iteration dependences in two dimensions from 
iterations (\plc{i-1, j}) and (\plc{i, j-1}) to iteration (\plc{i, j}).  
Likewise, the \code{doacross(sink:j-1,i)} and \code{doacross(sink:j,i-1)} clauses 
in the Fortran code define cross-iteration dependences from iterations 
(\plc{j-1, i}) and (\plc{j, i-1}) to iteration (\plc{j, i}).

\cexample[5.2]{doacross}{2}

\ffreeexample[5.2]{doacross}{2}


The following example shows the incorrect use of the \code{ordered} 
directive with a \code{doacross} clause.  There are two issues with the code.  
The first issue is a missing \code{ordered}~\code{doacross(source:)} directive,
which could cause a deadlock.  
The second issue is the \code{doacross(sink:i+1,j)} and \code{doacross(sink:i,j+1)} 
clauses define dependences on lexicographically later 
source iterations (\plc{i+1, j}) and (\plc{i, j+1}), which could cause 
a deadlock as well since they may not start to execute until the current iteration completes.

\cexample[5.2]{doacross}{3}

\ffreeexample[5.2]{doacross}{3}


The following example illustrates the use of the \code{collapse} clause for
a \plc{doacross loop nest}.  The \plc{i} and \plc{j} loops are the associated
loops for the collapsed loop as well as for the \plc{doacross loop nest}.
The example also shows a compliant usage of the dependence source
directive placed before the corresponding sink directive.
Checking the completion of computation from previous iterations at the sink point can occur after the source statement.

\cexample[5.2]{doacross}{4}

\ffreeexample[5.2]{doacross}{4}

%\pagebreak
\section{Lock Routines}
\label{sec:locks}

This section is about the use of lock routines for synchronization.

\subsection{\code{omp\_init\_lock} Routine}
\label{subsec:init_lock}

\index{routines!omp_init_lock@\scode{omp_init_lock}}
\index{omp_init_lock routine@\scode{omp_init_lock} routine}
The following example demonstrates how to initialize an array of locks in a \code{parallel} 
region by using \code{omp\_init\_lock}.

\cppexample{init_lock}{1}

\fexample{init_lock}{1}


%\pagebreak
\subsection{\code{omp\_init\_lock\_with\_hint} Routine}
\label{subsec:init_lock_with_hint}

\index{routines!omp_init_lock_with_hint@\scode{omp_init_lock_with_hint}}
\index{omp_init_lock_with_hint routine@\scode{omp_init_lock_with_hint} routine}
The following example demonstrates how to initialize an array of locks in a \code{parallel} region by using \code{omp\_init\_lock\_with\_hint}.
Note, hints are combined with an \code{|} or \code{+} operator in C/C++ and a \code{+} operator in Fortran.

\cppexample[5.0]{init_lock_with_hint}{1}

\fexample[5.0]{init_lock_with_hint}{1}
 
\subsection{Ownership of Locks}
\label{subsec:lock_owner}

\index{routines!omp_unset_lock@\kcode{omp_unset_lock}}
\index{omp_unset_lock routine@\kcode{omp_unset_lock} routine}
Ownership of locks has changed since OpenMP 2.5. In OpenMP 2.5, locks are owned 
by threads; so a lock released by the \kcode{omp_unset_lock} routine must be 
owned by the same thread executing the routine.  Beginning with OpenMP 3.0, locks are owned 
by tasks; so a lock released by the \kcode{omp_unset_lock} routine in
a task must be owned by the same task.

This change in ownership requires extra care when using locks. The following program 
is conforming in OpenMP 2.5 because the thread that releases the lock \ucode{lck} 
in the \kcode{parallel} region is the same thread that acquired the lock in the sequential
part of the program (primary thread of \kcode{parallel} region and the initial thread are
the same). However, it is not conforming beginning with OpenMP 3.0, because the task 
region that releases the lock \ucode{lck} is different from the task region that 
acquires the lock.

\cexample[5.1]{lock_owner}{1}

\fexample[5.1]{lock_owner}{1}



\subsection{Simple Lock Routines}
\label{subsec:simple_lock}

\index{routines!omp_set_lock@\scode{omp_set_lock}}
\index{omp_set_lock routine@\scode{omp_set_lock} routine}
\index{routines!omp_test_lock@\scode{omp_test_lock}}
\index{omp_test_lock routine@\scode{omp_test_lock} routine}
In the following example, the lock routines cause the threads to be idle while 
waiting for entry to the first critical section, but to do other work while waiting 
for entry to the second. The \code{omp\_set\_lock} function blocks, but the \scode{omp_test_lock} 
function does not, allowing the work in \code{skip} to be done. 

Note that the argument to the lock routines should have type 
\scode{omp_lock_t} (or \scode{omp_lock_kind} in Fortran), 
and that there is no need to flush the lock variable (\plc{lck}). 

\cexample{simple_lock}{1}

\fexample{simple_lock}{1}



\subsection{Nestable Lock Routines}
\label{subsec:nestable_lock}

\index{nestable lock}
The following example demonstrates how a nestable lock can be used to synchronize 
updates both to a whole structure and to one of its members.

\cexample{nestable_lock}{1}

\fexample{nestable_lock}{1}




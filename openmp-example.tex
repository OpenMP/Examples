% Welcome to openmp-examples.tex.
% This is the master LaTex file for the OpenMP Examples document.
%
% The files in this set include:
%
%    openmp-examples.tex              - this file, the master file
%    Makefile                         - makes the document
%    openmp.sty                       - the main style file
%    Title_Page.tex                   - the title page
%    openmplogo.png                   - the logo
%    Introduction_Chapt.tex           - unnumbered introductory chapter
%    Examples_Chapt.tex               - unnumbered chapter
%    Examples_Sects.tex               - examples
%    sources/*.c, *.f                 - C/C++/Fortran example source files
%
% When editing this file:
%
%    1. To change formatting, appearance, or style, please edit openmp.sty.
%
%    2. Custom commands and macros are defined in openmp.sty.
%
%    3. Be kind to other editors -- keep a consistent style by copying-and-pasting to
%       create new content.
%
%    4. We use semantic markup, e.g. (see openmp.sty for a full list):
%         \code{}     % for bold monospace keywords, code, operators, etc.
%         \plc{}      % for italic placeholder names, grammar, etc.
%
%    5. Other recommendations:
%         Use the convenience macros defined in openmp.sty for the minor headers
%         such as Comments, Syntax, etc.
%
%         To keep items together on the same page, prefer the use of 
%         \begin{samepage}.... Avoid \parbox for text blocks as it interrupts line numbering.
%         When possible, avoid \filbreak, \pagebreak, \newpage, \clearpage unless that's
%         what you mean. Use \needspace{} cautiously for troublesome paragraphs.
%
%         Avoid absolute lengths and measures in this file; use relative units when possible.
%         Vertical space can be relative to \baselineskip or ex units. Horizontal space
%         can be relative to \linewidth or em units.
%
%         Prefer \emph{} to italicize terminology, e.g.:
%             This is a \emph{definition}, not a placeholder.
%             This is a \plc{var-name}.
%

% The following says letter size, but the style sheet may change the size
\documentclass[10pt,letterpaper,twoside,makeidx,hidelinks]{scrreprt}

% Text to appear in the footer on even-numbered pages:
\newcommand{\VER}{5.0.0}
\newcommand{\PVER}{\VER{}p1}
\newcommand{\VERDATE}{February 2018}
\newcommand{\footerText}{OpenMP Examples Version \PVER{} - \VERDATE}

% Unified style sheet for OpenMP documents:
% This is openmp.sty, the preamble and style definitions for the OpenMP specification.
% This is an include file. Please see the main file for more information.
%
% When editing this file:
%
%    1. To change formatting, appearance, or style, please edit openmp.sty.
%
%    2. Custom commands and macros are defined in openmp.sty.
%
%    3. Be kind to other editors -- keep a consistent style by copying-and-pasting to
%       create new content.
%
%    4. We use semantic markup, e.g. (see openmp.sty for a full list):
%         \code{}     % for bold monospace keywords, code, operators, etc.
%         \plc{}      % for italic placeholder names, grammar, etc.
%
%    5. Other recommendations:
%         Use the convenience macros defined in openmp.sty for the minor headers
%         such as Comments, Syntax, etc.
%
%         To keep items together on the same page, prefer the use of 
%         \begin{samepage}.... Avoid \parbox for text blocks as it interrupts line numbering.
%         When possible, avoid \filbreak, \pagebreak, \newpage, \clearpage unless that's
%         what you mean. Use \needspace{} cautiously for troublesome paragraphs.
%
%         Avoid absolute lengths and measures in this file; use relative units when possible.
%         Vertical space can be relative to \baselineskip or ex units. Horizontal space
%         can be relative to \linewidth or em units.
%
%         Prefer \emph{} to italicize terminology, e.g.:
%             This is a \emph{definition}, not a placeholder.
%             This is a \plc{var-name}.
%
% Quick list of the environments, commands and macros supported. Search below for more details.
%
%    \binding                   % makes header of the same name
%    \comments
%    \constraints
%    \crossreferences
%    \descr
%    \effect
%    \format
%    \restrictions
%    \summary
%    \syntax
%    
%    \code{}                    % monospace, bold
%    \plc{}                     % for any kind of placeholder: italic
%    \begin{codepar}            % for blocks of verbatim code: monospace, bold
%    \begin{boxedcode}          % outlined verbatim code for syntax definitions, prototypes, etc.
%    \begin{indentedcodelist}   % used with,e.g., "where clause is one of the following:"
%
%    \specref{}                 % formats the cross-reference "Section X on page Y"
%    
%    \begin{note}               % black horizontal rule for Notes
%    \end{note}
%    
%    \begin{cspecific}          % blue horizontal rule for C-specific text
%    \end{cspecific}
%    
%    \begin{cppspecific}        % blue horizontal rule for C++ -specific text
%    \end{cppspecific}
%    
%    \begin{ccppspecific}       % blue horizontal rule for C/C++ -specific text
%    \end{ccppspecific}
%    
%    \begin{fortranspecific}    % blue horizontal rule for Fortran-specific text
%    \end{fortranspecific}
%    
%    \glossaryterm              % for use in formatting glossary entries
%    \glossarydefstart
%    \glossarydefend
%
%    \compactitem               % single-spaced itemized lists for the Examples doc
%    \cexample                  % C/C++ code example for the Examples doc
%    \fexample                  % Fortran code example for the Examples doc


\usepackage{comment}            % allow use of \begin{comment}
\usepackage{ifpdf,ifthen}       % allow conditional tests in LaTeX definitions
\usepackage{makecell}           % Allows common formatting in cells with \thread & \makecell


%%%%%%%%%%%%%%%%%%%%%%%%%%%%%%%%%%%%%%%%%%%%%%%%%%%%%%%%%%%%%%%%%%%%%%%%%%%%%%%%%%%%%%%%%%%%%%
% Document data
%
\author{}


%%%%%%%%%%%%%%%%%%%%%%%%%%%%%%%%%%%%%%%%%%%%%%%%%%%%%%%%%%%%%%%%%%%%%%%%%%%%%%%%%%%%%%%%%%%%%
% Fonts

\usepackage{amsmath}
\usepackage{amsfonts}
\usepackage{amssymb}
\usepackage{courier}
\usepackage{helvet}
\usepackage[utf8]{inputenc}

% Main body serif font:
\usepackage{tgtermes}
\usepackage[T1]{fontenc}


%%%%%%%%%%%%%%%%%%%%%%%%%%%%%%%%%%%%%%%%%%%%%%%%%%%%%%%%%%%%%%%%%%%%%%%%%%%%%%%%%%%%%%%%%%%%%
% Graphic elements

\usepackage{graphicx}
\usepackage{framed}    % for making boxes with \begin{framed}
\usepackage{tikz}      % for flow charts, diagrams, arrows
\usepackage{subcaption} % For subfigure
\usepackage{float}      % To fix location of figure: \begin{figure}[H] for no float

%%%%%%%%%%%%%%%%%%%%%%%%%%%%%%%%%%%%%%%%%%%%%%%%%%%%%%%%%%%%%%%%%%%%%%%%%%%%%%%%%%%%%%%%%%%%%
% Page formatting
%
% The PDF and book version need different margin spaces.  The bookbuild
% macro is set by the build system (see Makefile) to determine whether
% we are building the PDF or the print-on-demand book version of the spec.
%
\ifdefined\bookbuild
  % we are building the book version of the spec, so we need to have a bit
  % more margin for the publisher to print
  \usepackage[paperwidth=8in, paperheight=10in,
    top=1.25in, bottom=1.5in, left=1.65in, right=0.85in]{geometry}
  \setlength{\oddsidemargin}{0.875in}
  \setlength{\evensidemargin}{0.385in}
\else
  % we are building the PDF version of the spec, so we can use the default
  % margins
  \usepackage[paperwidth=7.5in, paperheight=9in,
    top=0.75in, bottom=1.0in, left=1.4in, right=0.6in]{geometry}
  \setlength{\oddsidemargin}{0.45in}
  \setlength{\evensidemargin}{0.185in}
\fi
\usepackage{changepage}   % allows left/right-page margin readjustments

\raggedbottom


%%%%%%%%%%%%%%%%%%%%%%%%%%%%%%%%%%%%%%%%%%%%%%%%%%%%%%%%%%%%%%%%%%%%%%%%%%%%%%%%%%%%%%%%%%%%%
% Paragraph formatting

\usepackage{setspace}     % allows use of \singlespacing, \onehalfspacing
\usepackage{needspace}    % allows use of \needspace to keep lines together
\usepackage{parskip}      % removes paragraph indenting

\raggedright
\usepackage[raggedrightboxes]{ragged2e}  % is this needed?

\lefthyphenmin=60         % only hyphenate if the left part is >= this many chars
\righthyphenmin=60        % only hyphenate if the right part is >= this many chars


%%%%%%%%%%%%%%%%%%%%%%%%%%%%%%%%%%%%%%%%%%%%%%%%%%%%%%%%%%%%%%%%%%%%%%%%%%%%%%%%%%%%%%%%%%%%%%
% Bulleted (itemized) lists
%    Align bullets with section header
%    Align text left
%    Small bullets
%    \compactitem for single-spaced lists (used in the Examples doc)

\usepackage{enumitem}     % for setting margins on lists
\setlist{leftmargin=*}    % don't indent bullet items
\renewcommand{\labelitemi}{{\normalsize$\bullet$}} % bullet size

% There is a \compactitem defined in package parlist (and perhaps others), however,
% we'll define our own version of compactitem in terms of package enumitem that
% we already use:
\newenvironment{compactitem}
{\begin{itemize}[itemsep=-1.2ex]}
{\end{itemize}}


%%%%%%%%%%%%%%%%%%%%%%%%%%%%%%%%%%%%%%%%%%%%%%%%%%%%%%%%%%%%%%%%%%%%%%%%%%%%%%%%%%%%%%%%%%%%%%
% Tables

% This allows tables to flow across page breaks, headers on each new page, etc.
\usepackage{supertabular}
\usepackage{caption}


%%%%%%%%%%%%%%%%%%%%%%%%%%%%%%%%%%%%%%%%%%%%%%%%%%%%%%%%%%%%%%%%%%%%%%%%%%%%%%%%%%%%%%%%%%%%%
% Line numbering

\usepackage[pagewise]{lineno}       % for line numbers on left side of the page
\pagewiselinenumbers
\setlength\linenumbersep{6em}
\renewcommand\linenumberfont{\normalfont\small\sffamily}
\nolinenumbers            % start with line numbers off


%%%%%%%%%%%%%%%%%%%%%%%%%%%%%%%%%%%%%%%%%%%%%%%%%%%%%%%%%%%%%%%%%%%%%%%%%%%%%%%%%%%%%%%%%%%%%
% Footers

\usepackage{fancyhdr}     % makes right/left footers
\pagestyle{fancy}
\fancyhead{} % clear all header fields
\cfoot{}
\renewcommand{\headrulewidth}{0pt}

% Left side on even pages:
% This requires that \footerText be defined in the main document:
\fancyfoot[LE]{\bfseries \thepage \mdseries \hspace{2em} \footerText}
\fancyhfoffset[E]{4em}

% Right side on odd pages:
\fancyfoot[RO]{\mdseries  \leftmark \hspace{2em} \bfseries \thepage}


%%%%%%%%%%%%%%%%%%%%%%%%%%%%%%%%%%%%%%%%%%%%%%%%%%%%%%%%%%%%%%%%%%%%%%%%%%%%%%%%%%%%%%%%%%%%%
% Section header format - we use four levels: \chapter \section \subsection \subsubsection.

\usepackage{titlesec}     % format headers with \titleformat{}
\usepackage{tocloft}

% Format and spacing for chapter, section, subsection, and subsubsection headers:

\setcounter{secnumdepth}{4}          % show numbers down to subsubsection level

\titleformat{\chapter}[hang]%
{\normalfont\sffamily\upshape\Huge\bfseries\fontsize{20}{20}\selectfont}%
{\thechapter}{0.5em}{}
\titlespacing{\chapter}{0ex}{0em plus 1em minus 1em}{2em plus 1em minus 0em}[10em]

\titleformat{\section}[hang]{\huge\bfseries\sffamily\fontsize{16}{16}\selectfont}{\thesection}{0.5em}{}
\titlespacing{\section}{0em}{3em plus 1em minus 1em}{1em plus 0.5em minus 0em}[10em]

\titleformat{\subsection}[hang]{\LARGE\bfseries\sffamily\fontsize{14}{14}\selectfont}{\thesubsection}{0.5em}{}
\titlespacing{\subsection}{0em}{3em plus 1em minus 1em}{0.75em plus 0.5em minus 0em}[10em]

\titleformat{\subsubsection}[hang]{\needspace{1\baselineskip}%
\Large\bfseries\sffamily\fontsize{12}{12}\selectfont}{\thesubsubsection}{0.5em}{}
\titlespacing{\subsubsection}{0em}{3em plus 1em minus 1em}{0.5em plus 0.5em minus 0em}[10em]

\setlength{\cftbeforetoctitleskip}{1.0ex}
\setlength{\cftaftertoctitleskip}{3.0ex}
\setlength{\cftbeforeloftitleskip}{1.0ex}
\setlength{\cftafterloftitleskip}{3.0ex}
\setlength{\cftbeforelottitleskip}{1.0ex}
\setlength{\cftafterlottitleskip}{3.0ex}
\renewcommand{\cftchapaftersnum}{}
\makeatletter
\renewcommand{\l@section}{\@dottedtocline{1}{1.5em}{2.6em}}
\renewcommand{\l@subsection}{\@dottedtocline{2}{4.1em}{3.4em}}
\makeatother

%%%%%%%%%%%%%%%%%%%%%%%%%%%%%%%%%%%%%%%%%%%%%%%%%%%%%%%%%%%%%%%%%%%%%%%%%%%%%%%%%%%%%%%%%%%%%%
% Macros for minor headers: Summary, Syntax, Description, etc.
% These headers are defined in terms of \paragraph

\titleformat{\paragraph}[block]{\large\bfseries\sffamily\fontsize{11}{11}\selectfont}{}{}{}
\titlespacing{\paragraph}{0em}{1.5em plus 0.55em minus 0.5em}{0.0em plus 0.55em minus 0.0em}

% Use one of the convenience macros below, or \littleheader{} for an arbitrary header
\newcommand{\littleheader}[1] {\paragraph*{#1}}

\newcommand{\binding} {\littleheader{Binding}}
\newcommand{\comments} {\littleheader{Comments}}
\newcommand{\constraints} {\littleheader{Constraints on Arguments}}
\newcommand{\crossreferences} {\littleheader{Cross References}}
\newcommand{\descr} {\littleheader{Description}}
\newcommand{\effect} {\littleheader{Effect}}
\newcommand{\format} {\littleheader{Format}}
\newcommand{\restrictions} {\littleheader{Restrictions}}
\newcommand{\summary} {\littleheader{Summary}}
\newcommand{\syntax} {\littleheader{Syntax}}

\usepackage{scrlayer}
\DeclareNewLayer[foreground,textarea,contents={
\phantom{a}
\emph{This page intentionally left blank}
    }
]{intentionally.text}
\DeclareNewPageStyleByLayers{intentionally}{intentionally.text}
\renewcommand{\cleardoublepage}{\cleardoubleoddpageusingstyle{intentionally}}
\newcommand{\chapdirname}{}
\newcommand{\cchapter}[2] {\cleardoublepage\chapter{#1}%
   \renewcommand{\chapdirname}{#2}}
\newcommand{\bchapter}[1] {\chapter*{#1}%
   \addcontentsline{toc}{chapter}{\protect\numberline{}#1}}
%\newcommand{\sinput}[1] {\input{\chapdirname/#1}}


%%%%%%%%%%%%%%%%%%%%%%%%%%%%%%%%%%%%%%%%%%%%%%%%%%%%%%%%%%%%%%%%%%%%%%%%%%%%%%%%%%%%%%%%%%%%%%
% Code and placeholder semantic tagging.
%
% When possible, prefer semantic tags instead of typographic tags. The
% following semantics tags are defined here:
%
%     \code{}     % for bold monospace keywords, code, operators, etc.
%     \plc{}      % for italic placeholder names, grammar, etc.
%
% For function prototypes or other code snippets, you can use \code{} as
% the outer wrapper, and use \plc{{} inside. Example:
%
%     \code{\#pragma omp directive ( \plc{some-placeholder-identifier} :}
%
% To format text in italics for emphasis (rather than text as a placeholder),
% use the generic \emph{} command. Example:
%
%     This sentence \emph{emphasizes some non-placeholder words}.

% Enable \alltt{} for formatting blocks of code:
\usepackage{alltt}
\usepackage{toolbox}         % for \toolboxReplace

% This sets the default \code{} font to tt (monospace) and bold:
\newcommand\code[1]{\texttt{\textbf{#1}}}
\newcommand\scode[1]{\protect\textbf{\protect\texttt{\protect\detokenize{#1}}}}

% This defines the \plc{} placeholder font to be tt normal slanted:
\newcommand\plc[1]{\textrm{\textmd{\itshape{#1}}}}
\newcommand\splc[1]{\protect\textit{\protect\textrm{\protect\detokenize{#1}}}}

% This is an updated set of macros for code style work
% kcode - keywords, vcode - value, bcode - base language,
% pvar - variables, pout - program outputs
\protected\def\DoReplaceU#1{\def\utexttmp{#1}%
   \toolboxReplace{_}{\_}\utexttmp\utexttmp}
\protected\def\myreplacedmt#1#2{\def\stexttmp{#1}%
   \toolboxReplace{_}{\_}\stexttmp%
   \toolboxReplace{ }{\rmfamily{ }\ttfamily#2}\stexttmp%
   {\ttfamily#2\stexttmp}}
\newcommand\kcode[1]{\myreplacedmt{#1}{\bfseries\upshape}}
\newcommand\vcode[1]{\myreplacedmt{#1}{\mdseries\upshape}}
\newcommand\bcode[1]{\kcode{#1}}
\newcommand\ucode[1]{\myreplacedmt{#1}{\mdseries\slshape}}
\newcommand\pvar[1]{\ucode{#1}}
\newcommand\pout[1]{\vcode{#1}}
\newcommand\docref[1]{\textrm{\mdseries\itshape{#1}}}
\newcommand\example[1]{\splc{#1}}

\newcommand\examplesrepo{https://github.com/OpenMP/Examples}
\newcommand\examplestree[2]{\href{\examplesrepo/tree/v#1}{#2}}
\newcommand\examplesref[1]{\examplestree{#1}{#1}}
\newcommand\examplesblob[1]{\href{\examplesrepo/blob/#1}{#1}}

% Environment for a paragraph of literal code, single-spaced, no outline, no indenting:
\usepackage{listings}
\lstnewenvironment{codepar}{%
  }{}
%\newenvironment{codepar}[1]
%{\begin{alltt}\bfseries #1}
%{\end{alltt}}

% For blocks of code inside a box frame:
\lstnewenvironment{boxedcode}{%
  \lstset{framesep=1.2ex,frame=l,framerule=3pt,
          backgroundcolor=\color{white!90!black}}}{}
\lstnewenvironment{boxeducode}{%
  \lstset{framesep=1.2ex,frame=l,framerule=3pt,
          basicstyle=\ttfamily\mdseries\slshape,
          backgroundcolor=\color{white!90!black}}}{}
%\newenvironment{boxedcode}[1]
%{\vspace{0.25em plus 5em minus %0.25em}\begin{framed}\begin{minipage}[t]{\textwidth}\begin{alltt}\bfseries #1}
%{\end{alltt}\end{minipage}\end{framed}\vspace{0.25em plus 5em minus 0.25em}}

% This sets the margins in the framed box:
\setlength{\FrameSep}{0.6em}

% For indented lists of verbatim code at a relaxed line spacing,
% e.g., for use after "where clause is one of the following:"
\usepackage{setspace}
\lstnewenvironment{indentedcodelist}{%
  \lstset{xleftmargin=0.25in}}{}
%\newenvironment{indentedcodelist}{%
%\begin{adjustwidth}{0.25in}{}\vspace{-0.2\baselineskip}\begin{spacing}{1.2}\beg%in{alltt}\bfseries}
%    {\end{alltt}\end{spacing}\vspace{-0.2\baselineskip}\end{adjustwidth}}

\lstdefinestyle{openmp}{
  showstringspaces=false,
  basicstyle=\ttfamily\bfseries,
  linewidth=.99\linewidth,
  xleftmargin=0.01\linewidth,
  columns=fullflexible,
  keepspaces=true,
  escapechar=@,
  belowskip=\smallskipamount,
  aboveskip=\smallskipamount,
  morecomment=[l][\color{red}\sout]{\%DIF\ <},         % deleted empty lines
  morecomment=[l][\color{blue}\uwave]{\%DIF\ >},       % added empty lines
  moredelim=[il][\color{red}\sout]{\%DIF\ <\ },        % deleted lines
  moredelim=[il][\color{blue}\uwave]{\%DIF\ >\ },      % added lines
  moredelim=**[is][\rmfamily\mdseries\itshape]{\\plc\{}{\}},
  moredelim=**[is][\textsubscript]{\\textsubscript\{}{\}},
  moredelim=**[is][]{\\textnormal\{}{\}},
  moredelim=**[is][\rmfamily\mdseries\itshape]{\\textsl\{}{\}},
  moredelim=**[is][\ttfamily\mdseries\slshape]{\\ucode\{}{\}},
  moredelim=**[is][\ttfamily\bfseries\upshape]{\\kcode\{}{\}},
  moredelim=**[is][]{\\code\{}{\}},
  moredelim=**[is][]{\\scode\{}{\}},
  moredelim=*[is][\color{red}\sout]{*!----}{----!*},
  moredelim=*[is][\color{blue}\uwave]{*!++++}{++++!*},
  moredelim=**[is][\mdseries\rmfamily]{\\text\{}{\}},
}
\lstset{style=openmp}


%%%%%%%%%%%%%%%%%%%%%%%%%%%%%%%%%%%%%%%%%%%%%%%%%%%%%%%%%%%%%%%%%%%%%%%%%%%%%%%%%%%%%%%%%%%%%%
% Macros for the black and blue lines and arrows delineating language-specific
% and notes sections. Example:
%
%   \fortranspecificstart
%   This is text that applies to Fortran.
%   \fortranspecificend

% local parameters for use \linewitharrows and \notelinewitharrows:
\newlength{\sbsz}\setlength{\sbsz}{0.05in}  % size of arrows
\newlength{\sblw}\setlength{\sblw}{1.35pt}  % line width (thickness)
\newlength{\sbtw}                           % text width
\newlength{\sblen}                          % total width of horizontal rule
\newlength{\sbht}                           % height of arrows
\newlength{\sbhadj}                         % vertical adjustment for aligning arrows with the line
\newlength{\sbns}\setlength{\sbns}{7\baselineskip}       % arg for \needspace for downward arrows

% \notelinewitharrows is a helper command that makes a black Note marker:
%     arg 1 = 1 or -1 for up or down arrows
%     arg 2 = solid or dashed or loosely dashed, etc.
\newcommand{\notelinewitharrows}[2]{%
    \needspace{0.1\baselineskip}%
    \vbox{\begin{tikzpicture}%
        \setlength{\sblen}{\linewidth}%
        \setlength{\sbht}{#1\sbsz}\setlength{\sbht}{1.4\sbht}%
        \setlength{\sbhadj}{#1\sblw}\setlength{\sbhadj}{0.25\sbhadj}%
        \filldraw (\sblen, 0) -- (\sblen - \sbsz, \sbht) -- (\sblen - 2\sbsz, 0) -- (\sblen, 0);
        \draw[line width=\sblw, #2] (2\sbsz - \sblw, \sbhadj) -- (\sblen - 2\sbsz + \sblw, \sbhadj);
        \filldraw (0, 0) -- (\sbsz, \sbht) -- (0 + 2\sbsz, 0) -- (0, 0);
    \end{tikzpicture}}}

% \linewitharrows is a helper command that makes a blue horizontal line, up or down arrows, and some text:
% arg 1 = 1 or -1 for up or down arrows
% arg 2 = solid or dashed or loosely dashed, etc.
% arg 3 = text
% arg 4 = text width
\newcommand{\linewitharrows}[4]{%
    \needspace{0.1\baselineskip}%
    \vbox to 1\baselineskip {\begin{tikzpicture}%
        \setlength{\sbtw}{#4}%
        \setlength{\sblen}{\linewidth}%
        \setlength{\sbht}{#1\sbsz}\setlength{\sbht}{1.4\sbht}%
        \setlength{\sbhadj}{#1\sblw}\setlength{\sbhadj}{0.25\sbhadj}%
        \filldraw[color=blue!40] (\sblen, 0) -- (\sblen - \sbsz, \sbht) -- (\sblen - 2\sbsz, 0) -- (\sblen, 0);
        \draw[line width=\sblw, color=blue!40, #2] (2\sbsz - \sblw, \sbhadj) -- (0.5\sblen - 0.5\sbtw, \sbhadj);
        \draw[line width=\sblw, color=blue!40, #2] (0.5\sblen + 0.5\sbtw, \sbhadj) -- (\sblen - 2\sbsz + \sblw, \sbhadj);
        \filldraw[color=blue!40] (0, 0) -- (\sbsz, \sbht) -- (0 + 2\sbsz, 0) -- (0, 0);
        \node[color=blue!90] at (0.5\sblen, 0) {\large  \textsf{\textup{#3}}};
    \end{tikzpicture}}}

\newcommand{\VSPb}{\vspace{0.5ex plus 5ex minus 0.25ex}}
\newcommand{\VSPa}{\vspace{0.25ex plus 5ex minus 0.25ex}}

% remove language marker definition if either ccpp or fortran is undefined
\ifthenelse{\boolean{ccpp}\and\boolean{fortran}}{}%
{\renewcommand{\linewitharrows}[4]{\par}}
\newcommand{\langselect}{}
\ifccpp\else\renewcommand{\langselect}{Fortran~}\fi
\iffortran\else\renewcommand{\langselect}{C/C++~}\fi

% C
\ifccpp
\newenvironment{cspecific}[1][0ex]{\vspace{#1}\cspecificstart\vspace{-#1}}{\cspecificend}
\else
\excludecomment{cspecific}
\fi
\newcommand{\cspecificstart}{\needspace{\sbns}\linewitharrows{-1}{solid}{C}{3em}}
\newcommand{\cspecificend}{\linewitharrows{1}{solid}{C}{3em}\bigskip}

% C/C++
\ifccpp
\newenvironment{ccppspecific}[1][0ex]{\vspace{#1}\ccppspecificstart\vspace{-#1}}{\ccppspecificend}
\else
\excludecomment{ccppspecific}
\fi
\newcommand{\ccppspecificstart}{\VSPb\linewitharrows{-1}{solid}{C / C++}{6em}\VSPa}
\newcommand{\ccppspecificend}{\VSPb\linewitharrows{1}{solid}{C / C++}{6em}\VSPa}

% C++
\ifccpp
\newenvironment{cppspecific}[1][0ex]{\vspace{#1}\cppspecificstart\vspace{-#1}}{\cppspecificend}
\else
\excludecomment{cppspecific}
\fi
\newcommand{\cppspecificstart}{\needspace{\sbns}\linewitharrows{-1}{solid}{C++}{6em}}
\newcommand{\cppspecificend}{\linewitharrows{1}{solid}{C++}{6em}\bigskip}

% C90
\newenvironment{cNinetyspecific}{\cNinetyspecificstart}{\cNinetyspecificend}
\newcommand{\cNinetyspecificstart}{\needspace{\sbns}\linewitharrows{-1}{solid}{C90}{4em}}
\newcommand{\cNinetyspecificend}{\linewitharrows{1}{solid}{C90}{4em}\bigskip}

% C99
\newenvironment{cNinetyNinespecific}{\cNinetyNinespecificstart}{\cNinetyNinespecificend}
\newcommand{\cNinetyNinespecificstart}{\needspace{\sbns}\linewitharrows{-1}{solid}{C99}{4em}}
\newcommand{\cNinetyNinespecificend}{\linewitharrows{1}{solid}{C99}{4em}\bigskip}

% Fortran
\iffortran
\newenvironment{fortranspecific}[1][0ex]{\vspace{#1}\fortranspecificstart\vspace{-#1}}{\fortranspecificend}
\else
\excludecomment{fortranspecific}
\fi
\newcommand{\fortranspecificstart}{\VSPb\linewitharrows{-1}{solid}{Fortran}{6em}\VSPa}
\newcommand{\fortranspecificend}{\VSPb\linewitharrows{1}{solid}{Fortran}{6em}\VSPa}

% Note
\newenvironment{note}{\notestart}{\noteend}
\newcommand{\notestart}{\VSPb\notelinewitharrows{-1}{solid}\VSPa}
\newcommand{\noteend}{\VSPb\notelinewitharrows{1}{solid}\VSPa}

% convenience macro for formatting the word "Note:" at the beginning of note blocks:
\newcommand{\noteheader}{{\textrm{\textsf{\textbf\textup\normalsize{{{{Note }}}}}}}}

% blue line floater at top of a page for "Name (cont.)"
\newcommand{\topmarker}[1]{%
  \begin{figure}[t!]
  \linewitharrows{-1}{dashed}{#1 (cont.)}{8em}
  \end{figure}}


%%%%%%%%%%%%%%%%%%%%%%%%%%%%%%%%%%%%%%%%%%%%%%%%%%%%%%%%%%%%%%%%%%%%%%%%%%%%%%%%%%%%%%%%%%%%%%
% Glossary formatting

\newcommand{\glossaryterm}[1]{\needspace{1ex}
\begin{adjustwidth}{-0.75in}{0.0in}
\nolinenumbers\parbox[b][-0.95\baselineskip][t]{1.4in}{\flushright \textbf{#1}}
\end{adjustwidth}\linenumbers}

\newcommand{\glossarydefstart}{
\begin{adjustwidth}{0.79in}{0.0in}}

\newcommand{\glossarydefend}{
\end{adjustwidth}\vspace{-1.5\baselineskip}}


%%%%%%%%%%%%%%%%%%%%%%%%%%%%%%%%%%%%%%%%%%%%%%%%%%%%%%%%%%%%%%%%%%%%%%%%%%%%%%%%%%%%%%%%%%%%%
% Indexing and Table of Contents

\usepackage{makeidx}
\usepackage[nodotinlabels]{titletoc}   % required for its [nodotinlabels] option

% Clickable links in TOC and index:
\usepackage[hyperindex=true,linktocpage=true]{hyperref}
\hypersetup{
  bookmarksnumbered = true,
  bookmarksopen     = false,
  colorlinks  = true, % Colors links instead of red boxes
  urlcolor    = blue, % Color for external links
  linkcolor   = blue  % Color for internal links
}


%%%%%%%%%%%%%%%%%%%%%%%%%%%%%%%%%%%%%%%%%%%%%%%%%%%%%%%%%%%%%%%%%%%%%%%%%%%%%%%%%%%%%%%%%%%%%
% Formats a cross reference label as "Section X on page Y".

\newcommand{\nspecref}[2]{#1~\ref{#2} on page~\pageref{#2}}
\newcommand{\specref}[1]{\nspecref{Section}{#1}}

% For caption for supertabular and figure, by yanyh15
\captionsetup[table]{labelfont={sf,sc,bf},textfont=normalfont,singlelinecheck=off,labelformat=simple,labelsep=colon,aboveskip=00pt,belowskip=10pt}

\captionsetup[figure]{labelfont={sf,sc,bf},textfont=normalfont,singlelinecheck=off,labelformat=simple,labelsep=colon}


%%%%%%%%%%%%%%%%%%%%%%%%%%%%%%%%%%%%%%%%%%%%%%%%%%%%%%%%%%%%%%%%%%%%%%%%%%%%%%%%%%%%%%%%%%%%%
% Code example formatting for the Examples document
% This defines:
%     \cexample       formats blue markers, caption, and code for C examples
%     \cppexample     formats blue markers, caption, and code for C++ examples
%     \fexample       formats blue markers, caption, and code for Fortran (fixed) examples
%     \ffreeexample   formats blue markers, caption, and code for Fortran90 (free) examples

\usepackage{color,fancyvrb}  % for \VerbatimInput
\usepackage{xargs}           % for optional args

\renewcommand\theFancyVerbLine{\normalfont\footnotesize\sffamily S-\arabic{FancyVerbLine}}

\newcommand{\escstr}[1]{\DoReplaceU{#1}}
\newcommandx*\verlabel[2][1=]{(\code{\small{}#1omp\_#2})}

\newcommand{\exampleheader}[6]{%
   \ifthenelse{ \equal{#1}{} }{
      \def\cname{#2}
      \def\ename\cname
   }{
      \def\cname{#1.#2.#3}
% Use following line for old numbering
%      \def\ename{\thechapter.#2.#3}
% Use following for mneumonics
      \def\ename{\escstr{#1}.\escstr{#2}.#3}
   }
   \newcount\cnt
   \cnt=#4
   \ifthenelse{ \equal{#5}{0} }{}{\global\advance\cnt by #5}

   \def\cverno{\ref{ex_vtag:\cname}}
   \ifthenelse{ \equal{\cverno}{pre_3.0} }{\def\cverno{}}{}
   \ifthenelse{ \equal{\cverno}{??} }{\def\cverno{#6}}{}
   \def\stagcnt{\pageref{ex_vtag:\cname}}
   \ifthenelse{ \equal{\cverno}{} }{
      \global\advance\cnt by 1
      \def\vername{\;\;\verlabel[pre\_]{3.0}}
   }{
      \def\myver##1{\toolboxSplitAt{##1}{_}\lefttext\righttext
         \lefttext\toolboxIfElse{\ifx\righttext\undefined}%
         {\global\advance\cnt by 1}{\expandafter{\righttext}}}
      \def\vername{\;\;\verlabel{\myver{\cverno}}}
   }
   \def\fcnt{\the\cnt}
%   \def\fcnt{\stagcnt}
   \noindent
   \underline{\hypertarget{ex:\cname}{\textit{Example \ename}}\vername}
   %\vspace*{-3mm}
   \code{\VerbatimInput[numbers=left,numbersep=5ex,firstnumber=1,firstline=\fcnt,fontsize=\small]%
      {\chapdirname/sources/\cname}} 
}

\newcommandx*\cnexample[4][1=,4=0]{%
   \exampleheader{#2}{#3}{c}{7}{#4}{#1}
}

\newcommandx*\cppnexample[4][1=,4=0]{%
   \exampleheader{#2}{#3}{cpp}{7}{#4}{#1}
}

\newcommandx*\fnexample[4][1=,4=0]{%
   \exampleheader{#2}{#3}{f}{5}{#4}{#1}
}

\newcommandx*\ffreenexample[4][1=,4=0]{%
   \exampleheader{#2}{#3}{f90}{5}{#4}{#1}
}

\newcommandx*\srcnexample[5][1=,5=0]{%
   \exampleheader{#2}{#3}{#4}{0}{#5}{#1}
}

\newcommandx*\cexample[4][1=,4=0]{%
\ifccpp
\needspace{5\baselineskip}\begin{ccppspecific}
\cnexample[#1]{#2}{#3}[#4]
\end{ccppspecific}
\fi
}

\newcommandx*\cppexample[4][1=,4=0]{%
\ifccpp
\needspace{5\baselineskip}\begin{cppspecific}
\cppnexample[#1]{#2}{#3}[#4]
\end{cppspecific}
\fi
}

\newcommandx*\fexample[4][1=,4=0]{%
\iffortran
\needspace{5\baselineskip}
\begin{fortranspecific}
\fnexample[#1]{#2}{#3}[#4]
\end{fortranspecific}
\fi
}

\newcommandx*\ffreeexample[4][1=,4=0]{%
\iffortran
\needspace{5\baselineskip}
\begin{fortranspecific}
\ffreenexample[#1]{#2}{#3}[#4]
\end{fortranspecific}
\fi
}

\newcommandx*\hexentry[4][1=c,3=]{%
  \hyperlink{ex:#2.#1}{\splc{#2.#1}}%
    \ifthenelse{ \equal{#3}{} }{}{,~\hyperlink{ex:#2.#3}{\plc{#3}}}%
    & #4%:~\splc{same name}
}
\newcommandx*\hexmentry[5][1=c,3=]{%
  \hyperlink{ex:#2.#1}{\splc{#2.#1}}%
    \ifthenelse{ \equal{#3}{} }{}{,~\hyperlink{ex:#2.#3}{\plc{#3}}}%
    & #4:~\splc{#5.#1}\ifthenelse{ \equal{#3}{} }{}{,~\plc{#3}}
}


% Set default fonts:
\rmfamily\mdseries\upshape\normalsize


% This is the end of openmp.sty of the OpenMP specification.



\begin{document}
    \pagenumbering{roman}

    \setcounter{page}{0}
    \setcounter{tocdepth}{2}


    % Uncomment the next line to enable line numbering on the main body text:
    \linenumbers\pagewiselinenumbers

    \newpage\pagenumbering{arabic}

    \setcounter{chapter}{0}  % start chapter numbering here

  % \input{Chap_Single}
       \input{Example}

   %\setcounter{chapter}{0}  % restart chapter numbering with "letter A"
   %\renewcommand{\thechapter}{\Alph{chapter}}%
   %\appendix
   %\cchapter{Document Revision History}{history}
\label{chap:history}

%=====================================
\section{Changes from 5.2.1 to 5.2.2}
\label{sec:history_521_to_522}

\begin{itemize}
\item To improve the style of the document, a set of macros was introduced
  and consistently used for language keywords, names, concepts, and user codes
  in the text description of the document.  Refer to the content of 
  \examplesblob{v5.2.2/Contributions.md}
  for details.

\item Added the following examples:
\begin{itemize}
  \item Orphaned and nested \kcode{loop} constructs (\specref{sec:loop})
  \item \kcode{all} variable category for the \kcode{defaultmap} clause
    (\specref{sec:defaultmap})
  \item \kcode{target update} construct using a custom mapper
    (\specref{subsec:target_update_mapper})
  \item \kcode{indirect} clause for indirect procedure calls in a
    \kcode{target} region (\specref{subsec:indirect})
  \item \kcode{omp_target_memcpy_async} routine with depend object
    (\specref{subsec:target_mem_and_device_ptrs})
  \item Synchronization hint for atomic operation (\specref{sec:atomic_hint})
  \item Implication of passing shared variable to a procedure
    in Fortran (\specref{sec:fort_shared_var})
  \item Assumption directives for providing additional information
    about program properties (\specref{sec:assumption})
  \item Mapping behavior of scalars, pointers, references (C++) and associate names
        (Fortran) when unified shared memory is required
    (\specref{sec:requires})
  \item \kcode{begin declare variant} paired with \kcode{end declare variant}
    example to show use of nested declare variant 
    directives (\specref{subsec:declare_variant})
  \item Explicit scoring in context selectors
    (\specref{subsec:context_selector_scoring})
\end{itemize}

\item Miscellaneous changes:
\begin{itemize}
  \item Included a general statement in Introduction about the number of 
    threads used throughout the examples document (\specref{sec:examples})
  \item Clarified the mapping of virtual functions in \kcode{target} regions
    (\specref{sec:virtual_functions})
  \item Added missing \kcode{declare target} directive for procedures
    called inside \kcode{target} region in \example{Examples} 
    \example{declare_mapper.1.f90} (\specref{sec:declare_mapper}),
    \example{target_reduction.*.f90} (\specref{subsec:target_reduction}), 
    and \example{target_task_reduction.*.f90}
    (\specref{subsec:target_task_reduction})
  \item Added missing \kcode{end target} directive in 
    \example{Example declare_mapper.3.f90}
    (\specref{sec:declare_mapper})
  \item Removed example for \kcode{flush} without a list from Synchronization 
    since the example is confusing and the use of \kcode{flush} is already
    covered in other examples
    (\specref{chap:synchronization})
  \item \docref{declare variant Directive} and \docref{Metadirective} sections were moved to
         subsections in the new \docref{Context-based Variant Selection} section, 
         with a section introduction on context selectors.
    (\specref{sec:context_based_variants})
  \item Fixed a typo (`\kcode{for}' $\rightarrow$ `\kcode{do}') in 
    \example{Example metadirective.4.f90}
    (\specref{subsec:metadirective})
\end{itemize}

\end{itemize}

%=====================================
\section{Changes from 5.2 to 5.2.1}
\label{sec:history_52_to_521}

\begin{itemize}
\item General changes:
\begin{itemize}
\item Updated source metadata tags for all examples to use an improved form
 (see \examplesblob{v5.2.1/Contributions.md})
\item Explicitly included the version tag \verlabel[pre\_]{3.0} in those 
 examples that did not contain a version tag previously
\end{itemize}

\item Added the following examples for the 5.2 features:
\begin{itemize}
  \item \kcode{uses_allocators} clause for the use of allocators in 
  \kcode{target} regions (\specref{sec:allocators})
\end{itemize}
\item Added the following examples for the 5.1 features:
\begin{itemize}
  \item The \kcode{inoutset} dependence type (\specref{subsec:task_concurrent_depend})
  \item Atomic compare and capture (\specref{sec:cas})
\end{itemize}
\item Added the following examples for the 5.0 features:
\begin{itemize}
  \item \kcode{declare target} directive with \kcode{device_type(nohost)}
     clause (\specref{subsec:declare_target_device_type})
  \item \kcode{omp_pause_resource} and \kcode{omp_pause_resource_all}
     routines (\specref{sec:pause_resource})
\end{itemize}

\item Miscellaneous fixes:
\begin{itemize}
  \item Cast to implementation-defined enum type \kcode{omp_event_handle_t} 
    now uses \bcode{uintptr_t} (not \bcode{void *}) in
    \example{Example task_detach.2.c}
    (\specref{sec:task_detachment})
  \item Moved Fortran \kcode{requires} directive into program main (\ucode{rev_off}), 
    the program unit, in \example{Example target_reverse_offload.7.f90}
    (\specref{subsec:target_reverse_offload})
  \item Fixed an inconsistent use of mapper in \example{Example target_mapper.3.f90}
    (\specref{sec:declare_mapper})
  \item Added a missing semicolon at end of \ucode{XOR1} class definition in
    \example{Example declare_target.2a.cpp}
    (\specref{subsec:declare_target_class})
  \item Fixed the placement of \kcode{declare simd} directive in 
    \example{Examples linear_modifier.*.f90} (\specref{sec:linear_modifier})
    and added a general statement about where a Fortran declarative
    directive can appear (\specref{chap:directive_syntax})
  \item Fixed mismatched argument list in \example{Example fort_sa_private.5.f}
    (\specref{sec:fort_sa_private})
  \item Moved the placement of \kcode{declare target enter}
    directive after function declaration
    (\specref{subsec:target_task_reduction})
  \item Fixed an incorrect use of \kcode{omp_in_parallel} routine in 
    \example{Example metadirective.4}
    (\specref{subsec:metadirective})
  \item Fixed an incorrect value for \kcode{at} clause
    (\specref{subsec:error})
\end{itemize}

\end{itemize}

%=====================================
\section{Changes from 5.1 to 5.2}
\label{sec:history_51_to_52}

\begin{itemize}
\item General changes:
\begin{itemize}
\item Included a description of the semantics for OpenMP directive syntax
 (see \specref{chap:directive_syntax})
\item Reorganized the Introduction Chapter and moved the Feature 
Deprecation Chapter to Appendix~\ref{chap:deprecated_features}
\item Included a list of examples that were updated for feature deprecation
and replacement in each version (see Appendix~\ref{sec:Updated Examples})
\item Added Index entries
\end{itemize}

\item Updated the examples for feature deprecation and replacement in OpenMP 5.2.
See Table~\ref{tab:Deprecated Features} and 
Table~\ref{tab:Updated Examples 5.2} for details.

\item Added the following examples for the 5.2 features:
\begin{itemize}
  \item Mapping class objects with virtual functions
     (\specref{sec:virtual_functions})
  \item \kcode{allocators} construct for Fortran \bcode{allocate} statement
     (\specref{sec:allocators})
  \item Behavior of reallocation of variables through OpenMP allocator in
     Fortran (\specref{sec:allocators})
\end{itemize}

\item Added the following examples for the 5.1 features:
\begin{itemize}
  \item Clarification of optional \kcode{end} directive for strictly structured 
      block in Fortran (\specref{sec:fortran_free_format_comments})
  \item \kcode{filter} clause on \kcode{masked} construct (\specref{sec:masked})
  \item \kcode{omp_all_memory} reserved locator for specifying task dependences
      (\specref{subsec:depend_undefer_task})
  \item Behavior of Fortran allocatable variables in \kcode{target} regions
      (\specref{sec:fort_allocatable_array_mapping})
  \item Device memory routines in Fortran
      (\specref{subsec:target_mem_and_device_ptrs})
  \item Partial tiles from \kcode{tile} construct
      (\specref{sec:incomplete_tiles})
  \item Fortran associate names and selectors in \kcode{target} region
      (\specref{sec:associate_target})
  \item \kcode{allocate} directive for variable declarations and 
      \kcode{allocate} clause on \kcode{task} constructs
      (\specref{sec:allocators})
  \item Controlling concurrency and reproducibility with \kcode{order} clause
      (\specref{sec:reproducible_modifier})
\end{itemize}

\item Added other examples:
\begin{itemize}
  \item Using lambda expressions with \kcode{target} constructs
     (\specref{sec:lambda_expressions})
  \item Target memory and device pointer routines
     (\specref{subsec:target_mem_and_device_ptrs})
  \item Examples to illustrate the ordering properties of 
     the \plc{flush} operation (\specref{sec:mem_model})
  \item User selector in the \kcode{metadirective} directive
     (\specref{subsec:metadirective})
\end{itemize}

\end{itemize}

%=====================================
\section{Changes from 5.0.1 to 5.1}
\label{sec:history_501_to_51}

\begin{itemize}
\item General changes:
\begin{itemize}
  \item Replaced \kcode{master} construct example with equivalent \kcode{masked} construct example (\specref{sec:masked})
  \item Primary thread is now used to describe thread number 0 in the current team
  \item \kcode{primary} thread affinity policy is now used to specify that every 
      thread in the team is assigned to the same place as the primary thread (\specref{subsec:affinity_primary})
  \item The \kcode{omp_lock_hint_*} constants have been renamed \kcode{omp_sync_hint_*} (\specref{sec:critical}, \specref{sec:locks})
\end{itemize}

\item Added the following new chapters:
\begin{itemize}
  \item Deprecated Features (on page~\pageref{chap:deprecated_features})
  \item Directive Syntax (\specref{chap:directive_syntax})
  \item Loop Transformations (\specref{chap:loop_transformations})
  \item OMPT Interface (\specref{chap:ompt_interface})
\end{itemize}

\item Added the following examples for the 5.1 features:
\begin{itemize}
  \item OpenMP directives in C++ \plc{attribute} specifiers
    (\specref{sec:attributes})
  \item Directive syntax adjustment to allow Fortran \bcode{BLOCK} ... 
    \bcode{END BLOCK} as a structured block 
      (\specref{sec:fortran_free_format_comments})
  \item \kcode{omp_target_is_accessible} API routine
      (\specref{sec:pointer_mapping})
  \item Fortran allocatable array mapping in \kcode{target} regions (\specref{sec:fort_allocatable_array_mapping})
  \item \kcode{begin declare target} (with
        \kcode{end declare target}) directive
      (\specref{subsec:declare_target_class})
  \item \kcode{tile} construct             (\specref{sec:tile})
  \item \kcode{unroll} construct           (\specref{sec:unroll})
  \item Reduction with the \kcode{scope} construct
      (\specref{subsec:reduction_scope})
  \item  \kcode{metadirective} directive with dynamic \kcode{condition} selector
      (\specref{subsec:metadirective})
  \item \kcode{interop} construct  (\specref{sec:interop})
  \item Environment display with the \kcode{omp_display_env} routine
      (\specref{subsec:display_env})
  \item \kcode{error} directive  (\specref{subsec:error})
\end{itemize}

\item Included additional examples for the 5.0 features:
\begin{itemize}
  \item \kcode{collapse} clause for non-rectangular loop nest
      (\specref{sec:collapse})
  \item \kcode{detach} clause for tasks (\specref{sec:task_detachment})
  \item Pointer attachment for a structure member (\specref{sec:structure_mapping})
  \item Host and device pointer association with the \kcode{omp_target_associate_ptr} routine (\specref{sec:target_associate_ptr})
  
  \item Sample code on activating the tool interface 
      (\specref{sec:ompt_start})
\end{itemize}

\item Added other examples:
\begin{itemize}
  \item The \kcode{omp_get_wtime} routine (\specref{subsec:get_wtime})
\end{itemize}
\end{itemize}


%=====================================
\section{Changes from 5.0.0 to 5.0.1}
\label{sec:history_50_to_501}

\begin{itemize}
\item Added version tags \verlabel{\plc{x.y}} in example labels 
and the corresponding source codes for all examples that feature
OpenMP 3.0 and later. 

\item Included additional examples for the 5.0 features:

\begin{itemize}
\item Extension to the \kcode{defaultmap} clause
(\specref{sec:defaultmap})
\item Transferring noncontiguous data with the \kcode{target update} directive in Fortran (\specref{sec:array-shaping})
\item \kcode{conditional} modifier for the \kcode{lastprivate} clause                             (\specref{sec:lastprivate})
\item \kcode{task} modifier for the \kcode{reduction} clause                                      (\specref{subsec:task_reduction})
\item Reduction on combined target constructs                                                     (\specref{subsec:target_reduction})
\item Task reduction with \kcode{target} constructs 
   (\specref{subsec:target_task_reduction})
\item \kcode{scan} directive for returning the \emph{prefix sum} of a reduction                  (\specref{sec:scan})

\end{itemize}

\item Included additional examples for the 4.x features:

\begin{itemize}
\item Dependence for undeferred tasks
(\specref{subsec:depend_undefer_task})
\item \kcode{ref}, \kcode{val}, \kcode{uval} modifiers for \kcode{linear} clause (\specref{sec:linear_modifier})

\end{itemize}

\item Clarified the description of pointer mapping and pointer attachment in 
\specref{sec:pointer_mapping}.
\item Clarified the description of memory model examples
in \specref{sec:mem_model}.

\end{itemize}


\section{Changes from 4.5.0 to 5.0.0}
\label{sec:history_45_to_50}

\begin{itemize}
\item Added the following examples for the 5.0 features:

\begin{itemize}
\item Extended \kcode{teams} construct for host execution     (\specref{sec:host_teams})
\item \kcode{loop} and \kcode{teams loop} constructs specify loop iterations that can execute concurrently 
                                                             (\specref{sec:loop})
\item Task data affinity is indicated by \kcode{affinity} clause of \kcode{task} construct
                                                             (\specref{sec: task_affinity})
\item Display thread affinity with \kcode{OMP_DISPLAY_AFFINITY} environment variable or \kcode{omp_display_affinity()} API routine
                                                             (\specref{sec:affinity_display})
\item \kcode{taskwait} with dependences                       (\specref{subsec:taskwait_depend})
\item \kcode{mutexinoutset} task dependences                  (\specref{subsec:task_dep_mutexinoutset})
\item Multidependence Iterators (in \kcode{depend} clauses)   (\specref{subsec:depend_iterator})
\item Combined constructs: \kcode{parallel master taskloop} and \kcode{parallel master taskloop simd}
                                                                                  (\specref{sec:parallel_masked_taskloop})
\item Reverse Offload through \kcode{ancestor} modifier of \kcode{device} clause.    (\specref{subsec:target_reverse_offload})
\item Pointer Mapping  - behavior of mapped pointers                              (\specref{sec:pointer_mapping})   %Example_target_ptr_map*
\item Structure Mapping  - behavior of mapped structures                          (\specref{sec:structure_mapping}) %Examples_target_structure_mapping.tex target_struct_map*
\item Array Shaping with the \plc{shape-operator}                                 (\specref{sec:array-shaping})
\item The \kcode{declare mapper} directive                                  (\specref{sec:declare_mapper})
\item Acquire and Release Semantics Synchronization: Memory ordering 
      clauses \kcode{acquire}, \kcode{release}, and \kcode{acq_rel} were added
      to flush and atomic constructs
                                                                                  (\specref{sec:acquire_and_release_semantics})
\item \kcode{depobj} construct provides dependence objects for subsequent use in \kcode{depend} clauses
                                                                                  (\specref{sec:depobj})
\item \kcode{reduction} clause for \kcode{task} construct                           (\specref{subsec:task_reduction})
\item \kcode{reduction} clause for \kcode{taskloop} construct                       (\specref{subsec:taskloop_reduction})
\item \kcode{reduction} clause for \kcode{taskloop simd} construct           (\specref{subsec:taskloop_reduction})
\item Memory Allocators for making OpenMP memory requests with traits             (\specref{sec:allocators})
\item \kcode{requires} directive specifies required features of implementation     (\specref{sec:requires})
\item \kcode{declare variant} directive - for function variants
(\specref{subsec:declare_variant})
\item \kcode{metadirective} directive  - for directive variants
(\specref{subsec:metadirective})
\item \kcode{OMP_TARGET_OFFLOAD} Environment Variable  - controls offload behavior (\specref{sec:target_offload})
\end{itemize}

\item Included the following additional examples for the 4.x features:
\begin{itemize}
\item more taskloop examples (\specref{sec:taskloop})
\item user-defined reduction (UDR) (\specref{subsec:UDR})
%NEW 5.0
%\item \code{target} \code{enter} and \code{exit} \code{data} unstructured data constructs  (\specref{sec:target_enter_exit_data}) %Example_target_unstructured_data.* ?

\end{itemize}
\end{itemize}

\section{Changes from 4.0.2 to 4.5.0}
\begin{itemize}
\item Reorganized into chapters of major topics
\item Included file extensions in example labels to indicate source type
\item Applied the explicit \kcode{map(tofrom)} for scalar variables 
      in a number of examples to comply with 
      the change of the default behavior for scalar variables from 
      \kcode{map(tofrom)} to \kcode{firstprivate} in the 4.5 specification
\item Added the following new examples:

\begin{itemize}
\item \kcode{linear} clause in loop constructs     (\specref{sec:linear_in_loop})
\item \kcode{priority} clause for \kcode{task} construct (\specref{sec:task_priority})
\item \kcode{taskloop} construct                   (\specref{sec:taskloop})
\item \plc{directive-name} modifier in multiple \kcode{if} clauses on
a combined construct                              (\specref{subsec:target_if})
\item unstructured data mapping                   (\specref{sec:target_enter_exit_data})
\item \kcode{link} clause for \kcode{declare target} directive 
                                                  (\specref{subsec:declare_target_link})
\item asynchronous target execution with \kcode{nowait} clause (\specref{sec:async_target_exec_depend})
\item device memory routines and device pointers  (\specref{subsec:target_mem_and_device_ptrs})
\item doacross loop nest                          (\specref{sec:doacross})
\item locks with hints                            (\specref{sec:locks})
\item C/C++ array reduction                       (\specref{subsec:reduction})
\item C++ reference types in data sharing clauses (\specref{sec:cpp_reference})
\end{itemize}

\end{itemize}

\section{Changes from 4.0.1 to 4.0.2}

\begin{itemize}
\item Names of examples were changed from numbers to mnemonics
\item Added SIMD examples                         (\specref{sec:SIMD})
\item Applied miscellaneous fixes in several source codes
\item Added the revision history
\end{itemize}

\section{Changes from 4.0 to 4.0.1}

Added the following new examples:
\begin{itemize}
\item the \kcode{proc_bind} clause   (\specref{sec:affinity})
\item the \kcode{taskgroup} construct (\specref{sec:taskgroup})
\end{itemize}

\section{Changes from 3.1 to 4.0}

\begin{itemize}
\item Beginning with OpenMP 4.0, examples were placed in a separate document
      from the specification document.
\item Version 4.0 added the following new examples:

\begin{itemize}
\item task dependences                       (\specref{sec:task_depend})
\item \kcode{target} construct                (\specref{sec:target})
\item array sections in device constructs    (\specref{sec:array_sections})
\item \kcode{target data} construct    (\specref{sec:target_data})
\item \kcode{target update} construct  (\specref{sec:target_update})
\item \kcode{declare target} directive (\specref{sec:declare_target})
\item \kcode{teams} constructs                (\specref{sec:teams})
\item asynchronous execution of a \kcode{target} region using tasks (\specref{subsec:async_target_with_tasks})
\item device runtime routines                (\specref{sec:device})
\item Fortran ASSOCIATE construct            (\specref{sec:associate})
\item cancellation constructs                (\specref{sec:cancellation})
\end{itemize}

\end{itemize}


\end{document}


\cchapter{SIMD}{SIMD}
\label{chap:simd}

Single instruction, multiple data (SIMD) is a form of parallel execution 
in which the same operation is performed on multiple data elements 
independently in hardware vector processing units (VPU), also called SIMD units.
The addition of two vectors to form a third vector is a SIMD operation.
Many processors have SIMD (vector) units that can perform simultaneously 
2, 4, 8 or more executions of the same operation (by a single SIMD unit). 

Loops without loop-carried backward dependency (or with dependency preserved using 
ordered simd) are candidates for vectorization by the compiler for 
execution with SIMD units. In addition, with state-of-the-art vectorization 
technology and \code{declare simd} directive extensions for function vectorization
in the OpenMP 4.5 specification, loops with function calls can be vectorized as well. 
The basic idea is that a scalar function call in a loop can be replaced by a vector version 
of the function, and the loop can be vectorized simultaneously by combining a loop 
vectorization (\code{simd} directive on the loop) and a function 
vectorization (\code{declare simd} directive on the function).

A \code{simd} construct states that SIMD operations be performed on the
data within the loop.  A number of clauses are available to provide
data-sharing attributes (\code{private}, \code{linear}, \code{reduction} and 
\code{lastprivate}).  Other clauses provide vector length preference/restrictions 
(\code{simdlen} / \code{safelen}), loop fusion (\code{collapse}), and data 
alignment (\code{aligned}).

The \code{declare simd} directive designates
that a vector version of the function should also be constructed for 
execution within loops that contain the function and have a \code{simd} 
directive.  Clauses provide argument specifications (\code{linear},
\code{uniform}, and \code{aligned}), a requested vector length 
(\code{simdlen}), and designate whether the function is always/never 
called conditionally in a loop (\code{notinbranch}/\code{inbranch}). 
The latter is for optimizing performance.

Also, the \code{simd} construct has been combined with the worksharing loop 
constructs (\code{for simd} and \code{do simd}) to enable simultaneous thread 
execution in different SIMD units.  
%Hence, the \code{simd} construct can be 
%used alone on a loop to direct vectorization (SIMD execution), or in 
%combination with a parallel loop construct to include thread parallelism 
%(a parallel loop sequentially followed by a \code{simd} construct,
%or a combined construct such as \code{parallel do simd} or 
%\code{parallel for simd}).


%===== Examples Sections =====
%\pagebreak
\section{\kcode{simd} and \kcode{declare simd} Directives}
\label{sec:SIMD}

\index{constructs!simd@\kcode{simd}}
\index{simd construct@\kcode{simd} construct}
The following example illustrates the basic use of the \kcode{simd} construct 
to assure the compiler that the loop can be vectorized.

\cexample[4.0]{SIMD}{1}

\ffreeexample[4.0]{SIMD}{1}
 

\index{directives!declare simd@\kcode{declare simd}}
\index{declare simd directive@\kcode{declare simd} directive}
\index{clauses!uniform@\kcode{uniform}}
\index{uniform clause@\kcode{uniform} clause}
\index{clauses!linear@\kcode{linear}}
\index{linear clause@\kcode{linear} clause}
When a function can be inlined within a loop the compiler has an opportunity to 
vectorize the loop. By guaranteeing SIMD behavior of a function's operations, 
characterizing the arguments of the function and privatizing temporary 
variables of the loop, the compiler can often create faster, vector code for 
the loop. In the examples below the \kcode{declare simd} directive is 
used on the \ucode{add1} and \ucode{add2} functions to enable creation of their 
corresponding SIMD function versions for execution within the associated SIMD 
loop. The functions characterize two different approaches of accessing data 
within the function: by a single variable and as an element in a data array, 
respectively. The \ucode{add3} C function uses dereferencing.

The \kcode{declare simd} directives also illustrate the use of 
\kcode{uniform} and \kcode{linear} clauses.  The \kcode{uniform(\ucode{fact})} clause 
indicates that the variable \ucode{fact} is invariant across the SIMD lanes. In 
the \ucode{add2} function \ucode{a} and \ucode{b} are included in the \kcode{uniform} 
list because the C pointer and the Fortran array references are constant.  The 
\ucode{i} index used in the \ucode{add2} function is included in a \kcode{linear} 
clause with a constant-linear-step of 1, to guarantee a unity increment of the 
associated loop. In the \kcode{declare simd} directive for the \ucode{add3} 
C function the  \kcode{linear(\ucode{a,b:1})} clause instructs the compiler to generate 
unit-stride loads across the SIMD lanes; otherwise,  costly \emph{gather} 
instructions would be generated for the unknown sequence of access of the 
pointer dereferences.

In the \kcode{simd} constructs for the loops the \kcode{private(\ucode{tmp})} clause is 
necessary to assure that each vector operation has its own \ucode{tmp}
variable.

\cexample[4.0]{SIMD}{2}

\ffreeexample[4.0]{SIMD}{2}

%\pagebreak
\index{clauses!private@\kcode{private}}
\index{private clause@\kcode{private} clause}
\index{clauses!reduction@\kcode{reduction}}
\index{reduction clause@\kcode{reduction} clause}
\index{reductions!reduction clause@\kcode{reduction} clause}
A thread that encounters a SIMD construct executes a vectorized code of the 
iterations. Similar to the concerns of a worksharing loop a loop vectorized 
with a SIMD construct must assure that temporary and reduction variables are 
privatized and declared as reductions with clauses.  The example below 
illustrates the use of \kcode{private} and \kcode{reduction} clauses in a SIMD 
construct.

\cexample[4.0]{SIMD}{3}

\ffreeexample[4.0]{SIMD}{3}


%\pagebreak
\index{clauses!safelen@\kcode{safelen}}
\index{safelen clause@\kcode{safelen} clause}
A \kcode{safelen(\ucode{N})} clause in a \kcode{simd} construct assures the compiler that 
there are no loop-carried dependences for vectors of size \ucode{N} or below. If
the \kcode{safelen} clause is not specified, then the default safelen value is 
the number of loop iterations.
 
The \kcode{safelen(\ucode{16})} clause in the example below guarantees that the vector 
code is safe for vectors up to and including size 16.  In the loop, \ucode{m} can 
be 16 or greater, for correct code execution.  If the value of \ucode{m} is less 
than 16, the behavior is undefined.

\cexample[4.0]{SIMD}{4}

\ffreeexample[4.0]{SIMD}{4}

%\pagebreak
\index{clauses!collapse@\kcode{collapse}}
\index{collapse clause@\kcode{collapse} clause}
The following SIMD construct instructs the compiler to collapse the \ucode{i} and 
\ucode{j} loops into a single SIMD loop in which SIMD chunks are executed by 
threads of the team. Within the workshared loop chunks of a thread, the SIMD 
chunks are executed in the lanes of the vector units.

\cexample[4.0]{SIMD}{5}

\ffreeexample[4.0]{SIMD}{5}


%%% section
\section{\kcode{inbranch} and \kcode{notinbranch} Clauses}
\label{sec:SIMD_branch}
\index{clauses!inbranch@\kcode{inbranch}}
\index{inbranch clause@\kcode{inbranch} clause}
\index{clauses!notinbranch@\kcode{notinbranch}}
\index{notinbranch clause@\kcode{notinbranch} clause}

The following examples illustrate the use of the \kcode{declare simd} 
directive with the \kcode{inbranch} and \kcode{notinbranch} clauses. The 
\kcode{notinbranch} clause informs the compiler that the function \ucode{foo} is 
never called conditionally in the SIMD loop of the function \ucode{myaddint}. On 
the other hand, the \kcode{inbranch} clause for the function goo indicates that 
the function is always called conditionally in the SIMD loop inside 
the function \ucode{myaddfloat}.

\cexample[4.0]{SIMD}{6}

\ffreeexample[4.0]{SIMD}{6}


In the code below, the function \ucode{fib()} is called in the main program and 
also recursively called in the function \ucode{fib()} within an \bcode{if} 
condition. The compiler creates a masked vector version and a non-masked vector 
version for the function \ucode{fib()} while retaining the original scalar 
version of the \ucode{fib()} function.

\cexample[4.0]{SIMD}{7}

\ffreeexample[4.0]{SIMD}{7}



%%% section
%\pagebreak
\section{Loop-Carried Lexical Forward Dependence}
\label{sec:SIMD_forward_dep}
\index{dependences!loop-carried lexical forward}


 The following example tests the restriction on an SIMD loop with the loop-carried lexical forward-dependence. This dependence must be preserved for the correct execution of SIMD loops.

A loop can be vectorized even though the iterations are not completely independent when it has loop-carried dependences that are forward lexical dependences, indicated in the code below by the read of \ucode{A[j+1]} and the write to \ucode{A[j]} in C/C++ code (or \ucode{A(j+1)} and \ucode{A(j)} in Fortran). That is, the read of \ucode{A[j+1]} (or \ucode{A(j+1)} in Fortran) before the write to \ucode{A[j]} (or \ucode{A(j)} in Fortran) ordering must be preserved for each iteration in \ucode{j} for valid SIMD code generation.

This test assures that the compiler preserves the loop-carried lexical forward-dependence for generating a correct SIMD code.

\cexample[4.0]{SIMD}{8}

\ffreeexample[4.0]{SIMD}{8}


%%% section
\section{\code{ref}, \code{val}, \code{uval} Modifiers for \code{linear} Clause}
\label{sec:linear_modifier}
\index{modifiers, linear@modifiers, \code{linear}!ref@\code{ref}}
\index{modifiers, linear@modifiers, \code{linear}!val@\code{val}}
\index{modifiers, linear@modifiers, \code{linear}!uval@\code{uval}}
\index{clauses!linear@\code{linear}}
\index{linear clause@\code{linear} clause}

When generating vector functions from \code{declare}~\code{simd} directives, 
it is important for a compiler to know the proper types of function arguments in
order to generate efficient codes.
This is especially true for C++ reference types and Fortran arguments.

In the following example, the function \plc{add\_one2} has a C++ reference
parameter (or Fortran argument) \plc{p}.  Variable \plc{p} gets incremented by 1 in the function.
The caller loop \plc{i} in the main program passes 
a variable \plc{k} as a reference to the function \plc{add\_one2} call.   
The \code{ref} modifier for the \code{linear} clause on the 
\code{declare}~\code{simd} directive specifies that the 
reference-type parameter \plc{p} is to match the property of the variable 
\plc{k} in the loop.  
This use of reference type is equivalent to the second call to 
\plc{add\_one2} with a direct passing of the array element \plc{a[i]}.  
In the example, the preferred vector 
length 8 is specified for both the caller loop and the callee function.

When \code{linear(p:~ref)} is applied to an argument passed by reference, 
it tells the compiler that the addresses in its vector argument are consecutive,
and so the compiler can generate a single vector load or store instead of 
a gather or scatter. This allows more efficient SIMD code to be generated with 
less source changes.

\cppexample[5.2]{linear_modifier}{1}
\ffreeexample[5.2]{linear_modifier}{1}
\clearpage

 
The following example is a variant of the above example. The function \plc{add\_one2} 
in the C++ code includes an additional C++ reference parameter \plc{i}. 
The loop index \plc{i} of the caller loop \plc{i} in the main program 
is passed as a reference to the function \plc{add\_one2} call.   
The loop index \plc{i} has a uniform address with
linear value of step 1 across SIMD lanes. 
Thus, the \code{uval} modifier is used for the \code{linear} clause 
to specify that the C++ reference-type parameter \plc{i} is to match 
the property of loop index \plc{i}.

In the corresponding Fortran code the arguments \plc{p} and
\plc{i} in the routine \plc{add\_on2} are passed by references. 
Similar modifiers are used for these variables in the \code{linear} clauses
to match with the property at the caller loop in the main program.

When \code{linear(i:~uval)} is applied to an argument passed by reference, it 
tells the compiler that its addresses in the vector argument are uniform 
so that the compiler can generate a scalar load or scalar store and create 
linear values. This allows more efficient SIMD code to be generated with 
less source changes.

\cppexample[5.2]{linear_modifier}{2}
\ffreeexample[5.2]{linear_modifier}{2}

In the following example, the function \plc{func} takes arrays \plc{x} and \plc{y} 
as arguments, and accesses the array elements referenced by the index \plc{i}.
The caller loop \plc{i} in the main program passes a linear copy of
the variable \plc{k} to the function \plc{func}.
The \code{val} modifier is used for the \code{linear} clause 
in the \code{declare}~\code{simd} directive for the function
\plc{func} to specify that the argument \plc{i} is to match the property of 
the actual argument \plc{k} passed in the SIMD loop.
Arrays \plc{x} and \plc{y} have uniform addresses across SIMD lanes.

When \code{linear(i:~val,step(1))} is applied to an argument, 
it tells the compiler that its addresses in the vector argument may not be 
consecutive, however, their values are linear (with stride 1 here). When the value of \plc{i} is used 
in subscript of array references (e.g., \plc{x[i]}), the compiler can generate 
a vector load or store instead of a gather or scatter. This allows more 
efficient SIMD code to be generated with less source changes.

\cexample[5.2]{linear_modifier}{3}
\ffreeexample[5.2]{linear_modifier}{3}






\pagebreak
\section{\kcode{masked} Construct}
\label{sec:masked}
\index{constructs!masked@\kcode{masked}}
\index{masked construct@\kcode{masked} construct}
\index{masked construct@\kcode{masked} construct!filter clause@\kcode{filter} clause}
\index{clauses!filter@\kcode{filter}}
\index{filter clause@\kcode{filter} clause}

The following example demonstrates the \kcode{masked} construct. 
In the example, the primary thread (thread number 0) 
keeps track of how many iterations have been executed and prints out 
a progress report in the iteration loop.
The other threads skip the \kcode{masked} region without waiting. 
The \kcode{filter} clause can be used to specify a thread number other 
than the primary thread to execute a structured block, as illustrated by
the second \kcode{masked} construct after the iteration loop.
If the thread specified in a \kcode{filter} clause does not exist 
in the team then the structured block is not executed by any thread.

\cexample[5.1]{masked}{1}

\fexample[5.1]{masked}{1}



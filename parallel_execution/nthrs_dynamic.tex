%\pagebreak
\section{Interaction Between the \kcode{num_threads} Clause and \kcode{omp_set_dynamic}}
\label{sec:nthrs_dynamic}
\index{clauses!num_threads@\kcode{num_threads}}
\index{num_threads clause@\kcode{num_threads} clause}
\index{routines!omp_set_dynamic@\kcode{omp_set_dynamic}}
\index{omp_set_dynamic routine@\kcode{omp_set_dynamic} routine}

The following example demonstrates the \kcode{num_threads} clause  and the effect 
of the \\
\kcode{omp_set_dynamic} routine  on it.

The call to the \kcode{omp_set_dynamic} routine with argument \ucode{0} in 
C/C++, or \ucode{.FALSE.} in Fortran, disables the dynamic adjustment of the number 
of threads in OpenMP implementations that support it. In this case, 10 threads 
are provided. Note that in case of an error the OpenMP implementation is free to 
abort the program or to supply any number of threads available.

\cexample{nthrs_dynamic}{1}

\fexample{nthrs_dynamic}{1}

%\pagebreak
The call to the \kcode{omp_set_dynamic} routine with a non-zero argument in 
C/C++, or \ucode{.TRUE.} in Fortran, allows the OpenMP implementation to choose 
any number of threads between 1 and 10.

\cexample{nthrs_dynamic}{2}

\fexample{nthrs_dynamic}{2}

It is good practice to set the \plc{dyn-var} ICV explicitly by calling the \kcode{omp_set_dynamic} 
routine, as its default setting is implementation defined.



\pagebreak
\section{Controlling Concurrency and Reproducibility with the \code{order} Clause}
\label{sec:order_clause}

The \code{order} clause is used for controlling the parallel execution of loop
iterations for one or more loops that are associated with a directive. It is
specified with a clause argument and optional modifier. The only supported
argument as of OpenMP 5.2 is the keyword \code{concurrent} which indicates that
the loop iterations may execute concurrently, including iterations in the same
chunk per the loop schedule. Because of the relaxed execution permitted with an
\code{order(concurrent)} clause, codes must not assume that any cross-iteration
data dependences would be preserved or that any two iterations may execute on
the same thread.

The first example in this section demonstrates the use of the
\code{order(concurrent)} clause, without any modifiers, for controlling the
parallel execution of loop iterations.

\cexample[5.1]{order}{1}

\fexample[5.1]{order}{1}

Modifiers to the \code{order} clause may be specified to control the
reproducibility of the loop schedule for the associated loop(s). A reproducible
loop schedule will consistently yield the same mapping of iterations to threads
(or SIMD lanes) if the directive name, loop schedule, iteration space, and
binding region remain the same. The \code{reproducible} modifier indicates the
loop schedule must be reproducible, while the \code{unconstrained} modifier
indicates that the loop schedule is not reproducible.

The next example demonstrates the use of the \code{order(concurrent)} clause
with modifiers for additionally controlling the reproducibility of a loop's
schedule.

\cexample[5.1]{order}{2}

\fexample[5.1]{order}{2}

\pagebreak
\section{\kcode{omp_get_num_threads} Routine}
\label{sec:get_nthrs}
\index{routines!omp_get_num_threads@\kcode{omp_get_num_threads}}
\index{omp_get_num_threads routine@\kcode{omp_get_num_threads} routine}

In the following example, the \kcode{omp_get_num_threads} call returns 1 in 
the sequential part of the code, so \ucode{np} will always be equal to 1. To determine 
the number of threads that will be deployed for the \kcode{parallel} region, the 
call should be inside the \kcode{parallel} region.

\cexample{get_nthrs}{1}

\fexample{get_nthrs}{1}

\pagebreak
The following example shows how to rewrite this program without including a query 
for the number of threads:

\cexample{get_nthrs}{2}

\fexample{get_nthrs}{2}



%\pagebreak
\section{\kcode{loop} Construct}
\label{sec:loop}
\index{constructs!loop@\kcode{loop}}
\index{loop construct@\kcode{loop} construct}

The following example illustrates the use of the OpenMP 5.0 \kcode{loop}
construct for the execution of a loop.
The \kcode{loop} construct asserts to the compiler that the iterations 
of the loop are free of data dependencies and may be executed concurrently.
It allows the compiler to use heuristics to select the parallelization scheme
and compiler-level optimizations for the concurrency. 

\cexample[5.0]{loop}{1}
\ffreeexample[5.0]{loop}{1}

The following example shows the use of the orphaned \kcode{loop} construct. Since the 
function \ucode{foo()} is not lexically nested inside of the \kcode{teams} region it needs to specify 
the \kcode{bind} clause. The first \kcode{loop} construct binds to the \kcode{teams} region 
from where the function \ucode{foo} is called. Binding to \kcode{teams} allows thread-level 
parallelism to be available for the second \kcode{loop} construct.
The loop iterations can be executed concurrently, 
thus allowing implementations to perform various loop nest optimizations including
reordering of the \ucode{i}  and \ucode{j} loops. The \kcode{loop} construct can be
implemented using any parallelism-generating mechanism, which allows better use
of hardware resources while also allowing sequential optimizations, reordering,
tiling etc.

For example, the first \kcode{loop} construct could be implemented as if it was specified as 
\kcode{distribute parallel for} and the second \kcode{loop} construct as if it was specified as 
\kcode{simd} if the hardware can support SIMD operations.
% \\
% OR
% \\
% For example, the first \kcode{loop} construct could be implemented as 
% \kcode{distribute parallel for} and the second \kcode{loop} construct as \kcode{simd} 
% if the hardware can support SIMD operations.

\cexample[5.0]{loop}{2}
\ffreeexample[5.0]{loop}{2}

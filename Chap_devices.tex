\cchapter{Devices}{devices}
\label{chap:devices}

The \kcode{target} construct consists of a \kcode{target} directive 
and an execution region. The \kcode{target} region is executed on
the default device or the device specified in the \kcode{device} 
clause. 

In OpenMP version 4.0, by default, all variables within the lexical
scope of the construct are copied \plc{to} and \plc{from} the
device, unless the device is the host, or the data exists on the
device from a previously executed data-type construct that
has created space on the device and possibly copied host
data to the device storage.

The constructs that explicitly
create storage, transfer data, and free storage on the device
are categorized as structured and unstructured. The
\kcode{target data} construct is structured. It creates
a data region around \kcode{target} constructs, and is
convenient for providing persistent data throughout multiple
\kcode{target} regions. The \kcode{target enter data} and 
\kcode{target exit data} constructs are unstructured, because 
they can occur anywhere and do not support a ``structure''
(a region) for enclosing \kcode{target} constructs, as does the
\kcode{target data} construct. 

The \kcode{map} clause is used on \kcode{target} 
constructs and the data-type constructs to map host data. It 
specifies the device storage and data movement \plc{to} and \plc{from}
the device, and controls on the storage duration.

There is an important change in the OpenMP 4.5 specification
that alters the data model for scalar variables and C/C++ pointer variables.
The default behavior for scalar variables and C/C++ pointer variables
in a 4.5 compliant code is \kcode{firstprivate}. Example
codes that have been updated to reflect this new behavior are
annotated with a description that describes changes required
for correct execution. Often it is a simple matter of mapping
the variable as \kcode{tofrom} to obtain the intended 4.0 behavior.

In OpenMP version 4.5 the mechanism for target
execution is specified as occurring through a \plc{target task}. 
When the \kcode{target} construct is encountered a new 
target task is generated. The target task 
completes after the \kcode{target} region has executed and all data 
transfers have finished.

This new specification does not affect the execution of 
pre-4.5 code; it is a necessary element for asynchronous 
execution of the \kcode{target} region when using the new \kcode{nowait} 
clause introduced in OpenMP 4.5.


%===== Examples Sections =====
\pagebreak
\section{\code{target} Construct}
\label{sec:target}

\subsection{\code{target} Construct on \code{parallel} Construct}
\label{subsec:target_parallel}
\index{constructs!target@\code{target}}
\index{target construct@\code{target} construct}
\index{target construct@\code{target} construct!implicit mapping}

This following example shows how the \code{target} construct offloads a code 
region to a target device. The variables \plc{p}, \plc{v1}, \plc{v2}, and \plc{N} are implicitly mapped 
to the target device.

\cexample[4.0]{target}{1}

\ffreeexample[4.0]{target}{1}

\subsection{\code{target} Construct with \code{map} Clause}
\label{subsec:target_map}
\index{target construct@\code{target} construct!map clause@\code{map} clause}
\index{clauses!map@\code{map}}
\index{map clause@\code{map} clause}

This following example shows how the \code{target} construct offloads a code 
region to a target device. The variables \plc{p}, \plc{v1} and \plc{v2} are explicitly mapped to the 
target device using the \code{map} clause. The variable \plc{N} is implicitly mapped to 
the target device.

\cexample[4.0]{target}{2}

\ffreeexample[4.0]{target}{2}

\subsection{\code{map} Clause with \code{to}/\code{from} map-types}
\label{subsec:target_map_tofrom}
\index{map clause@\code{map} clause!to map-type@\code{to} map-type}
\index{map clause@\code{map} clause!from map-type@\code{from} map-type}

The following example shows how the \code{target} construct offloads a code region 
to a target device. In the \code{map} clause, the \code{to} and \code{from} 
map-types define the mapping between the original (host) data and the target (device) 
data. The \code{to} map-type specifies that the data will only be read on the 
device, and the \code{from} map-type specifies that the data will only be written 
to on the device. By specifying a guaranteed access on the device, data transfers 
can be reduced for the \code{target} region.

The \code{to} map-type indicates that at the start of the \code{target} region 
the variables \plc{v1} and \plc{v2} are initialized with the values of the corresponding variables 
on the host device, and at the end of the \code{target} region the variables 
\plc{v1} and \plc{v2} are not assigned to their corresponding variables on the host device.

The \code{from} map-type indicates that at the start of the \code{target} region 
the variable \plc{p} is not initialized with the value of the corresponding variable 
on the host device, and at the end of the \code{target} region the variable \plc{p} 
is assigned to the corresponding variable on the host device.

\cexample[4.0]{target}{3}

The \code{to} and \code{from} map-types allow programmers to optimize data 
motion. Since data for the \plc{v} arrays are not returned, and data for the \plc{p} array 
are not transferred to the device, only one-half of the data is moved, compared 
to the default behavior of an implicit mapping.

\ffreeexample[4.0]{target}{3}

\subsection{\code{map} Clause with Array Sections}
\label{subsec:target_array_section}
\index{map clause@\code{map} clause!array sections in}

The following example shows how the \code{target} construct offloads a code region 
to a target device. In the \code{map} clause, map-types are used to optimize 
the mapping of variables to the target device. Because variables \plc{p}, \plc{v1} and \plc{v2} are 
pointers, array section notation must be used to map the arrays. The notation \code{:N} 
is equivalent to \code{0:N}.

\cexample[4.0]{target}{4}
\clearpage

In C, the length of the pointed-to array must be specified. In Fortran the extent 
of the array is known and the length need not be specified. A section of the array 
can be specified with the usual Fortran syntax, as shown in the following example. 
The value 1 is assumed for the lower bound for array section \plc{v2(:N)}.

\ffreeexample[4.0]{target}{4}

A more realistic situation in which an assumed-size array is passed to \code{vec\_mult} 
requires that the length of the arrays be specified, because the compiler does 
not know the size of the storage. A section of the array must be specified with 
the usual Fortran syntax, as shown in the following example. The value 1 is assumed 
for the lower bound for array section \plc{v2(:N)}.

\ffreeexample[4.0]{target}{4b}

\subsection{\code{target} Construct with \code{if} Clause}
\label{subsec:target_if}
\index{target construct@\code{target} construct!if clause@\code{if} clause}
\index{clauses!if@\code{if}}
\index{if clause@\code{if} clause}

The following example shows how the \code{target} construct offloads a code region 
to a target device.

The \code{if} clause on the \code{target} construct indicates that if the variable 
\plc{N} is smaller than a given threshold, then the \code{target} region will be executed 
by the host device.

The \code{if} clause on the \code{parallel} construct indicates that if the 
variable \plc{N} is smaller than a second threshold then the \code{parallel} region 
is inactive.

\cexample[4.0]{target}{5}

\ffreeexample[4.0]{target}{5}

The following example is a modification of the above \plc{target.5} code to show the combined \code{target}
and parallel loop directives. It uses the \plc{directive-name} modifier in multiple \code{if}
clauses to specify the component directive to which it applies. 

The \code{if} clause with the \code{target} modifier applies to the \code{target} component of the 
combined directive, and the \code{if} clause with the \code{parallel} modifier applies 
to the \code{parallel} component of the combined directive.    

\cexample[4.5]{target}{6}

\ffreeexample[4.5]{target}{6}

\subsection{Target Reverse Offload}
\label{subsec:target_reverse_offload}
\index{target reverse offload!reverse_offload clause@\scode{reverse_offload} clause}
\index{target reverse offload!requires directive@\code{requires} directive}
\index{requires directive@\code{requires} directive!reverse_offload clause@\scode{reverse_offload} clause}
\index{directives!requires@\code{requires}}
\index{clauses!reverse_offload@\scode{reverse_offload}}
\index{reverse_offload clause@\scode{reverse_offload} clause}
\index{target construct@\code{target} construct!device clause@\code{device} clause}
\index{clauses!device@\code{device}}
\index{device clause@\code{device} clause!ancestor modifier@\code{ancestor} modifier}
\index{ancestor modifier@\code{ancestor} modifier}
\index{declare target directive@\code{declare}~\code{target} directive!device_type clause@\scode{device_type} clause}
\index{clauses!device_type@\scode{device_type}}
\index{device_type clause@\scode{device_type} clause}
\index{clauses!enter@\code{enter}}
\index{enter clause@\code{enter} clause}

\index{directives!declare target@\code{declare}~\code{target}}
\index{declare target directive@\code{declare}~\code{target} directive}

Beginning with OpenMP 5.0, implementations are allowed to
offload back to the host (reverse offload).

In the example below the \plc{error\_handler} function
is executed back on the host, if an erroneous value is
detected in the \plc{A} array on the device.

This is accomplished by specifying the \plc{device-modifier}
\code{ancestor} modifier, along with a device number of \code{1},
to indicate that the execution is to be performed on the
immediate parent (\plc{1st ancestor})-- the host.

The \code{requires} directive (another 5.0 feature)
uses the \code{reverse\_offload} clause to guarantee
that the reverse offload is implemented.

Note that the \code{declare}~\code{target} directive uses the
\code{device\_type} clause (another 5.0 feature) to specify that
the \plc{error\_handler} function is compiled to
execute on the \plc{host} only. This ensures that no
attempt will be made to create a device version of the
function.  This feature may be necessary if the function
exists in another compile unit.


\cexample[5.2]{target_reverse_offload}{7}

\ffreeexample[5.0]{target_reverse_offload}{7}

\pagebreak
\section{\code{defaultmap} Clause}
\label{sec:defaultmap}
\index{target construct@\code{target} construct!defaultmap clause@\code{defaultmap} clause}
\index{clauses!defaultmap@\code{defaultmap}}
\index{defaultmap clause@\code{defaultmap} clause}
\index{defaultmap clause@\code{defaultmap} clause!implicit behavior}
\index{defaultmap clause@\code{defaultmap} clause!variable category}

The implicitly determined data-mapping and data-sharing attribute
rules of variables referenced in a \code{target} construct can be
changed by the \code{defaultmap} clause. 
As of OpenMP 5.0, the implicit behavior is specified as
\code{alloc}, \code{to}, \code{from}, \code{tofrom},
\code{firstprivate}, \code{none}, \code{default} or \code{present},
and is optionally applied to a variable category specified as \code{scalar}, \code{aggregate}, \code{allocatable},
or \code{pointer}. 

A referenced variable that is in a specified ``category`` is treated as having
the specified implicit behavior. In C/C++, \code{scalar} refers to
base-language scalar variables, except pointers.  In Fortran it refers to a
scalar variable, as defined by the base language, of intrinsic type but
excluding the character type. The \code{aggregate} category refers to arrays and
structures (which includes variables of any derived type and of character type for Fortran). Fortran
has the additional category of \code{allocatable} for variables that have the
allocatable attribute. The \code{pointer} category refers to pointers, which
for Fortran are variables that have the pointer attribute.

In the example below, the first \code{target} construct uses  \code{defaultmap}
clauses to set data-mapping and possibly data-sharing attributes that reproduce 
the default rules for implicitly determined data-mapping and data-sharing
attributes for variables in the construct. That is, if the \code{defaultmap}
clauses were removed, the results would be identical.

In the second \code{target} construct all implicit behavior is removed
by specifying the \code{none} implicit behavior in the \code{defaultmap} clause.
Hence, all variables that don't have predetermined attributes must be given an
explicit data-mapping or data-sharing attribute. A scalar (\texttt{s}), an
array (\texttt{A}) and a structure (\texttt{S} for the C/C++ example and
\texttt{D} for the Fortran example) are explicitly mapped with the
\code{tofrom} map type.

The third \code{target} construct shows another usual case for using the
\code{defaultmap} clause.  The default mapping for (non-pointer) scalar
variables is specified.  Here, the default implicit mapping for \texttt{s3} is
\code{tofrom} as specified in the \code{defaultmap} clause, while \texttt{s1}
and \texttt{s2} are instead explicitly treated as \code{firstprivate}.

In the fourth \code{target} construct all arrays and structures are given
\code{firstprivate} implicit behavior by default with the use of the
\code{aggregate} variable category.  For the Fortran example, the
\code{allocatable} category is used in a separate \code{defaultmap} clause to
specify default \code{firstprivate} implicit behavior for referenced
allocatable variables (in this case, \texttt{H}).
% (Common use cases for C/C++ heap storage can be found in
% \specref{sec:pointer_mapping}.)

\cexample[5.0]{target_defaultmap}{1}

\ffreeexample[5.0]{target_defaultmap}{1}

\pagebreak
\section{Pointer Mapping}
\label{sec:pointer_mapping}
\index{mapping!pointer}
\index{mapping!pointer attachment}
\index{pointer attachment}

Pointers that contain host addresses require that those addresses are translated to device addresses for them to be useful in the context of a device data environment. Broadly speaking, there are two scenarios where this is important.

The first scenario is where the pointer is mapped to the device data environment, such that references to the pointer inside a \code{target} region are to the corresponding pointer. Pointer attachment ensures that the corresponding pointer will contain a device address when all of the following conditions are true:
\begin{itemize}
 \item the pointer is mapped by directive $A$ to a device;
 \item a list item that uses the pointer as its base pointer (call it the \emph{pointee}) is mapped, to the same device, by directive $B$, which may be the same as $A$;
 \item the effect of directive $B$ is to create either the corresponding pointer or pointee in the device data environment of the device.
\end{itemize}

Given the above conditions, pointer attachment is initiated as a result of directive $B$ and subsequent references to the pointee list item in a target region that use the pointer will access the corresponding pointee. The corresponding pointer remains in this \emph{attached} state until it is removed from the device data environment.

The second scenario, which is only applicable for C/C++, is where the pointer is implicitly privatized inside a \code{target} construct when it appears as the base pointer to a list item on the construct and does not appear explicitly as a list item in a \code{map} clause, \code{is\_device\_ptr} clause, or data-sharing attribute clause. This scenario can be further split into two cases: the list item is a zero-length array section (e.g., \plc{p[:0]}) or it is not.

If it is a zero-length array section, this will trigger a runtime check on entry to the \code{target} region for a previously mapped list item where the value of the pointer falls within the range of its base address and ending address. If such a match is found the private pointer is initialized to the device address corresponding to the value of the original pointer, and otherwise it is initialized to NULL (or retains its original value if the \code{unified\_address} requirement is specified for that compilation unit).

If the list item (again, call it the \emph{pointee}) is not a zero-length array section, the private pointer will be initialized such that references in the \code{target} region to the pointee list item that use the pointer will access the corresponding pointee.

The following example shows the basics of mapping pointers with and without
associated storage on the host.

Storage for pointers \plc{ptr1} and \plc{ptr2} is created on the host. 
To map storage that is associated with a pointer on the host, the data can be
explicitly mapped as an array section so that the compiler knows 
the amount of data to be assigned in the device (to the ``corresponding'' data storage area).
On the \code{target} construct array sections are mapped; however, the pointer \plc{ptr1}
is mapped, while \plc{ptr2} is not. Since \plc{ptr2} is not explicitly mapped, it is
firstprivate.  This creates a subtle difference in the way these pointers can be used.

As a firstprivate pointer, \plc{ptr2} can be manipulated on the device;
however, as an explicitly mapped pointer, 
\plc{ptr1} becomes an \emph{attached} pointer and cannot be manipulated.
In both cases the host pointer is not updated with the device pointer 
address---as one would expect for distributed memory. 
The storage data on the host is updated from the corresponding device
data at the end of the \code{target} region.

As a comparison, note that the \plc{aray} array is automatically mapped, 
since the compiler knows the extent of the array. 

The pointer \plc{ptr3} is used inside the \code{target} construct, but it does
not appear in a data-mapping or data-sharing clause. Nor is there a
\code{defaultmap} clause on the construct to indicate what its implicit
data-mapping or data-sharing attribute should be. For such a case, \plc{ptr3}
will be implicitly privatized within the construct and there will be a runtime
check to see if the host memory to which it is pointing has corresponding memory
in the device data environment. If this runtime check passes, the private
\plc{ptr3} would be initialized to point to the corresponding memory. But in
this case the check does not pass and so it is initialized to null.
Since \plc{ptr3} is private, the value to which it is assigned in the
\code{target} region is not returned into the original \plc{ptr3} on the host.

\cexample[5.0]{target_ptr_map}{1}

\index{directives!begin declare target@\code{begin}~\code{declare}~\code{target}}
\index{begin declare target directive@\code{begin}~\code{declare}~\code{target} directive}

In the following example the global pointer \plc{p} appears in a
declare target directive.  Hence, the pointer \plc{p} will 
persist on the device throughout executions in all \code{target} regions.

The pointer is also used in an array section of a \code{map} clause on 
a \code{target} construct.  When the pointer of storage associated with 
a declare target directive
is mapped, as for the array section \plc{p[:N]} in the
\code{target} construct, the array section on the device is \emph{attached}
to the device pointer \plc{p} on entry to the construct, and
the value of the device pointer \plc{p} becomes undefined on exit. 
(Of course, storage allocation for
the array section on the device will occur before the 
pointer on the device is \emph{attached}.)
% For globals with declare target is there such a things a
% original and corresponding?

\cexample[5.1]{target_ptr_map}{2}

\index{directives!begin declare target@\code{begin}~\code{declare}~\code{target}}
\index{begin declare target directive@\code{begin}~\code{declare}~\code{target} directive}

The following two examples illustrate subtle differences in pointer attachment
to device address because of the order of data mapping.

In example \plc{target\_ptr\_map.3a} 
the global pointer \plc{p1} points to array \plc{x} and \plc{p2} points to
array \plc{y} on the host.
The array section \plc{x[:N]} is mapped by the \code{target}~\code{enter}~\code{data} directive while array \plc{y} is mapped
on the \code{target} construct. 
Since the \code{begin}~\code{declare}~\code{target} directive is applied to the declaration
of \plc{p1}, \plc{p1} is a treated like a mapped variable on the \code{target}
construct and references to \plc{p1} inside the construct will be to the
corresponding \plc{p1} that exists on the device.  However, the corresponding
\plc{p1} will be undefined since there is no pointer attachment for it. Pointer
attachment for \plc{p1} would require that (1) \plc{p1} (or an lvalue
expression that refers to the same storage as \plc{p1}) appears as a base
pointer to a list item in a \code{map} clause, and (2) the construct that has
the \code{map} clause causes the list item to transition from \emph{not mapped}
to \emph{mapped}. The conditions are clearly not satisfied for this example.

The problem for \plc{p2} in this example is also subtle. It will be privatized
inside the \code{target} construct, with a runtime check for whether the memory
to which it is pointing has corresponding memory that is accessible on the
device. If this check is successful, then the \plc{p2} inside the construct
would be appropriately initialized to point to that corresponding memory.
Unfortunately, despite there being an implicit map of the array \plc{y} (to
which \plc{p2} is pointing) on the construct, the order of this map relative to
the initialization of \plc{p2} is unspecified. Therefore, the initial value of
\plc{p2} will also be undefined.

Thus, referencing values via either \plc{p1} or \plc{p2} inside
the \code{target} region would be invalid.

\cexample[5.1]{target_ptr_map}{3a}

In example \plc{target\_ptr\_map.3b} the mapping orders for arrays \plc{x}
and \plc{y} were rearranged to allow proper pointer attachments.
On the \code{target} construct, the \code{map(x)} clause triggers pointer
attachment for \plc{p1} to the device address of \plc{x}. 
Pointer \plc{p2} is assigned the device address of the previously mapped
 array \plc{y}.
Referencing values via either \plc{p1} or \plc{p2} inside the \code{target} region is now valid.

\cexample[5.1]{target_ptr_map}{3b}
%\clearpage

\index{routines!omp_target_is_accessible@\scode{omp_target_is_accessible}}
\index{omp_target_is_accessible routine@\scode{omp_target_is_accessible} routine}
In the following example, storage allocated on the host is not mapped in a \code{target}
region if it is determined that the host memory is accessible from the device.
On platforms that support host memory access from a target device, 
it may be more efficient to omit map clauses and avoid the potential memory allocation 
and data transfers that may result from the map.
%For discrete memory storage on host and devices, explicit mapping may be required, whereas for
%Unified Shared Memory platforms it may be optimal to avoid using map clauses,
%because re-allocation of the space may occur when map clauses are present.
The \code{omp\_target\_is\_accessible} API routine is used to determine if the 
host storage of size \plc{buf\_size} is accessible on the device, and a metadirective
is used to select the directive variant (a \code{target} with/without a \code{map} clause).

The \code{omp\_target\_is\_accessible} routine will return true if the storage indicated 
by the first and second arguments is accessible on the target device. In this case, 
the host pointer \plc{ptr} may be directly dereferenced in the subsequent 
\code{target} region to access this storage, rather than mapping an array section based 
off the pointer. By explicitly specifying the host pointer in a \code{firstprivate} 
clause on the construct, its original value will be used directly in the \code{target} region. 
In OpenMP 5.1, removing the \code{firstprivate} clause will result in an implicit presence 
check of the storage to which \plc{ptr} points, and since this storage is not mapped by the 
program, \plc{ptr} will be NULL-initialized in the \code{target} region. 
In the next version of the OpenMP Specification, a false presence check without 
the \code{firstprivate} clause will cause the pointer to retain its original value. 

\cexample[5.2]{target_ptr_map}{4}

\index{mapping!deep copy}
Similar to the previous example, the \code{omp\_target\_is\_accessible} routine is used to
discover if a deep copy is required for the platform.  Here, the \plc{deep\_copy} map,
defined in the \code{declare}~\code{mapper} directive, is used if the host storage referenced by 
\plc{s.ptr} (or \plc{s\%ptr} in Fortran) is not accessible from the device.

\cexample[5.2]{target_ptr_map}{5}
\ffreeexample[5.2]{target_ptr_map}{5}

\pagebreak
\section{Structure Mapping}
\label{sec:structure_mapping}
\index{mapping!structure}

\index{directives!begin declare target@\code{begin}~\code{declare}~\code{target}}
\index{begin declare target directive@\code{begin}~\code{declare}~\code{target} directive}


%%%%%%%%%%%%%%%%%%%%%%%%%%%%%%%%%%%%%%%%%%%%%%%%%%%%%%%%%%%%%%%%%%%%%%%%%%%%%%%%%
In the example below, only structure elements \plc{S.a}, \plc{S.b} and \plc{S.p} 
of the \plc{S} structure appear in \code{map} clauses of a \code{target} construct.
Only these components have corresponding variables and storage on the device.  
Hence, the large arrays, \plc{S.buffera} and \plc{S.bufferb}, and the \plc{S.x} component have no storage 
on the device and cannot be accessed.  

Also, since the pointer member \plc{S.p} is used in an array section of a 
\code{map} clause, the array storage of the array section on the device, 
\plc{S.p[:N]}, is \emph{attached} to the pointer member \plc{S.p} on the device.
Explicitly mapping the pointer member \plc{S.p} is optional in this case.

Note: The buffer arrays and the \plc{x} variable have been grouped together, so that
the components that will reside on the device are all together (without gaps).
This allows the runtime to optimize the transfer and the storage footprint on the device.

\cexample[5.1]{target_struct_map}{1}


%%%%%%%%%%%%%%%%%%%%%%%%%%%%%%%%%%%%%%%%%%%%%%%%%%%%%%%%%%%%%%%%%%%%%%%%%%%%%%%%%
The following example is a slight modification of the above example for 
a C++ class.  In the member function \plc{SAXPY::driver} 
the array section \plc{p[:N]} is \emph{attached} to the pointer member \plc{p}
on the device.
 
\cppexample[5.1]{target_struct_map}{2}

%%%%%%%%%%%%%%%%%%%%%%%%%%%%%%%%%%%%%%%%%%%%%%%%%%%%%%%%%%%%%%%%%%%%%%%%%%%%%%%%%

%In this example a pointer, \plc{p}, is mapped in a 
%\code{target}~\code{data} construct (\code{map(p)}) and remains 
%persistent throughout the \code{target}~\code{data} region. The address stored
%on the host is not assigned to the device pointer variable, and 
%the device value is not copied back to the host at the end of the
%region (for a pointer, it is as though \code{map(alloc:p}) is effectively
%used).  The array section, \plc{p[:N]}, is mapped on both \code{target}
%constructs, and the pointer \plc{p} on the device is attached at the
%beginning and detached at the end of the regions to the newly created
%array section on the device.
%
%Also, in the following example the global variable, \plc{a}, becomes 
%allocated when it is first used on the device in a \code{target} region, 
%and persists on the device for all target regions.  The value on the
%device and host may be different, as shown by the print statements.
%The values may be made consistent with the \code{update} construct,
%as shown in the \plc{declare\_target.3.c} and \plc{declare\_target.3.f90} 
%examples.
%
%\cexample{target_struct_map}{2}

%%%%%%%%%%%%%%%%%%%%%%%%%%%%%%%%%%%%%%%%%%%%%%%%%%%%%%%%%%%%%%%%%%%%%%%%%%%%%%%%%

The next example shows two ways in which the structure may be
\emph{incorrectly} mapped.

In Case 1, the array section \plc{S1.p[:N]} is first mapped in an enclosing
\code{target}~\code{data} construct, and the \code{target} construct then
implicitly maps the structure \plc{S1}. The initial map of the array section
does not map the base pointer \plc{S1.p} -- it only maps the elements of the
array section.  Furthermore, the implicit map is not sufficient to ensure
pointer attachment for the structure member \plc{S1.p} (refer to the conditions
for pointer attachment described in Section~\ref{sec:pointer_mapping}).
Consequentially, the dereference operation \plc{S1.p[i]} in the call to
\plc{saxpyfun} will probably fail because \plc{S1.p} contains a host address.

In Case 2, again an array section is mapped on an enclosing
\code{target}~\code{data} construct. This time, the nested \code{target}
construct explicitly maps \plc{S2.p}, \plc{S2.a}, and \plc{S2.b}. But as in
Case 1, this does not satisfy the conditions for pointer attachment since the
construct must map a list item for which \splc{S2.p} is a base pointer, and it
must do so when the \splc{S2.p} is already present on the device or will be
created on the device as a result of the same construct.

 
\cexample[5.1]{target_struct_map}{3}

%%%%%%%%%%%%%%%%%%%%%%%%%%%%%%%%%%%%%%%%%%%%%%%%%%%%%%%%%%%%%%%%%%%%%%%%%%%%%%%%%

The following example correctly implements pointer attachment cases that
involve implicit structure maps.

In Case 1, members \splc{p}, \splc{a}, and \splc{b} of the structure \splc{S1}
are explicitly mapped by the \code{target}~\code{data} construct, to avoid
mapping parts of \splc{S1} that aren't required on the device. The mapped
\splc{S1.p} is attached to the array section \splc{S1.p[:N]}, and remains
attached while it exists on the device (for the duration of
\code{target}~\code{data} region).  Due to the \splc{S1} reference inside the
nested \code{target} construct, the construct implicitly maps \splc{S1} so that
the reference refers to the corresponding storage created by the enclosing
\code{target}~\code{data} region. Note that only the members \splc{a},
\splc{b}, and \splc{p} may be accessed from this storage.

In Case 2, only the storage for the array section \splc{S2.p[:N]} is mapped
by the \code{target}~\code{data} construct.  The nested \code{target}
construct explicitly maps \splc{S2.a} and \splc{S2.b} and explicitly
maps an array section for which \splc{S2.p} is a base pointer. This satisfies
the conditions for \splc{S2.p} becoming an attached pointer. The array
section in this case is zero-length, but the effect would be the same if the
length was a positive integer less than or equal to \splc{N}. There is also an
implicit map of the containing structure \splc{S2}, again due to the reference
to \splc{S2} inside the construct. The effect of this implicit map permits
access only to members \splc{a}, \splc{b}, and \splc{p}, as for Case 1. 

In Case 3, there is no \code{target}~\code{data} construct. The \code{target}
construct explicitly maps \splc{S3.a} and \splc{S3.b} and explicitly
maps an array section for which \splc{S3.p} is a base pointer. Again, there is
an implicit map of the structure referenced in the construct, \splc{S3}. This
implicit map also causes \splc{S3.p} to be implicitly mapped, because no other
part of \splc{S3} is present prior to the construct being encountered. The
result is an attached pointer \splc{S3.p} on the device. As for Cases 1 and 2,
this implicit map only ensures that storage for the members \splc{a}, \splc{b},
and \splc{p} are accessible within the corresponding \splc{S3} that is created
on the device.

\cexample[5.1]{target_struct_map}{4}


%\pagebreak
\section{Fortran Allocatable Array Mapping}
\label{sec:fort_allocatable_array_mapping}
\index{mapping!allocatable array, Fortran}


%%%%%%%%%%%%%%%%%%%%%%%%%%%%%%%%%%%%%%%%%%%%%%%%%%%%%%%%%%%%%%%%%%%%%%%%%%%%%%%%%
The following examples illustrate the use of Fortran allocatable arrays in \kcode{target} regions.

In the first example, allocatable variables (\ucode{a} and \ucode{b}) are first allocated
on the host, and then mapped onto a device in the Target 1 and 2 sections, respectively.
For \ucode{a} the map is implicit and for \ucode{b} an explicit map is used.
Both are mapped with the default \kcode{tofrom} map type.
The user-level behavior is similar to non-allocatable arrays.
However, the mapping operations include creation of the allocatable variable,
creation of the allocated storage, setting the allocation status to allocated,
and making sure the allocatable variable references the storage.

In Target 3 and 4 sections, allocatable variables are mapped in two
different ways before they are allocated on the host and subsequently used on the device.
In one case, a \kcode{target data} construct creates an enclosing region for
the allocatable variable to persist, and in the other case a
\kcode{declare target} directive maps the allocation variable for all device executions.
In both cases the new array storage is mapped \kcode{tofrom} with the \kcode{always} modifier.
An explicit map is used here with an \kcode{always} modifier to ensure that the allocatable
variable status is updated on the device.

Note: OpenMP 5.1 specifies that an \kcode{always} map modifier guarantees the
allocation status update for an existing allocatable variable on the device.
%In OpenMP 6.0, this restriction may be relaxed to also guarantee updates
%without the \kcode{always} modifier.

In Target 3 and 4 sections, the behavior of an allocatable variable is very
much like a Fortran pointer, in which a pointer can be mapped to a device with an associated
or disassociated status, and associated storage can be mapped and attached as needed.
For allocatable variables, the update of the allocation status to allocated (allowing
reference to allocated storage) on the device, is similar to pointer attachment.


\ffreeexample[5.1]{target_fort_allocatable_map}{1}

Once an allocatable variable has been allocated on the host,
its allocation status may not be changed in a \kcode{target} region, either
explicitly or implicitly. The following example illustrates typical
operations on allocatable variables that violate this restriction.
Note, an assignment that reshapes or reassigns (causing a deallocation
and allocation) in a \kcode{target} region is not conforming.
Also, an initial intrinsic assignment of an allocatable variable 
requires deallocation before the \kcode{target} region ends.

\ffreeexample[5.1]{target_fort_allocatable_map}{2}

\newpage
The next example illustrates a corner case of this restriction (allocatable status
change in a \kcode{target} region).
Two allocatable arrays are passed to a subroutine within a \kcode{target}
region. The dummy-variable arrays are declared \bcode{allocatable}.
Also, the \ucode{ain} variable has the \bcode{intent(in)} attribute, and \ucode{bout}
has the \bcode{intent(out)} attribute. 
For the dummy argument with the attributes \bcode{allocatable} and \bcode{intent(out)}, 
the compiler will deallocate the associated actual argument when the subroutine is invoked. 
(However, the allocation on procedure entry can be avoided by specifying the intent 
as \bcode{intent(inout)}, making the intended use conforming.)

\ffreeexample[5.1]{target_fort_allocatable_map}{3}

\pagebreak
\section{Array Sections in Device Constructs}
\label{sec:array_sections}
\index{array sections!in map clause@in \code{map} clause}
\index{map clause@\code{map} clause}
\index{clauses!map@\code{map}}
\index{target construct@\code{target} construct}
\index{constructs!target@\code{target}}
\index{target data construct@\code{target}~\code{data} construct}
\index{constructs!target data@\code{target}~\code{data}}

The following examples show the usage of array sections in \code{map} clauses 
on \code{target} and \code{target} \code{data} constructs.

This example shows the invalid usage of two separate sections of the same array 
inside of a \code{target} construct.

\cexample[4.0]{array_sections}{1}

\ffreeexample[4.0]{array_sections}{1}

\pagebreak
This example shows the invalid usage of two separate sections of the same array 
inside of a \code{target} construct.

\cexample[4.0]{array_sections}{2}

\ffreeexample[4.0]{array_sections}{2}

\pagebreak
This example shows the valid usage of two separate sections of the same array inside 
of a \code{target} construct.

\cexample[4.0]{array_sections}{3}

\ffreeexample[4.0]{array_sections}{3}

\pagebreak
This example shows the valid usage of a wholly contained array section of an already 
mapped array section inside of a \code{target} construct.

\cexample[4.0]{array_sections}{4}

\ffreeexample[4.0]{array_sections}{4}


\pagebreak
\section{Unified Shared Memory}
\label{sec:usm}

\index{requires directive@\kcode{requires} directive}
\index{directives!requires@\kcode{requires}}

\index{unified_shared_memory clause@\kcode{unified_shared_memory} clause}
\index{clauses!unified_shared_memory@\kcode{unified_shared_memory}}

The following examples show behavior of scalars, pointers, references (C++) and 
associate names (Fortran) in \kcode{target} constructs when
unified shared memory (USM) is required throughout the scope of the program by the 
\kcode{unified_shared_memory} clause in a \kcode{requires} directive.
USM assumes a unified address space.

\begin{cppspecific}
In the C++ code of the first example, a scalar (\ucode{x}), a pointer 
(\ucode{ptr}), and a reference (\ucode{ref}) are used in a \kcode{target} construct in Cases 1, 2 and 3, respectively.
For the scalar variable \ucode{x}, the predetermined data-sharing attribute is still
firstprivate under the USM requirement and, hence, any manipulation of the local variable on the device is never reflected on the host.
With the USM requirement, pointers always refer to the same location in memory on the host and devices.
Hence, the value of \ucode{x} (in the host data environment) can be modified with \ucode{ptr} in 
the \kcode{target} construct, as seen in Case 2. 
For the reference \ucode{ref}, the object to which it refers is mapped for  
the \kcode{target} construct, as seen in Case 3. 

\cppnexample[5.2]{usm_scalar_ptr_ref_asc}{1}
\end{cppspecific}

\begin{fortranspecific}
In Case 1 of the Fortran example, the scalar \ucode{x} is firstprivate under the USM requirement
in the \kcode{target} construct, and modification of the local variable on the device is 
never updated to the host data environment.
Also, in Case 2, the use of \ucode{ax}, which is associated with \ucode{x}, 
will update the \ucode{x} value in the host data environment.
In Case 3, the Fortran pointer \ucode{ptr} and its target \ucode{y} are not firstprivate, 
but implicitly mapped.  Hence, updates to the value of \ucode{y} appear in the host data environment.

%Hence, updates to \ucode{y} in the \kcode{target} construct appear in the data environment of the host.

%\pagebreak
\ffreenexample[5.2]{usm_scalar_ptr_ref_asc}{1}
\end{fortranspecific}

%\pagebreak
\begin{cppspecific}[4ex]
\section{C++ Virtual Functions}
\label{sec:virtual_functions}

\index{mapping!virtual functions, C++}

The 5.2 OpenMP Specification clarified restrictions on the use 
of polymorphic classes and virtual functions when used within 
\kcode{target} regions. The following examples illustrate 
use cases and the limitations imposed by restrictions for
references and the use of Unified Shared Memory. 

The first example illustrates two simple cases of using a virtual function
through a pointer and reference without Unified Shared Memory. 

A class, \ucode{D}, is derived from class \ucode{A}.  
Function \ucode{vf} in class \ucode{A} is declared
virtual, and the function \ucode{vf} in class \ucode{D} is declared with override.
An object, \ucode{d} of type \ucode{D}, is created and mapped through a \kcode{target data}
directive.

In the first case, a pointer of type \ucode{A}, \ucode{ap}, is assigned to point 
to the derived object \ucode{d}, and in the first \kcode{target} region the pointer
is used to call the \ucode{vf} function of the derived class, \ucode{D}.

In the second case, the reference variable \ucode{ar} of type \ucode{A} references 
the derived object \ucode{d}).  The use of the reference variable \ucode{ar} 
in the \kcode{target} region is illegal due to the restriction that the static and dynamic 
types must match when mapping an object for the first time.
That is, the behavior of the implicit map of \ucode{ar} 
is non-conforming -- its static type \ucode{A} doesn't match its dynamic type \ucode{D}.  
Hence the behavior of the access to the virtual functions is unspecified.

\topmarker{C++}
\cppnexample[5.2]{virtual_functions}{1}

The second example illustrates the restriction:

\emph{``Invoking a virtual member function of an object on a device other than the device on which the
object was constructed results in unspecified behavior, unless the object is accessible and was
constructed on the host device.''}

In the first case, an instantiation \ucode{ap} of a polymorphic class (\ucode{A}) occurs in the 
\kcode{target} region, and access of its virtual function
is incorrectly attempted on the host (another device).

In the second case, the object \ucode{ap} is instantiated on the host; access of \ucode{ap} within
the next \kcode{target} region is permitted. (Unified Shared Memory is
used here to minimize mapping concerns.)

\topmarker{C++}
\cppnexample[5.2]{virtual_functions}{2}
\end{cppspecific}

\section{Array Shaping}
\label{sec:array-shaping}
\index{array shaping!in motion-clause@in \plc{motion-clause}}
\index{constructs!target update@\kcode{target update}}
\index{target update construct@\kcode{target update} construct!to clause@\kcode{to} clause}
\index{target update construct@\kcode{target update} construct!from clause@\kcode{from} clause}

\index{directives!declare target@\kcode{declare target}}
\index{declare target directive@\kcode{declare target} directive}

\index{directives!begin declare target@\kcode{begin declare target}}
\index{begin declare target directive@\kcode{begin declare target} directive}

\begin{ccppspecific}
A pointer variable can be shaped to a multi-dimensional array to facilitate
data access. This is achieved by a \plc{shape-operator} casted in front of 
a pointer (lvalue expression):
\begin{description}
\item[]\hspace*{5mm}\code{([$s_1$][$s_2$]...[$s_n$])}\plc{pointer}
\end{description}
where each $s_i$ is an integral-type expression of positive value.
The shape-operator can appear in either the \plc{motion-clause}
of the \kcode{target update} directive or the \kcode{depend} clause.

The following example shows the use of the shape-operator in the 
\kcode{target update} directive. The shape-operator \ucode{([nx][ny+2])}
casts pointer variable \ucode{a} to a 2-dimensional array of size
%\ucode{nx}$\times$\ucode{(ny+2)}.  The resulting array is then accessed as
\ucode{nx}*\ucode{(ny+2)}.  The resulting array is then accessed as
array sections (such as \ucode{[0:nx][1]} and \ucode{[0:nx][ny]}) 
in the \kcode{from} or \kcode{to} clause for transferring two columns of 
noncontiguous boundary data from or to the device.  
Note the use of additional parentheses
around the shape-operator and \ucode{a} to ensure the correct precedence 
over array-section operations.

\cnexample[5.1]{array_shaping}{1}
\end{ccppspecific}
%\clearpage

\begin{fortranspecific}
The shape operator is not defined for Fortran.  Explicit array shaping
of procedure arguments can be used instead to achieve a similar goal.
Below is the Fortran equivalent of the above example that illustrates
the support of transferring two rows of noncontiguous boundary
data in the \kcode{target update} directive.
 
\ffreenexample[5.2]{array_shaping}{1}
\end{fortranspecific}

\pagebreak
\section{\kcode{declare mapper} Directive}
\label{sec:declare_mapper}
\index{directives!declare mapper@\kcode{declare mapper}}
\index{declare mapper directive@\kcode{declare mapper} directive}

The following examples show how to use the \kcode{declare mapper}
directive to prescribe a map for later use.
It is also quite useful for pre-defining partitioned and nested 
structure elements.

In the first example the \kcode{declare mapper} directive specifies 
that any structure of type \ucode{myvec_t} for which implicit data-mapping
rules apply will be mapped according to its \kcode{map} clause.
The variable \ucode{v} is used for referencing the structure and its 
elements within the \kcode{map} clause. 
Within the \kcode{map} clause the \ucode{v} variable specifies that all
elements of the structure are to be mapped.  Additionally, the
array section \ucode{v.data[0:v.len]} specifies that the dynamic 
storage for data is to be mapped. 

Within the main program the \ucode{s} variable is typed as \ucode{myvec_t}.
Since the variable is found within the \kcode{target} region and the type has a mapping prescribed by
a \kcode{declare mapper} directive, it will be automatically mapped according to its prescription: 
full structure, plus the dynamic storage of the \ucode{data} element. 

%Note: By default the mapping is \kcode{tofrom}. 
%The associated Fortran allocatable \ucode{data} array is automatically mapped with the derived
%type, it does not require an array section as in the C/C++ example.

\cexample[5.0]{target_mapper}{1}

\ffreeexample[5.0]{target_mapper}{1}

%\pagebreak
\index{mapping!deep copy}
\index{map clause@\kcode{map} clause!mapper modifier@\kcode{mapper} modifier}
\index{mapper modifier@\kcode{mapper} modifier}
The next example illustrates the use of the \plc{mapper-identifier} and deep copy within a structure. 
The structure, \ucode{dzmat_t},  represents a complex matrix, 
with separate real (\ucode{r_m}) and imaginary (\ucode{i_m}) elements.
Two map identifiers are created for partitioning the \ucode{dzmat_t} structure.

For the C/C++ code the first identifier is named \ucode{top_id} and maps the top half of
two matrices of type \ucode{dzmat_t}; while the second identifier, \ucode{bottom_id},
maps the lower half of two matrices. 
Each identifier is applied to a different \kcode{target} construct,
as  \kcode{map(mapper(\ucode{top_id}), tofrom: \ucode{a,b})} 
and \kcode{map(mapper(\ucode{bottom_id}), tofrom: \ucode{a,b})}.
Each target offload is allowed to execute concurrently on two different devices 
(\ucode{0} and \ucode{1}) through the \kcode{nowait} clause.
%The OpenMP 5.1 \kcode{parallel masked} construct creates a region of two threads
%for these \kcode{target} constructs, with a single thread (\plc{primary}) generator.

The Fortran code uses the \ucode{left_id} and \ucode{right_id} map identifiers in the
\kcode{map(mapper(\ucode{left_id}),tofrom: \ucode{a,b})} and \kcode{map(mapper(\ucode{right_id}),tofrom: \ucode{a,b})} map clauses.  
The array sections for these left and right contiguous portions of the matrices 
were defined previously in the \kcode{declare mapper} directive.

Note, the \ucode{is} and \ucode{ie} scalars are firstprivate 
by default for a \kcode{target} region, but are declared firstprivate anyway
to remind the user of important firstprivate data-sharing properties required here.

\cexample[5.0]{target_mapper}{2}

\ffreeexample[5.0]{target_mapper}{2}

%\pagebreak
In the third example \ucode{myvec} structures are
nested within a \ucode{mypoints} structure. The \ucode{myvec_t} type is mapped
as in the first example.  Following the \ucode{mypoints} structure declaration, 
the \ucode{mypoints_t} type is mapped by a \kcode{declare mapper} directive. 
For this structure the \ucode{hostonly_data} element will not be mapped;
also the array section of \ucode{x} (\ucode{v.x[:1]}) and \ucode{x} will be mapped; and
\ucode{scratch} will be allocated and used as scratch storage on the device.
The default map-type mapping, \kcode{tofrom}, applies to the \ucode{x} array section,
but not to \ucode{scratch} which is explicitly mapped with the \kcode{alloc} map-type. 
Note: the variable \ucode{v} is not included in the map list (otherwise
the \ucode{hostonly_data} would be mapped)-- just the elements 
to be mapped are listed.

The two mappers are combined when a \ucode{mypoints_t} structure type is mapped,
because the mapper \ucode{myvec_t} structure type is used within a \ucode{mypoints_t}
type structure.
%Note, in the main program \ucode{P} is an array of \ucode{mypoints_t} type structures, 
%and hence every element of the array is mapped with the mapper prescription.

\cexample[5.0]{target_mapper}{3}

\ffreeexample[5.0]{target_mapper}{3}


\pagebreak
\section{\code{target} \code{data} Construct}
\label{sec:target_data}

\subsection{Simple \code{target} \code{data} Construct}
\label{subsec:target_data_simple}
\index{constructs!target data@\code{target}~\code{data}}
\index{target data construct@\code{target}~\code{data} construct}

This example shows how the \code{target} \code{data} construct maps variables 
to a device data environment. The \code{target} \code{data} construct creates 
a new device data environment and maps the variables \plc{v1}, \plc{v2}, and \plc{p} to the new device 
data environment. The \code{target} construct enclosed in the \code{target} 
\code{data} region creates a new device data environment, which inherits the 
variables \plc{v1}, \plc{v2}, and \plc{p} from the enclosing device data environment. The variable 
\plc{N} is mapped into the new device data environment from the encountering task's data 
environment.

\cexample[4.0]{target_data}{1}

\pagebreak
The Fortran code passes a reference and specifies the extent of the arrays in the 
declaration. No length information is necessary in the map clause, as is required 
with C/C++ pointers.

\ffreeexample[4.0]{target_data}{1}

\subsection{\code{target} \code{data} Region Enclosing Multiple \code{target} Regions}
\label{subsec:target_data_multiregion}

The following examples show how the \code{target} \code{data} construct maps 
variables to a device data environment of a \code{target} region. The \code{target} 
\code{data} construct creates a device data environment and encloses \code{target} 
regions, which have their own device data environments. The device data environment 
of the \code{target} \code{data} region is inherited by the device data environment 
of an enclosed \code{target} region. The \code{target} \code{data} construct 
is used to create variables that will persist throughout the \code{target} \code{data} 
region.

In the following example the variables \plc{v1} and \plc{v2} are mapped at each \code{target} 
construct. Instead of mapping the variable \plc{p} twice, once at each \code{target} 
construct, \plc{p} is mapped once by the \code{target} \code{data} construct.

\cexample[4.0]{target_data}{2}


The Fortran code uses reference and specifies the extent of the \plc{p}, \plc{v1} and \plc{v2} arrays. 
No length information is necessary in the \code{map} clause, as is required with 
C/C++ pointers. The arrays \plc{v1} and \plc{v2} are mapped at each \code{target} construct. 
Instead of mapping the array \plc{p} twice, once at each target construct, \plc{p} is mapped 
once by the \code{target} \code{data} construct.

\ffreeexample[4.0]{target_data}{2}

\index{target data construct@\code{target}~\code{data} construct!map clause@\code{map} clause}
\index{target construct@\code{target} construct!map clause@\code{map} clause}
\index{target construct@\code{target} construct!implicit mapping}
\index{map clause@\code{map} clause!tofrom map-type@\code{tofrom} map-type}
In the following example, the array \plc{Q} is mapped once at the enclosing 
\code{target}~\code{data} region instead of at each \code{target} construct. 
In OpenMP 4.0, a scalar variable is implicitly mapped with the \code{tofrom} map-type.
But since OpenMP 4.5, a scalar variable, such as the \plc{tmp} variable, has to be explicitly mapped with 
the \code{tofrom} map-type at the first \code{target} construct in order to return 
its reduced value from the parallel loop construct to the host.
The variable defaults to firstprivate at the second \code{target} construct.

\cexample[4.0]{target_data}{3}

\ffreeexample[4.0]{target_data}{3}

\subsection{\code{target} \code{data} Construct with Orphaned Call}

The following two examples show how the \code{target} \code{data} construct 
maps variables to a device data environment. The \code{target} \code{data} 
construct's device data environment encloses the \code{target} construct's device 
data environment in the function \code{vec\_mult()}.

\index{map clause@\code{map} clause!alloc map-type@\code{alloc} map-type}
When the type of the variable appearing in an array section is pointer, the pointer 
variable and the storage location of the corresponding array section are mapped 
to the device data environment. The pointer variable is treated as if it had appeared 
in a \code{map} clause with a map-type of \code{alloc}. The array section's 
storage location is mapped according to the map-type in the \code{map} clause 
(the default map-type is \code{tofrom}).

The \code{target} construct's device data environment inherits the storage locations 
of the array sections \plc{v1[0:N]}, \plc{v2[:n]}, and \plc{p0[0:N]} from the enclosing \code{target}~\code{data}
construct's device data environment. Neither initialization nor assignment is performed 
for the array sections in the new device data environment.

The pointer variables \plc{p1}, \plc{v3}, and \plc{v4} are mapped into the \code{target} construct's device 
data environment with an implicit map-type of alloc and they are assigned the address 
of the storage location associated with their corresponding array sections. Note 
that the following pairs of array section storage locations are equivalent (\plc{p0[:N]}, 
\plc{p1[:N]}), (\plc{v1[:N]},\plc{v3[:N]}), and (\plc{v2[:N]},\plc{v4[:N]}).

\cexample[4.0]{target_data}{4}

The Fortran code maps the pointers and storage in an identical manner (same extent, 
but uses indices from 1 to \plc{N}).

The \code{target} construct's device data environment inherits the storage locations 
of the arrays \plc{v1}, \plc{v2} and \plc{p0} from the enclosing \code{target} \code{data} constructs's 
device data environment. However, in Fortran the associated data of the pointer 
is known, and the shape is not required.

The pointer variables \plc{p1}, \plc{v3}, and \plc{v4} are mapped into the \code{target} construct's 
device data environment with an implicit map-type of \code{alloc} and they are 
assigned the address of the storage location associated with their corresponding 
array sections. Note that the following pair of array storage locations are equivalent 
(\plc{p0},\plc{p1}), (\plc{v1},\plc{v3}), and (\plc{v2},\plc{v4}).

\ffreeexample[4.0]{target_data}{4}


In the following example, the variables \plc{p1}, \plc{v3}, and \plc{v4} are references to the pointer 
variables \plc{p0}, \plc{v1} and \plc{v2} respectively. The \code{target} construct's device data 
environment inherits the pointer variables \plc{p0}, \plc{v1}, and \plc{v2} from the enclosing \code{target} 
\code{data} construct's device data environment. Thus, \plc{p1}, \plc{v3}, and \plc{v4} are already 
present in the device data environment.

\cppexample[4.0]{target_data}{5}

In the following example, the usual Fortran approach is used for dynamic memory. 
The \plc{p0}, \plc{v1}, and \plc{v2} arrays are allocated in the main program and passed as references 
from one routine to another. In \code{vec\_mult}, \plc{p1}, \plc{v3} and \plc{v4} are references to the 
\plc{p0}, \plc{v1}, and \plc{v2} arrays, respectively. The \code{target} construct's device data 
environment inherits the arrays \plc{p0}, \plc{v1}, and \plc{v2} from the enclosing target data construct's 
device data environment. Thus, \plc{p1}, \plc{v3}, and \plc{v4} are already present in the device 
data environment.

\ffreeexample[4.0]{target_data}{5}

\subsection{\code{target} \code{data} Construct with \code{if} Clause}
\label{subsec:target_data_if}
\index{target data construct@\code{target}~\code{data} construct!if clause@\code{if} clause}
\index{clauses!if@\code{if}}
\index{if clause@\code{if} clause}

The following two examples show how the \code{target} \code{data} construct 
maps variables to a device data environment.

In the following example, the if clause on the \code{target} \code{data} construct 
indicates that if the variable \plc{N} is smaller than a given threshold, then the \code{target} 
\code{data} construct will not create a device data environment.

The \code{target} constructs enclosed in the \code{target} \code{data} region 
must also use an \code{if} clause on the same condition, otherwise the pointer 
variable \plc{p} is implicitly mapped with a map-type of \code{tofrom}, but the storage 
location for the array section \plc{p[0:N]} will not be mapped in the device data environments 
of the \code{target} constructs.

\cexample[4.0]{target_data}{6}

\pagebreak
The \code{if} clauses work the same way for the following Fortran code. The \code{target} 
constructs enclosed in the \code{target} \code{data} region should also use 
an \code{if} clause with the same condition, so that the \code{target} \code{data} 
region and the \code{target} region are either both created for the device, or 
are both ignored.

\ffreeexample[4.0]{target_data}{6}

\pagebreak
In the following example, when the \code{if} clause conditional expression on 
the \code{target} construct evaluates to \plc{false}, the target region will 
execute on the host device. However, the \code{target} \code{data} construct 
created an enclosing device data environment that mapped \plc{p[0:N]} to a device data 
environment on the default device. At the end of the \code{target} \code{data} 
region the array section \plc{p[0:N]} will be assigned from the device data environment 
to the corresponding variable in the data environment of the task that encountered 
the \code{target} \code{data} construct, resulting in undefined values in \plc{p[0:N]}.

\cexample[4.0]{target_data}{7}

\pagebreak
The \code{if} clauses work the same way for the following Fortran code. When 
the \code{if} clause conditional expression on the \code{target} construct 
evaluates to \plc{false}, the \code{target} region will execute on the host 
device. However, the \code{target} \code{data} construct created an enclosing 
device data environment that mapped the \plc{p} array (and \plc{v1} and \plc{v2}) to a device data 
environment on the default target device. At the end of the \code{target} \code{data} 
region the \plc{p} array will be assigned from the device data environment to the corresponding 
variable in the data environment of the task that encountered the \code{target} 
\code{data} construct, resulting in undefined values in \plc{p}.

\ffreeexample[4.0]{target_data}{7}


%\pagebreak
\section{\kcode{target enter data} and \kcode{target exit data} Constructs}
\label{sec:target_enter_exit_data}
%\section{Simple target enter data and target exit data Constructs}
\index{constructs!target enter data@\kcode{target enter data}}
\index{constructs!target exit data@\kcode{target exit data}}
\index{target enter data construct@\kcode{target enter data} construct}
\index{target exit data construct@\kcode{target exit data} construct}

The structured data construct (\kcode{target data}) provides persistent data on a
device for subsequent \kcode{target} constructs as shown in the 
\kcode{target data} examples above. This is accomplished by creating a single
\kcode{target data} region containing \kcode{target} constructs.

The unstructured data constructs allow the creation and deletion of data on
the device at any appropriate point within the host code, as shown below 
with the \kcode{target enter data} and \kcode{target exit data} constructs.

\index{map clause@\kcode{map} clause!alloc map-type@\kcode{alloc} map-type}
\index{map clause@\kcode{map} clause!delete map-type@\kcode{delete} map-type}
\index{alloc map-type@\kcode{alloc} map-type}
\index{delete map-type@\kcode{delete} map-type}
The following C++ code creates/deletes a vector in a constructor/destructor 
of a class. The constructor creates a vector with \kcode{target enter data}
and uses an \kcode{alloc} modifier in the \kcode{map} clause to avoid copying values
to the device. The destructor deletes the data (\kcode{target exit data})
and uses the \kcode{delete} modifier in the \kcode{map} clause to avoid copying data
back to the host. Note, the stand-alone \kcode{target enter data} occurs 
after the host vector is created, and the \kcode{target exit data}
construct occurs before the host data is deleted.

\cppexample[4.5]{target_unstructured_data}{1}

%\pagebreak
The following C code allocates and frees the data member of a \ucode{Matrix} structure.
The \ucode{init_matrix} function allocates the memory used in the structure and
uses the \kcode{target enter data} directive to map it to the target device. The
\ucode{free_matrix} function removes the mapped array from the target device
and then frees the memory on the host.  Note, the stand-alone 
\kcode{target enter data} occurs after the host memory is allocated, and the 
\kcode{target exit data} construct occurs before the host data is freed.

\cexample[4.5]{target_unstructured_data}{1}

%\pagebreak
The following Fortran code allocates and deallocates a module array, \ucode{A}.  The
\ucode{initialize} subroutine allocates the module array and uses the
\kcode{target enter data} directive to map it to the target device. The
\ucode{finalize} subroutine removes the mapped array from the target device and
then deallocates the array on the host.  Note, the stand-alone 
\kcode{target enter data} occurs after the host memory is allocated, and the 
\kcode{target exit data} construct occurs before the host data is deallocated.

\ffreeexample[4.5]{target_unstructured_data}{1}
%end


%\pagebreak
\section{\kcode{target update} Construct}
\label{sec:target_update}

\subsection{Simple \kcode{target data} and \kcode{target update} Constructs}
\label{subsec:target_data_and_update}
\index{constructs!target data@\kcode{target data}}
\index{target data construct@\kcode{target data} construct}
\index{constructs!target update@\kcode{target update}}
\index{target update construct@\kcode{target update} construct}
\index{target update construct@\kcode{target update} construct!to clause@\kcode{to} clause}
\index{target update construct@\kcode{target update} construct!from clause@\kcode{from} clause}
\index{target update construct@\kcode{target update} construct!motion-clause@\plc{motion-clause}}
\index{clauses!motion-clause@\plc{motion-clause}}
\index{clauses!to@\kcode{to}}
\index{clauses!from@\kcode{from}}
\index{motion-clause@\plc{motion-clause}!to clause@\kcode{to} clause}
\index{motion-clause@\plc{motion-clause}!from clause@\kcode{from} clause}

The following example shows how the \kcode{target update} construct updates 
variables in a device data environment.

The \kcode{target data} construct maps array sections \ucode{v1[:N]} and \ucode{v2[:N]} 
(arrays \ucode{v1} and \ucode{v2} in the Fortran code) into a device data environment.

The task executing on the host device encounters the first \kcode{target} region 
and waits for the completion of the region.

After the execution of the first \kcode{target} region, the task executing on 
the host device then assigns new values to \ucode{v1[:N]} and \ucode{v2[:N]} (\ucode{v1} and \ucode{v2} arrays 
in Fortran code) in the task's data environment by calling the function \ucode{init_again()}.

The \kcode{target update} construct assigns the new values of \ucode{v1} and 
\ucode{v2} from the task's data environment to the corresponding mapped array sections 
in the device data environment of the \kcode{target data} construct.

The task executing on the host device then encounters the second \kcode{target} 
region and waits for the completion of the region.

The second \kcode{target} region uses the updated values of \ucode{v1[:N]} and \ucode{v2[:N]}.

\cexample[4.0]{target_update}{1}

\ffreeexample[4.0]{target_update}{1}

\subsection{\kcode{target update} Construct with \kcode{if} Clause}
\label{subsec:target_update_if}
\index{target update construct@\kcode{target update} construct!if clause@\kcode{if} clause}
\index{clauses!if@\kcode{if}}
\index{if clause@\kcode{if} clause}

The following example shows how the \kcode{target update} construct updates 
variables in a device data environment.

The \kcode{target data} construct maps array sections \ucode{v1[:N]} and \ucode{v2[:N]} 
(arrays \ucode{v1} and \ucode{v2} in the Fortran code) into a device data environment. In between 
the two \kcode{target} regions, the task executing on the host device conditionally 
assigns new values to \ucode{v1} and \ucode{v2} in the task's data environment. The function \ucode{maybe_init_again()} 
returns \vcode{true} if new data is written.

When the conditional expression (the return value of \ucode{maybe_init_again()}) in the 
\kcode{if} clause is \plc{true}, the \kcode{target update} construct 
assigns the new values of \ucode{v1} and \ucode{v2} from the task's data environment to the corresponding 
mapped array sections in the \kcode{target data} construct's device data 
environment.

\cexample[4.0]{target_update}{2}

\ffreeexample[4.0]{target_update}{2}

\subsection{\kcode{target update} Construct with Mapper}
\label{subsec:target_update_mapper}
\index{target update construct@\kcode{target update} construct!modifier@mapper}

The following example shows how the \kcode{target update} construct can be used with a \kcode{mapper} (\ucode{custom}). 
The \ucode{custom} mapper maps members of structure \ucode{T} with different map-type modifiers. Inside the 
\kcode{target data} region the \kcode{target update} with the \kcode{to} \plc{motion-clause} is equivalent to an update of \ucode{x} on the device. After the \kcode{target} region the \kcode{target update} with the \kcode{from} motion-clause is equivalent to an update of \ucode{y} on the host.

\cexample[5.1]{target_update}{3}

\ffreeexample[5.1]{target_update}{3}


\pagebreak
\section{Declare Target Directive}
\label{sec:declare_target}

%\index{declare target directive@\code{declare}~\code{target} directive!enter clause@\code{enter} clause}
%\index{enter clause@\code{enter} clause}
%\index{clauses!enter@\code{enter}}

\subsection{Declare Target Directive for a Procedure}

\label{subsec:declare_target_function}

\index{directives!declare target@\code{declare}~\code{target}}
\index{declare target directive@\code{declare}~\code{target} directive}

\index{directives!begin declare target@\code{begin}~\code{declare}~\code{target}}
\index{begin declare target directive@\code{begin}~\code{declare}~\code{target} directive}

The following example shows how the declare target directive 
is used to indicate that the corresponding call inside a \code{target} region 
is to a \code{fib} function that can execute on the default target device.

A version of the function is also available on the host device. When the \code{if} 
clause conditional expression on the \code{target} construct evaluates to \plc{false}, 
the \code{target} region (thus \code{fib}) will execute on the host device.

For the following C/C++ code the declaration of the function \code{fib} appears between the 
\code{begin}~\code{declare}~\code{target} and \code{end}~\code{declare}~\code{target} directives. 
In the corresponding Fortran code, the \code{declare}~\code{target} directive appears at the
end of the specification part of the subroutine.

\cexample[5.1]{declare_target}{1}

The Fortran \code{fib} subroutine contains a \code{declare} \code{target} declaration 
to indicate to the compiler to create an device executable version of the procedure. 
The subroutine name has not been included on the \code{declare} \code{target} 
directive and is, therefore, implicitly assumed.

The program uses the \code{module\_fib} module, which presents an explicit interface to 
the compiler with the \code{declare} \code{target} declarations for processing 
the \code{fib} call.

\ffreeexample[4.0]{declare_target}{1}

The next Fortran example shows the use of an external subroutine. As the subroutine
is neither use associated nor an internal procedure, the \code{declare} \code{target}
declarations within a external subroutine are unknown to the main program unit; 
therefore, a \code{declare} \code{target} must be provided within the program
scope for the compiler to determine that a target binary should be available.

\ffreeexample[4.0]{declare_target}{2}

\subsection{Declare Target Directive for Class Type}
\label{subsec:declare_target_class}

\index{directives!begin declare target@\code{begin}~\code{declare}~\code{target}}
\index{begin declare target directive@\code{begin}~\code{declare}~\code{target} directive}

The following example shows the use of the \code{begin}~\code{declare}~\code{target}
and \code{end}~\code{declare}~\code{target} pair to designate the beginning and
end of the affected declarations, as introduced in OpenMP 5.1.
The \code{begin}~\code{declare}~\code{target} directive was defined
to symmetrically complement the terminating (``end'') directive.

\cppspecificstart

The example also shows 3 different ways to use a declare target directive for a 
class and an external member-function definition (for the \plc{XOR1}, \plc{XOR2}, 
and \plc{XOR3} classes and definitions for their corresponding \plc{foo} member functions).

For \plc{XOR1}, a \code{begin}~\code{declare}~\code{target} and 
\code{end}~\code{declare}~\code{target} directive
enclose both the class and its member function definition. The compiler immediately
knows to create a device version of the function for execution in a \code{target} region.

For \plc{XOR2}, the class member function definition is not specified with a
declare target directive.
An implicit declare target is created for the member function definition.
The same applies if this declaration arrangement for the class and function 
are included through a header file.

For \plc{XOR3}, the class and its member function are not enclosed by \code{begin}~\code{declare}~\code{target}
and \code{end}~\code{declare}~\code{target} directives,
but there is an implicit declare target since the class, its function
and the \code{target} construct are in the same file scope. That is, the class
and its function are treated as if delimited by a declare target directive.
The same applies if the class and function are included through a header file.

\cppnexample[5.1]{declare_target}{2a}

% blue line floater at top of this page for "C++, cont."
\begin{figure}[t!]
\linewitharrows{-1}{dashed}{C++ (cont.)}{8em}
\end{figure}

Often class definitions and their function definitions are included in separate files,
as shown in \splc{declare_target.2b_classes.hpp} and \splc{declare_target.2b_functions.cpp} below.
In this case, it is necessary to specify in a declare target directive for the classes.
However, as long as the \splc{2b_functions.cpp} file includes the corresponding declare target classes,
there is no need to specify the functions with a declare target directive.
The functions are treated as if they are specified with a declare target directive.
Compiling the \splc{declare_target.2b_functions.cpp} and \splc{declare_target.2b_main.cpp} files 
separately and linking them, will create appropriate executable device functions for the target device.

\srcnexample[5.1]{declare_target}{2b_classes}{hpp}
\smallskip
\cppnexample[5.1]{declare_target}{2b_functions}
\smallskip
\cppnexample[5.1]{declare_target}{2b_main}

%\cppspecificend
% blue line floater at top of this page for "C++, cont."
\begin{figure}[t!]
\linewitharrows{-1}{dashed}{C++ (cont.)}{8em}
\end{figure}

%\cppspecificstart
The following example shows how the \code{begin}~\code{declare} \code{target} and \code{end} 
\code{declare} \code{target} directives are used to enclose the declaration 
of a variable \plc{varY} with a class type \code{typeY}. 
%Prior to OpenMP 5.0, the member function \code{typeY::foo()} cannot 
%be accessed on a target device because its declaration did not appear between \code{begin}~\code{declare} 
%\code{target} and \code{end} \code{declare} \code{target} directives.

This example shows pre-OpenMP 5.0 behavior for the \plc{varY.foo()} function call (an error).
The member function \code{typeY::foo()} cannot be accessed on a target device because its 
declaration does not appear between \code{begin}~\code{declare}~\code{target} and 
\code{end}~\code{declare}~\code{target} directives. As of OpenMP 5.0, the
function is implicitly declared with a declare target directive 
and will successfully execute the function on the device.  See previous examples.
%as if it were included in list or block of a declare target directive,

\cppnexample[5.1]{declare_target}{2c}
\cppspecificend

\subsection{Declare Target Directive for Variables}
\label{subsec:declare_target_variables}

\index{directives!declare target@\code{declare}~\code{target}}
\index{declare target directive@\code{declare}~\code{target} directive}

\index{directives!begin declare target@\code{begin}~\code{declare}~\code{target}}
\index{begin declare target directive@\code{begin}~\code{declare}~\code{target} directive}

The following examples show how the declare target directive is used to indicate that
global variables are mapped to the implicit device data environment of each target device.

In the following example, the declarations of the variables \plc{p}, \plc{v1}, and \plc{v2} appear 
between \code{begin}~\code{declare}~\code{target} and \code{end}~\code{declare}~\code{target} 
directives indicating that the variables are mapped to the implicit device data 
environment of each target device. The \code{target} \code{update} directive 
is then used to manage the consistency of the variables \plc{p}, \plc{v1}, and \plc{v2} between the 
data environment of the encountering host device task and the implicit device data 
environment of the default target device.

\cexample[5.1]{declare_target}{3}

The Fortran version of the above C code uses a different syntax. Fortran modules 
use a list syntax on the \code{declare} \code{target} directive to declare 
mapped variables.

\ffreeexample[4.0]{declare_target}{3}

The following example also indicates that the function \plc{Pfun()} is available on the 
target device, as well as the variable \plc{Q}, which is mapped to the implicit device 
data environment of each target device. The \code{target} \code{update} directive 
is then used to manage the consistency of the variable \plc{Q} between the data environment 
of the encountering host device task and the implicit device data environment of 
the default target device.

In the following example, the function and variable declarations appear between 
the \code{begin}~\code{declare}~\code{target} and \code{end}~\code{declare}~\code{target} 
directives.

\cexample[5.1]{declare_target}{4}

The Fortran version of the above C code uses a different syntax. In Fortran modules 
a list syntax on the \code{declare} \code{target} directive is used to declare 
mapped variables and procedures. The \plc{N} and \plc{Q} variables are declared as a comma 
separated list. When the \code{declare} \code{target} directive is used to 
declare just the procedure, the procedure name need not be listed -- it is implicitly 
assumed, as illustrated in the \plc{Pfun()} function.

\ffreeexample[4.0]{declare_target}{4}

\subsection{Declare Target Directive with \code{declare}~\code{simd}}
\label{subsec:declare_target_simd}

\index{directives!declare target@\code{declare}~\code{target}}
\index{declare target directive@\code{declare}~\code{target} directive}

\index{directives!begin declare target@\code{begin}~\code{declare}~\code{target}}
\index{begin declare target directive@\code{begin}~\code{declare}~\code{target} directive}

\index{directives!declare simd@\code{declare}~\code{simd}}
\index{declare simd directive@\code{declare}~\code{simd} directive}

The following example shows how the \code{begin}~\code{declare}~\code{target} and 
\code{end}~\code{declare}~\code{target} directives are used to indicate that a function 
is available on a target device. The \code{declare} \code{simd} directive indicates 
that there is a SIMD version of the function \plc{P()} that is available on the target 
device as well as one that is available on the host device.

\cexample[5.1]{declare_target}{5}

The Fortran version of the above C code uses a different syntax. Fortran modules 
use a list syntax of the \code{declare} \code{target} declaration for the mapping. 
Here the \plc{N} and \plc{Q} variables are declared in the list form as a comma separated list. 
The function declaration does not use a list and implicitly assumes the function 
name. In this Fortran example row and column indices are reversed relative to the 
C/C++ example, as is usual for codes optimized for memory access.

\ffreeexample[4.0]{declare_target}{5}


\subsection{Declare Target Directive with \code{link} Clause}
\label{subsec:declare_target_link}

\index{directives!declare target@\code{declare}~\code{target}}
\index{declare target directive@\code{declare}~\code{target} directive}

\index{directives!begin declare target@\code{begin}~\code{declare}~\code{target}}
\index{begin declare target directive@\code{begin}~\code{declare}~\code{target} directive}

\index{clauses!link@\code{link}}
\index{link clause@\code{link} clause}

In the OpenMP 4.5 standard the declare target directive was extended to allow static
data to be mapped, \emph{when needed}, through a \code{link} clause.

Data storage for items listed in the \code{link} clause becomes available on the device
when it is mapped implicitly or explicitly in a \code{map} clause, and it persists for the scope of
the mapping (as specified by a \code{target} construct, 
a \code{target}~\code{data} construct, or 
\code{target}~\code{enter/exit}~\code{data} constructs).

Tip: When all the global data items will not fit on a device and are not needed
simultaneously, use the \code{link} clause and map the data only when it is needed.

The following C and Fortran examples show two sets of data (single precision and double precision)
that are global on the host for the entire execution on the host; but are only used
globally on the device for part of the program execution. The single precision data
are allocated and persist only for the first \code{target} region. Similarly, the
double precision data are in scope on the device only for the second \code{target} region.

\cexample[5.1]{declare_target}{6}
\ffreeexample[4.5]{declare_target}{6}


\pagebreak
\section{Lambda Expressions}
\label{sec:lambda_expressions}

\index{lambda expressions}


\cppspecificstart
The following example illustrates the usage of lambda expressions and their
corresponding closure objects within a \scode{target} region.

In CASE 1, a lambda expression is defined inside a \scode{target} construct
that implicitly maps the structure \textit{s}. Inside the construct, the
lambda captures (by reference) the corresponding \textit{s}, and the resulting
closure object is assigned to \textit{lambda1}.  When the call operator is
invoked on \textit{lambda1}, the captured reference to \textit{s} is used in
the call.  The modified \textit{s} is then copied back to the host device on
exit from the \scode{target} construct.

In CASE 2, a lambda expression is instead defined before the \scode{target}
construct and captures (by copy) the pointer \textit{sp}. A
\scode{target}~\scode{data} construct is used to first map the structure, and
then the \scode{target} construct implicitly maps the closure object
referenced by \textit{lambda2}, a zero-length array section based on the
structure pointer \textit{sp}, and a zero-length array section based on the
captured pointer in the closure object. The implicit maps result in attached
pointers to the corresponding structure. The call for \textit{lambda2} inside
the \scode{target} construct will access \textit{sp->a} and \textit{sp->b}
from the corresponding structure.

CASE 3 is similar to CASE 2, except \textit{s} is instead captured by
reference by the lambda expression. As for CASE 2, the structure is first
mapped by an enclosing \scode{target}~\scode{data} construct, and then the
\scode{target} construct implicitly maps \textit{s} and the closure object
referenced by \textit{lambda3}. The effect of the map is to make the
the call for \textit{lambda3} refer to the corresponding \textit{s} inside the
\scode{target} construct rather than the original \textit{s}.

In CASE 4, the program defines a static variable \textit{ss} of the same
structure type as \textit{s}. While the body of the lambda expression refers
to \textit{ss}, it is not captured. In order for \textit{lambda4} to be
callable in the \scode{target} region, the reference to \textit{ss} should be
to a device copy of \textit{ss} that also has static storage. This is achieved
with the use of the \scode{declare}~\scode{target} directive.  Inside the
\scode{target} construct, all references to \textit{ss}, including in the
\textit{lambda4()} call, will refer to the corresponding \textit{ss} that
results from the \scode{declare}~\scode{target} directive. The \scode{always}
modifier is used on the \scode{map} clause to transfer the updated values for
the structure back to the host device.

\cppnexample[5.0]{lambda_expressions}{1}
\cppspecificend

%\pagebreak
\section{\kcode{teams} Construct and Related Combined Constructs}
\label{sec:teams}

\subsection{\kcode{target} and \kcode{teams} Constructs with \kcode{omp_get_num_teams}\\
and \kcode{omp_get_team_num} Routines}
\label{subsec:teams_api}
\index{constructs!target@\kcode{target}}
\index{target construct@\kcode{target} construct}
\index{constructs!teams@\kcode{teams}}
\index{teams construct@\kcode{teams} construct}
\index{combined constructs!target teams@\kcode{target teams}}
\index{teams construct@\kcode{teams} construct!num_teams clause@\kcode{num_teams} clause}
\index{clauses!num_teams@\kcode{num_teams}}
\index{num_teams clause@\kcode{num_teams} clause}
\index{routines!omp_get_num_teams@\kcode{omp_get_num_teams}}
\index{routines!omp_get_team_num@\kcode{omp_get_team_num}}
\index{omp_get_num_teams routine@\kcode{omp_get_num_teams} routine}
\index{omp_get_team_num routine@\kcode{omp_get_team_num} routine}

The following example shows how the \kcode{target} and \kcode{teams} constructs 
are used to create a \plc{league} of thread teams that execute a region. The \kcode{teams} 
construct creates a league of at most two teams where the primary thread of each 
team executes the \kcode{teams} region.

The \kcode{omp_get_num_teams} routine returns the number of teams executing in a \kcode{teams} 
region. The \kcode{omp_get_team_num} routine returns the team number, which is an integer 
between 0 and one less than the value returned by \kcode{omp_get_num_teams}. The following 
example manually distributes a loop across two teams.

\cexample[4.0]{teams}{1}

\ffreeexample[4.0]{teams}{1}

\subsection{\kcode{target}, \kcode{teams}, and \kcode{distribute} Constructs}
\label{subsec:teams_distribute}
\index{constructs!distribute@\kcode{distribute}}
\index{distribute construct@\kcode{distribute} construct}

The following example shows how the \kcode{target}, \kcode{teams}, and \kcode{distribute} 
constructs are used to execute a loop nest in a \kcode{target} region. The \kcode{teams} 
construct creates a league and the primary thread of each team executes the \kcode{teams} 
region. The \kcode{distribute} construct schedules the subsequent loop iterations 
across the primary threads of each team.

The number of teams in the league is less than or equal to the variable \ucode{num_blocks}. 
Each team in the league has a number of threads less than or equal to the variable 
\ucode{block_threads}. The iterations in the outer loop are distributed among the primary 
threads of each team.

When a team's primary thread encounters the parallel loop construct before the inner 
loop, the other threads in its team are activated. The team executes the \kcode{parallel} 
region and then workshares the execution of the loop.

\index{reduction clause@\kcode{reduction} clause!on teams construct@on \kcode{teams} construct}

Each primary thread executing the \kcode{teams} region has a private copy of the 
variable \ucode{sum} that is created by the \kcode{reduction} clause on the \kcode{teams} construct. 
The primary thread and all threads in its team have a private copy of the variable 
\ucode{sum} that is created by the \kcode{reduction} clause on the parallel loop construct. 
The second private \ucode{sum} is reduced into the primary thread's private copy of \ucode{sum} 
created by the \kcode{teams} construct. At the end of the \kcode{teams} region, 
each primary thread's private copy of \ucode{sum} is reduced into the final \ucode{sum} that is 
implicitly mapped into the \kcode{target} region.

\cexample[4.0]{teams}{2}
%\clearpage

\ffreeexample[4.0]{teams}{2}

\subsection{\kcode{target teams}, and Distribute Parallel Loop Constructs}
\label{subsec:teams_distribute_parallel}

The following example shows how the \kcode{target teams} and distribute 
parallel loop constructs are used to execute a \kcode{target} region. 
The \kcode{target teams} construct creates a league of teams where the primary thread of each 
team executes the \kcode{teams} region.

The distribute parallel loop construct schedules the loop iterations across the 
primary threads of each team and then across the threads of each team.

\cexample[4.5]{teams}{3}

\ffreeexample[4.5]{teams}{3}

\subsection{\kcode{target teams} and Distribute Parallel Loop 
Constructs with Scheduling Clauses}
\label{subsec:teams_distribute_parallel_schedule}
\index{distribute construct@\kcode{distribute} construct!dist_schedule clause@\kcode{dist_schedule} clause}
\index{clauses!dist_schedule@\kcode{dist_schedule}}
\index{dist_schedule clause@\kcode{dist_schedule} clause}
\index{worksharing-loop constructs!schedule clause@\kcode{schedule} clause}
\index{clauses!schedule@\kcode{schedule}}
\index{schedule clause@\kcode{schedule} clause}

The following example shows how the \kcode{target teams} and \kcode{distribute parallel}
constructs are used to execute a \kcode{target} region. The \kcode{teams} 
construct creates a league of at most eight teams where the primary thread of each 
team executes the \kcode{teams} region. The number of threads in each team is 
less than or equal to \ucode{16}.

The \kcode{distribute} parallel loop construct schedules the subsequent loop iterations 
across the primary threads of each team and then across the threads of each team.

The \kcode{dist_schedule} clause on the distribute parallel loop construct indicates 
that loop iterations are distributed to the primary thread of each team in chunks 
of \ucode{1024} iterations.

The \kcode{schedule} clause indicates that the 1024 iterations distributed to 
a primary thread are then assigned to the threads in its associated team in chunks 
of \ucode{64} iterations.

\cexample[4.0]{teams}{4}

\ffreeexample[4.0]{teams}{4}

\subsection{\kcode{target teams} and \kcode{distribute simd} Constructs}
\label{subsec:teams_distribute_simd}

The following example shows how the \kcode{target teams} and \kcode{distribute simd} constructs are used to execute a loop in a \kcode{target} region. 
The \kcode{target teams} construct creates a league of teams where the 
primary thread of each team executes the \kcode{teams} region.

The \kcode{distribute simd} construct schedules the loop iterations across 
the primary thread of each team and then uses SIMD parallelism to execute the iterations.

\cexample[4.0]{teams}{5}

\ffreeexample[4.0]{teams}{5}

\subsection{\kcode{target teams} and Distribute Parallel Loop SIMD Constructs}
\label{subsec:teams_distribute_parallel_simd}

The following example shows how the \kcode{target teams} and the distribute 
parallel loop SIMD constructs are used to execute a loop in a \kcode{target teams} 
region. The \kcode{target teams} construct creates a league of teams 
where the primary thread of each team executes the \kcode{teams} region.

The distribute parallel loop SIMD construct schedules the loop iterations across 
the primary thread of each team and then across the threads of each team where each 
thread uses SIMD parallelism.

\cexample[4.0]{teams}{6}

\ffreeexample[4.0]{teams}{6}

\subsection{Evaluation of \kcode{num_teams} Clause that Appears inside \kcode{target} Region}
\label{subsec:target_teams_num_teams}

The following example shows the evaluation of the \kcode{num_teams} clause when the \kcode{teams} construct is closely nested inside \kcode{target} construct. The code is non-conforming since value of \ucode{x} for the clause may be different from different devices. As of OpenMP 6.0, it is the user's responsibility to ensure identical values for the clause expression for nested as well as combined directive cases for \kcode{target} and \kcode{teams} constructs. This permits implementations to evaluate the \kcode{num_teams} argument on the host rather than the target device. For the program to be conforming, the program must update the host value so that \ucode{x} will have the same value when evaluated on the host or target device.

\cexample[6.0]{teams}{7}

\ffreeexample[6.0]{teams}{7}



\pagebreak
\section{Asynchronous \code{target} Execution and Dependences}
\label{sec:async_target_exec_depend}

Asynchronous execution of a \code{target} region can be accomplished
by creating an explicit task around the \code{target} region. Examples
with explicit tasks are shown at the beginning of this section. 

As of OpenMP 4.5 and beyond the \code{nowait} clause can be used on the
\code{target} directive for asynchronous execution. Examples with 
\code{nowait} clauses follow the explicit \code{task} examples.

This section also shows the use of \code{depend} clauses to order 
executions through dependences.

\subsection{Asynchronous \kcode{target} with Tasks}
\label{subsec:async_target_with_tasks}
\index{target construct@\kcode{target} construct}
\index{task construct@\kcode{task} construct}

\index{directives!declare target@\kcode{declare target}}
\index{declare target directive@\kcode{declare target} directive}

\index{directives!begin declare target@\kcode{begin declare target}}
\index{begin declare target directive@\kcode{begin declare target} directive}

The following example shows how the \kcode{task} and \kcode{target} constructs 
are used to execute multiple \kcode{target} regions asynchronously. The task that 
encounters the \kcode{task} construct generates an explicit task that contains 
a \kcode{target} region. The thread executing the explicit task encounters a task 
scheduling point while waiting for the execution of the \kcode{target} region 
to complete, allowing the thread to switch back to the execution of the encountering 
task or one of the previously generated explicit tasks.

\cexample[5.1]{async_target}{1}

\pagebreak
\index{directives!declare target@\kcode{declare target}}
\index{declare target directive@\kcode{declare target} directive}
The Fortran version has an interface block that contains the \kcode{declare target}. 
An identical statement exists in the function declaration (not shown here).

\ffreeexample[4.0]{async_target}{1}

The following example shows how the \kcode{task} and \kcode{target} constructs 
are used to execute multiple \kcode{target} regions asynchronously. The task dependence 
ensures that the storage is allocated and initialized on the device before it is 
accessed.

\cexample[5.1]{async_target}{2}

The Fortran example below is similar to the C version above. Instead of pointers, though, it uses
the convenience of Fortran allocatable arrays on the device. In order to preserve the arrays 
allocated on the device across multiple \kcode{target} regions, a \kcode{target data} region 
is used in this case.

If there is no shape specified for an allocatable array in a \kcode{map} clause, only the array descriptor
(also called a dope vector) is mapped. That is, device space is created for the descriptor, and it
is initially populated with host values. In this case, the \ucode{v1} and \ucode{v2} arrays will be in a
non-associated state on the device. When space for \ucode{v1} and \ucode{v2} is allocated on the device
in the first \kcode{target} region the addresses to the space will be included in their descriptors.

At the end of the first \kcode{target} region, the arrays \ucode{v1} and \ucode{v2} are preserved on the device 
for access in the second \kcode{target} region. At the end of the second \kcode{target} region, the data 
in array \ucode{p} is copied back, the arrays \ucode{v1} and \ucode{v2} are not.

\index{task construct@\kcode{task} construct!depend clause@\kcode{depend} clause}
\index{clauses!depend@\kcode{depend}}
\index{depend clause@\kcode{depend} clause}
A \kcode{depend} clause is used in the \kcode{task} directive to provide a wait at the beginning of the second 
\kcode{target} region, to insure that there is no race condition with \ucode{v1} and \ucode{v2} in the two tasks.
It would be noncompliant to use \ucode{v1} and/or \ucode{v2} in lieu of \ucode{N} in the \kcode{depend} clauses, 
because the use of non-allocated allocatable arrays as list items in a \kcode{depend} clause would 
lead to unspecified behavior. 

\noteheader{--} This example is not strictly compliant with the OpenMP 4.5 specification since the allocation status
of allocatable arrays \ucode{v1} and \ucode{v2} is changed inside the \kcode{target} region, which is not allowed.
(See the restrictions for the \kcode{map} clause in the \docref{Data-mapping Attribute Rules and Clauses} 
section of the specification.)
However, the intention is to relax the restrictions on mapping of allocatable variables in the next release
of the specification so that the example will be compliant.

\ffreeexample[4.0]{async_target}{2}

\subsection{\code{nowait} Clause on \code{target} Construct}
\label{subsec:target_nowait_clause}
\index{target construct@\code{target} construct!nowait clause@\code{nowait} clause}
\index{clauses!nowait@\code{nowait}}
\index{nowait clause@\code{nowait} clause}

The following example shows how to execute code asynchronously on a 
device without an explicit task. The \code{nowait} clause on a \code{target} 
construct allows the thread of the \plc{target task} to perform other
work while waiting for the \code{target} region execution to complete. 
Hence, the \code{target} region can execute asynchronously on the 
device (without requiring a host thread to idle while waiting for 
the \plc{target task} execution to complete).

In this example the product of two vectors (arrays), \plc{v1}
and \plc{v2}, is formed. One half of the operations is performed
on the device, and the last half on the host, concurrently.

After a team of threads is formed the primary thread generates 
the \plc{target task} while the other threads can continue on, without a barrier,
to the execution of the host portion of the vector product.
The completion of the \plc{target task} (asynchronous target execution) is 
guaranteed by the synchronization in the implicit barrier at the end of the 
host vector-product worksharing loop region. See the \code{barrier} 
glossary entry in the OpenMP specification for details.

The host loop scheduling is \code{dynamic}, to balance the host thread executions, since 
one thread is being used for offload generation. In the situation where 
little time is spent by the \plc{target task} in setting 
up and tearing down the target execution, \code{static} scheduling may be desired. 

\cexample[5.1]{async_target}{3}

\ffreeexample[5.1]{async_target}{3}

%begin 
\subsection{Asynchronous \kcode{target} with \kcode{nowait} and \kcode{depend} Clauses}
\label{subsec:async_target_nowait_depend}
\index{target construct@\kcode{target} construct!nowait clause@\kcode{nowait} clause}
\index{target construct@\kcode{target} construct!depend clause@\kcode{depend} clause}
\index{nowait clause@\kcode{nowait} clause}
\index{depend clause@\kcode{depend} clause}
\index{clauses!nowait@\kcode{nowait}}
\index{clauses!depend@\kcode{depend}}

More details on dependences can be found in \specref{sec:task_depend}, Task 
Dependences. In this example, there are three flow dependences.  In the first two dependences the
target task does not execute until the preceding explicit tasks have finished.   These 
dependences are produced by arrays \ucode{v1} and \ucode{v2}  with the \kcode{out} dependence type in the first two tasks, and the \kcode{in} dependence type in the \plc{target task}.   

The last dependence is produced by array \ucode{p}  with the \kcode{out} dependence type in the target task, and the \kcode{in} dependence type in the last task.  The last task does not execute until the target task finishes.  

The \kcode{nowait} clause on the \kcode{target} construct creates a deferrable target task, allowing the encountering task to continue execution without waiting for the completion of the target task.

\cexample[4.5]{async_target}{4}

\ffreeexample[4.5]{async_target}{4}

%end

\subsection{Conditionally Asynchronous \kcode{target} Using the \kcode{nowait} Clause}
\label{subsec:async_target_nowait_arg}
\index{target construct@\kcode{target} construct!nowait clause@\kcode{nowait} clause}
\index{nowait clause@\kcode{nowait} clause}
\index{clauses!nowait@\kcode{nowait}}

In OpenMP 6.0, \kcode{nowait} takes an OpenMP logical type argument to specify if the generated \plc{task} is an included task or a deferred task. In the following example, the \kcode{nowait} clause is used with an argument on the \kcode{target} directive. In a practical situation, the value of \ucode{is_deferred} can be chosen based on the time taken for some work on host or device that can be performed asynchronously after the target task is scheduled. If the target task is deferred, it must be synchronized by a \kcode{taskwait} before the value of \ucode{x} is used. Prior to 6.0, the same effect would require the use of a \plc{metadirective} or an \bcode{if-else} statement that duplicates the \kcode{target} construct.

\cexample[6.0]{async_target}{5}

\ffreeexample[6.0]{async_target}{5}

%\pagebreak
\section{Device Routines}
\label{sec:device}

\subsection{\kcode{omp_is_initial_device} Routine}
\label{subsec:device_is_initial}
\index{routines!omp_is_initial_device@\kcode{omp_is_initial_device}}
\index{omp_is_initial_device routine@\kcode{omp_is_initial_device} routine}

\index{directives!declare target@\kcode{declare target}}
\index{declare target directive@\kcode{declare target} directive}

\index{directives!begin declare target@\kcode{begin declare target}}
\index{begin declare target directive@\kcode{begin declare target} directive}

The following example shows how the \kcode{omp_is_initial_device} runtime library routine 
can be used to query if a code is executing on the initial host device or on a 
target device. The example then sets the number of threads in the \kcode{parallel} 
region based on where the code is executing.

\cexample[5.1]{device}{1}

\ffreeexample[4.0]{device}{1}

\subsection{\kcode{omp_get_num_devices} Routine}
\label{subsec:device_num_devices}

The following example shows how the \kcode{omp_get_num_devices} runtime library routine 
can be used to determine the number of devices.

\cexample[4.0]{device}{2}

\ffreeexample[4.0]{device}{2}

\subsection{\kcode{omp_set_default_device} and \\
\kcode{omp_get_default_device} Routines}
\label{subsec:device_is_set_get_default}
\index{routines!omp_set_default_device@\kcode{omp_set_default_device}}
\index{omp_set_default_device routine@\kcode{omp_set_default_device} routine}

The following example shows how the \kcode{omp_set_default_device} and \kcode{omp_get_default_device} 
runtime library routines can be used to set the default device and determine the 
default device respectively.

\cexample[4.0]{device}{3}

\ffreeexample[4.0]{device}{3}


%-------------
%\newpage
\subsection{Device and Host Memory Association}
\label{subsec:target_associate_ptr}
\label{sec:target_associate_ptr}
\index{routines!omp_target_associate_ptr@\scode{omp_target_associate_ptr}}
\index{omp_target_associate_ptr routine@\scode{omp_target_associate_ptr} routine}

\index{routines!omp_target_alloc@\scode{omp_target_alloc}}
\index{omp_target_alloc routine@\scode{omp_target_alloc} routine}
The association of device memory with host memory
can be established by calling the \scode{omp_target_associate_ptr} 
API routine as part of the mapping.
The following example shows the use of this routine
to associate device memory of size \splc{CS}, 
allocated by the \scode{omp_target_alloc} routine and
pointed to by the device pointer \splc{dev_ptr}, 
with a chunk of the host array \splc{arr} starting at index \splc{ioff}.
In Fortran, the intrinsic function \scode{c_loc} is called
to obtain the corresponding C pointer (\splc{h_ptr}) of \splc{arr(ioff)} 
for use in the call to the API routine.

\index{constructs!target update@\code{target}~\code{update}}
\index{target update construct@\code{target}~\code{update} construct}
\index{map clause@\code{map} clause!always modifier@\code{always} modifier}
\index{always modifier@\code{always} modifier}
Since the reference count of the resulting mapping is infinite,
it is necessary to use the \scode{target}~\scode{update} directive (or
the \scode{always} modifier in a \scode{map} clause) to accomplish a
data transfer between host and device.
The explicit mapping of the array section \splc{arr[ioff:CS]} 
(or \splc{arr(ioff:ioff+CS-1)} in Fortran) on the \scode{target}
construct ensures that the allocated and associated device memory is used 
when referencing the array \splc{arr} in the \scode{target} region.
The device pointer \splc{dev_ptr} cannot be accessed directly 
after a call to the \scode{omp_target_associate_ptr} routine.

\index{routines!omp_target_disassociate_ptr@\scode{omp_target_disassociate_ptr}}
\index{omp_target_disassociate_ptr routine@\scode{omp_target_disassociate_ptr} routine}
\index{routines!omp_target_free@\scode{omp_target_free}}
\index{omp_target_free routine@\scode{omp_target_free} routine}
After the \scode{target} region, the device pointer is disassociated from
the current chunk of the host memory by calling the \scode{omp_target_disassociate_ptr} routine before working on the next chunk.
The device memory is freed by calling the \scode{omp_target_free}
routine at the end.

\cexample[4.5]{target_associate_ptr}{1}

\ffreeexample[5.1]{target_associate_ptr}{1}


%-------------

\subsection{Target Memory and Device Pointers Routines}
\label{subsec:target_mem_and_device_ptrs}
\index{routines!omp_target_alloc@\kcode{omp_target_alloc}}
\index{omp_target_alloc routine@\kcode{omp_target_alloc} routine}
\index{routines!omp_target_memcpy@\kcode{omp_target_memcpy}}
\index{omp_target_memcpy routine@\kcode{omp_target_memcpy} routine}
\index{routines!omp_target_free@\kcode{omp_target_free}}
\index{omp_target_free routine@\kcode{omp_target_free} routine}

The following example shows how to create space on a device, transfer data
to and from that space, and free the space, using API calls. The API calls
directly execute allocation, copy and free operations on the device, without invoking
any mapping through a \kcode{target} directive. The \kcode{omp_target_alloc} routine allocates space
and returns a device pointer for referencing the space in the \kcode{omp_target_memcpy}
API routine on the host. The \kcode{omp_target_free} routine frees the space on the device.

\index{target construct@\kcode{target} construct!is_device_ptr clause@\kcode{is_device_ptr} clause}
\index{is_device_ptr clause@\kcode{is_device_ptr} clause}
\index{clauses!is_device_ptr@\kcode{is_device_ptr}}
The example also illustrates how to access that space
in a \kcode{target} region by exposing the device pointer in an \kcode{is_device_ptr} clause.

The example creates an array of cosine values on the default device, to be used
on the host device. The function fails if a default device is not available.

\cexample[4.5]{device}{4}

\index{routines!omp_target_is_present@\kcode{omp_target_is_present}}
\index{omp_target_is_present routine@\kcode{omp_target_is_present} routine}
\index{routines!omp_target_associate_ptr@\kcode{omp_target_associate_ptr}}
\index{omp_target_associate_ptr routine@\kcode{omp_target_associate_ptr} routine}
The following Fortran example illustrates how to use the \kcode{omp_target_alloc}
and \kcode{omp_target_memcpy} functions to directly allocate device
storage and transfer data to and from a device. It also shows how to check for
the presence of device data with the \kcode{omp_target_is_present} function and
to associate host and device storage with the \kcode{omp_target_associate_ptr} function.

In Section 1 of the code, 40 bytes of storage are allocated on the default device
with the \kcode{omp_target_alloc} function, which returns a value (of type
\bcode{C_PTR}) that contains the device address of the storage.
%A Fortran pointer (\ucode{fp}) is associated by the Fortran \splc{iso_c_binding}
%\bcode{c_f_pointer} routine with the target of the C pointer (\ucode{cp}).
In the subsequent \kcode{target} construct, \ucode{cp} is specified on the
\kcode{is_device_ptr} clause to instruct the compiler that \ucode{cp} is
a device pointer.
The device pointer (\ucode{cp}) is then associated with the Fortran pointer
(\ucode{fp}) via the \bcode{c_f_pointer} routine inside the \kcode{target}
construct.
As a result, \ucode{fp} points to the storage on the device that is allocated
by the \kcode{omp_target_alloc} routine.
In the \kcode{target} region, the value 4 is assigned to the storage on the device,
using the Fortran pointer.
A trivial test checks that all values were correctly assigned.
The Fortran pointer (\ucode{fp}) is nullified before the end of the \kcode{target} region.
After the \kcode{target} construct, the space on the device is freed with the
\kcode{omp_target_free} function, using the device \ucode{cp} pointer
which is set to null after the call.

In Section 2, the content of the storage allocated on the host is directly copied
to the OpenMP allocated storage on the device.
First, storage is allocated for the device and host using \kcode{omp_target_alloc}.
Next, on the host the device pointer, returned from the allocation
\kcode{omp_target_alloc} function, is associated with a Fortran pointer, and
values are assigned to the storage. Similarly, values are assigned on the device
to the device storage, after associating a Fortran pointer (\ucode{fp_dst})
with the device's storage pointer (\ucode{cp_dst}).

Next the \kcode{omp_target_memcpy} function directly copies the host data
to the device storage, specified by the respective host and device pointers.
This copy will overwrite -1 values in the device storage, and is checked in the
next \kcode{target} construct.
Keyword arguments are used here for clarity.
(A positional argument list is used in the next Section.)

In Section 3, space is allocated (with a Fortran \bcode{ALLOCATE} statement) and initialized using a
host Fortran pointer (\ucode{h_fp}), and the address of the storage is directly assigned to a
host C pointer (\ucode{h_cp}).
The following \kcode{omp_target_is_present} function returns \ucode{0} (\plc{false}, of \bcode{integer(C_INT)} type)
to indicate that \ucode{h_cp} does not have any corresponding storage on the default device.

Next, the same amount of space is allocated on the default device with
the \kcode{omp_target_alloc} function, which returns a device pointer (\ucode{d_cp}).
The device pointer \ucode{d_cp} and host pointer \ucode{h_cp}
are then associated using the \kcode{omp_target_associate_ptr} function.
The device storage to which \ucode{d_cp} points becomes the corresponding storage of
the host storage to which \ucode{h_cp} points.
The following \kcode{omp_target_is_present} call confirms this, by returning
a non-zero value of \bcode{integer(C_INT)} type for true.

After the association, the content of the host storage
is copied to the device using the \kcode{omp_target_memcpy} function.
In the final \kcode{target} construct an array section of \ucode{h_fp} 
is mapped to the device, and evaluated for correctness.
The mapping establishes a connection of \ucode{h_fp} with
the corresponding device data in the \kcode{target} construct,
but does not produce an update on the device because the previous \kcode{omp_target_associate_ptr} routine sets the 
reference count of the mapped object to infinity, meaning a mapping 
without the \kcode{always} modifier will not 
update the device object.

\ffreeexample[5.0]{device}{4}

\index{routines!omp_target_memcpy_async@\kcode{omp_target_memcpy_async}}
\index{omp_target_memcpy_async routine@\kcode{omp_target_memcpy_async} routine}
The following example illustrates the use of the \kcode{omp_target_memcpy_async}
routine to perform asynchronous memory copies.
The routine acts as if it is a deferrable task so that
a \kcode{taskwait} construct can be used to wait for the completion
of the deferrable task.
In the example the \kcode{omp_target_memcpy_async} routine copies host data 
(\ucode{h_buf}) to device (\ucode{d_buf}).
The Fortran code uses the intrinsic \bcode{c_loc} function to get
the corresponding C pointer (\ucode{c_hbuf}) for
passing to the \kcode{omp_target_memcpy_async} routine.
The last two arguments (\ucode{0} and \ucode{NULL}) to the routine
indicate that there is no specified dependence associated with the call.
The Fortran code omits the unused last argument.

\cexample[5.2]{device}{5}
\ffreeexample[5.2]{device}{5}

\index{directives!depobj@\kcode{depobj}}
\index{depobj directive@\kcode{depobj} directive}
The following is a more complicated example that shows the use of 
the \kcode{omp_target_memcpy_async} routine with a depend object \ucode{obj} to 
overlap the memory copy with computation performed by \ucode{do_work}.
The depend object \ucode{obj} was created by the \kcode{depobj} directive
and initialized to an \kcode{out} dependence on the data \ucode{d_buf[0:N]} 
(or \ucode{d_buf(1:N)} for Fortran) in advance.
The \kcode{depend(depobj: \ucode{obj})} (or alternatively 
\kcode{depend(in: \ucode{d_buf[0:N]})}) clause on the \kcode{target} construct
ensures the asynchronous memory copy is complete before the data \ucode{d_buf}
can be used in the \kcode{target} region.

\cexample[5.2]{device}{6}
\ffreeexample[5.2]{device}{6}


\pagebreak
\section{Traits for Specifying Devices}
\label{sec:device_env_traits}

\index{environment variables!OMP_AVAILABLE_DEVICES@\kcode{OMP_AVAILABLE_DEVICES}}
\index{OMP_AVAILABLE_DEVICES@\kcode{OMP_AVAILABLE_DEVICES}}
\index{environment variables!OMP_DEFAULT_DEVICE@\kcode{OMP_DEFAULT_DEVICE}}
\index{OMP_DEFAULT_DEVICE@\kcode{OMP_DEFAULT_DEVICE}}

Environment variables \kcode{OMP_AVAILABLE_DEVICES} and
\kcode{OMP_DEFAULT_DEVICE} can take traits to specify the available
devices and the default device, respectively.
In addition, \kcode{OMP_DEFAULT_DEVICE} can also take an integer
as a device number to specify the default device.

The following examples show how traits are used to specify devices
for these environment variables.

Only GPU non-host devices are available to program:
\begin{boxedcode}
export OMP_AVAILABLE_DEVICES=\ucode{"kind(gpu)"} 
\end{boxedcode}

Order of available devices would be all vendor \ucode{A} GPUs, then 
the rest of the non-host devices as specified by "\ucode{*}":
\begin{boxedcode}
export OMP_AVAILABLE_DEVICES=\ucode{"kind(gpu)&&vendor(A),*"}
\end{boxedcode}

Available devices would be all non-gpu devices from vendor \ucode{A}:
\begin{boxedcode}
export OMP_AVAILABLE_DEVICES=\ucode{"!kind(gpu)&&vendor(A)"}
\end{boxedcode}

Available devices start with 1 vendor \ucode{A} GPU device, then 
2 vendor \ucode{B} GPU devices, and then the rest of the non-host devices:
\begin{boxedcode}
export OMP_AVAILABLE_DEVICES=\ucode{"(kind(gpu)&&vendor(A))[0],}
                              \ucode{(kind(gpu)&&vendor(B))[0:2],*"}
\end{boxedcode}
The device number range is specified by the C/C++ array section syntax
\ucode{[0:2]} where "\ucode{0}" is the first index and "\ucode{2}"
is the length.

Three available devices are re-ordered with "\ucode{uid-gpu3}" corresponding 
to device 0, "\ucode{uid-gpu2}" to device 1 and "\ucode{uid-gpu1}"
to device 2:
\begin{boxedcode}
export OMP_AVAILABLE_DEVICES=\ucode{"uid(uid-gpu3),uid(uid-gpu2),}
                              \ucode{uid(uid-gpu1)"}
\end{boxedcode}

The default device will be some visible vendor \ucode{A} GPU device. 
If not available, then set to initial device:
\begin{boxedcode}
export OMP_DEFAULT_DEVICE=\ucode{"kind(gpu)&&vendor(A),initial"}
\end{boxedcode}

The default device will be some visible vendor \ucode{A} GPU device. 
If not available, then set to invalid device so that upon first use of default 
device the program will error out:
\begin{boxedcode}
export OMP_DEFAULT_DEVICE=\ucode{"kind(gpu)&&vendor(A),invalid"}
\end{boxedcode}




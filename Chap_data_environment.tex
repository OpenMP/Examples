\cchapter{Data Environment}{data_environment}
\label{chap:data_environment}
The OpenMP \plc{data environment} contains data attributes of variables and
objects.  Many constructs (such as \code{parallel}, \code{simd}, \code{task}) 
accept clauses to control \plc{data-sharing} attributes
of referenced variables in the construct, where \plc{data-sharing} applies to
whether the attribute of the variable is \plc{shared}, 
is \plc{private} storage, or has special operational characteristics 
(as found in the \code{firstprivate}, \code{lastprivate}, \code{linear}, or \code{reduction} clause).

The data environment for a device (distinguished as a \plc{device data environment})
is controlled on the host by \plc{data-mapping} attributes, which determine the
relationship of the data on the host, the \plc{original} data, and the data on the
device, the \plc{corresponding} data.

\bigskip
DATA-SHARING ATTRIBUTES

Data-sharing attributes of variables can be classified as being \plc{predetermined},
\plc{explicitly determined} or \plc{implicitly determined}.

Certain variables and objects have predetermined attributes.  
A commonly found case is the loop iteration variable in associated loops 
of a \code{for} or \code{do} construct. It has a private data-sharing attribute.
Variables with predetermined data-sharing attributes cannot be listed in a data-sharing clause; but there are some
exceptions (mainly concerning loop iteration variables).

Variables with explicitly determined data-sharing attributes are those that are
referenced in a given construct and are listed in a data-sharing attribute
clause on the construct. Some of the common data-sharing clauses are:
\code{shared}, \code{private}, \code{firstprivate}, \code{lastprivate}, 
\code{linear}, and \code{reduction}. % Are these all of them?

Variables with implicitly determined data-sharing attributes are those
that are referenced in a given construct, do not have predetermined
data-sharing attributes, and are not listed in a data-sharing
attribute clause of an enclosing construct.
For a complete list of variables and objects with predetermined and
implicitly determined attributes, please refer to the
\plc{Data-sharing Attribute Rules for Variables Referenced in a Construct}
subsection of the OpenMP Specifications document.  

\bigskip
DATA-MAPPING ATTRIBUTES

The \code{map} clause on a device construct explicitly specifies how the list items in
the clause are mapped from the encountering task's data environment (on the host)
to the corresponding item in the device data environment (on the device).
The common \plc{list items} are arrays, array sections, scalars, pointers, and
structure elements (members). 

Procedures and global variables have predetermined data mapping if they appear
within the list or block of a \code{declare}~\code{target} directive. Also, a C/C++ pointer
is mapped as a zero-length array section, as is a C++ variable that is a reference to a pointer.
% Waiting for response from Eric on this.

Without explicit mapping, non-scalar and non-pointer variables within the scope of the \code{target}
construct are implicitly mapped with a \plc{map-type} of \code{tofrom}.
Without explicit mapping, scalar variables within the scope of the \code{target}
construct are not mapped, but have an implicit firstprivate data-sharing
attribute. (That is, the value of the original variable is given to a private
variable of the same name on the device.) This behavior can be changed with
the \code{defaultmap} clause.

The \code{map} clause can appear on \code{target}, \code{target data} and 
\code{target enter/exit data} constructs.  The operations of creation and
removal of device storage as well as assignment of the original list item 
values to the corresponding list items may be complicated when the list 
item appears on multiple constructs or when the host and device storage 
is shared. In these cases the item's reference count, the number of times
it has been referenced (+1 on entry and -1 on exited) in nested (structured)
map regions and/or accumulative (unstructured) mappings, determines the operation.
Details of the \code{map} clause and reference count operation are specified 
in the \plc{map Clause} subsection of the OpenMP Specifications document.


%===== Examples Sections =====
%\pagebreak
\section{\kcode{threadprivate} Directive}
\label{sec:threadprivate}
\index{directives!threadprivate@\kcode{threadprivate}}
\index{threadprivate directive@\kcode{threadprivate} directive}

The following examples demonstrate how to use the \kcode{threadprivate} directive 
 to give each thread a separate counter.

\cexample{threadprivate}{1}

\fexample{threadprivate}{1}

\pagebreak
\ccppspecificstart
The following example uses \kcode{threadprivate} on a static variable:

\cnexample{threadprivate}{2}

The following example demonstrates unspecified behavior for the initialization 
of a \kcode{threadprivate} variable. A \kcode{threadprivate}  variable is initialized 
once at an unspecified point before its first reference. Because \ucode{a} is 
constructed using the value of \ucode{x}  (which is modified by the statement 
\ucode{x++}), the value of \ucode{a.val}  at the start of the \kcode{parallel} 
region could be either 1 or 2. This problem is avoided for \ucode{b}, which uses 
an auxiliary \bcode{const} variable and a copy-constructor.

\cppnexample{threadprivate}{3}
\ccppspecificend

The following examples show non-conforming uses and correct uses of the \kcode{threadprivate} 
directive. 

\fortranspecificstart
The following example is non-conforming because the common block is not declared 
local to the subroutine that refers to it:

\fnexample{threadprivate}{2}

The following example is also non-conforming because the common block is not declared 
local to the subroutine that refers to it:

\fnexample{threadprivate}{3}

The following example is a correct rewrite of the previous example:

\fnexample{threadprivate}{4}

The following is an example of the use of \kcode{threadprivate} for local variables:
\topmarker{Fortran}

\fnexample{threadprivate}{5}

The above program, if executed by two threads, will print one of the following 
two sets of output: 

\pout{a = 11 12 13}
\\
\pout{ptr = 4}
\\
\pout{i = 15}

\pout{A is not allocated}
\\
\pout{ptr = 4}
\\
\pout{i = 5}

or

\pout{A is not allocated}
\\
\pout{ptr = 4}
\\
\pout{i = 15}

\pout{a = 1 2 3}
\\
\pout{ptr = 4}
\\
\pout{i = 5}

The following is an example of the use of \kcode{threadprivate} for module variables:
\topmarker{Fortran}

\fnexample{threadprivate}{6}
\fortranspecificend

\cppspecificstart
The following example illustrates initialization of \kcode{threadprivate} variables 
for class-type \ucode{T}. \ucode{t1} is default constructed, \ucode{t2} is constructed 
taking a constructor accepting one argument of integer type, \ucode{t3} is copy 
constructed with argument \ucode{f()}:

\cppnexample{threadprivate}{4}

The following example illustrates the use of \kcode{threadprivate} for static 
class members. The \kcode{threadprivate} directive for a static class member must 
be placed inside the class definition.

\cppnexample{threadprivate}{5}
\cppspecificend


\pagebreak
\section{\code{default(none)} Clause}
\label{sec:default_none}
\index{clauses!default(none)@\code{default(none)}}
\index{default(none) clause@\code{default(none)} clause}

The following example distinguishes the variables that are affected by the \code{default(none)} 
clause from those that are not. 

\ccppspecificstart
Beginning with OpenMP 4.0, variables with \code{const}-qualified type and no mutable member 
are no longer predetermined shared.  Thus, these variables (variable \plc{c} in the example) 
need to be explicitly listed
in data-sharing attribute clauses when the \code{default(none)} clause is specified.

\cnexample{default_none}{1}
\ccppspecificend

\fexample{default_none}{1}



\pagebreak
\section{\code{private} Clause}
\label{sec:private}
\index{clauses!private@\code{private}}
\index{private clause@\code{private} clause}

In the following example, the values of original list items \plc{i} and \plc{j} 
are retained on exit from the \code{parallel} region, while the private list 
items \plc{i} and \plc{j} are modified within the \code{parallel} construct. 

\cexample{private}{1}

\fexample{private}{1}

In the following example, all uses of the variable \plc{a} within the loop construct 
in the routine \plc{f} refer to a private list item \plc{a}, while it is 
unspecified whether references to \plc{a} in the routine \plc{g} are to a 
private list item or the original list item.

\cexample{private}{2}

\fexample{private}{2}

The following example demonstrates that a list item that appears in a \code{private} 
 clause in a \code{parallel} construct may also appear in a \code{private} 
 clause in an enclosed worksharing construct, which results in an additional private 
copy.

\cexample{private}{3}

\fexample{private}{3}



%\pagebreak
\section{Fortran Private Loop Iteration Variables}
\label{sec:fort_loopvar}
\fortranspecificstart
\index{loop variables, Fortran}

In general loop iteration variables will be private, when used in the \plc{do-loop} 
of a \kcode{do} and \kcode{parallel do} construct or in sequential loops in a 
\kcode{parallel} construct (see the \docref{Loop Construct} section and 
the \docref{Data-sharing Attribute Rules} section of 
the OpenMP 4.0 specification). In the following example of a sequential 
loop in a \kcode{parallel} construct the loop iteration variable \ucode{I} will 
be private.

\ffreenexample{fort_loopvar}{1}

In exceptional cases, loop iteration variables can be made shared, as in the following 
example:

\ffreenexample{fort_loopvar}{2}

Note however that the use of shared loop iteration variables can easily lead to 
race conditions.
\fortranspecificend


\pagebreak
\section{Fortran Restrictions on \code{shared} and \code{private} Clauses with Common Blocks}
\fortranspecificstart
\label{sec:fort_sp_common}
\index{clauses!private@\code{private}}
\index{clauses!shared@\code{shared}}
\index{private clause@\code{private} clause!common blocks, Fortran}
\index{shared clause@\code{shared} clause!common blocks, Fortran}

When a named common block is specified in a \code{private}, \code{firstprivate}, 
or \code{lastprivate} clause of a construct, none of its members may be declared 
in another data-sharing attribute clause on that construct. The following examples 
illustrate this point. 

The following example is conforming:

\fnexample{fort_sp_common}{1}

The following example is also conforming:

\fnexample{fort_sp_common}{2}
% blue line floater at top of this page for "Fortran, cont."
%\begin{figure}[t!]
%\linewitharrows{-1}{dashed}{Fortran (cont.)}{8em}
%\end{figure}
\clearpage

The following example is conforming:

\fnexample{fort_sp_common}{3}

The following example is non-conforming because \code{x} is a constituent element 
of \code{c}:

\fnexample{fort_sp_common}{4}

The following example is non-conforming because a common block may not be declared 
both shared and private:

\fnexample{fort_sp_common}{5}
\fortranspecificend



%\pagebreak
\begin{fortranspecific}[4ex]
\section{Fortran Restrictions on Storage Association with the \kcode{private} Clause}
\label{sec:fort_sa_private}
\index{clauses!private@\kcode{private}}
\index{private clause@\kcode{private} clause!storage association, Fortran}

The following non-conforming examples illustrate the implications of the \kcode{private} 
clause rules with regard to storage association. 

\pagebreak
\fnexample{fort_sa_private}{1}

\topmarker{Fortran}
\fnexample{fort_sa_private}{2}

\fnexample{fort_sa_private}{3}

\fnexample{fort_sa_private}{4}

\topmarker{Fortran}
\fnexample[5.1]{fort_sa_private}{5}
\end{fortranspecific}


%\pagebreak
\section{C/C++ Arrays in a \kcode{firstprivate} Clause}
\ccppspecificstart
\label{sec:carrays_fpriv}
\index{clauses!firstprivate@\kcode{firstprivate}}
\index{firstprivate clause@\kcode{firstprivate} clause!C/C++ arrays in}

The following example illustrates the size and value of list items of array or 
pointer type in a \kcode{firstprivate} clause. The size of new list items is 
based on the type of the corresponding original list item, as determined by the 
base language.

In this example:

\begin{compactitem}
\item The type of \ucode{A} is array of two arrays of two \bcode{int}s.

\item  The type of \ucode{B} is adjusted to pointer to array of \ucode{n} 
\bcode{int}s, because it is a function parameter.

\item  The type of \ucode{C} is adjusted to pointer to \bcode{int}, because 
it is a function parameter.

\item  The type of \ucode{D} is array of two arrays of two \bcode{int}s.

\item  The type of \ucode{E} is array of \ucode{n} arrays of \ucode{n} 
\bcode{int}s.
\end{compactitem}

Note that  \ucode{B} and \ucode{E} involve variable length array types.

The new items of array type are initialized as if each integer element of the original 
array is assigned to the corresponding element of the new array. Those of pointer 
type are initialized as if by assignment from the original item to the new item.

\cnexample{carrays_fpriv}{1}
\ccppspecificend



%\pagebreak
\section{\kcode{lastprivate} Clause}
\label{sec:lastprivate}
\index{clauses!lastprivate@\kcode{lastprivate}}
\index{lastprivate clause@\kcode{lastprivate} clause}

Correct execution sometimes depends on the value that the last iteration of a loop 
assigns to a variable. Such programs must list all such variables in a \kcode{lastprivate} 
clause  so that the values of the variables are the same as when the loop is executed 
sequentially.

\cexample{lastprivate}{1}

\fexample{lastprivate}{1}

\index{lastprivate clause@\kcode{lastprivate} clause!conditional modifier@\kcode{conditional} modifier}
\index{conditional modifier@\kcode{conditional} modifier}
The next example illustrates the use of the \kcode{conditional} modifier in
a \kcode{lastprivate} clause to return the last value when it may not come from
the last iteration of a loop.
That is, users can preserve the serial equivalence semantics of the loop.
The conditional lastprivate ensures the final value of the variable after the loop 
is as if the loop iterations were executed in a sequential order.

\cexample[5.0]{lastprivate}{2}

\ffreeexample[5.0]{lastprivate}{2}

\pagebreak

\section{Reduction}
\label{sec:reduction}

This section covers ways to perform reductions in parallel, task, taskloop, and SIMD regions.

\subsection{\code{reduction} Clause}
\label{subsec:reduction}
\index{clauses!reduction@\code{reduction}}
\index{reduction clause@\code{reduction} clause}
\index{reductions!reduction clause@\code{reduction} clause}

The following example demonstrates the \code{reduction} clause; note that some 
reductions can be expressed in the loop in several ways, as shown for the \code{max} 
and \code{min} reductions below:

\cexample[3.1]{reduction}{1}

\pagebreak

\ffreeexample{reduction}{1}

A common implementation of the preceding example is to treat it as if it had been 
written as follows:

\cexample{reduction}{2}

\fortranspecificstart
\ffreenexample{reduction}{2}

The following program is non-conforming because the reduction is on the 
\emph{intrinsic procedure name} \code{MAX} but that name has been redefined to be the variable 
named \code{MAX}.

\ffreenexample{reduction}{3}
% blue line floater at top of this page for "Fortran, cont."
\begin{figure}[t!]
\linewitharrows{-1}{dashed}{Fortran (cont.)}{8em}
\end{figure}

The following conforming program performs the reduction using the 
\emph{intrinsic procedure name} \code{MAX} even though the intrinsic \code{MAX} has been renamed 
to \code{REN}.

\ffreenexample{reduction}{4}

The following conforming program performs the reduction using 
\plc{intrinsic procedure name} \code{MAX} even though the intrinsic \code{MAX} has been renamed 
to \code{MIN}.

\ffreenexample{reduction}{5}
\fortranspecificend

%\pagebreak
The following example is non-conforming because the initialization (\code{a = 
0}) of the original list item \code{a} is not synchronized with the update of 
\code{a} as a result of the reduction computation in the \code{for} loop. Therefore, 
the example may print an incorrect value for \code{a}.

To avoid this problem, the initialization of the original list item \code{a} 
should complete before any update of \code{a} as a result of the \code{reduction} 
clause. This can be achieved by adding an explicit barrier after the assignment 
\code{a = 0}, or by enclosing the assignment \code{a = 0} in a \code{single} 
directive (which has an implied barrier), or by initializing \code{a} before 
the start of the \code{parallel} region.

\cexample[5.1]{reduction}{6}

\fexample[5.1]{reduction}{6}

The following example demonstrates the reduction of array \plc{a}.  In C/C++ this is illustrated by the explicit use of an array section \plc{a[0:N]} in the \code{reduction} clause.  The corresponding Fortran example uses array syntax supported in the base language.  As of the OpenMP 4.5 specification the explicit use of array section in the \code{reduction} clause in Fortran is not permitted.  But this oversight has been fixed in the OpenMP 5.0 specification.


\cexample[4.5]{reduction}{7}

\ffreeexample{reduction}{7}

\subsection{Task Reduction}
\label{subsec:task_reduction}
\index{clauses!task_reduction@\scode{task_reduction}}
\index{task_reduction clause@\scode{task_reduction} clause}
\index{reductions!task_reduction clause@\scode{task_reduction} clause}
\index{clauses!in_reduction@\scode{in_reduction}}
\index{in_reduction clause@\scode{in_reduction} clause}
\index{reductions!in_reduction clause@\scode{in_reduction} clause}

In OpenMP 5.0 the \code{task\_reduction} clause was created for the \code{taskgroup} construct, 
to allow reductions among explicit tasks that have an \code{in\_reduction} clause.

In the \plc{task\_reduction.1} example below a reduction is performed as the algorithm
traverses a linked list. The reduction statement is assigned to be an explicit task using
a \code{task} construct and is specified to be a reduction participant with 
the \code{in\_reduction} clause.
A \code{taskgroup} construct encloses the tasks participating in the reduction, and
specifies, with the \code{task\_reduction} clause, that the taskgroup has tasks participating
in a reduction.  After the \code{taskgroup} region the original variable will contain 
the final value of the reduction.

Note: The \plc{res} variable is private in the \plc{linked\_list\_sum} routine
and is not required to be shared (as in the case of a \code{parallel} construct
reduction).


\cexample[5.0]{task_reduction}{1}

\ffreeexample[5.0]{task_reduction}{1}

\index{reduction clause@\code{reduction} clause!task modifier@\code{task} modifier}
\index{task modifier@\code{task} modifier}
In OpenMP 5.0 the \code{task} \plc{reduction-modifier} for the \code{reduction} clause was
introduced to provide a means of performing reductions among implicit and explicit tasks.

The \code{reduction} clause of a \code{parallel} or worksharing construct may
specify the \code{task} \plc{reduction-modifier} to include explicit task reductions
within their region, provided the reduction operators (\plc{reduction-identifiers})
and variables (\plc{list items}) of the participating tasks match those of the
implicit tasks.

There are 2 reduction use cases (identified by USE CASE \#) in the \plc{task\_reduction.2} example below.  

In USE CASE 1 a \code{task} modifier in the \code{reduction} clause 
of the \code{parallel} construct is used to include the reductions of any 
participating tasks, those with an \code{in\_reduction} clause and matching 
\plc{reduction-identifiers} (\code{+}) and list items (\code{x}).  

Note, a \code{taskgroup} construct (with a \code{task\_reduction} clause) in not
necessary to scope the explicit task reduction (as seen in the example above). 
Hence, even without the implicit task reduction statement (without the C \code{x++\;}  
and Fortran \code{x=x+1} statements), the \code{task} \plc{reduction-modifier} 
in a \code{reduction} clause of the \code{parallel} construct
can be used to avoid having to create a \code{taskgroup} construct 
(and its \code{task\_reduction} clause) around the task generating structure.

In USE CASE 2 tasks participating in the reduction are within a
worksharing region (a parallel worksharing-loop construct).
Here, too, no \code{taskgroup} is required, and the \plc{reduction-identifier} (\code{+})
and list item (variable \code{x}) match as required.


\cexample[5.0]{task_reduction}{2}

\ffreeexample[5.0]{task_reduction}{2}


\subsection{Reduction on Combined Target Constructs}
\label{subsec:target_reduction}
\index{reduction clause@\code{reduction} clause!on target construct@on \code{target} construct}
\index{constructs!target@\code{target}}
\index{target construct@\code{target} construct}

When a \code{reduction} clause appears on a combined construct that combines 
a \code{target} construct with another construct, there is an implicit map 
of the list items with a \code{tofrom} map type for the \code{target} construct. 
Otherwise, the list items (if they are scalar variables) would be 
treated as firstprivate by default in the \code{target} construct, which 
is unlikely to provide the intended behavior since the result of the
reduction that is in the firstprivate variable would be discarded 
at the end of the \code{target} region.

In the following example, the use of the \code{reduction} clause on \code{sum1}
or \code{sum2} should, by default, result in an implicit \code{tofrom} map for
that variable. So long as neither \code{sum1} nor \code{sum2} were already
present on the device, the mapping behavior ensures the value for
\code{sum1} computed in the first \code{target} construct is used in the
second \code{target} construct.

\cexample[5.0]{target_reduction}{1}

\ffreeexample[5.0]{target_reduction}{1}
%\clearpage

In next example,  the variables \code{sum1} and \code{sum2} remain on the
device for the duration of the \code{target}~\code{data} region so that it is
their device copies that are updated by the reductions. Note the significance
of mapping \code{sum1} on the second \code{target} construct; otherwise, it
would be treated by default as firstprivate and the result computed for
\code{sum1} in the prior \code{target} region may not be used. Alternatively, a
\code{target}~\code{update} construct could be used between the two
\code{target} constructs to update the host version of \code{sum1} with the
value that is in the corresponding device version after the completion of the
first construct.

\cexample[5.0]{target_reduction}{2}

\ffreeexample[5.0]{target_reduction}{2}


\subsection{Task Reduction with Target Constructs}
\label{subsec:target_task_reduction}
\index{in_reduction clause@\scode{in_reduction} clause}
\index{constructs!target@\code{target}}
\index{target construct@\code{target} construct}

\index{clauses!enter@\code{enter}}
\index{enter clause@\code{enter} clause}

The following examples illustrate how task reductions can apply to target tasks
that result from a \code{target} construct with the \code{in\_reduction}
clause. Here, the \code{in\_reduction} clause specifies that the target task
participates in the task reduction defined in the scope of the enclosing
\code{taskgroup} construct. Partial results from all tasks participating in the
task reduction will be combined (in some order) into the original variable
listed in the \code{task\_reduction} clause before exiting the \code{taskgroup}
region. 

\cexample[5.2]{target_task_reduction}{1}

\ffreeexample[5.2]{target_task_reduction}{1}
\clearpage

\index{reduction clause@\code{reduction} clause!task modifier@\code{task} modifier}
\index{task modifier@\code{task} modifier}
In the next pair of examples, the task reduction is defined by a
\code{reduction} clause with the \code{task} modifier, rather than a
\code{task\_reduction} clause on a \code{taskgroup} construct. Again, the
partial results from the participating tasks will be combined in some order
into the original reduction variable, \code{sum}.

\cexample[5.2]{target_task_reduction}{2a}

\ffreeexample[5.2]{target_task_reduction}{2a}

\index{in_reduction clause@\scode{in_reduction} clause!with target construct@with \code{target} construct}
\index{constructs!target@\code{target}}
\index{target construct@\code{target} construct}
Next, the \code{task} modifier is again used to define a task reduction over
participating tasks. This time, the participating tasks are a target task
resulting from a \code{target} construct with the \code{in\_reduction} clause,
and the implicit task (executing on the primary thread) that calls
\code{host\_compute}. As before, the partial results from these participating
tasks are combined in some order into the original reduction variable.

\cexample[5.2]{target_task_reduction}{2b}

\ffreeexample[5.2]{target_task_reduction}{2b}


\subsection{Taskloop Reduction}
\label{subsec:taskloop_reduction}
\index{reduction clause@\code{reduction} clause!on taskloop construct@on \code{taskloop} construct}
\index{constructs!taskloop@\code{taskloop}}
\index{taskloop construct@\code{taskloop} construct}

In the OpenMP 5.0 Specification the \code{taskloop} construct
was extended to include the reductions.

The following two examples show how to implement a reduction over an array
using taskloop reduction in two different ways.
In the first
example we apply the \code{reduction} clause to the \code{taskloop} construct. As it was
explained above in the task reduction examples, a reduction over tasks is
divided in two components: the scope of the reduction, which is defined by a
\code{taskgroup} region, and the tasks that participate in the reduction. In this
example, the \code{reduction} clause defines both semantics. First, it specifies that
the implicit \code{taskgroup} region associated with the \code{taskloop} construct is the scope of the
reduction, and second, it defines all tasks created by the \code{taskloop} construct as
participants of the reduction. About the first property, it is important to note
that if we add the \code{nogroup} clause to the \code{taskloop} construct the code will be
nonconforming, basically because we have a set of tasks that participate in a
reduction that has not been defined.

\cexample[5.0]{taskloop_reduction}{1}
\ffreeexample[5.0]{taskloop_reduction}{1}

%In the second example, we are computing exactly the same
%value but we do it in a very different way. The first thing that we do in the
%\plc{array\_sum} function is to create a \code{taskgroup} region that defines the scope of a
%new reduction using the \code{task\_reduction} clause.
%After that, we specify that a task and also the tasks generated
%by a taskloop will participate in that reduction using the \code{in\_reduction} clause
%on the \code{task} and \code{taskloop} constructs, respectively. Note that
%we also added the \code{nogroup} clause to the \code{taskloop} construct. This is allowed
%because what we are expressing with the \code{in\_reduction} clause is different
%from what we were expressing with the \code{reduction} clause. In one case we specify
%that the generated tasks will participate in a previously declared reduction
%(\code{in\_reduction} clause) whereas in the other case we specify that we want to
%create a new reduction and also that all tasks generated by the taskloop will
%participate on it.

The second example computes exactly the same value as in the preceding \plc{taskloop\_reduction.1} code section,
but in a very different way.
First, in the \plc{array\_sum} function a \code{taskgroup} region is created 
that defines the scope of a new reduction using the \code{task\_reduction} clause.
After that, a task and also the tasks generated by a taskloop participate in 
that reduction by using the \code{in\_reduction} clause on the \code{task}
and \code{taskloop} constructs, respectively. 
Note that the \code{nogroup} clause was added to the \code{taskloop} construct.
This is allowed because what is expressed with the \code{in\_reduction} clause
is different from what is expressed with the \code{reduction} clause.
In one case the generated tasks are specified to participate in a previously 
declared reduction (\code{in\_reduction} clause) whereas in the other case
creation of a new reduction is specified and also all tasks generated 
by the taskloop will participate on it.

\cexample[5.0]{taskloop_reduction}{2}
\ffreeexample[5.0]{taskloop_reduction}{2}
%\clearpage

In the OpenMP 5.0 Specification, \code{reduction} clauses for the
\code{taskloop}~\code{ simd} construct were also added. 

\index{reduction clause@\code{reduction} clause!on taskloop simd construct@on \code{taskloop}~\code{simd} construct}
\index{combined constructs!taskloop simd@\code{taskloop}~\code{simd}}
\index{taskloop simd construct@\code{taskloop}~\code{simd} construct}
The examples below compare reductions for the \code{taskloop} and the \code{taskloop}~\code{simd} constructs.
These examples illustrate the use of \code{reduction} clauses within 
"stand-alone" \code{taskloop} constructs, and the use of \code{in\_reduction} clauses for tasks of taskloops to participate
with other reductions within the scope of a parallel region.

\textbf{taskloop reductions:}

In the \plc{taskloop reductions} section of the example below, 
\plc{taskloop 1} uses the \code{reduction} clause 
in a \code{taskloop} construct for a sum reduction, accumulated in \plc{asum}. 
The behavior is as though a \code{taskgroup} construct encloses the 
taskloop region with a \code{task\_reduction} clause, and each taskloop
task has an \code{in\_reduction} clause with the specifications 
of the \code{reduction} clause.
At the end of the taskloop region \plc{asum} contains the result of the reduction.

The next taskloop, \plc{taskloop 2}, illustrates the use of the 
\code{in\_reduction} clause to participate in a previously defined
reduction scope of a \code{parallel} construct.

The task reductions of \plc{task 2} and \plc{taskloop 2} are combined
across the \code{taskloop} construct and the single \code{task} construct, as specified
in the \code{reduction(task,}~\code{+:asum)} clause of the \code{parallel} construct.
At the end of the parallel region \plc{asum} contains the combined result of all reductions.

\textbf{taskloop simd reductions:}

Reductions for the \code{taskloop}~\code{simd} construct are shown in the second half of the code.
Since each component construct, \code{taskloop} and \code{simd}, 
can accept a reduction-type clause, the \code{taskloop}~\code{simd} construct
is a composite construct, and the specific application of the reduction clause is defined
within the \code{taskloop}~\code{simd} construct section of the OpenMP 5.0 Specification.
The code below illustrates use cases for these reductions.

In the \plc{taskloop simd reduction} section of the example below,
\plc{taskloop simd 3} uses the \code{reduction} clause 
in a \code{taskloop}~\code{simd} construct for a sum reduction within a loop.
For this case a \code{reduction} clause is used, as one would use 
for a \code{simd} construct.
The SIMD reductions of each task are combined, and the results of these tasks are further 
combined just as in the \code{taskloop} construct with the \code{reduction} clause for \plc{taskloop 1}.
At the end of the taskloop region \plc{asum} contains the combined result of all reductions.

If a \code{taskloop}~\code{simd} construct is to participate in a previously defined 
reduction scope, the reduction participation should be specified with
a \code{in\_reduction} clause, as shown in the \code{parallel} region enclosing
\plc{task 4} and \plc{taskloop simd 4} code sections.  

Here the \code{taskloop}~\code{simd} construct's 
\code{in\_reduction} clause specifies participation of the construct's tasks as 
a task reduction within the scope of the parallel region.  
That is, the results of each task of the \code{taskloop} construct component 
contribute to the reduction in a broader level, just as in \plc{parallel reduction a} code section above.
Also, each \code{simd}-component construct
occurs as if it has a \code{reduction} clause, and the
SIMD results of each task are combined as though to form a single result for
each task (that participates in the \code{in\_reduction} clause).
At the end of the parallel region \plc{asum} contains the combined result of all reductions.

%Just as in \plc{parallel reduction a} the
%\code{taskloop simd} construct reduction results are combined 
%with the \code{task} construct reduction results
%as specified by the \code{in\_reduction} clause of the \code{task} construct
%and the \plc{task} reduction-modifier of the \code{reduction} clause of 
%the \code{parallel} construct.
%At the end of the parallel region \plc{asum} contains the combined result of all reductions.


\cexample[5.1]{taskloop_simd_reduction}{1}

\ffreeexample[5.1]{taskloop_simd_reduction}{1}


\subsection{Reduction with the \code{scope} Construct}
\label{subsec:reduction_scope}
\index{reduction clause@\code{reduction} clause!on scope construct@on \code{scope} construct}
\index{constructs!scope@\code{scope}}
\index{scope construct@\code{scope} construct}

The following example illustrates the use of the \code{scope} construct 
to perform a reduction in a \code{parallel} region. The case is useful for 
producing a reduction and accessing reduction variables inside a \code{parallel} region 
without using a worksharing-loop construct.

\cppexample[5.1]{scope_reduction}{1}
\clearpage

\ffreeexample[5.1]{scope_reduction}{1}


\subsection{User-Defined Reduction}
\label{subsec:UDR}
\index{reductions!user-defined}
\index{reductions!declare reduction directive@\code{declare}~\code{reduction} directive}
\index{declare reduction directive@\code{declare}~\code{reduction} directive}
\index{directives!declare reduction@\code{declare}~\code{reduction}}
\index{declare reduction directive@\code{declare}~\code{reduction} directive!initializer clause@\code{initializer} clause}
\index{declare reduction directive@\code{declare}~\code{reduction} directive!combiner}
\index{declare reduction directive@\code{declare}~\code{reduction} directive!OpenMP variable identifiers}
\index{OpenMP variable identifiers!omp_in@\scode{omp_in}}
\index{OpenMP variable identifiers!omp_out@\scode{omp_out}}
\index{OpenMP variable identifiers!omp_priv@\scode{omp_priv}}
\index{combiner}
\index{clauses!initializer@\code{initializer}}
\index{initializer clause@\code{initializer} clause}

The \code{declare}~\code{reduction} directive can be used to specify 
user-defined reductions (UDR) for user data types.

%The following examples show how user-defined reductions can be used to support user data types in the \code{reduction} clause.

%The following example computes the enclosing rectangle of a set of points. The point data structure (\code{struct}~\code{point}) is not supported by the \code{reduction} clause. Using two \code{declare}~\code{reduction} directives we define how a reduction for the point data structure is done for the \plc{min} and \plc{max} operations. Each \code{declare}~\code{reduction} directive calls the appropriate function that passes the two special variables that can be used in the user-defined reduction expression: \code{omp\_in}, which holds one of the two values to reduce, and \code{omp\_out}, which holds the other value and should hold also the result of the reduction once the expression has been executed. Note, also, that when defining the user-defined reduction for \plc{min} we specify how the private variables of each thread are to be initialized (that is, the neutral value). This is not the case for \plc{max} as the default values (that is, zero filling) are already adequate.


In the following example, \code{declare}~\code{reduction} directives are used to define
\plc{min} and \plc{max} operations for the \plc{point} data structure for computing
the rectangle that encloses a set of 2-D points.

Each \code{declare}~\code{reduction} directive defines new reduction identifiers,
\plc{min} and \plc{max}, to be used in a \code{reduction} clause. The next item in the
declaration list is the data type (\plc{struct} \plc{point}) used in the reduction,
followed by the combiner, here the functions \plc{minproc} and \plc{maxproc} perform
the min and max operations, respectively, on the user data (of type \plc{struct} \plc{point}).
In the function argument list are two special OpenMP variable identifiers, \code{omp\_in} and \code{omp\_out},
that denote the two values to be combined in the "real" function;
the \code{omp\_out} identifier indicates which one is to hold the result.

The initializer of the \code{declare}~\code{reduction} directive specifies
the initial value for the private variable of each implicit task.
The \code{omp\_priv} identifier is used to denote the private variable.

\cexample[4.0]{udr}{1}
%\clearpage

The following example shows the corresponding code in Fortran. 
The \code{declare}~\code{reduction} directives are specified as part of 
the declaration in subroutine \plc{find\_enclosing\_rectangle} and 
the procedures that perform the min and max operations are specified as subprograms.

\ffreeexample[4.0]{udr}{1}


The following example shows the same computation as \plc{udr.1} but it illustrates that you can craft complex expressions in the user-defined reduction declaration. In this case, instead of calling the \plc{minproc} and \plc{maxproc} functions we inline the code in a single expression.

\cexample[4.0]{udr}{2}

The corresponding code of the same example in Fortran is very similar
except that the assignment expression in the \code{declare}~\code{reduction}
directive can only be used for a single variable, in this case through
a type structure constructor \plc{point($\ldots$)}.

\ffreeexample[4.0]{udr}{2}


\index{OpenMP variable identifiers!omp_orig@\scode{omp_orig}}
The following example shows the use of special variables in arguments for combiner (\code{omp\_in} and \code{omp\_out}) and initializer (\code{omp\_priv} and \code{omp\_orig}) routines.  This example returns the maximum value of an array and the corresponding index value. The \code{declare}~\code{reduction} directive specifies a user-defined reduction operation \plc{maxloc} for data type \plc{struct} \plc{mx\_s}. The function \plc{mx\_combine} is the combiner and the function \plc{mx\_init} is the initializer.

\cexample[4.0]{udr}{3}

Below is the corresponding Fortran version of the above example.  The \code{declare}~\code{reduction} directive specifies the user-defined operation \plc{maxloc} for user-derived type \plc{mx\_s}.  The combiner \plc{mx\_combine} and the initializer \plc{mx\_init} are specified as subprograms.

\ffreeexample[4.0]{udr}{3}


The following example explains a few details of the user-defined reduction 
in Fortran through modules. The \code{declare}~\code{reduction} directive is declared in a module (\plc{data\_red}). 
The reduction-identifier \plc{.add.} is a user-defined operator that is
to allow accessibility in the scope that performs the reduction
operation.
The user-defined operator \plc{.add.} and the subroutine \plc{dt\_init} specified in the \code{initializer} clause are defined in the same subprogram.

The reduction operation (that is, the \code{reduction} clause) is in the main program.
The reduction identifier \plc{.add.} is accessible by use association.
Since \plc{.add.} is a user-defined operator, the explicit interface
should also be accessible by use association in the current
program unit.
Since the \code{declare}~\code{reduction} associated to this \code{reduction} clause
has the \code{initializer} clause, the subroutine specified on the clause
must be accessible in the current scoping unit.  In this case,
the subroutine \plc{dt\_init} is accessible by use association.

\ffreeexample[4.0]{udr}{4}


The following example uses user-defined reductions to declare a plus (+) reduction for a C++ class. As the \code{declare}~\code{reduction} directive is inside the context of the \plc{V} class the expressions in the \code{declare}~\code{reduction} directive are resolved in the context of the class. Also, note that the \code{initializer} clause uses a copy constructor to initialize the private variables of the reduction and it uses as parameter to its original variable by using the special variable \code{omp\_orig}.

\cppexample[4.0]{udr}{5}

The following examples shows how user-defined reductions can be defined for some STL containers. The first \code{declare}~\code{reduction} defines the plus (+) operation for \plc{std::vector<int>} by making use of the \plc{std::transform} algorithm. The second and third define the merge (or concatenation) operation for \plc{std::vector<int>} and \plc{std::list<int>}. 
%It shows how the same user-defined reduction operation can be defined to be done differently depending on the specified data type.
It shows how the user-defined reduction operation can be applied to specific data types of an STL.

\cppexample[4.0]{udr}{6}


%\pagebreak
\section{\kcode{scan} Directive}
\label{sec:scan}
\index{directives!scan@\kcode{scan}}
\index{scan directive@\kcode{scan} directive}
\index{reduction clause@\kcode{reduction} clause!inscan modifier@\kcode{inscan} modifier}
\index{inscan modifier@\kcode{inscan} modifier}

The following examples illustrate how to parallelize a loop that saves 
the \emph{prefix sum} of a reduction. This is accomplished by using 
the \kcode{inscan} modifier in the \kcode{reduction} clause for the input 
variable of the scan, and specifying with a \kcode{scan} directive whether 
the storage statement includes or excludes the scan input of the present 
iteration (\ucode{k}).

\index{scan directive@\kcode{scan} directive!inclusive clause@\kcode{inclusive} clause}
\index{scan directive@\kcode{scan} directive!exclusive clause@\kcode{exclusive} clause}
\index{clauses!inclusive@\kcode{inclusive}}
\index{inclusive clause@\kcode{inclusive} clause}
\index{clauses!exclusive@\kcode{exclusive}}
\index{exclusive clause@\kcode{exclusive} clause}
Basically, the \kcode{inscan} modifier connects a loop and/or SIMD reduction to 
the scan operation, and a \kcode{scan} construct with an \kcode{inclusive} or 
\kcode{exclusive} clause specifies whether the ``scan phase'' (lexical block 
before and after the directive, respectively) is to use an \plc{inclusive} or 
\plc{exclusive} scan value for the list item (\ucode{x}).

The first example uses the \plc{inclusive} scan operation on a composite
loop-SIMD construct. The \kcode{scan} directive separates the reduction 
statement on variable \ucode{x} from the use of \ucode{x} (saving to array \ucode{b}).
The order of the statements in this example indicates that
value \ucode{a[k]} (\ucode{a(k)} in Fortran) is included in the computation of 
the prefix sum \ucode{b[k]} (\ucode{b(k)} in Fortran) for iteration \ucode{k}.

\cexample[5.0]{scan}{1}

\ffreeexample[5.0]{scan}{1}

The second example uses the \plc{exclusive} scan operation on a composite
loop-SIMD construct. The \kcode{scan} directive separates the use of \ucode{x} 
(saving to array \ucode{b}) from the reduction statement on variable \ucode{x}.
The order of the statements in this example indicates that
value \ucode{a[k]} (\ucode{a(k)} in Fortran) is excluded from the computation 
of the prefix sum \ucode{b[k]} (\ucode{b(k)} in Fortran) for iteration \ucode{k}.

\cexample[5.0]{scan}{2}

\ffreeexample[5.0]{scan}{2}

%\pagebreak
\section{\kcode{copyin} Clause}
\label{sec:copyin}
\index{clauses!copyin@\kcode{copyin}}
\index{copyin clause@\kcode{copyin} clause}
\index{directives!threadprivate@\kcode{threadprivate}}
\index{threadprivate directive@\kcode{threadprivate} directive}

The \kcode{copyin} clause is used to initialize threadprivate data upon entry 
to a \kcode{parallel} region. The value of the threadprivate variable in the primary
thread is copied to the threadprivate variable of each other team member.

\cexample{copyin}{1}

\fexample{copyin}{1}



%\pagebreak
\section{\kcode{copyprivate} Clause}
\label{sec:copyprivate}
\index{clauses!copyprivate@\kcode{copyprivate}}
\index{copyprivate clause@\kcode{copyprivate} clause}

The \kcode{copyprivate} clause can be used to broadcast values acquired by a single 
thread directly to all instances of the private variables in the other threads. 
In this example, if the routine is called from the sequential part, its behavior 
is not affected by the presence of the directives. If it is called from a \kcode{parallel} 
region, then the actual arguments with which \ucode{a} and \ucode{b} are associated 
must be private. 

\index{constructs!single@\kcode{single}}
\index{single construct@\kcode{single} construct}
The thread that executes the structured block associated with the \kcode{single} 
 construct broadcasts the values of the private variables \ucode{a}, \ucode{b}, 
\ucode{x}, and 
\ucode{y} from its implicit task's data environment to the data environments 
of the other implicit tasks in the thread team. The broadcast completes before 
any of the threads have left the barrier at the end of the construct.

\cexample{copyprivate}{1}

\fexample{copyprivate}{1}

\index{constructs!masked@\kcode{masked}}
\index{masked construct@\kcode{masked} construct}
In this example, assume that the input must be performed by the primary thread. 
Since the \kcode{masked} construct does not support the \kcode{copyprivate} clause, 
it cannot broadcast the input value that is read. However, \kcode{copyprivate} 
is used to broadcast an address where the input value is stored. 

\cexample[5.1]{copyprivate}{2}

\fexample[5.1]{copyprivate}{2}

Suppose that the number of lock variables required within a \kcode{parallel} region 
cannot easily be determined prior to entering it. The \kcode{copyprivate} clause 
can be used to provide access to shared lock variables that are allocated within 
that \kcode{parallel} region.

\cexample{copyprivate}{3}

\begin{fortranspecific}
\fnexample{copyprivate}{3}

Note that the effect of the \kcode{copyprivate} clause on a variable with the 
\bcode{allocatable} attribute is different than on a variable with the \bcode{pointer} 
attribute. The value of \ucode{A} is copied (as if by intrinsic assignment) and 
the pointer \ucode{B} is copied (as if by pointer assignment) to the corresponding 
list items in the other implicit tasks belonging to the \kcode{parallel} region. 

\fnexample{copyprivate}{4}
\end{fortranspecific}



\begin{cppspecific}[4ex]
\section{C++ Reference in Data-Sharing Clauses}
\label{sec:cpp_reference}
\index{clauses!data-sharing, C++ reference in}
\index{data-sharing clauses, C++ reference in}

C++ reference types are allowed in data-sharing attribute clauses as of OpenMP 4.5, except
for the \kcode{threadprivate}, \kcode{copyin} and \kcode{copyprivate} clauses.  
(See the \docref{Data-Sharing Attribute Clauses} section of the 4.5 OpenMP specification.)
When a variable with C++ reference type is privatized, the object the reference refers to is privatized in addition to the reference itself.
The following example shows the use of reference types in data-sharing clauses in the usual way.
Additionally it shows how the data-sharing of formal arguments with a C++ reference type on an orphaned task generating construct is determined implicitly. (See the \docref{Data-sharing Attribute Rules for Variables Referenced in a Construct} section of the 4.5 OpenMP specification.)


\cppnexample[4.5]{cpp_reference}{1}
\end{cppspecific}

\pagebreak
\section{Fortran \code{ASSOCIATE} Construct}
\fortranspecificstart
\label{sec:associate}
\index{ASSOCIATE construct, Fortran@\code{ASSOCIATE} construct, Fortran}

The following is an invalid example of specifying an associate name on a data-sharing attribute 
clause. The constraint in the Data Sharing Attribute Rules section in the OpenMP 
4.0 API Specifications states that an associate name preserves the association 
with the selector established at the \code{ASSOCIATE} statement. The associate 
name \plc{b} is associated with the shared variable \plc{a}. With the predetermined data-sharing 
attribute rule, the associate name \plc{b} is not allowed to be specified on the \code{private} 
clause.

\fnexample[4.0]{associate}{1}

In next example, within the \code{parallel} construct, the association name \plc{thread\_id} 
is associated with the private copy of \plc{i}. The print statement should output the 
unique thread number.

\fnexample[4.0]{associate}{2}

The following example illustrates the effect of specifying a selector name on a data-sharing 
attribute clause. The associate name \plc{u} is associated with \plc{v} and the variable \plc{v} 
is specified on the \code{private} clause of the \code{parallel} construct. 
The construct association is established prior to the \code{parallel} region. 
The association between \plc{u} and the original \plc{v} is retained (see the Data Sharing 
Attribute Rules section in the OpenMP 4.0 API Specifications). Inside the \code{parallel} 
region, \plc{v} has the value of -1 and \plc{u} has the value of the original \plc{v}.

\pagebreak
\ffreenexample[4.0]{associate}{3}

% blue line floater at top of this page for "Fortran, cont."
\begin{figure}[t!]
\linewitharrows{-1}{dashed}{Fortran (cont.)}{8em}
\end{figure}
\label{sec:associate_target}

\bigskip
The following example illustrates mapping behavior for a Fortran
associate name and its selector for a \scode{target} construct.

For the first 3 \scode{target} constructs the associate name \splc{a_aray} is
associated with the selector \splc{aray}, an array.  
For the \scode{target} construct of code block TARGET 1 just the selector
\splc{aray} is used and is implicitly mapped,
likewise for the associate name \splc{a_aray} in the TARGET 2 block.  
However, mapping an associate name and its selector is not valid for the same
\scode{target} construct.  Hence the TARGET 3 block is non-conforming.


In TARGET 4, the \splc{scalr} selector used in the \scode{target} region 
has an implicit data-sharing attribute of firstprivate since it is a scalar.
Hence, the assigned value is not returned.
In TARGET 5, the associate name \splc{a_scalr} is implicitly mapped and the
assigned value is returned to the host (default \scode{tofrom} mapping behavior).
In TARGET 6, the use of the associate name and its selector in the \scode{target}
region is conforming because the scalar firstprivate behavior of the selector 
and the implicit mapping of the associate name are allowed.  
At the end of the \scode{target} region only the 
associate name's value is returned to the host. 
In TARGET 7, the selector and associate name appear in
an explicit mapping for the same \scode{target} construct, 
hence the code block is non-conforming.

\ffreenexample[5.1]{associate}{4}
\fortranspecificend




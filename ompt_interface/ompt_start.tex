\pagebreak
\section{OMPT Start}
\label{sec:ompt_start}

There are three steps an OpenMP implementation takes to activate a tool.
This section explains how the tool and an OpenMP implementation interact to accomplish tool activation.

Step 1. \emph{Determine Whether to Initialize}
\begin{adjustwidth}{2.5em}{0pt}
A tool is activated by the OMPT interface when it returns a non-NULL pointer to an \code{ompt\_start\_tool\_result\_t} structure on a call to \code{ompt\_start\_tool} by the OpenMP implementation.
There are three ways that a tool can provide a definition of \code{ompt\_start\_tool} to an OpenMP implementation:
(1) Statically linking the tool's definition of \code{ompt\_start\_tool} into an OpenMP application.
(2) Introducing a dynamically linked library that includes the tool's definition of
\code{ompt\_start\_tool} into the application's address space.
(3) Providing the name of a dynamically linked library appropriate for the architecture
and operating system used by the application in the \code{tool-libraries-var} ICV.
\end{adjustwidth}

Step 2. \emph{Initializing a First-Party tool}
\begin{adjustwidth}{2.5em}{0pt}
If a tool-provided implementation of \code{ompt\_start\_tool} returns a non-NULL pointer
to an \code{ompt\_start\_tool\_result\_t} structure, the OpenMP implementation will invoke
the tool initializer specified in this structure prior to the occurrence of any OpenMP event.
\end{adjustwidth}


Step 3. \emph{Monitoring Activity on the Host}
\begin{adjustwidth}{2.5em}{0pt}
To monitor execution of an OpenMP program on the host device, a tool's initializer
must register to receive notification of events that occur as an OpenMP program executes.
A tool can register callbacks for OpenMP events using the runtime entry point known
as \code{ompt\_set\_callback}, which has the following possible return codes: \hfill \break
 \code{ompt\_set\_error},
 \code{ompt\_set\_never},
 \code{ompt\_set\_impossible},
 \code{ompt\_set\_sometimes},
 \code{ompt\_set\_sometimes\_paired},
 \code{ompt\_set\_always}.

If the \code{ompt\_set\_callback} runtime entry point is called outside a tool’s initializer,
registration of supported callbacks may fail with a return code of \code{ompt\_set\_error}.
All callbacks registered with \code{ompt\_set\_callback} or returned by \code{ompt\_get\_callback}
use the dummy type signature \code{ompt\_callback\_t}. While this is a compromise, it is
better than providing unique runtime entry points with precise type signatures
to set and get the callback for each unique runtime entry point type signature.
\end{adjustwidth}

----------------

To use the OMPT interface a tool must provide a globally-visible implementation
of the \code{ompt\_start\_tool} function.
The function returns a pointer to an \code{ompt\_start\_tool\_result\_t} structure 
that contains callback pointers for tool initialization and finalization as well 
as a data word, \plc{tool\_data}, that is to be passed by reference to these callbacks.
A \code{NULL} return indicates the tool will not use the OMPT interface.
The runtime execution of \code{ompt\_start\_tool} is triggered by the first OpenMP 
directive or OpenMP API routine call.


In the example below, the user-provided \code{ompt\_start\_tool} function
performs a check to make sure the runtime OpenMP version that OMPT supports 
(provided by the \texttt{omp\_version} argument) is identical to the 
OpenMP implementation (compile-time) version.
Also, a \code{NULL} is returned to indicate that the OMPT interface is not
used (no callbacks and tool data are specified). 

\emph{Note}: The \texttt{omp-tools.h} file is included.

\cexample[5.0]{ompt_start}{1}

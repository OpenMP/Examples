\section{Examples Organization}
\label{chap:examples}
\label{sec:examples}
\index{example label}
\index{example label!omp_verno@\scode{omp_}\plc{verno}}
\index{omp_verno@\scode{omp_}\plc{verno}}

This document includes examples of the OpenMP API directives, constructs, and routines.

Each example is labeled as \plc{ename.seqno.ext}, where \plc{ename} is 
the example name, \plc{seqno} is the sequence number in a section, and 
\plc{ext} is the source file extension to indicate the code type and 
source form.  \plc{ext} is one of the following:
\begin{description}[noitemsep,labelindent=5mm,widest=f90]
\item[\plc{c}] -- \ C code,
\item[\plc{cpp}] -- \ C++ code,
\item[\plc{f}] -- \ Fortran code in fixed form, and
\item[\plc{f90}] -- \ Fortran code in free form.
\end{description}

Some of the example labels may include version information 
(\code{\small{}omp\_\plc{verno}}) to indicate features that are illustrated
by an example for a specific OpenMP version, such as ``\plc{scan.1.c} 
\;(\code{\small{}omp\_5.0}).''

\ccppspecificstart
A statement following a directive is compound only when necessary, and a 
non-compound statement is indented with respect to a directive preceding it.
\ccppspecificend

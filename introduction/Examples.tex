\section{Examples Organization}
\label{chap:examples}
\label{sec:examples}
\index{example label}
\index{example label!omp_verno@\kcode{omp_\plc{verno}}}
\index{omp_verno@\kcode{omp_\plc{verno}}}

This document includes examples of the OpenMP API directives, constructs, and routines.

Each example is labeled with \plc{ename.seq-id.ext}, where \plc{ename} is 
the example name, \plc{seq-id} is the sequence identifier in a section, and 
\plc{ext} is the source file extension to indicate the code type and 
source form.  \plc{ext} is one of the following:
\begin{description}[noitemsep,labelindent=5mm,widest=f90]
\item[\plc{c}] -- \ C code,
\item[\plc{cpp}] -- \ C++ code,
\item[\plc{f}] -- \ Fortran code in fixed form, and
\item[\plc{f90}] -- \ Fortran code in free form.
\end{description}

Example labels include version information of the form
\verlabel{\plc{verno}} to indicate features that are illustrated
by an example for a specific OpenMP version, such as 
``\example{scan.1.c} \;\verlabel{5.0}.''
Some of the example labels include version information of the form
\verlabel[pre\_]{3.0} to indicate features that are specified 
prior to OpenMP version 3.0, such as
``\example{ploop.1.c} \;\verlabel[pre\_]{3.0}.''

Language markers may be used to indicate text or codes that are specific 
to a particular base language.
\ccppspecificstart
This is C/C++ specific: 
A statement following a directive is compound only when necessary, and a 
non-compound statement is indented with respect to a directive preceding it.
\ccppspecificend
\fortranspecificstart
This is Fortran specific...
\fortranspecificend

Throughout the examples document we assume that the number of threads 
used for a \kcode{parallel} region is the same as 
the number of threads requested, unless explicitly specified otherwise.

